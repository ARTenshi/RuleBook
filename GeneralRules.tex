%% %%%%%%%%%%%%%%%%%%%%%%%%%%%%%%%%%%%%%%%%%%%%%%%%%%%%%%%%%%%%%%%%%%%%%%%%%%%
%%
%%          $Id: general_rules.tex 420 2013-04-08 15:30:35Z holz $
%%    author(s): RoboCupAtHome Technical Committee(s)
%%  description: description of the GENERAL RULES
%%
%% %%%%%%%%%%%%%%%%%%%%%%%%%%%%%%%%%%%%%%%%%%%%%%%%%%%%%%%%%%%%%%%%%%%%%%%%%%%
\chapter{General Rules \& Regulations}
\label{chap:rules}

These are the general rules and regulations for the competition in the RoboCup@Home league.
Every rule in this section can be considered to implicitly include the
term \emph{\enquote{unless stated otherwise}}, meaning that additional or contrary rules in particular
test specifications have a higher priority than those mentioned herein in the general rules and regulations.

%%%%%%%%%%%%%%%%%%%%%%%%%%%%%%%%%%%%%%%%%%%%%%%%%%%%%%%%%
\section{Team Registration and Qualification}


\subsection{Registration and Qualification Process}
\label{rule:participation}

Each year there are four phases in the process toward participation:
\begin{enumerate}
	\item \iterm{Intention of Participation} (optional)
	\item \iterm{Preregistration}
	\item \iterm{Qualification} announcements
	\item Final \iterm{Registration} for qualified teams
\end{enumerate}
Positions 1 and 2 will be announced by a call on the \iterm{RoboCup@Home mailing list}. Preregistration requires a \iterm{team description paper}, a \Term{video}{qualification video} and a \Term{website}{Team Website}.

\subsection{Qualification Video}
As a proof of running hardware, each team has to provide a \iterm{qualification video} showing at least two from the following abilities (minimum requirement):
\begin{itemize}
	\item Human-Robot interaction
	\item Navigation (safe, indoors with obstacle avoidance).
	\item Object detection \& manipulation.
	\item People detection
	\item Speech recognition.
	\item speech synthesis (clear and loud).
\end{itemize}

Showing some of the following abilities is recommended:
\begin{itemize}
	\item Activity recognition
	\item Complex speech recognition
	\item Complex action planning
	\item Gesture recognition
\end{itemize}


Videos should be self-explicative and designed for a general audience, showing the  robot solving complex tasks. The minimum to qualify requires proving the robot is able to solve successfully at least one test of the current or last year's rulebook. For robots moving slowly, we suggest to speed-up videos. When doing so, please specify the speed factor being used (e.g.~2x, 5X, 10X). The same applies for slow motion scenes.

Please notice that the videos should not last longer than the average time for a test (max.~\SI{10}{\minute}).

\paragraph{Important note to Standard Platform Leagues:} The qualification video must show an unmodified robot in normal operation (See~\refsec{rule:spl-mods}).

\subsection{Team Website}

The \iterm{Team Website} should be designed for a broader audience, but also including scientific material and access to open source code being developed. Requirements are as follows:

\begin{enumerate}

	\item \textbf{Multimedia: } Please include as many photos and videos of the robot(s) as possible.

	\item \textbf{Language: } The team website has to be in English. Other languages may be also available, but English must be default language.

	\item \textbf{Team: } List of the team members including brief profiles.

	\item \textbf{RoboCup:} Link to the league website and previous participation of the team in RoboCup.

	\item \textbf{Scientific approach: } The team website has to include research lines, description of the approaches, and information on scientific achievements.

	\item \textbf{Publications: } Relevant \iterm{publications} from 5 years up to date. Downloadable publications are scored higher during the qualification process.

	\item \textbf{Open source material: } Blueprints, designs, repositories or any kind of contribution to the league is highly scored during qualification process.
\end{enumerate}


\subsection{Team Description Paper}
\label{rule:website_tdp}
The \iaterm{team description paper}{TDP} is an 8-pages long scientific paper which must have a explained description of your main research, including the scientific contribution, goals, scope, and results.

Preferably, it should also contain the following:
\begin{itemize}
	\item the focus of research and the contributions in the respective fields,
	\item innovative technology (if any),
	\item re-usability of the system for other research groups
	\item applicability of the robot in the real world
	\item photo(s) of the robot(s)
\end{itemize}

~\\\noindent As addendum in the 9th page (after references) please include:
\begin{itemize}
	\item Team name
	\item Contact information
	\item Website url
	\item Team members' names
	\item photo(s) of the robot(s), unless included before.
	\item description of the hardware used
	\item Brief, compact list of \iterm{external devices} (See~\refsec{rule:robot_external_computing}), if any.
	\item Brief, compact list of 3rd party reused software packages (e.g.~ROS' \texttt{object\_recognition} should be listed, but not OpenCV).
	\item \textbf{[Open Platform League only]} Brief description of the hardware used by the robot(s).
\end{itemize}

~\\\noindent The TDP has to be in English, up to eight pages in length and formatted according to the guidelines of the RoboCup International Symposium without altering margins or spacing. It goes into detail about the technical and scientific approach.

Please notice that, during qualification process, TDP will be scored by its scientific value, novelty and contributions.


%% %%%%%%%%%%%%%%%%%%%%%%%%%%%%%%%%%%%%%%%
\subsection{Qualification}
\label{rule:qualification}

During the \iterm{qualification process} a selection will be made by the \iaterm{Organizing Committee}{OC} Taken into account and evaluated in this decision process are:
\begin{itemize}
	\item The content on the team website, scoring higher publications and open source resources;
	\item the number of abilities shown in the qualification video,
	\item the complexity of the tasks shown in the qualification video, and
	\item the scientific value, novelty and contributions in the \iterm{team description paper}. %, and
	% \item the information in the \iterm{RoboCup\char64Home Wiki} (added by the team).
\end{itemize}
(Additional) evaluation criteria are:
\begin{itemize}
	\item the performance in previous competitions,
	\item the relevant scientific contributions and publications, and
	\item the contributions to the RoboCup@Home league.
\end{itemize}

\paragraph{Important note to Standard Platform Leagues:} Only unmodified robots may compete in Standard Platform Leagues. Any \textit{slight} modification made to the robot found in the Qualification Material will automatically disqualify the team, for which registration to the international competition will not be possible  (See~\refsec{rule:spl-mods}).

% For getting considered in the evaluation, be sure to insert your team's name when adding information to the \iterm{RoboCup\char64Home Wiki}.


% Local Variables:
% TeX-master: "../Rulebook"
% End:


\section{Audience interaction}
Direct interaction with the audience is not a part of most challenges, though some explicitly require it in an effort to make robots step out of the laboratory.

Informing the audience however is important for the league.
\subsection{Vizbox}
\label{vizbox}

The objective of RoboCup is to \quotes{promote robotics and AI research, by offering a publicly appealing, but formidable challenge} \footnote{\url{http://robocup.org/objective}}.

Part of making RoboCup@Home appealing, is to show the audience what is going on, what the robots should do and what they are doing.

To this end, robots in RoboCup@Home are expected run the RoboCup@Home \href{https://github.com/LoyVanBeek/vizbox}{VizBox}\footnote{\url{https://github.com/LoyVanBeek/vizbox}}.

This is a web server to be run on a robot during a challenge. The page it serves can be displayed on a screen, visible to the audience, via a secondary computer in or around the arena, connected to the web server via the wireless network.

All robots are expected to run the \iterm{VizBox}; the audience expects to know what all the robots are doing and what each challenge entails.

The \iterm{VizBox}'s code is hosted \url{https://github.com/LoyVanBeek/vizbox}.
We want to show the audience a consistent presentation, so ideally, all teams run the same VizBox code.
Sharing your changes back in the form of a Pull Request is much appreciated so all teams can benefit.

The \iterm{VizBox} has the following visualization capabilities:
\begin{itemize}
	\item Images of what the robot sees or a visualization of the robot's world model, eg. camera images, it's map, anything to make clear what is going on to the audience.
	\item Show an outline of the current challenge and where the robot is in the story of the current challenge.
	\item Subtitles of what the robot and operator just said; their conversation
\end{itemize}

Additionally, the \iterm{VizBox} offers a way to \textbf{input} a text command to the robot, to bypass automatic speech recognition if need be.

The exact documentation is maintained in the repository of the \iterm{VizBox} itself.

%%%%%%%%%%%%%%%%%%%%%%%%%%%%%%%%%%%%%%%%%%%%%%%%%%%%%%%%%
\section{Scenario}
\label{sec:scenario}

The tests take place in the \iterm{RoboCup@Home arena}. In addition, particular tests are situated outside the arena, e.g., in a previously unknown public place. The following rules are related to the \iterm{RoboCup@Home arena} and its contents. 

\subsection{RoboCup@Home arena}
The \iterm{RoboCup@Home arena} is a realistic home setting (apartment) consisting of inter-connected rooms like, for instance, a living room, a kitchen, a bath room, and a bed room. 
Depending on the Local Organization, there may be multiple apartments which may be different to each other.
Robot must be prepared to perform any task in any arena, not the same arena every time. 

The arena is decorated and dressed to resemble a home in which one could live, with as much of the necessities and decorations one might find in a normal home. 
Please do note that what is considered as \quotes{normal} may greatly vary by culture and on the location where the RoboCup event is hosted. 
For some examples on items one may find in the arena, see \refsec{chap:arena-decorations-appendix}

% \subsection{Team area}\label{rule:scenario_team_area}

% \todo{remove? does not depend on the rules, but on local organization }
% The maximum number of people to register per team is unlimited, but
% the organization only provides space for \emph{four} (4) persons to
% work at tables in the team area. 
% \todo{this is actually more an additional note for the registration information}

\subsection{Walls, doors and floor}
\label{rule:scenario_walls}

The indoor home setting will be surrounded by high and low \Term{walls}{Arena walls}. These walls will be built up using standard fair construction material.

\begin{enumerate}
	\item \textbf{Walls:} Walls have a minimum height of \SI{60}{\centi\meter}. A maximum height is not specified, but should be chosen so that the audience is able to watch the competition.\\
	Walls will be fixed and are likely to be not modified during the competition (see \refsec{rule:scenario_changes}). 

	\item \textbf{Doors:} There will be at least two entry/exit \Term{doors}{Arena doors} connecting the outside of the scenario. These doors are used as starting points for the robots (see \refsec{rule:start_position}).
	% At least one of the entrances will be a door with a handle (not a knob).\
	There will be also another door inside the scenario with a handle (not a knob) between any two rooms. Doors with handle (not a knob) may be closed at any time, it is expected robots be able to open them.

	\item \textbf{Floor:} The floor of the arena as well as the doorways of the arena are even. That is, there will be no significant steps or even stairways. However, minor unevenness such as carpets, transitions in floor covering between different areas, and minor gaps (especially at doorways) must be expected.

	\item \textbf{Appearance:} Floor and walls are mainly uni-colored but can contain texture, e.g., a carpet on the floor, or a poster or picture on the wall.\\
	Although being unlikely at the moment, transparent elements are also possible. 
\end{enumerate}


\subsection{Furniture}
\label{rule:scenario_furniture}

The arena will be equipped with typical objects (furniture) that are not specified in quantity and kind. The minimal configuration consists of 
\begin{itemize}
	\item a small dinner table with two chairs, 
	\item a couch, 
	\item an open cupboard or small table with a television and remote control, 
	\item a cupboard or shelf (with some books inside), and
	\item a refrigerator in the kitchen (with some cans and plastic bottles inside). 
\end{itemize}
A typical arena setup is shown in \reffig{fig:scenario_arena}.

\begin{figure}[tbp]
	\centering
	\subfloat[Typical arena]{\label{fig:scenario_arena}\includegraphics[height=46mm]{images/typical_arena.jpg}} ~ 
	\subfloat[Typical objects]{\label{fig:scenario_objects}\includegraphics[height=46mm]{images/typical_objects.jpg}}
	\caption{Scenario examples: (a) a typical arena, and (b) typical objects.}
	\label{fig:arena}
\end{figure}



\subsection{Changes to the arena}
\label{rule:scenario_changes}

Since the robots should be able to function in the real world the scenario is not fixed and might change without further notice.
\begin{enumerate}
	\item \textbf{Major changes:} 
	The arena is meant to be a simulated apartment. 
	The furniture might be moved around between tests. 
	This includes furniture that is a named location (see \refsec{rule:scenario_names}).
	As in a normal home, furniture is not very likely to move from one room to another and is unlikely to be moved to the other side of a room.
	However, a couch or table may be rotated, moved to its side etc. 
	Walls will stay in place and rooms will not change function.
	Passages might be blocked and cleared. 
	One hour before a test slot begins no \iterm{major changes} will be made.
	This time will be shortened in the future. 
	\item \textbf{Minor changes:} In contrast to major changes, \iterm{minor changes} like, for instance, slightly moved chairs cannot be avoided and may happen at any time (even during a test). 
\end{enumerate}


%%%%%%%%%%%%%%%%%%%%%%%%%%%%%%%%%%%%%%%%%%%%%%%%%%%%%%%%%%%%%%%%%%
%
% Objects section.
%
% Revisited by Mauricio Matamoros for 2015
%
%%%%%%%%%%%%%%%%%%%%%%%%%%%%%%%%%%%%%%%%%%%%%%%%%%%%%%%%%%%%%%%%%%
\def\NumObjects{10\ }
\def\NumLocations{20\ }
\def\NumNames{20\ }

\subsection{Objects}
\label{rule:scenario_objects}
Some tests in the RoboCup@Home league involve object manipulation and recognition. These \iterm{objects} resemble items usually found in household environments like, for instance, soda cans, coffee mugs or books. An example of objects used in a previous competition can be seen in \reffig{fig:scenario_objects}.

Objects are divided in five main groups:

\begin{enumerate}
	\item \textbf{\iterm{Known objects}:} Objects with no noticeable difference among peers. \textit{Known objects} tend to be artificial and regular shaped, such as coke cans, beer bottles, cereal boxes, etc.~A set of copies of these objects is provided before the competition for training.

	\item \textbf{\iterm{Alike objects}:} Objects with slight differences among peers (e.g.~color, size, shape). \textit{Alike objects} tend to be natural and similar to each other, but not equal; for example: apples, bananas, rags, etc.~A specimen of these objects is provided before the competition for training.

	\item \textbf{\iterm{Containers}:} Objects which can contain, transport or be filled with other objects or their content, such as baskets, bowls, bags, trays, etc.~. As with \textit{known objects}, \textit{containers} are known beforehand with no noticeable difference among peers, and a copy is provided before the competition for training.

	\item \textbf{\iterm{Special objects}:} Objects require a proper identification and special handling (not necesarily grasping), operation or interaction for accomplishing a particular task. Examples of special objects are: door handles, chairs, walking sticks, poles, etc.~Notice that a copy of these objects may not be available beforehand for previous training.

	\item \textbf{\iterm{Unknown objects}:} Any other object that is not known beforehand but can be grasped or handled.
\end{enumerate}

The following general rules for objects apply:

\begin{enumerate}
	\item \textbf{Object category:} Each object will be assigned to an \iterm{object category}. The objects \quotes{apple} and \quotes{banana} may be of class \quotes{fruits} for example.

	\item \textbf{Object (category) locations:} An \iterm{object location} will be assigned to each \iterm{object category}. For example, objects categorized as \quotes{fruits} may be usually found on the \quotes{kitchen table}, and unknown objects \quotes{unknown} may be usually found in the \quotes{trash bin}.

	\item \textbf{Announcement:} The TC makes the set of \iterm{objects}, including their names, categories, and usual locations; available during the setup days. 
	
	\item \textbf{Placement:} \nterm{object placement} Unless stated otherwise, in manipulation tasks, the objects will be positioned at \iterm{manipulation locations} and less than \SI{15}{\centi\meter} away from the border of the surface they are located at. There will be at least \SI{5}{\centi\meter} space around each object.
\end{enumerate}

\paragraph*{Important note:} It is not allowed to modify any of the objects provided for training. Also, teams are not allowed to keep more than 5 the objects provided for training at a time nor retaining it for more than one hour.

\subsubsection{Containers}
The TC will provide at least three different types of containers to be used in the tests.

\begin{itemize}
	\item \textbf{Pouring containers:} Such as a bowls, glasses, or other objects in which liquids and grains can be poured.

	\item \textbf{Storage containers:} Such as bags or boxes in which objects can be stored or retrieved.\\
	Bags used during the competition are rigid and with clearly visible standing handles; more likely made of paper and in bright colors (See Figure \ref{fig:scenario_container_bag}).

	\item \textbf{Transport containers:} Such as trays in which objects can be neatly arranged for transport.
\end{itemize}

Although there are no restrictions on a container size, appearance or weight; however, it can be expected that the selected containers be lightweight, with handles, and easily manipulable by a human using either one or both hands.

\begin{figure}[H]
	\centering
	\subfloat[Bright-colored paper bags]{
		\label{fig:scenario_container_bag}\includegraphics[width=0.33\textwidth]{images/container_paper_bag.png}}~
	\subfloat[Cereal bowls]{
		\label{fig:scenario_container_bowl}\includegraphics[width=0.33\textwidth]{images/container_bowl.png}}~
	\subfloat[Serving tray]{
		\label{fig:scenario_container_tray}\includegraphics[width=0.33\textwidth]{images/container_tray.png}}
	\caption{Example of object containers}
	\label{fig:scenario_containers}
\end{figure}

\paragraph*{Custom containers.}
\label{rule:custom_containers}
It is allowed that a team provide a \iterm{custom container} adapted to be used by the robot, considering the following:
\begin{enumerate}
	\item Custom containers must be approved by the TC during during the \iterm{Robot Inspection} (see \refsec{sec:robot_inspection}).
	\item Custom containers must \emph{not} have any kind of artificial marks, sensors, or electronic devices.
	\item Penalties may apply for the use of custom containers. The TC may establish special penalties during the \iterm{Robot Inspection}. The default penalties applicable to any task involving a container are as follows.
	\begin{itemize}
		\item Special color on an otherwise unmodified two-hand manipulable container: 75\% of the points.
		\item Special color on an otherwise unmodified single-hand manipulable container: 50\% of the points.
		\item Specially designed or adapted two-hand manipulable container (e.g.~special handles): 50\% of the points.
		\item Specially designed or adapted single-hand manipulable container (e.g.~special handle): 25\% of the points.
		\item Two-hand manipulable container adapted to be used \textit{single-handed}: 25\% of the points.
		\item On-robot mounted container: 0 points.
	\end{itemize}
	\textbf{Notes:} Trays are considered two-hand manipulable containers, while most bowls and dishes are considered single-hand manipulable container unless they are too big. Color patterns are allowed as long as they look natural (e.g.~\textit{barber sign colored} handles are allowed, but black and white bar-code like handles are not). Penalties does not stack, the most meaningful modification is considered. 
\end{enumerate}

\subsubsection{Predefined objects}
The TC will compile a list of at least \NumObjects objects (including both \iterm{known objects} and \iterm{alike objects}) which will be available for training. There are no restrictions on an object size, appearance or weight; however, it can be expected that the selected objects are easily manipulable by a human using a single hand.

Note that, any object not previously announced by the TC is automatically considered an unknown object for scoring purposes (e.g.~ornamentation).

%%%%%%%%%%%%%%%%%%%%%%%%%%%%%%%%%%%%%%%%%%%%%%%%%%%%%%%%%%%%%%%%%%
%
% Predefined locations section.
%
%%%%%%%%%%%%%%%%%%%%%%%%%%%%%%%%%%%%%%%%%%%%%%%%%%%%%%%%%%%%%%%%%%

\subsection{Predefined locations}
\label{rule:scenario_locations}

Some tests in the RoboCup@Home league involve \iterm{predefined locations}. 
These may include places like a \quotes{bookshelf} or a \quotes{dining table}, as well as certain objects such as a \quotes{television}, or the \quotes{front door}. 

\begin{enumerate}
	\item \textbf{Definition:} The TC will compile a list of predefined locations. There are no restrictions on which parts of the arena will be selected as a predefined location.

	\item \textbf{Location classes:} Each location will be assigned to a \iterm{location class}. The objects \quotes{couch} and \quotes{arm chair} may be of class \quotes{seat} for example. 

	\item \textbf{Announcement:} The TC makes the set of locations (and their names and classes) available during the setup days.

	\item \textbf{Position:} The positions of locations are \emph{not} necessarily fixed (see \refsec{rule:scenario_changes}).

	\item \textbf{Manipulation locations:} The TC will mark at least \NumLocations locations out of the set of predefined locations as being \iterm{manipulation locations}. Whenever a test involves manipulation, the object to manipulate will be placed at one of the manipulation locations. 
\end{enumerate}



\subsection{Predefined rooms}
\label{rule:scenario_rooms}
Some tests in the RoboCup@Home league involve \iterm{predefined rooms}. 
\begin{enumerate}
	\item \textbf{Definition:} The TC will compile a list of room names.
	\item \textbf{Announcement:} The TC makes the set of rooms available during the setup days.
\end{enumerate}



\subsection{Predefined (person) names}\label{rule:scenario_names}

Some tests in the RoboCup@Home league involve \iterm{predefined names} of people. 

\begin{enumerate}
	\item \textbf{Definition:} The TC will compile a list of \NumNames predefined names. The names are \SI{50}{\percent} male and \SI{50}{\percent} female, and taken from the (current) most common first names in the United States.\\
	In order to ease speech recognition, it is tried to select names to be phonetically different from each other.

	\item \textbf{Announcement:} The TC makes the set of names available during the setup days.
	\item \textbf{Assignment:} When a test involves interacting with persons (using a person's name), all involved persons are assigned names by the referees before the test. 
\end{enumerate}

Typical names are, for example, James, John, Robert, Michael and William as male names; Mary, Patricia, Linda, Barbara and Elizabeth as female names.

% MAURICIO @2017
% Separated file for better control
%% %%%%%%%%%%%%%%%%%%%%%%%%
\subsection{Wireless network}
\label{rule:scenario_wifi}

For wireless communication, an \iterm{arena network} is provided. The actual infrastructure depends on the local organization.

\begin{itemize}
	\item To avoid interference with other leagues, this \iterm{arena network} has to be used for communication only. It is not allowed to use the above or any other WiFi network for personal use at the venue.
	\item During the competitions, only the active team is allowed to use the \iterm{arena network}.
	\item The organizers cannot guarantee reliability and performance of wireless communication. Therefore, teams are required to be ready to setup, start their robots and run the tests even if, for any reason, network is not working properly.
\end{itemize}

Preferred situation:
\begin{itemize}
	\item The \iterm{arena network} consists of of several Virtual Local Area Networks (VLANs), one for each team.
	\item The traffic from the robot inside the arena is separated into the corresponding team's VLAN as soon as possible, e.g.~at the wireless acces point.This may require that each team has it's own SSID, each of which gets routed into the corresponding VLAN. Each team has a network cable routed to their team area, which is also connected to the teams VLAN. On this cable, the team can set up their own router/switch/hub etc. which will be inside the team's VLAN. This way, one team's traffic and devices are completely separated from any other team, while any team can set up their own DHCP server etc. if they desire.
	\item An Internet connection is preferably also available for every team.
\end{itemize}
Each team has to bring its own LAN hub/switch and cables for routing inside the team area.

In case the \iterm{arena network} is not functioning at the end of the first setup day, teams are allowed to set up their own networking equipment and wireless networks.

\paragraph*{Important note:} Different countries have different regulations for wireless equipment and the \iterm{arena network} has to obey these.
It is up to the teams to have networking equipment that also adheres to these regulations. For example, if due to local regulations various WiFi channels are prohibited, it is a team's responsibility to be able to use different, allowed channels.

\paragraph*{Important note:} Any unapproved wireless device may be removed by the TC at any time.

% Local Variables:
% TeX-master: "../Rulebook"
% End:

% MAURICIO @2017
% We are not really using SmartHome devices anymore.
%\subsection{Smart Home Devices}
\label{rule:smarthomedevices}

The Organizing and Technical Committees in coordination with the Local Organization will compile a list of \iterm{Smart~Home} official devices that will be available in the arena and can be used in some tests for additional score.

At any time, only the Smart~Home devices provided by the Local Organization and approved by the Technical Committee may be used during competition.

\subsubsection{Smart~Home devices list announcement}
The list if Smart~Home devices will be provided to teams as soon as it becomes available and has been granted by the Local Organization and approved by the Technical Committee.

This list must be announced at least one month prior the competition. In case that this list does not become available for that date, Smart~Home devices may still be present at the arena for testing, but no additional score can be achieved for its use. This is to maintain fair conditions among all teams.

\subsubsection{Technical specifications}
The list of \iterm{Smart~Home} official devices will include as much technical information as possible. However, before it becomes available you may assume the following considerations:

\begin{enumerate}
	\item \textbf{Interface:} Most Smart~Home devices interface wireless, often via R/F transmitters. When possible, the OC will provide an official interface via the \iterm{arena network}.
	\item \textbf{Operating voltage:} The operating voltage used will be the standard for the place of the competition (e.g.120V/60Hz for North America and 220V/50Hz for Europe). Please note that devices designed for other voltages/frecuencies may burn when plugged to the outlet.
	\item \textbf{Type of devices:} Mostly Smart~Home switches will be used (set on/off, read can not be guaranteed). For high bandwidth devices such as microphones or video cameras, an official interface (such as a ROS topic or web service) will be provided via the \iterm{arena network}.
\end{enumerate}

\subsubsection{Availability \& Scoring}
All test has been designed to optionally allow the use of Smart~Home devices and even grant bonus scoring for using this option. However, robots must be able to continue operating normally when there are no Smart~Home devices available. Therefore, it is unacceptable that a robot gets stuck or in some sort of deadlock while trying to operate those devices.

As stated in~\refsec{rule:scenario_wifi}, organizers cannot guarantee reliability and performance of wireless communication. Therefore, in case of malfunction or communication problems with the Smart~Home devices, or any other issue which may affect scoring, no claims will be accepted by the EC/TC/OC, nor test will be repeated. The decision on if a team given points for using \iterm{Smart~Home} devices, is conducted by the \iaterm{Technical Committee}{TC}, and it reserves the rights of discarding Smart~Home related scoring.


% Local Variables:
% TeX-master: "../Rulebook"
% End:


% Local Variables:
% TeX-master: "../Rulebook"
% End:


%%%%%%%%%%%%%%%%%%%%%%%%%%%%%%%%%%%%%%%%%%%%%%%%%%%%%%%%%
\section{Robots}
\label{rule:robots}

\subsection{Number of robots}
\label{rule:robots_number}

\begin{enumerate}
	\item \textbf{Registration:} The maximum \term{number of robots} per team is \emph{two} (2).
	\item \textbf{Regular Tests:} Only one robot is allowed per test. For different tests different robots can be used.
	% \item \textbf{Open Demonstrations:} In the \iterm{Open Challenge} and the \iterm{Finals} both robots can be used simultaneously.
\end{enumerate}

\subsection{Appearance and safety}
\label{rule:robot_appearance}

Robots should have a nice product-like appearance, be safe to operate, and should not annoy people. The following rules apply to all robots and are part of the \iterm{Robot Inspection} test (see~\refsec{sec:robot_inspection}).
\begin{enumerate}
	\item \textbf{Cover:} The robot's internal hardware (electronics and cables) should be covered in an appealing way. The use of (visible) duct tape is strictly prohibited.
	\item \textbf{Loose cables:} Loose cables hanging out of the robot are not permitted.
	\item \textbf{Safety:} The robot must not have sharp edges or elements that might harm people.
	\item \textbf{Annoyance:} The robot must not be continuously making loud noises or use blinding lights.
	\item \textbf{Marks:} The robot may not exhibit any kind of artificial marks or patterns.
	\item \textbf{Driving:} To be safe, the robots should be careful when driving (obstacle avoidance is mandatory).
\end{enumerate}

\subsection{Standard Platform Leagues}
\label{sec:rules:robotappearance_spl}
For Robots competing in a \SPL{}, modifications and alterations to the robots are strictly forbidden. This includes, but is not limited to, attaching, connecting, plugging, gluing, and taping components into and onto the robot, as well as, modifying or altering the robot structure. Not complying with this rule, leads to an immediate disqualification and penalization of the team (see~\ref{sec:rules:penaltiesbonuses}).

Robots are allowed to \enquote{wear} clothes, have stickers (e.g., a sticker exhibiting the logo of a sponsor), and be painted as long as they are compliant with section \ref{sec:rules:robotappearance}.

\subsubsection{DSPL Modifications}
\label{sec:rules:mountingbracket}
In the \DSPL{}, some modifications to the \HSR{} are allowed. An official \MountingBracket{} is provided by Toyota for the \HSR{}. Any laptop fitting inside the \MountingBracket{} can be used as additional on board computing. Furthermore, teams are allowed to attach the following devices to the robot or the laptop in the \MountingBracket{}:
\begin{enumerate}
	\item \textbf{Audio:} USB audio output device, e.g. USB-powered speaker, possibly with sound card.
	\item \textbf{Wi-Fi Adapter:} USB-powered IEEE 802.11ac (or newer) compliant device.
	\item \textbf{Ethernet Switch:} USB-powered IEEE 802.3ab (or newer) compliant device.
\end{enumerate}

\noindent A maximum of three such devices can be attached, they cannot increase the robot's dimension.

\subsection{Robot Specifications for the Open Platform League }
Robots competing in the RoboCup@Home Open Platform League must comply with security specifications in order to avoid causing any harm while operating in human environments.

\subsubsection{Size and weight of robots}
\label{rule:robots_size}

\begin{enumerate}
	\item \textbf{Dimensions:} The dimensions of a robot should not exceed the limits of an average door, which is \SI{200}{\centi\meter} by \SI{70}{\centi\meter} in most countries.\\
	The TC may allow the qualification and registration of larger robots, but due to the international character of the competition it cannot be guaranteed that the robots can actually enter the arena. In case of doubt, contact the local organization.
	\item \textbf{Weight:} There is no specific weight restriction. However, the weight of the robot and the pressure it exerts on the floor should not exceed local regulations for the construction of buildings which are used for living and/or offices in the country where the competitions is being held.
	\item \textbf{Transportation:} Team members are responsible for quickly moving the robot out of the arena.	If the robot cannot move by itself (for any reason), the team members must be able to transport the robot away with an easy and fast procedure.
\end{enumerate}



\subsubsection{Emergency stop button}
\label{rule:robots_emergency_button}

\begin{enumerate}
	\item \textbf{Accessibility and visibility:} Every robot has to provide an easily accessible and visible \iterm{emergency stop} button.
	\item \textbf{Color:} It must be coloured red, and preferably be the only red button on the robot. If it is not the only red button, the TC may ask the team to tape over or remove the other red button.
	\item \textbf{Robot behavior:} When pressing this button, the robot and all parts of it have to stop moving immediately.
	\item \textbf{Inspection:} The emergency stop button is tested during the \iterm{Robot Inspection} test (see~\refsec{sec:robot_inspection}).
\end{enumerate}




\subsubsection{Start button}
\label{rule:start_button}

\begin{enumerate}
	\item \textbf{Requirements:} As stated in~\refsec{rule:start_signal}, teams that aren't able to carry out the default start signal (opening the door) have to provide a \iterm{start button} that can be used to start tests. The team needs to announce this to the TC before every test that involves a start signal, including \iterm{Robot Inspection}.
	\item \textbf{Definition:} The start button can be any \quotes{one-button procedure} that can be easily executed by a referee.  This includes, for example, the release of the \iterm{emergency button} (\refsec{rule:robots_emergency_button}), a hardware button different from the \iterm{emergency button} (e.g., a green button), or a software button in a Graphical User Interface.
	\item \textbf{Inspection:} It is during the the \iterm{Robot Inspection} test (see~\refsec{sec:robot_inspection}) that the procedure for the start button, if needed, is announced to the TC and inspected. The start button for a robot should be the same for all the tests.
	\item \textbf{Penalty for using start button:} If a team needs to use the start button in a test where opening the door is the start signal, it may receive a penalty (see~\refsec{rule:start_signal}).
\end{enumerate}




\subsubsection{Audio output plug}
\label{rule:roobt_audio_out}

\begin{enumerate}
	\item \textbf{Mandatory plug:} Either the robot or some external device connected to it \emph{must} have a \iterm{speaker output plug}. It is used to connect the robot to the sound system so that the audience and the referees can hear and follow the robot's speech output.
	\item \textbf{Inspection:} The output plug needs to be presented to the TC during the \iterm{Robot Inspection} test (see~\refsec{sec:robot_inspection}).
	\item \textbf{Audio during tests:} Audio (and speech) output of the robot during a test have to be understood at least by the referees and the operators.
	\begin{compactitem}
		\item It is the responsibility of the teams to plug in the transmitter before a test, to check the sound system, and to hand over the transmitter to next team.
		\item Do not rely on the sound system! For fail-safe operation and interacting with operators make sure that the sound system is not needed, e.g., by having additional speakers directly on the robot.
\end{compactitem}
\end{enumerate}




\subsubsection{Appearance}
\label{rule:robots_appearance}
Open Platform Robots should have a neat appearance that resembles more a safe and finished product than an early stage prototype, paying special attention in completely cover the robot's internal hardware (electronics and cables) in an appealing way.
% However, teams must keep in mind that no artificial markers are allowed when personalizing the appearance or a robot. This includes, but is not limited to bar codes, QR codes, OpenCV markers, fluorescent and phosphorescent colors, and reflective stickers.
Although covering the robot's internal hardware with a T-Shirt is not forbidden (for now) it is strongly unadvised.



% Local Variables:
% TeX-master: "../Rulebook"
% End:


% \section{Data Recording}
  \label{rule:datarecording}
  In order to benchmark robots and software outside the RoboCup@Home arena, the teams are asked to contribute to a public dataset.
  This will consist of audio, imagery and other data obtained and generated by the robots during RoboCup@Home tests.
  Contributing to this dataset gives a small bonus as an incentive.
  The bonus will be proportional to the points gathered normally:
    if 50\% of points are gathered, 50\% of the data collection points are awarded.

  \subsection{Collected data}
    During a test, specific data can be gathered and stored on a USB stick.
    After all attempts at a test are made, the USB stick must be given to the TC, which will copy the data to the public dataset.
    The recordings themselves are not used for scoring and may be post-processed manually to be more useful, before handing over to the TC.
    Not all types of data are interesting for each test and thus each test will list which data to record.

    \begin{itemize}
    \item \textbf{Audio: } A .wav file of conversation or commands given by any operator and the result of the automatic speech recognition, if applicable.
      The recording must be made of the same signals that are input to the automatic speech recognition software.
      \begin{itemize}
	\item \textbf{Format: } TeamName\_SensorName\_Timestamp.wav
	\item \textbf{Format: } PCM Wav 44.1 kHz 16 bit stereo
      \end{itemize}

    \item \textbf{Commands: } A text file with the commands as received by the robot.
      This may be the output of speech recognition or the outcome of any form of bypassing it via the CONTINUE rule (see~\refsec{rule:asrcontinue}).
      Include a timestamp and then the command.
      \begin{itemize}
	\item \textbf{Format: } TeamName\_commands\_Timestamp.csv
	\item \textbf{Format: } csv-file. First column has command timestamp, second column the command in \enquote{quotes}.
      \end{itemize}

    \item \textbf{Images: } 2D and/or 3D RGB(D) images from the robot's camera(s) while doing any sort of recognition task.
			    Record the full field of view.
    \begin{itemize}
      \item
	\textbf{Color images: }
	\begin{itemize}
	  \item \textbf{Filename: } TeamName\_SensorName\_Timestamp\_rgb.png
	  \item \textbf{Format: } Standard PNG 24bit
	\end{itemize}
      \item
	\textbf{Depth images: }
	\begin{itemize}
	  \item \textbf{Filename: } TeamName\_SensorName\_Timestamp\_depth.png
	  \item \textbf{Format: } Standard PNG 16bit grayscale
	\end{itemize}
    \end{itemize}

    \item \textbf{Mapping data: } Record the data the robot uses for mapping its surroundings and obstacle avoidance plus the resulting map.
      For many robots this will be 2D laser scans of an Laser Range Finder but other means are possible.

    \item \textbf{Plans: } Any plan generated by the robot. This includes navigation paths, arm trajectories and action plans.
      If possible, plans are preferably annoted with whether is was succesfully executed or not.
    \end{itemize}

    For ROS-based robots, the most convenient data format for mapping data (laser scans, occupancy grids etc.) and motion plans are their ROS messages recorded into a ROS Bag file.
    This ROSBag should then contain:
     \begin{itemize}
       \item \textbf{Laser scans: } sensor\_msgs/LaserScan
       \item \textbf{Path(s): } nav\_msgs/Path
       \item \textbf{Map(s): } nav\_msgs/OccupancyGrid
       \item \textbf{Robot pose: } geometry\_msgs/PoseStamped
       \item \textbf{Transformation tree: } tf2\_msgs/TFMessage or equivalent
       \item \textbf{Odometry: } nav\_msgs/Odometry
      \end{itemize}
    Although not all robots use ROS, this serves as a guideline of the type of data that may be interesting for others.
    The RoCKIn robot competition provides a conversion tool that converts to ROS Bag files:
    \url{http://rockinrobotchallenge.eu/rockin_d2.1.3.pdf}, section 3.4 and
    \url{https://github.com/rockin-robot-challenge/benchmark_and_scoring_converter}

%
%   5
% In the following, ‘offline’ identifies data produced by the robot that will be collected by the referees when
% the execution of the benchmark ends (e.g., as files on a USB stick), while ‘online’ identifies data that the robot
% has to transmit to the testbed during the execution of the benchmark. Data marked neither with ‘offline’ nor
% with ‘online’ is generated outside the robot. NOTE: the online data should also be displayed by the robot on its
% computer screen, for redundancy purposes, in case problems with wireless communications arise.
%
% 6Speech files from all teams and all benchmarks (both Task benchmarks and Functional benchmarks) will be
% collected and used to build a public dataset. The audio files in the dataset will therefore include all the defects of
% real-world audio capture using robot hardware (e.g., electrical and mechanical noise, limited bandwidth, harmonic
% distortion). Such files will be usable to test speech recognition software, or (possibly) to act as input during the
% execution of speech recognition benchmarks.
%
% Catering to Granny_annie's comfort:
% • On the robot, the audio signals of the conversations between Annie and the robot. [offline]
% • The final command produced during the natural language analysis process. [online]
% • The pose of the robot while moving in the environment.
% • The pose of the robot while moving in the environment, as perceived by the robot. [offline]
% • The sensorial data of the robot when recognizing the object to be operated. [offline]
% • The results of the robot’s attempts to execute Annie’s commands.
%
% Welcoming Visitors:
% The event/command causing the activation of the robot. [online]
% • The video signal from the door camera.
% • The pose of the robot during the execution of the task.
% • The pose of the robot while moving in the environment, as perceived by the robot. [offline]
% • The results of any attempts by the robot to detect and classify a visitor. [online]
% • The audio signals of the conversations with the visitors. [offline]
% • Any notifications from the robot (e.g., alarm if a visitor shows anomalous behavior). [online]
% • The results of any actions taken by the robot, including opening or closing the front door,
% guiding visitors into and around the apartment, manipulating objects, etc.
%
% Getting to know my home:
% The output files produced by the robot, as described by section 4.3.4. [offline]
% • The pose of the robot during the execution of the task.
% • The pose of the robot while moving in the environment, as perceived by the robot. [offline]
% • The result (success/failure) of the command issued to the robot.

% %% %%%%%%%%%%%%%%%%%%%%%%%%%%%%%%%%%%%%%%%%%%%%%%%%%%%%%%%%%
% 
% External Devices
% 
% %% %%%%%%%%%%%%%%%%%%%%%%%%%%%%%%%%%%%%%%%%%%%%%%%%%%%%%%%%%

\section{External devices}
\label{rule:robot_external_devices}
Everything which is not part of the robot is considered an \iterm{external device}.
All external devices must be authorized by the \iaterm{Technical Committee}{TC} during the \iterm{Robot Inspection} test (see~\refsec{sec:robot_inspection}).
The \iaterm{Technical Committee}{TC} specifies whether an external device can be used freely, under referee supervision, and its impact on scoring.
In general, external devices must be removed quickly after the test.
	
\noindent \textbf{Remark:} The use of \iterm{wireless devices} is strictly prohibited. \iterm{External microphones}, hand microphones, and headsets are not allowed in OPL and it use is discouraged in DSPL and SSPL.

\subsection{On-site external computing}
Computing resources that are not physical attached to the robot are considered \iterm{external computing resources}.
The use of up to 5 external computing resources is allowed, but only through the arena network (see \refsec{rule:scenario_wifi}) and with the previous approval of the \iaterm{Technical Committee}{TC}.
Teams must announce the use of any external computing resource at least 1 month before the competition to the \iaterm{Technical Committee}{TC}.

External Computing Devices must be placed in the \iaterm{\textbf{E}xternal \textbf{C}omputing \textbf{R}esource \textbf{A}rea}{ECRA} which is announced by the \iaterm{Technical Committee}{TC} during setup days.
A switch connected to the arena wireless network will be available to teams in the ECRA.
It is strictly forbidden to connect any kind of device or peripheral (e.g. screens, mouses, keyboards, etc.) to the computers in the ECRA during the competition.

A maximum of two laptops and two people from different teams is allowed at any time in the ECRA.
Teams using laptops as External Computing Devices must remove the device immediately after the test.
Once a test has started, all people must stay at least 1 meter from the ECRA.
Interacting with computers in the ECRA after the Referee has given the start signal will cause the immediate disqualification of the team.

\noindent \textbf{Remark:} Robot operation must be able to operate safely when \iterm{external computing resources} are unavailable.



% On-line devices
\subsection{On-line external computing}
\label{rule:robot_external_computing_online}
Robots are allowed to use \enquote{Cloud services}, \enquote{Internet API's}, and any other type of \iterm{external computing resource}.
Same restrictions for on-site external computing resources apply.

\noindent \textbf{Remark:} The competition organization doesn't guarantee or take any responsibility regarding the availability or reliability of neither the network nor Internet connection.
Teams' use of external computing resources is at their own risk.



% DSPL laptop
\subsection{Official Standard Laptop for DSPL}
\label{rule:osl_dspl}

In the Domestic Standard Platform League, teams may use the \iaterm{Official Standard Laptop}{OSL} connected to the Toyota HSR via Ethernet cable, safely located in the TOYOTA HSR \iterm{Mounting Bracket} provided by TOYOTA for this purpose.

\subsubsection{Technical Specifications}
The technical specifications for the Official Standard Laptop in the Domestic Standard Platform League are the following:


 \begin{itemize}
  \item \textbf{Brand and model:} DELL Alienware 15 or 17
  \item \textbf{CPU:} Core-i7 series
  \item \textbf{RAM:} 16GB or 32GB
  \item \textbf{GPU:} NVIDIA GeForce GTX 1070 or 1080
  \item \textbf{Storage:} Unrestricted.
\end{itemize}

No other brands or models will be accepted. There are no constrains regarding the software installed in the OSL but no additional hardware is allowed.

The referees, Technical Committee, and Organizing Committee members may run random checks anytime during the competition prior to the test to verify that the laptop in the TOYOTA HSR \iterm{Mounting Bracket} has no additional hardware plugged in, and matches the authorized specifications.


% Local Variables:
% TeX-master: "../Rulebook"
% End:


%%%%%%%%%%%%%%%%%%%%%%%%%%%%%%%%%%%%%%%%%%%%%%%%%%%%%%%%%
\section{External computing}\label{rule:robot_external_computing}
Robots are allowed to use some form of external computing, for example in the form of so-called \quotes{Cloud services} and/or \quotes{Internet API's} etc. 
\begin{enumerate}
	\item \textbf{Definition:} Computing resources that are not physical part of the robot are \iterm{external computing resources}. 
	\item \textbf{Inspection:} In general, external computers are not allowed unless explained to and allowed by the \iaterm{Technical Committee}{TC}.
	  A team must announce to the TC at least 1 month in advance the external computing resources they want to use, for what purpose and how to reach the resources (e.g. specify the URL or IP address and port). Inspected software must meet the following \textbf{requirements:}
	  \begin{itemize}
	  	\item The software must be open source (BSD/GPL/etc), or
        \item Detailed information about the propietary product must be provided (e.g. vendor, patent number, licencing, pricing, etc.), as well as publishing the interface for scientific use.
	  \end{itemize}
	All relevant information must be specified in the team description paper.
	\item \textbf{Connection:} The robot may connect to \iterm{external computing resources} via a network connection, e.g. the Internet. 
	  The competition organisation cannot make any guarantees concerning availability, connectivity and performance of the connection. 
	  The robot should still be functional (albeit limited perhaps) if the \iterm{external computing resources} cannot be used for some reason.
	  This is the team's responsibility. 
	\item \textbf{Autonomy:} The robot has to maintain full autonomy when using \iterm{external computing resources}, 
	  meaning there may not be a human giving the robot any kind of instructions via \iterm{external computing resources}.
	  It is up to the team to prove to the \iaterm{Technical Committee}{TC} that there was no cheating introduced via the \iterm{external computing resources}. 
	  For example, the use of Amazon Mechanical Turk to classify and recognize objects during a competition will be considered cheating, since effectively a human will do the classification.
	  Remote control or tele-operation is also considered cheating. 
	\item \textbf{Availability:} The resources must be publicly available, for use by robots of other teams, well before and after the competition.
	\item \textbf{Recognition:} In case the resources are not developed by the team itself, the creators must be properly credited in the Team Description Paper (See \refsec{rule:website_tdp}).
	\item \textbf{Limit:} A robot is limited to use up to 5 \iterm{external computing resources}. 
\end{enumerate}

\textbf{Remark:} Teams are allowed to use their own software in the external computing devices (not only cloud services). This software must be publicly available to other teams for scientific purposes (evaluation, test, and benchmarking), as well as for TC for inspection. Although open-sourcing the software is not mandatory, this practice is advised and encouraged by the league.

\subsection{External Computing Devices on site for SSPL (Hardware)}
\label{rule:robot_external_computing_devices_sspl}

\begin{itemize}
  \item \textbf{Location:} all external computing devices, approved by the \iaterm{Technical Committee}{TC}
  		(see~\ref{rule:robot_external_computing}), must be placed at a designated location close to the arena.
  		The location features a table (or otherwise suitable furnishing) and an ethernet switch (wired)
        that connects to the arena WIFI. In the remainder of this document, this location is called
  		\textbf{E}xternal \textbf{C}omputing \textbf{R}esource \textbf{A}rea (ECRA).
  \item \textbf{Procedure:} before a teams' test run, external comp. devices must be setup in time.
  		Immediately after a teams' test slot, all equipment must be removed from the ECRA in order to give ensuing
        teams adequate space and time to setup their devices. No team should ever occupy the ECRA continuously.
  		As soon as the referee indicates the teams' test slot starts, team members are strictly forbidden to touch
        their external comp. devices until the current test run is over. Breaking this ``hands-off'' rule
        is penalized with 0 (zero) points for the corresponding test. Using external comp. devices during the test
        from any other location than the ECRA will be penalized with \textbf{at least 0 (zero) points for the test}.
  \item \textbf{Limitations:} external computing devices currently have no limitation concerning computation
  		power and form factor. However, in order to maintain enough space and a	clear arrangement (e.g. wrt
        inspectability for referees), stand-alone screens are \textbf{not allowed}. Mice and keyboards connected
        to external computing devices are allowed.
  \item \textbf{Recommendations:} show up in time and setup up your devices early enough. If your team is up last,
        do not setup first. Consider the ECRA a FIFO queue. There will be enough space at the ECRA to establish a
        buffer (of teams), so if you show up in time and everyone removes their equipment the transition will be smooth.
\end{itemize}

The referees, Technical Committee, and Organizing Committee members may run random checks anytime during the competition
to check if a team occupies the ECRA continuously or devices outside the ECRA are used to control the robot.


\subsection{Official Standard Laptop for DSPL}
\label{rule:osl_dspl}
% 
% Mandatory mounting bracket
% 
% In the Domestic Standard Platform League, teams are required to have the \iaterm{Official Standard Laptop}{OSL} connected to the Toyota HSR via Ethernet cable, safely located in the TOYOTA HSR \iterm{Mounting Bracket} provided by TOYOTA for this purpose.

% The robot must wear the \iterm{Mounting Bracket} with the OSL in all the tests during the competition and the OSL must be connected to the robot whether it's used or not. Since the use of the \iterm{Mounting Bracket} and the presence of the OSL is compulsory, teams missing this requirement will not be allowed to compete.

% 
% Optional mounting bracket
% 
In the Domestic Standard Platform League, teams may use the \iaterm{Official Standard Laptop}{OSL} connected to the Toyota HSR via Ethernet cable, safely located in the TOYOTA HSR \iterm{Mounting Bracket} provided by TOYOTA for this purpose.

\subsubsection{Technical Specifications}
The technical specifications for the Official Standard Laptop in the Domestic Standard Platform League are the following:


 \begin{itemize}
  \item \textbf{Brand and model:} DELL Alienware 15 or 17
  \item \textbf{CPU:} Core-i7 series
  \item \textbf{RAM:} 16GB or 32GB
  \item \textbf{GPU:} NVIDIA GeForce GTX 1070 or 1080
  \item \textbf{Storage:} Unrestricted.
\end{itemize}

No other brands or models will be accepted. There are no constrains regarding the software installed in the OSL but no additional hardware is allowed.

The referees, Technical Committee, and Organizing Committee members may run random checks anytime during the competition prior to the test to verify that the laptop in the TOYOTA HSR \iterm{Mounting Bracket} has no additional hardware plugged in, and matches the authorized specifications.

% Local Variables:
% TeX-master: "../Rulebook"
% End:


%%%%%%%%%%%%%%%%%%%%%%%%%%%%%%%%%%%%%%%%%%%%%%%%%%%%%%%%%
\section{Organization of the competition}
\label{sec:procedure_during_competition}

\subsection{Stage system}\label{rule:stages}

The competition features a \iterm{stage system}. It is organized in two stages each consisting of a number of specific tasks. It ends with the \iterm{Finals}.

Each \iaterm{stage} comprehends a set of tasks grouped in two thematic scenarios.
% \iaterm{House Cleaner} and \iaterm{Party Host}.
The \iaterm{Housekeeper} scenario features tasks related to cleaning, organizing, and giving maintenance; while the \iaterm{Party Host} scenario focuses in attending guests needs and providing general assistance during a party.

\begin{enumerate}
	\item \textbf{Robot Inspection:} For security, robots are inspected during setup days.
  A robot must pass \iterm{Robot Inspection} test (see~\refsec{sec:robot_inspection}) in order to compete.

	\item \textbf{Stage~I:} The first days of the competition called \iterm{Stage~I}.
	All qualified teams can participate in \iterm{Stage~I}.
	The same task can be performed multiple times (See~\refsec{rule:score_system}).

	\item \textbf{Stage~II:} The best \emph{50\% of teams}\footnotemark (after Stage~I) advance to \iterm{Stage~II}.
	Here, tasks require more complex abilities or combinations of abilities.\\
	\footnotetext{If the total number of teams is less than 12, up to 6 teams may advance to Stage~II}

	\item \textbf{Final demonstration:} The best \emph{two teams} of each league, the ones with the highest score after Stage~II, advance to the final round.
	The final round features only a single task integrating all tested abilities.
	In order to participate in the Finals, a team must have solved at least one task of the Stage~II.
\end{enumerate}

In case of having no considerable score deviation between a team advancing to the next stage and a team dropping out, the TC may announce additional teams advancing to the next stage.


%%%%%%%%%%%%%%%%%%%%%%%%%%%%%%%%%%%%%%%%%%%%%%%%%%%%%%%%%
\subsection{Schedule}
\label{rule:schedule}

\begin{enumerate}
	\item \textbf{Thematic scenario blocks:} Each \iterm{thematic scenario} or \iterm{theme} is split in two \iterm{blocks}.
	At least two blocks are scheduled per day, having each block an assigned theme and lasting no less than two hours.
	The \iaterm{Organizing Committee}{OC} announces the schedule during the setup days (see Table \ref{tbl:schedule}).

	\item \textbf{Slots:} The \iaterm{Organizing Committee}{OC} assigns at least two \iterm{test slots} of 5 minutes to each team in each block.
   The maximum number of \iterm{tests slots} will be announced during setup days by the \iaterm{Technical Committee}{TC} based on the available time and the number of participating teams.
	A team can solve any task during its test slot.
	Remaining block time can be used to assign additional testing slots to interested teams.
	Testing slots are randomly assigned to teams in each block.

	\item \textbf{Tests:} Teams must inform the OC in advance which task(s) will try to solve.
	Only one task can be attempted per test slot.

	\item \textbf{Participation is default:} Teams have to indicate to the \iaterm{Organizing Committee}{OC} when they are \emph{skipping} a test slot. Without such indication, they may receive a penalty when not attending (see~\refsec{rule:not_attending}).
\end{enumerate}

% Please add the following required packages to your document preamble:
% \usepackage[table,xcdraw]{xcolor}
% If you use beamer only pass "xcolor=table" option, i.e. \documentclass[xcolor=table]{beamer}
\begin{table}[h]
	\centering\small
	\newcommand{\teams}[3]{%
		\tiny
		\begin{tabular}{c}%
			\textit{Test slot 1, team $#1$}\\
			\textit{Test slot 2, team $#2$}\\
			$\vdots$\\
			\textit{Test slot $n$, team $#3$}\\
		\end{tabular}
	}
	\newcommand{\wcell}[2]{%
		\parbox[c]{2.5cm}{%
			\vspace{#1}%
			\centering%
			#2%
			\vspace{#1}%
		}%
	}
	\newcommand{\cell}[1]{\wcell{0.2\baselineskip}{#1}}


	\begin{tabular}{
		>{\centering\arraybackslash}m{2.5cm}|%
		>{\columncolor[HTML]{9AFF99}}c |%
		>{\columncolor[HTML]{9AFF99}}c |%
		>{\columncolor[HTML]{CBCEFB}}c |%
		>{\columncolor[HTML]{CBCEFB}}c |%
	}
	\multicolumn{1}{ c }{}
		& \multicolumn{1}{ c }{\cellcolor{white} Day 1 }
		& \multicolumn{1}{ c }{\cellcolor{white} Day 2 }
		& \multicolumn{1}{ c }{\cellcolor{white} Day 3 }
		& \multicolumn{1}{ c }{\cellcolor{white} Day 4 }
		\\\cline{2-5}
	\cell{Block 1\\\footnotesize(9:00 - 12:00)}
		& \cell{Housekeeper\\\teams{i}{j}{i}}
		& \cell{Party Host\\\teams{k}{i}{k}}
		& \cell{Housekeeper\\\teams{i}{j}{i}}
		& \cell{Party Host\\\teams{j}{k}{j}}\\\cline{2-5}

	\multicolumn{1}{ c }{}
		& \multicolumn{4}{ c }{\wcell{0.5\baselineskip}{\color{gray}Lunch}}\\\cline{2-5}

	\cell{Block 2\\\footnotesize(14:00 - 17:00)}
		& \cell{Housekeeper\\\teams{i}{k}{i}}
		& \cell{Party Host\\\teams{k}{j}{k}}
		& \cell{Party Host\\\teams{i}{i}{k}}
		& \cell{Housekeeper\\\teams{k}{j}{j}}\\\cline{2-5}

	\multicolumn{1}{ c }{}
		& \multicolumn{2}{ c }{\wcell{0.5\baselineskip}{\color[HTML]{029734}Stage 1}}
		& \multicolumn{2}{ c }{\wcell{0.5\baselineskip}{\color[HTML]{6668e5}Stage 2}}\\
	\end{tabular}

	\caption{Example schedule.
		Each team has assigned at least two test slots in every block.
		At least two blocks are scheduled per day with an assigned theme.
		A team can choose a different task in each test, meaning at least 4 different tests per stage.
	}
	\label{tbl:schedule}
\end{table}


\subsection{Score system}
\label{rule:score_system}
Each task has a main objective and a set of scoring bonuses.
To score in a test, a team must successfully accomplish the main objective of the task; bonuses are not considered otherwise.
Overall scoring is calculated as the sum of the maximum score obtained in each ability.

The \iaterm{score system} has the following constrains
\begin{enumerate}

	\item \textbf{Stage~I:} The maximum total score per task in \iterm{Stage~I} is \scoring{1000 points}.
	
	\item \textbf{Stage~II:} The maximum total score per task in \iterm{Stage~I} is \scoring{2000 points}.

	\item \textbf{\iterm{Finals}:} Final score is normalized and a special evaluation is used.

	\item \textbf{Minimum score:} The minimum total score per test in \iterm{Stage~I} and \iterm{Stage~II} is \scoring{0 points}.
	Teams cannot receive negative points.

	\item \textbf{Penalties:} An exception to \emph{minimum score} rule are penalties.
	Both penalties for not attending (see~\refsec{rule:not_attending}) and extraordinary penalties (see~\refsec{rule:extraordinary_penalties}) can cause a total negative score.
\end{enumerate}




% Local Variables:
% TeX-master: "../Rulebook"
% End:


%%%%%%%%%%%%%%%%%%%%%%%%%%%%%%%%%%%%%%%%%%%%%%%%%%%%%%%%%
\section{Procedure during Tests}

\subsection{Safety First!}
\label{rule:safetyfirst}
\begin{enumerate}
	\item \textbf{Emergency Stop:} At any time when operating the robot inside and outside the scenario the owners have to stop the robot immediately if there is a possibility of dangerous behavior towards people and/or objects.
	\item \textbf{Stopping on request:} If a referee, member of the Technical or Organizational committee, an Executive or Trustee of the federation stops the robot (by pressing the emergency button) there will be no discussion. Similarly if they tell the team to stop the robot, the robot must be stopped \emph{immediately}.
	\item \textbf{Penalties:} If the team does not comply, the team and its members will be excluded from the ongoing competition immediately by a decision of the RoboCup@Home \iaterm{Technical Committee}{TC}. 	Furthermore, the team and its members may be banned from future competitions for a period not less than a year by a decision of the RoboCup Federation Trustee Board.
\end{enumerate}

\subsection{Maximum number of team members}
\label{rule:number_of_people}
\begin{enumerate}
	\item \textbf{Regular Tests:} During a regular test, the maximum number of team members allowed inside the \Arena{} is \emph{one} (1).
	Exceptions are tests that explicitly require volunteer assistance.
	\item \textbf{Setup:} During the setup of a test, the number of team members inside the \Arena{} is not limited.
	% \item \textbf{Open Demonstrations:} During the \iterm{Open Challenge} \iterm{Demo Challenge}, and the \iaterm{final demonstration}{Finals}, the number of team members inside the arena is not limited.
	%\item \textbf{Open Demonstrations:} During the \iterm{Open Challenge}, and the \iaterm{final demonstration}{Finals}, the number of team members inside the arena is not limited.
	\item \textbf{\FINAL:} During the \FINAL, the number of team members inside the \Arena{} is not limited.
	\item \textbf{Moderation:} During a regular test, one team member \emph{must} be available to host and comment the test (see~\refsec{rule:moderator}).
\end{enumerate}

\subsection{Fair play}
\label{rule:fairplay}
\iterm{Fair Play} and cooperative behavior is expected from all teams during the entire competition, in particular:
\begin{itemize}
	\item while evaluating other teams,
	\item while refereeing, and
	\item when having to interact with other teams' robots.
\end{itemize}
This also includes:
\begin{itemize}
	\item not trying to cheat (e.g., pretending autonomous behavior where there is none),
	\item not trying to exploit the rules (e.g., not trying to solve the task but trying to score), and
	\item not trying to make other robots fail on purpose.
	\item not modifying robots in standard platforms.
\end{itemize}
Disregard of this rule can lead to penalties in the form of negative scores, disqualification for a test, or even for the entire competition.

\subsection{Expected Robot's Behavior}
Unless stated otherwise, it is expected that the robot always behave and react in the same way a polite and friendly human being would do.
This applies also to how robots try solve the assigned task
As rule of thumb, one may ask any non-scientist how she would solve the task.

Please consider that average users will not know the specific procedure to operate a robot.
Hence, interaction should be as with any other human being.


\subsection{Robot Autonomy and Remote Control}
\begin{enumerate}
	\item \textbf{No touching:} During a test, the participants are not allowed to make contact with the robot(s), unless it is in a \enquote{natural} way and required by the task.

	\item \textbf{Natural interaction:} The only allowed means to interact with the robot(s) are gestures and speech.

	\item \textbf{Natural commands:} Anything that resembles direct control is forbidden.

	\item \textbf{Remote Control:} Remotely controlling the robot(s) is strictly prohibited.
	This also includes pressing buttons, or influencing sensors on purpose.

	\item \textbf{Penalties:} Disregard of these rules will lead to disqualification for a test or for the entire competition.
\end{enumerate}



\subsection{Collisions}
\begin{enumerate}
	\item \textbf{\iterm{Touching}:} Gently \emph{touching} objects is tolerated but unadvised.
	However, robots are not allowed to crash with something.
	The \enquote{safety first} rule (\refsec{rule:safetyfirst}) overrides any other rule.

	\item \textbf{\iterm{Major collisions}:} If a robot crushes into something during a test, the robot is immediately stopped.	Additional penalties may apply.

	\item \textbf{\iterm{Functional touching}:} Robots are allowed to apply pressure on objects, push away furniture and, in general, interact with the environment using structural parts other than their manipulators.
	This is known as \iaterm{functional touching}.
	However, the robot must clearly announce the collision-like interaction and kindly request not being stopped.\\
	\textbf{Remark: } Referees can (and will) immediately stop a robot in case or suspicion of \emph{dangerous} behavior.

	\item \textbf{Robot-Robot avoidance:} If two robots encounter each other, they both have to actively try to avoid the other robot.
	\begin{enumerate}
		\item A robot which is not going for a different route within a reasonable amount of time (e.g., \SI{30}{\second}) is removed.
		\item A non-moving robot blocking the path of another robot for longer than a reasonable amount of time (e.g., \SI{30}{\second}) is removed.
	\end{enumerate}
\end{enumerate}



\subsection{Removal of robots}
\label{rule:robot_removal}
Robots not obeying the rules are stopped and removed from the \Arena{}.
\begin{enumerate}
	\item It is the decision of the referees and the TC member monitoring the test if and when to remove a robot.

	\item When told to do so by the referees or the TC member monitoring the test, the team must immediately stop the robot, and remove it from the \Arena{} without disturbing the ongoing test.

\end{enumerate}


\subsection{Start signal}
\label{rule:start_signal}
The default \iterm{start signal} (unless stated otherwise) is \iterm{door opening}.
Other start signals are allowed but must be authorized by the \iaterm{Technical Committee}{TC} during the Robot Inspection (see~\refsec{sec:robot_inspection}).

\begin{enumerate}
	\item \textbf{Door opening:} The robot is waiting behind the door, outside the \Arena{} and accompanied by a team member.
	The test starts when a referee (not a team member) opens the door.

	\item \textbf{Start button:} If the robot is not able to automatically start after the door is open, the team may start the robot using a start button.
	\begin{enumerate}[nosep]
		\item It must be a physical button on the robot (e.g., a dedicated one or releasing the eStop).
		\item It is allowed to use the robot's contact/pressure sensors (e.g., pushing the head or an arm joint).
		\item Using a start button needs to be announced to the referees before the test starts.
		\item There may be penalties for using a start button in some tests
	\end{enumerate}

	\item \textbf{Ad-hoc start signal:} Other means of triggering robot to action are allowed but must be approved by the \iaterm{Technical Committee}{TC} during the Robot Inspection (see~\refsec{sec:robot_inspection}).
	These include:
	\begin{itemize}[nosep]
		\item QR Codes
		\item Verbal instructions
		\item Custom HRI interfaces (apps, software, etc.)
	\end{itemize}
	\textbf{Remark:} There may be penalties for using Ad-hoc start signals in some tests. The use of mouses, keyboards, and devices attached to ECRA computers is strictly forbidden.

\end{enumerate}


\subsection{Entering and leaving the \Arena{}}
\label{rule:start_position}
\begin{enumerate}

	\item \textbf{Start position:} Unless stated otherwise, the robot starts outside of the \Arena{}.
	\item \textbf{Entering:} The robot must autonomously enter the \Arena{}.
	\item \textbf{Success:} The robot is said to \emph{have entered} when the door used to enter can be closed again, and the robot is not blocking the passage.
\end{enumerate}



\subsection{Gestures}
\label{rule:gestures}
Hand gestures may be used to control the robot in the following way:
\begin{enumerate}
	\item \textbf{Definition:} The teams define the hand gestures by themselves.

	\item \textbf{Approval:} Gestures need to be approved by the referees and TC member monitoring the test. Gestures should not involve more than the movement of both arms. This includes, e.g., expressions of sign language or pointing gestures.

	\item \textbf{Instructing operators:} It is the responsibility of the team to instruct operators.
	\begin{enumerate}
		\item The team may only instruct the operator when told to so by a referee.
		\item The team may only instruct the operator in the presence of a referee.
		\item The team may only instruct the robot for as long as allowed by the referee.
		\item When the robot has to instruct the operator, it is the robot that instructs the operator and \emph{not} the team. The team is not allowed to additionally guide the operator, e.g., tell the operator to come closer, speak louder, or to repeat a command.
		\item The robot is allows to instruct the operator at any time.
	\end{enumerate}

	\item \textbf{Receiving gestures:} Unless stated otherwise, it is not allowed to use a speech command to set the robot into a special mode for receiving gestures.
\end{enumerate}



\subsection{Referees}
\label{rule:referees}
All tests are monitored by a referee and one member of the \iaterm{Technical Committee}{TC}.
The following rules apply:

\begin{enumerate}
	\item \textbf{Selection:}
	\begin{itemize}
		\item Referees are chosen by EC/TC/OC.
		\item Referees are announced together with the schedule for the test slot.
	\end{itemize}

	\item \textbf{Not showing up:} Not showing up for refereeing (on time) will result in a penalty (see~\refsec{rule:extraordinary_penalties}).

	\item \textbf{TC monitoring:} A TC member acts as main referee.

	\item \textbf{Referee instructions:} Right before each test, referee instructions are conducted by the TC.
	The referees for all slots need to be present at the \Arena{} where the referee instructions are taking place.
	When and where referee instructions are taking place is announced together with the schedule for the slots.
\end{enumerate}


\subsection{Operators}
\label{rule:operator}
Unless stated otherwise, robots are operated by the referee or by a person selected by the referee.
If the robot fails to understand the default operator, the team may request the use of a custom operator.
Penalty may apply when using a custom operator.


\subsection{Moderator}
\label{rule:moderator}
The LOC is responsible of organizing test moderation in the local language.
The OC may request the participating teams to provide a team member for moderation.
Candidates have to be fluent in the moderation language (default is English).

\noindent\textbf{Responsibilities:} The moderators have to:
\begin{compactitem}
	\item Do \textbf{NOT interfere} with the performance
	\item Explain the tasks being performed
	\item Comment on the performance of the competitor
	\item Follow the instructions of the referee.
\end{compactitem}

\noindent \textbf{Not showing up:} Not showing up for moderation (on time) will result in a penalty (see~\refsec{rule:extraordinary_penalties}).


\subsection{Time limits}
\label{rule:time_limits}
\begin{enumerate}
	\item \textbf{Stage~I:} Unless stated otherwise, the time limit for each test in \iterm{Stage~I} is \timing{5 minutes}.

	\item \textbf{Stage~II:} Unless stated otherwise, the time limit for each test in \iterm{Stage~II} is \timing{10 minutes}.

	\item \textbf{Inactivity:} Robots are not allowed to stand still or get stuck into endless loops.
	A robot not progressing in the task execution (and obviously not trying to), is consider as inactive.
	Robots must be removed after 30 seconds of inactivity.

	\item \textbf{Requesting time:} A robot (not the team) can request referees to make exception from the 30-seconds inactivity time limit.
	In its request, the robot must clearly state for how long it will be performing a time-consuming process (e.g., 60~seconds).
	This time cannot exceed 3 minutes and cannot be used more than once per test.

	\item \textbf{Setup time:} Unless stated otherwise, there is no setup time.
	Robots need to be ready to enter the \Arena{} no later than one minute after the door has been closed to the former team.

	\item \textbf{Time-up:} When the time is up, the team must immediately remove their robot(s) from the  \Arena{}.
	No more additional score will be giving.

	\item \textbf{Show must go on:} On special cases, the referee may let the robot continue the test for demonstration purposes, but no additional points will be scored.
\end{enumerate}



\subsection{Restart}
\label{rule:restart}
Some tasks allow a single restart, a procedure in which the team is allowed to quickly fix any issue with the robot.
Restarts can be requested only when the test slot permits it, and when the amount of remaining time is greater than 50\% of the total.
The procedure is as follows:

\begin{enumerate}
	\item The team request a restart.
	\item The robot is taken to the initial position (e.g. outside the \Arena{}) and gets fixed.
	\item When the robot is ready, the team informs the referee.
\end{enumerate}

The following rules apply:
\begin{enumerate}
	\item \textbf{Number of restarts:} When allowed, the maximum number of restarts is one (1).

	\item \textbf{Early request:} Restart is \textbf{NOT} allowed after the first 50\% of the allotted time has elapsed.

	\item \textbf{Time:} The timer is neither restarted nor stopped.

	\item \textbf{One-minute Setup} The team has 1 minute to fix the robot, starting when the referee announces th restart.
	If the robot is not ready, the test is considered finished.

	\item \textbf{Scoring:} If the score of the second attempt is lower than the score of the first one, the average score of first and second run is taken.
\end{enumerate}

% Local Variables:
% TeX-master: "../Rulebook"
% End:


\section[Deus ex Machina]{Deus ex Machina: Bypassing features with human help \\ \small Because the show must go on}
\label{rule:continue}
Robots can't score unless they accomplish the main goal of a task.
However, in many real-life situations, a minor malfunction may prevent the robot from accomplishing a task.
To prevent this situation, while fostering awareness and human-robot interaction, robots are allowed to request human assistance during a test.

\subsection{Procedure}
\label{rule:continue_procedure}
The procedure to request human assistance while solving a task is as follows:

\begin{enumerate}
	\item \textbf{Request help:} The robot must indicate loud and clear that it requires human assistance. It must be clearly stated:
	\begin{compactitem}
		\item The nature of the assistance
		\item The particular goal or desired result
		\item How the action must be carried out (when necessary)
		\item Details about how to interact with the robot (when necessary)
	\end{compactitem}

	\item \textbf{Supervise:} The robot must be aware of the human's actions, being able to tell when the requested action has been completed, as well as guiding the human assistant (if necessary) during the process.

	\item \textbf{Acknowledge:} The robot must politely thank the human for the assistance provided.
\end{enumerate}

\subsection*{Example}
\label{rule:continue_example}
In this example the robot has to clean the table but is unable to grasp the spoon. 
\begin{itemize}[noitemsep]
	\small
	\item[\textcolor{gray}{R:}] \texttt{I am sorry but the spoon is too small for me to take.\\
	Could you please help me with it?\\
	Please say "robot yes" or "robot no" to confirm.}
	\item[\textcolor{gray}{H:}] \textit{Robot, yes!}
	\item[\textcolor{gray}{R:}] \texttt{Thank you! Please follow my instructions.\\
	Please take the purple spoon from the table. It is on my left.}
	\item[\textcolor{gray}{H:}] (Referee takes green fork)
	\item[\textcolor{gray}{R:}] \texttt{You took the wrong object.\\
	Please take the purple spoon from the table. It is on my left.}
	\item[\textcolor{gray}{H:}] (Referee takes purple spoon)
	\item[\textcolor{gray}{R:}] \texttt{I saw you took the spoon.\\
	Would you be so kind of following me to the kitchen?\\
	Please keep the spoon visible in front of you so I can track you. Thank you!}
	\item[\textcolor{gray}{R:}] \texttt{You can stop following me now.\\
	As you can see, the dishwasher is already open.\\
	Please place the spoon in the gray basket on the lower tray.}
	\item[\textcolor{gray}{R:}] \texttt{Lovely! Thanks for your help human.\\
	I'll let you know if I need further assistance.}
\end{itemize}



\subsection{Scoring}
\label{rule:continue_scoring}
There is no limit in the amount of times a robot can request human assistance, but score reduction applies every time it is requested.

\begin{enumerate}
	\item \textbf{Partial execution:} A reduction of 10\% of the maximum attainable score is applied when the robot request a partial solution (e.g. pointing to the person the robot is looking for or placing an object within grasping distance).
	The referee decides whether the requested action is simple enough to corresponds to a partial execution or not.

	\item \textbf{Full awareness:} A reduction of 20\% of the maximum attainable score is applied when the robot is able to track and supervise activity, detecting possible, and when the requested action has been completed.

	\item \textbf{No awareness:} A reduction of 30\% of the maximum attainable score is applied when the robot has to be told when the requested action has been completed.

	\item \textbf{Bonuses:} No bonus points can be scored when the robot requests help to solve part of a task that normally would grant a bonus.

	\item \textbf{Score reduction overlap:} The score reduction for multiple requests of the same kind do not stack, but overlap.
	The total reduction applied correspond to the worse execution (higher reduction of all akin help requests).
	This means, a robot won't be reduced again for requesting help to transport a second object, but a second reduction will apply when the robot asks for a door to be opened.
\end{enumerate}

\subsection{Bypassing Automatic Speech Recognition}
\label{rule:asrcontinue}
Giving commands to the robot is essential in many tests.
When the robot is not able to receive spoken commands, teams are allowed to provide means to bypass ASR via an Alternative method for HRI (see~\refsec{rule:asralternative}).
Nonetheless, Automatic Speech Recognition is preferred.

The following rules apply in addition to the ones specified in section \refsec{rule:continue_scoring}
\begin{enumerate}
	\item \textbf{ASR with Default Operator:} No score reduction.
	The command is given by the human operator who must speak (not shout) loud and clear.
	The \iterm{default operator} may repeat the command up to three times.

	\item \textbf{ASR with Custom Operator:} A reduction of 10\% of the maximum attainable score is applied when a \iterm{custom operator} is requested.
	The Team Leader chooses a person who gives the command \emph{exactly as instructed by the referee}.

	\item \textbf{Gestures:} A reduction of 20\% of the maximum attainable score is applied when a gesture (or set of gestures) is used to instruct the robot.

	\item \textbf{QR Codes:} A reduction of 30\% of the maximum attainable score is applied when a QR code is used to instruct the robot.

	\item \textbf{Alternative Input Method:} A reduction of up to 30\% of the maximum attainable score is applied when a \iterm{alternative HRI interface}, is used to instruct the robot.
	Alternative HRI interfaces (see~\refsec{rule:asralternative}) must be previously approved by the TC during the Robot Inspection (see~\refsec{sec:robot_inspection}).
\end{enumerate}


\subsubsection{Alternative interfaces for HRI}
\label{rule:asralternative}
Alternative methods and interfaces for HRI offer a way for a robot to start or complete a task.
Any reasonable method may be used, with the following criteria:
\begin{itemize}
	\item \textbf{Intuitive to use and self-explanatory:} a manual should not be needed. Teams are not allowed to explain how to interface with the robot. %you immediately know how to use it after a quick glance

	\item \textbf{Effortless use:} Must be as easy to use as uttering a command. %is as easy to use as it is uttering a command

	\item \textbf{Is smart and preemptive:} The interface adapts to the user input, displaying only the options that make sense or that the robot can actually perform.

	\item Exploits the best of the device being used (eg. touch screen, display area, speakers, etc.)
\end{itemize}

Preferably, the alternative HRI must be also adapted to the user.
Consider localization (with English as the default), but also potential users of service robots at their home.
For example: elderly people and people with physical disabilities.

\textbf{\textsc{Award:}} The best alternative is awarded the Best Human-Robot Interface award (\refsec{award:hri}).


% Local Variables:
% TeX-master: "../Rulebook"
% End:


%%%%%%%%%%%%%%%%%%%%%%%%%%%%%%%%%%%%%%%%%%%%%%%%%%%%%%%%%
\newcommand{\penaltybig}{500~}
\newcommand{\penaltysmall}{250~}


\section{Special penalties and bonuses}\label{sec:special_awards}

\subsection{Penalty for not attending}\label{rule:not_attending}
\begin{enumerate}
	\item \textbf{Automatic schedule:} All teams are automatically scheduled for all tests.

	\item \textbf{Announcement:} If a team cannot participate in a test (for any reason), the team leader has to announce this to the OC at least \timing{60 minutes} before the test slot begins.

	\item \textbf{Penalties:} A team that is not present at the start position when their scheduled test starts, the team is not allowed to participate in the test anymore.
	If the team has not announced that it is not going to participate, it gets a penalty of \scoring{\penaltysmall points}.
\end{enumerate}

\subsection{Extraordinary penalties}\label{rule:extraordinary_penalties}
\begin{enumerate}
	\item \textbf{Penalty for cheating:} If a team member is found cheating or breaking the Fair Play rule (see \refsec{rule:fairplay}), the team will be automatically disqualified of the running test, and a penalty of \scoring{\penaltybig points} is handed out.
	The \iaterm{Technical Committee}{TC} may also disqualify the team for the entire competition.

	\item \textbf{Penalty for faking robots:} If a team starts a test, but it does not solve any of the partial tasks (and is obviously not trying to do so), a penalty of \scoring{\penaltysmall points} is handed out.
	The decision is made by the referees and the monitoring TC member.

	\item \textbf{Extra penalty for collision:} In case of major, (grossly) negligent collisions the \iaterm{Technical Committee}{TC} may disqualify the team for a test (the team receives \scoring{0 points}), or for the entire competition.

	\item \textbf{Not showing up as referee or jury member:} If a team does not provide a referee or jury member (being at the \Arena{} on time), the team receives a penalty of \scoring{\penaltysmall points}, and will be remembered for qualification decisions in future competitions.\\
	Jury members missing a performance to evaluate are excluded from the jury, and the team is disqualified from the test (receives \scoring{0 points}).

	\item \textbf{Modifying or altering standard platform robots:} If any unauthorized modification is found on a Standard Platform League robot, the responsible team will be immediately disqualified for the entire competition while also receiving a penalty of \scoring{\penaltybig points} in the overall score. This behavior will be remembered for qualification decisions in future competitions.\\
\end{enumerate}

\subsection{Bonus for outstanding performance}\label{rule:outstanding_performance}
\begin{enumerate}
	\item For every regular test in \iterm{Stage~I} and \iterm{Stage~II}, the @Home \iaterm{Technical Committee}{TC} can decide to give an extra bonus for \iterm{outstanding performance} of up to 10\% of the maximum test score.

	\item This is to reward teams that do more than what is needed to solely score points in a test but show innovative and general approaches to enhance the scope of @Home.

	\item If a team thinks that it deserves this bonus, it should announce (and briefly explain) this to the \iaterm{Technical Committee}{TC} beforehand.

	\item It is the decision of the \iaterm{Technical Committee}{TC} if (and to which degree) the bonus score is granted.
\end{enumerate}


% Local Variables:
% TeX-master: "../Rulebook"
% End:


%%%%%%%%%%%%%%%%%%%%%%%%%%%%%%%%%%%%%%%%%%%%%%%%%%%%%%%%%
\section{General Instructions for Organizing Committee}
\label{sec:rules:ocinstructions}

In addition to the test specific instructions the \abb{OC} needs to: 
\begin{description}
	\item[During competition:] \hfill
	\begin{compactitem}
		\item Provide \abb{TC} and \Referee\textit{s} with scoring sheets, pens, clipboards, stopwatches and other material relevant of carrying out the scoring.
		\item Post time schedules and results.
	\end{compactitem}
	\item[1h before each test:] \hfill
	\begin{compactitem}
		\item Organize \Referee\textit{s}.
	\end{compactitem}
\end{description}


% Local Variables:
% TeX-master: "../Rulebook"
% End:



% Local Variables:
% TeX-master: "Rulebook"
% End:
