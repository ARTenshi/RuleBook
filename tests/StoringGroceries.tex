\section{Storing Groceries [DSPL \& OPL]}
The robot helps by storing newly bought groceries in the cupboard next to the objects of the same kind that are already there; for instance by placing fresh apples near other apples.

\subsection{Goal}
The robot has to correctly identify and manipulate objects at different heights, grouping them by category and likelihood.

\subsection{Focus}
This test focuses on the detection and recognition of objects and their features, as well as object manipulation.

% %% %%%%%%%%%%%%%%%%%%%%%%%%%%%%%%%%%%%%%%%%%%%%%%%%%%%%%%
%
% Setup
%
% %% %%%%%%%%%%%%%%%%%%%%%%%%%%%%%%%%%%%%%%%%%%%%%%%%%%%%%%
\begin{minipage}{0.70\textwidth}
	\subsection{Setup}
	\begin{enumerate}
		\item \textbf{Location:} This test can take place either inside or outside the arena. The testing area must have a bookcase or cupboard, and a nearby table. The maximum distance between the Table and the Cupboard is 2 meters.
		\item \textbf{Start position:} The robot starts between the cupboard and the table in a random orientation, but facing towards the Cupboard.
		\item \textbf{Cupboard:} The cupboard has 5 shelves between 0.0m and 1.80m from the ground and contains several objects grouped by category or likeliness (See \ref{rule:scenario_objects}). The cupboard has at least one free space for starting a new set.
		\begin{itemize}
		 	\item \textbf{Door:} The cupboard has a single door, which is closed initially.
		 	This door encloses some of the objects, covering up to one half of the cupboard (e.g. the left or bottom half), as indicated by the hatched area in Figure \ref{fig:storing_groceries_shelf}.
		\end{itemize}
		\item \textbf{Table:} The table has at least 5 objects (but no more than 10). If not all objects fit on the table, they will be added as the robot frees up space.
	\end{enumerate}
\end{minipage}\hfill
\begin{minipage}{0.25\textwidth}
	\begin{figure}[H]
		\centering
		\includegraphics[width=\textwidth]{images/storing_groceries.png}%
		\vspace{-10pt}
		%\caption{Example shelf where objects will be placed.}
		\caption{Shelf}
		\label{fig:storing_groceries_shelf}
	\end{figure}
\end{minipage}

% %% %%%%%%%%%%%%%%%%%%%%%%%%%%%%%%%%%%%%%%%%%%%%%%%%%%%%%%
%
% Task
%
% %% %%%%%%%%%%%%%%%%%%%%%%%%%%%%%%%%%%%%%%%%%%%%%%%%%%%%%%
\subsection{Task}
\begin{enumerate}
	\item \textbf{Evaluating the situation:} The robot inspects its surrounding and analyzing the best course of action. In any order, the robot has to:
	\begin{itemize}
		\item \textit{Inspect the cupboard} (locating and categorizing existing groceries).
		\item \textit{Open the cupboard's door.} If the robot can't open the door, it may ask the Referee to do it.
		\item \textit{Find the table}
		\item \textit{Inspect the table} (analyze the newly bought groceries, i.e. objects).
	\end{itemize}

	\item \textbf{Moving objects:} The robot moves as many objects as possible in the given time
	(only the first five score)
	from the Table to the Cupboard, allocating similar objects all together.
	Stacking is allowed.
	\begin{itemize}
		\item Objects of the same type (i.e. identical known objects or akin alike objects) must be placed one next to the other.
		\item If the Cupboard has no object of the same type, then objects must be grouped by category (e.g. drinks with drinks, snacks with snacks, etc)
		\item If the Cupboard has no similar object, the robot must clearly state its decision on how to solve the problem. For instance, the robot can start a new set in a free space for either all unknown objects or all objects sharing a particular feature (color, shape, function, etc.).
		\item Moving two objects at a time (2-handed manipulation) is allowed.
	\end{itemize}

	\textbf{Note:} Either before or after grasping an object the robot may announce the name of the object found.
	\item \textbf{Repeat:} This repeats until the time is up or all groceries are stored.
\end{enumerate}



% %% %%%%%%%%%%%%%%%%%%%%%%%%%%%%%%%%%%%%%%%%%%%%%%%%%%%%%%
%
% Additional Rules
%
% %% %%%%%%%%%%%%%%%%%%%%%%%%%%%%%%%%%%%%%%%%%%%%%%%%%%%%%%
\subsection{Additional rules and remarks}
\begin{enumerate}
	\item \textbf{Bypassing Manipulation:} Bypassing object manipulation via the CONTINUE rule (Section \refsec{rule:asrcontinue}) is not allowed during this test.
	\item \textbf{No setup:} There is no setup time.
	\item \textbf{Startup:} The robot can be started with a simple voice command or via a start button (Section \refsec{rule:start_signal}).
	\item \textbf{Single try:} The robot must be able to start from the first attempt. There is no restart for this test. If the robot is unable to start it must be removed immediately.
	\item \textbf{Collisions:} Slightly touching the cupboard is tolerated (but not advised). Crushing objects or any other form of a major collision terminates the test immediately (Section \refsec{rule:safetyfirst}).
	\item \textbf{Clear area:} The robot may assume that the direct vicinity of the cupboard and table are clear, and that the robot can move slightly backwards for its task.
	\item \textbf{Objects:} The 10 objects are evenly distributed in random fashion including
	3 known objects,
	3 alike objects,
	2 unknown objects, and
	2 special objects (bowl, cloth, dish, etc.).
	% \item \textbf{Timing:} The robot has to successfully place the first object within the first two minutes, otherwise the test is ended. If the robot opens the cupboard door by on its own, one additional minute is added to the 2-minutes limit. The maximum time for this test is 5 minutes.
\end{enumerate}

% \subsection{Data recording}
% Please record the following data (See \refsec{rule:datarecording}):
% \begin{itemize}
% 	\item Images
% 	\item Plans
% \end{itemize}

\subsection{OC instructions}

\textbf{2 hours before the test}
\begin{itemize}
    \item Announce the startup location for robots.
\end{itemize}

\subsection{Referee instructions}
The referee needs to
\begin{itemize}
	\item Place the objects in the cupboard and a few of the same class on the table. New items can be placed when there is room or the robot asks for more objects.
	\item Close the door of the cupboard.
	\item Put objects on the table and the corresponding objects in the cupboard: 3 known objects, 2 alike and 5 unknown objects.
\end{itemize}


\newpage
\subsection{Score sheet}

The maximum time for this test is 5 minutes.

\begin{scorelist}[attempts=4,%
datarecording=true,%
datarecordingbonus=5000,%
outstanding=true,%
outstandingpc=20,%
]
	\scoreheading{Main Goal}
	\scoreitem{500}{Move 5 objects next to their peers in the shelf}
	\penaltyitem[5]{-30}{Receiving human help (point at target location)}
	\penaltyitem[5]{-100}{Receiving human help (move object)}

	\scoreheading{Bonus rewards}
	\scoreitem{300}{Opening the shelf door without human help}
	\scoreitem{100}{Moving a \emph{tiny} object}
	\scoreitem{100}{Moving a \emph{heavy} object}

	% No longer necessary, computes automatically
	% \setTotalScore{1000}
\end{scorelist}


% Local Variables:
% TeX-master: "Rulebook"
% End:


% Local Variables:
% TeX-master: "Rulebook"
% End:
