\section{Procter \& Gamble Challenge [DSPL \& OPL]}

\definecolor{tpblue}{HTML}{182085}%
\definecolor{tporange}{HTML}{FD5B10}%
\newcommand{\TidePod}{%
	\textit{%
		\fontfamily{pvh}\selectfont%
		P\&G
		\fontfamily{phv}\selectfont%
		\textcolor{tpblue}{Tide}%
		\textcolor{tporange}{\textbf{\upshape POD}}%
	}%
}

The robot has to remove all dishes from a table (presumably after dinner) and place them into the dishwasher.

\subsection{Focus}
This test focuses on object perception, manipulation, and planning.

\subsection{Setup}
\begin{itemize}
	\item \textbf{Location:} This test takes place in the arena. A dining table is located close to the dishwasher.
	\item \textbf{Dishwasher:} The dishwasher is near the table, preferably located in the same room. The dish washer is open and with all racks out.
	\item \textbf{Tray:} A plastic tray is located either on top of the dishwasher, or onto one of its racks. The tray may have tableware and cutlery placed inside already.
	\item \textbf{Table setting:} The table has several objects disposed in a typical setting for a meal for one person. These objects include tableware (e.g. place mats, napkins, dishes, glasses), and silverware (e.g. forks, spoons, knives).
	\item \textbf{Spot:} There is a spot on the table next to the table setting.
\end{itemize}

\subsection{Task}
	\begin{enumerate}
		\item \textbf{Entering the arena:} The robot enters the arena and navigates to the designated location.
		\item \textbf{Fetching command:} The operator requests the robot to clean up the table.
		\item \textbf{Clean the table:} The robot takes all the tableware and cutlery to either the dishwasher or to the tray, as instructed (team's choice).
		\item \textbf{Filling the dishwasher:} If the robot placed the objects into the tray, it must proceed to put the tray onto one of the dishwasher racks.
		\item \textbf{Place the \TidePod:} The robot places the  \TidePod~into the dishwasher, preferrably inside the tab compartment.
		\item \textbf{Scrub spots and spills:} The robot detects spots and spills on the table and clean them up using the cleaning cloth or sponge it has retrieved previously.
		\item \textbf{Leave the arena:} The robot leaves the arena once it has finished cleaning.
	\end{enumerate}

\subsection{Additional rules and remarks}
\begin{enumerate}
	\item \textbf{Collisions:} Slightly touching the table is allowed, as well as slightly pushing some objects. However, driving over the objects or any other form of a major collision is not allowed, and the referees directly stop the robot (Section \refsec{rule:safetyfirst}).

	\item \textbf{Objects:} A total of 6 objects are used in this test following the distribution shown below:
	\begin{itemize}
		\item\textit{Silverware}: Any two objects.
		\item\textit{Tableware}: Any three objects, excluding silverware. At least one must be a dish.
		\item\textit{\TidePod}: One \TidePod.
	\end{itemize}
	All objects used in this test are taken from the list of standard objects (See \ref{rule:scenario_objects}). All of them are considered to be known to the robot.

	\item \textbf{Spots and Spills:} The referees must place a spot (e.g. jam or chocolate syrup) or spill some liquid (e.g.milk) on the table before the test starts. The substance used to create the spot shall be clearly visible and contrasting with the table. When cleaning it, it must be clear the robot has detected the spot and is trying actively to clean it. The selection of the cleaning tool (sponge or cloth) is made by the team.\\
	\textbf{Remark:} When possible, no tablecloth will be used to ease cleaning. If removing the tablecloth is not possible, a dry spot will be used instead (e.g. breadcrumbs or coffee powder).

	\item \textbf{Dishwasher:} Is up to the team to decide whether the robot will place the objects in the dishwasher's rack or in the official tray. When using the tray, it should be loaded into the dishwasher.

	\item \textbf{Human-Robot Interaction:} The robot is allowed to
	\begin{enumerate*}[label={\alph*)}]
	\item indicate the location of the cloth or sponge,
	\item ask the human operator what to do with food leftovers,
	and
	\item request operator's help to find the spot (e.g. pointing at it).
	\end{enumerate*}
	This interaction is extensible to any kind of reasonable request from the robot when attempting to solve the task.

\end{enumerate}

\subsection{Data recording}
Please record the following data (See \refsec{rule:datarecording}):
\begin{itemize}
	\item Images of recognized objects
	\item List of moved items
\end{itemize}

\subsection{Referee instructions}

The referee needs to
\begin{itemize}
	\item Place the table setting.
	\item Clean spots smudged by the previous robot.
	\item Place the new spot meant to be clean.
	\item Place the tray on the dishwasher or onto the rack, as requested by the team.
\end{itemize}

\subsection{OC instructions}
During Setup days:
\begin{itemize}
	\item Provide official cutlery and tableware for training.
\end{itemize}

2 hours before the test:
\begin{itemize}
	\item Announce the predefined location to take the command.
	\item Announce the predefined location of the \TidePod.
\end{itemize}

\newpage
\subsection{Score sheet}
The maximum time for this test is \textbf{10 minutes}.

\begin{scorelist}
	\scoreheading{Opening the dishwasher} %50 pts
	\scoreitem{50}{Autonomously opening the dishwasher}

	\scoreheading{Filling the dishwasher (direct)} %200 pts
	\scoreitem[3]{40}{Safely placing a tableware item in the dishwasher's rack}
	\scoreitem[2]{40}{Safely placing a cutlery item in the dishwasher's basket}

	\scoreheading{Filling the dishwasher (tray)} %200 pts
	\scoreitem[3]{30}{Safely placing a tableware item in the tray}
	\scoreitem[2]{35}{Safely placing a cutlery item in the tray}
	\scoreitem{40}{Placing the tray into the dishwasher}

	\scoreheading{Placing the cascade-pod} %40 pts
	\scoreitem{40}{Placing the cascade-pod in the dishwasher's soap compartment}
	\scoreitem{20}{Placing the cascade-pod in the dishwasher (somewhere else)}

	\scoreheading{Cleaning the table} %50 pts
	\scoreitem{50}{Successfully cleaning the spot}
	\scoreitem[-1]{20}{Receiving operator's assistance to find the spot}
	\scoreitem[-1]{20}{Smudging the spot while trying to clean it}

	\scoreheading{Leave the arena} %10 pts
	\scoreitem{10}{Autonomously leave the arena before the time elapses}

	\setTotalScore{350}
\end{scorelist}

% Local Variables:
% TeX-master: "Rulebook"
% End:


% Local Variables:
% TeX-master: "Rulebook"
% End:

