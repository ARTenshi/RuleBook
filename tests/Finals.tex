\chapter{Finals}

The competition ends with the Finals on the last day, where the four teams with the highest total score compete.
The \iterm{Finals} are conducted as a final open demonstration.
This demonstration does not have to be different from the Open Challenge. 
It does not have to be the same either.

To avoid logistical issues during the last day of the competition, the \iterm{Finals} are divided into two sets of demonstrations: the Bronze Competition and the RoboCup @Home Grand Finale.
The Bronze Competition is a set of demonstrations that are carried out before the RoboCup @home Grand Finale. Here, all the leagues run in parallel, with the fourth and third highest scored teams competing for the bronze.
Finally, the two teams with the highest score in each League present their demonstrations in a serialized manner during the RoboCup @Home Grand Finale.

Even though each league has its own first, second and third place, the RoboCup @Home Grand Finale is meant to show the best of all leagues to the jury members as well as the audience and, thus, warrants a single schedule slot.

\section{Evaluating Juries for Final Demonstrations}
Each set of final demonstrations is evaluated by a different combination of evaluating juries, here described.

\begin{enumerate}
\item\textbf{League-internal jury:} The league-internal jury is formed by the Executive Committee.
The evaluation of the league-internal jury is based on the following criteria:
  \begin{compactenum}
  \item Scientific contribution %(maybe taken from the OC)
  \item Contribution to @Home %(evaluated by Execs/TC)
  \item Relevance for @Home / Novelty of approaches %(evaluated by execs/TC)
  \item Presentation and performance in the finals.
  \end{compactenum}

\item \textbf{League-external jury:} The league-external jury consists of people not being involved in the RoboCup@Home league,
but having a related background (not necessarily robotics).
They are appointed by the Executive Committee.
The evaluation of the league-external jury is based on the following criteria:
  \begin{compactenum}
  \item Originality and Presentation
    (story-telling is to be rewarded)
  \item Usability / Human-robot interaction
  \item Multi-modality / System integration
  \item Difficulty and success of the performance
  \item Relevance / Usefulness for daily life
  \end{compactenum}

\item\textbf{Teams-based jury:} The teams-based jury is formed by members of the league's teams.
The evaluation of the teams-based jury is based on the following criteria:
  \begin{compactenum}
  \item Scientific contribution %(maybe taken from the OC)
  \item Contribution to @Home %(evaluated by Execs/TC)
  \item Relevance for @Home / Novelty of approaches %(evaluated by execs/TC)
  \item Presentation and performance in the finals.
  \end{compactenum}
\end{enumerate}


\section{Bronze Competition (4th and 3rd Highest Scoring Teams)}
The demonstration is evaluated by one member of the league-internal jury, by one member of the league-external jury and by the complete team-based jury.
The final score and ranking are determined by the jury evaluations and by the previous performance (in Stages I and II) of the team, in the following manner:

\begin{enumerate}
  \item The influence of the league-internal jury member to the final ranking is \SI{15}{\percent}.
  \item The influence of the league-external jury member to the final ranking is \SI{15}{\percent}.
  \item The influence of the teams-based jury to the final ranking is \SI{15}{\percent}.
  \item The influence of the total sum of points scored by the team in Stage I and II is \SI{55}{\percent}.
\end{enumerate}

These demonstrations are carried out in parallel, having each League perform their own Bronze Competition in their own arena at the same time to save time.

\section{RoboCup@Home Grand Finale (2nd and 1st Highest Scoring Teams)}
The demonstration is evaluated by the complete league-internal and the complete league-external jury.
The final score and ranking are determined by the jury evaluations and by the previous performance (in Stages I and II) of the team, in the following manner:
  
\begin{enumerate}
  \item The influence of the league-internal jury to the final ranking is \SI{25}{\percent}.
  \item The influence of the league-external jury to the final ranking is \SI{25}{\percent}.
  \item The influence of the total sum of points scored by the team in Stage I and II is \SI{50}{\percent}.
\end{enumerate}

These demonstrations are carried out in a serialized fashion, one League performing after another in one arena.


\section{Common Description of Final Demonstrations}
Teams can choose freely what to demonstrate, however it is expected that teams present the scientific and technical contributions they submitted in both \iterm{team description paper} and the \iterm{RoboCup\char64Home Wiki}.
In addition, teams may provide a printed document to the jury (max 2 pages) that summarizes the demonstrated robot capabilities and contributions.  

\subsection{Task}
The procedure for the demonstration and the timing of slots is as follows:
\OpenDemonstrationTask{ten}{five}

\OpenDemonstrationChanges

%% %%%%%%%%%%%%%%%%%%%%%%%%
\section{Final Ranking and Winner}

The winner of the competition is the team that gets the highest
ranking in the finals.

There will be an award for 1st, 2nd and 3rd place. All teams in the
Finals receive a certificate stating that they made it into the Finals
of the RoboCup@Home competition.


% Local Variables:
% TeX-master: "Rulebook"
% End:
