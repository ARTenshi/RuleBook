\chapter{Finals}
\label{chap:finals}
\raggedbottom
The competition ends with the \FINAL{} on the last day, where the two teams with the highest total score compete. The \FINAL{} are conducted as an open, themed demonstration. The \FINAL{} of each league is scheduled at different times.

\section{Structure and Theme}
\label{sec:finals:theme}

To give a better basis to the juries and the audience on how to evaluate the performances, each year the \FINAL{} have a theme which is stated as an objective. The themes for this year are: \\

\textbf{OPL/DSPL:} The robot serves food to a user. (TODO: new theme for this year)

\textbf{SSPL:} The robot naturally interacts with a non-expert user. (TODO: new theme for this year) \\

\noindent Teams are free to use anything from the \AtHome{} scenario (see~\ref{sec:rules:scenario}) and to chose participants from other teams, the jury, or the audience. The themes are open to allow teams to create a story around them which incorporates additional tasks and abilities the team wants to present. Teams may provide a printed document to the jury (max 1 page) that summarizes the demonstrated robot capabilities and contributions. However, teams are discouraged to provide any material that would distract from their demonstration. It is recommended to spend only a short amount of time to explain the demonstration and rather let it speak for itself.


\section{Juries and Evaluation Criteria}
\label{sec:finals:juries}
The \iterm{Finals} are evaluated by two juries:

\begin{enumerate}
\item\textbf{League-Internal Jury:} Formed by the \EC{}. The evaluation is based on the following criteria:
  \begin{compactenum}
  \item Efficacy/elegance of the solution.
  \item Innovation/contribution to the league.
  \item Difficulty of the overall demonstration.
  \item Fit to the Theme.
  \end{compactenum}

\item \textbf{League-External Jury:} Consists of people not being involved in \RoboCup\AtHome{}, but with related background (not necessarily robotics). They are appointed by the \abb{EC}. The evaluation of the league-external jury is based on the following criteria:
  \begin{compactenum}
  \item Originality and presentation (story-telling is to be rewarded).
  \item Relevance/usefulness to everyday life.
  \item Elegance/success of overall demonstration.
  \item Fit to the Theme.
  \end{compactenum}
\end{enumerate}


\section{Procedure}
\label{sec:finals:procedure}
\begin{enumerate}%
	\item \textbf{Schedule:} The team in 2nd place goes first, followed by the current 1st place.
	\item \textbf{Time:} The total time is \SI{15}{\minute}, 10 for setup and demonstration and 5 for questions and clean up.
	\item \textbf{Setup and Demonstration:} The team has to setup, introduce and present their demonstration. The \Arena{} will be in its default state except for the jury area which is announced beforehand.
	\item \textbf{Interview and Cleanup:} Members of both juries can ask questions. During this time, the team has to undo any changes made to the \Arena{} during their demonstration.
\end{enumerate}%


\section{Scoring}
\label{sec:finals:scoring}
The final score and ranking are determined by the jury evaluations and by the previous performance (in \SONE{} and \STWO{}) of the team, in the following manner:
  
\begin{enumerate}
  \item The influence of the league-internal jury to the final ranking is \SI{25}{\percent}.
  \item The influence of the league-external jury to the final ranking is \SI{25}{\percent}.
  \item The influence of the total sum of points scored by the team in Stage I and II is \SI{50}{\percent}.
\end{enumerate}


%% %%%%%%%%%%%%%%%%%%%%%%%%
\section{Final Ranking and Winner}
\label{sec:finals:ranking}

There will be an award for 1st, 2nd and 3rd place of each league.\\
The winner of the competition is the team that gets the highest ranking in the \FINAL{}.\\
The second place will be the team that got the second-highest ranking in the \FINAL{}.\\
The third place will be the team with the highest score that did not make it to the \FINAL{}.\\
Additional certificates are granted if:

\begin{enumerate}
  \item the number of teams in the league is above 11, a certificate will be awarded to the 4th ranked team.
  \item the number of teams in the league is above 14, a certificate will be awarded to the 5th ranked team.
\end{enumerate}


% Local Variables:
% TeX-master: "Rulebook"
% End:
