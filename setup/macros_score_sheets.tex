%%%%%%%%%%%%%%%%%%%%%%%%%%%%%%%%%%%%%%%%%%%%%%%%%%%%%%%%%%%%%%%%%%%%
% Macros for generating score sheets for RoboCup@Home              %
% to be used in the rulebook or during the competition             %
%                                                                  %
% Author: Dirk Holz & David Gossow                                 %
% Modif : Mauricio Matamoros                                       %
% $Id: macros_score_sheets.tex 429 2013-04-30 10:09:55Z holz $     %
%%%%%%%%%%%%%%%%%%%%%%%%%%%%%%%%%%%%%%%%%%%%%%%%%%%%%%%%%%%%%%%%%%%%
% chktex-file 1
% chktex-file 15
% chktex-file 21
% chktex-file 35
% chktex-file 44

% %%% %%%%%%%%%%%%%%%%%%%%%%%%%%%%%%%%%%%%%%%%%%%%%%%%%%%%%%%%%%%%%%
%                                                                  %
% GLOBAL OPTIONS                                                   %
%                                                                  %
% %%% %%%%%%%%%%%%%%%%%%%%%%%%%%%%%%%%%%%%%%%%%%%%%%%%%%%%%%%%%%%%%%

% Set \shortScoresheet to true for the rulebook version
% Set \shortScoresheet to false for the referee's scoresheet
\newcommand{\shortScoresheet}{true}

% The global number of attempts per test
\newcommand{\attempts}{3}

% Sets the total penalty for not showing up
\newcommand{\notattendingpenalty}{500}

% Set to true to display the "Using start button" penalty item
\newcommand{\startbuttonpenalized}{true}

% Sets the total penalty for not using start button instead of door
\newcommand{\startbuttonpenalty}{100}

% Set to true to display the the outstanding performance bonus item
\newcommand{\outstandingPerformanceBonus}{true}

% Percentage of the outstanding performance bonus (ommit % symbol)
\newcommand{\outstandingPerformanceBonusPercentage}{10}

% Set to true to display the data recording bonus item
\newcommand{\dataRecordingBonus}{false}

% Percentage of the data recording bonus (ommit % symbol)
\newcommand{\dataRecordingBonusPercentage}{10}

% Sets the name of the column for referee scoring when \attempts=1
\newcommand{\singleTryColumnCaption}{Single try}

% Sets the first column's name for referee scoring when \attempts=2
\newcommand{\firstTryColumnCaption}{First try}

% Sets the second column's name for referee scoring when \attempts=2
\newcommand{\secondTryColumnCaption}{Restart}

% Sets the name of the column for referee scoring when \attempts=1
\newcommand{\firstColumnCaption}{Action}

% Sets the second column's name for referee scoring when \attempts=2
\newcommand{\secondColumnCaption}{Score}

% %%% %%%%%%%%%%%%%%%%%%%%%%%%%%%%%%%%%%%%%%%%%%%%%%%%%%%%%%%%%%%%%%
%                                                                  %
% USAGE                                                            %
%                                                                  %
% %%% %%%%%%%%%%%%%%%%%%%%%%%%%%%%%%%%%%%%%%%%%%%%%%%%%%%%%%%%%%%%%%
%
% A scoresheet must be in a separated tex file. Scoring marks are
% presented as a list within the {scorelist} environment. Each
% mark is enlisted using the \scoreitem macro. Headings can be
% defined with the \scoreheading macro. Within the scoresheet
% booklet the {scorelist} environment shall be placed inside the
% {scoresheet} environment that adds the footer and heading required
% by the referee.
%
% -= Snippet (rulebook.tex) =-
%     \newpage%
%     \input{my_score_sheet.tex}
% -= End Snippet =-
%
% -= Snippet (score_sheets.tex) =-
%     \begin[options]{scoresheet}
%     \input{my_score_sheet.tex}
%     \end{scoresheet}
% -= End Snippet =-
%
% -= Snippet (my_score_sheet.tex) =-
%     \begin[options]{scorelist}
%       \scoreheading{Main goal}
%       \scoreitem[multiplier]{score}{Description}
%       % These do not contribute to automatic scoring calculation
%       \scorebonus[multiplier]{score}{Description}
%       \scorepenalty[multiplier]{score}{Description}
%     \end{scorelist}
% -= End Snippet =-
%
%
% scorelist options:
% The scorelist environment supports the following comma-separated
% optional arguments:
%   - score              Integer. Sets the test total score to an
%                        arbitrary value (disables autocalc)
%   - attempts           Integer. Number of attempts for the
%                        scoresheet (default is \global\attempts)
%   - continue           Not implemented
%   - datarecording      Boolean. Toggles the "Data Recording"
%                        item under Special penalties and standard
%                        bonuses
%   - datarecordingpc    Integer. Percentage for the data
%                        recording bonus
%   - datarecordingbonus Integer. Arbitrary value for the data
%                        recording bonus
%   - outstanding        Boolean. Toggles the "Outstanding
%                        Performance" item under Special penalties
%                        and standard bonuses
%   - outstandingpc      Integer. Percentage for the
%                        outstanding performance bonus
%   - outstandingbonus   Integer. Arbitrary value for the
%                        outstanding performance bonus
%   - startbutton        Boolean. Toggles the "Using start button"
%                        item under Special penalties and standard
%                        bonuses.
%   - startbuttonpenalty Integer. Arbitrary value for the Using
%                        start button penalty.
%   - firstcolcaption    String. Caption for the first column of
%                        the scoresheet (default="Action")
%   - secondcolcaption   String. Caption for the second column of
%                        the scoresheet (default="Score")
%   - singletrycc        String. Caption of the first column for
%                        referee scoring when attempts=1
%   - firsttrycc         String. Caption of the first column for
%                        referee scoring when attempts=2
%   - secondtrycc        String. Caption of the second column for
%                        referee scoring when attempts=2
%
%
%
% scoreitem, scorebonus, and scorepenalty arguments
%   #1 multiplier   A number indicating how many times the mark
%                   can be scored. It is printed at the left of
%                   the score followed by the \times symbol.
%   #2 score        Scoring points. Printed at the right of the
%                   description
%   #3 description  A description for the score mark
%
%
%
%
%
%
%
%
%
%
%
%
%
%
%
%
%
%
%
%
%
%
%
%
%
%
%
%
%
%
%
%
%
%
%
%
%
%
%
%
%
%
%
%
%
%
%
%
%
%
% %%% %%%%%%%%%%%%%%%%%%%%%%%%%%%%%%%%%%%%%%%%%%%%%%%%%%%%%%%%%%%%%%
%                                                                  %
% FROM HERE ON, THERE IS NOTHING TO CHANGE                         %
%                                                                  %
% %%% %%%%%%%%%%%%%%%%%%%%%%%%%%%%%%%%%%%%%%%%%%%%%%%%%%%%%%%%%%%%%%

%%% Counters / temp. variables %%%%%%%%%%%%%%%%%
\newcounter{currTestScore}
\newcounter{currTestScoreTotal}
\newcounter{currTestScoreTotalWithoutBonus}
\newcounter{currOutstandingBonus}
\newcounter{currDataRecordingBonus}


% set \continueAvailable to true for CONTINUE sections
\newcommand{\continueAvailable}{true}


% name of the current test, is set automatically in the rulebook
\newcommand{\currentTest}{}

% (internal) if-clause shortcut to switch between short rulebook version and full score sheet for referees
\newcommand{\ifShortScoresheet}[2]{%
	\ifthenelse{ \equal{\shortScoresheet}{true} }{#1}{#2}%
}

% (internal) draws the scoresheet line for score handwritting
\newcommand{\scoreline}[1][0.08]{\rule{#1\linewidth}{.2pt}}

% (internal) returns absolute value of argument
\newcommand{\absval}[1]{\ifnum#1<0 -\fi#1}
















% %%% %%%%%%%%%%%%%%%%%%%%%%%%%%%%%%%%%%%%%%%%%%%%%%%%%%%%%%%%%%%%%%
%                                                                  %
% ENVIRONMENT: scoresheet                                          %
% Scoresheet page layout                                           %
%                                                                  %
% %%% %%%%%%%%%%%%%%%%%%%%%%%%%%%%%%%%%%%%%%%%%%%%%%%%%%%%%%%%%%%%%%

\newenvironment{scoresheet}{%
% \begin{scoresheet}
	\newpage%
	%
	% Test, team, and referee info
	%
	\begin{minipage}[t]{0.85\textwidth}%
		\vspace{0pt}%
		{\huge \textbf{Score Sheet} }%
		\vspace{2 em}%

		\begin{tabular}{ @{} l l l}
			\textbf{Test:} & \currentTest \\[.9 em]%
			\textbf{Team name:} & \scoreline[0.6]\\[.9 em]%
			\textbf{Referee name:} & \scoreline[0.6]\\[.9 em]%
		\end{tabular}%
		\vspace{0.5 em}%

	\end{minipage}
	\hfill
	%
	% @Home Logo
	%
	\begin{minipage}[t]{0.15\textwidth}%
		\vspace{0pt}%
		\includegraphics[width=\textwidth]{images/logo_RoboCupAtHome.jpg}%
	\end{minipage}\\%
}{
% \end{scoresheet}
	\vspace{0.5 em}%
	\textbf{Remarks:}%

	%
	% Signatures of referee / team leader %%%%%%%%%%%%
	%
	\vfill
	\begin{tabular*}{\linewidth}{@{} @{\extracolsep{\fill}} l l l @{}}
		\scoreline[0.25] \hspace{0.05\linewidth}%
			& \scoreline[0.25] \hspace{0.05\linewidth}%
			& \scoreline[0.25]%
		\\
		\textit{Date \& time}%
			& \textit{Referee} %
			& \textit{Team leader}%
	\end{tabular*}

	\newpage
}














% %%% %%%%%%%%%%%%%%%%%%%%%%%%%%%%%%%%%%%%%%%%%%%%%%%%%%%%%%%%%%%%%%
%                                                                  %
% ENVIRONMENT: scorelist                                           %
% Score list table                                                 %
%                                                                  %
% %%% %%%%%%%%%%%%%%%%%%%%%%%%%%%%%%%%%%%%%%%%%%%%%%%%%%%%%%%%%%%%%%

\usepackage{pgfkeys}
\pgfkeys{
	/scorelist/.is family, /scorelist,
	default/.style={
		attempts = \attempts,
		continue = true,
		datarecording = \dataRecordingBonus,
		datarecordingpc = \dataRecordingBonusPercentage,
		datarecordingbonus = 0,
		outstanding = \outstandingPerformanceBonus,
		outstandingpc = \outstandingPerformanceBonusPercentage,
		outstandingbonus = 0,
		startbutton = \startbuttonpenalized,
		startbuttonpenalty = \startbuttonpenalty,
		firstcolcaption = \firstColumnCaption,
		secondcolcaption = \secondColumnCaption,
		singletrycc = \singleTryColumnCaption,
		firsttrycc = \firstTryColumnCaption,
		secondtrycc = \secondTryColumnCaption,
	},
	attempts/.estore in = \scorelistAttempts,
	continue/.estore in = \scorelistContinue,
	datarecording/.estore in = \scorelistDataRecording,
	datarecordingpc/.estore in = \scorelistDataRecordingPercentage,
	datarecordingbonus/.estore in = \scorelistDataRecordingBonus,
	outstanding/.estore in = \scorelistOutstanding,
	outstandingpc/.estore in = \scorelistOutstandingPercentage,
	outstandingbonus/.estore in = \scorelistOutstandingBonus,
	startbutton/.estore in = \scorelistStartButton,
	startbuttonpenalty/.estore in = \scorelistStartButtonPenalty,
	firstcolcaption/.estore in = \scorelistFirstColCaption,
	secondcolcaption/.estore in = \scorelistSecondColCaption,
	singletrycc/.estore in = \scorelistSingleTryCC,
	firsttrycc/.estore in = \scorelistFirstTryCC,
	secondtrycc/.estore in = \scorelistSecondTryCC,
}

\makeatletter%
\NewEnviron{scorelist}[1][]{
% \begin{scorelist}
	%%%%%%%%%%%%%%%%%%%%%%%%%%%%%%%%%%%%%%%%%%%%%%%%%%%%%%%%%%%%%%%
	% read options
	%%%%%%%%%%%%%%%%%%%%%%%%%%%%%%%%%%%%%%%%%%%%%%%%%%%%%%%%%%%%%%%
	\pgfkeys{/scorelist, default, #1}%

	%%%%%%%%%%%%%%%%%%%%%%%%%%%%%%%%%%%%%%%%%%%%%%%%%%%%%%%%%%%%%%%
	% init variables %%%%%%%%%%%%%%%%%%%%%%%%%%%%%%%%%%%%%%%%%%%%%%
	%%%%%%%%%%%%%%%%%%%%%%%%%%%%%%%%%%%%%%%%%%%%%%%%%%%%%%%%%%%%%%%
	\setcounter{currTestScore}{0}
	\setcounter{currOutstandingBonus}{\scorelistOutstandingBonus}
	\setcounter{currDataRecordingBonus}{\scorelistDataRecordingBonus}

	%%%%%%%%%%%%%%%%%%%%%%%%%%%%%%%%%%%%%%%%%%%%%%%%%%%%%%%%%%%%%%%
	% environment commands %%%%%%%%%%%%%%%%%%%%%%%%%%%%%%%%%%%%%%%%
	%%%%%%%%%%%%%%%%%%%%%%%%%%%%%%%%%%%%%%%%%%%%%%%%%%%%%%%%%%%%%%%

	% heading %%%%%%%%%%%%%%%%%%%%%%%%%%%%%%%%%%%%%%%%%%%%%%%%%%%%%
	\newcommand{\scoreheading}[1]{%
		\ifShortScoresheet{%
			\xdef\@scoreheadingcolspan{2}%
		}{%
			\xdef\@scoreheadingcolspan{\the\numexpr2+\scorelistAttempts\relax}%
		}%
		\phantom{.} \\[-12pt]%
		\multicolumn{\@scoreheadingcolspan}{@{}l}{\textbi{##1}}\\[0pt]%
	}

	\newcommand{\scoreitem}[3][1]{%
		\ifthenelse{ ##2 > 0 }{%
			\addtocounter{currTestScore}{ ##2 * ##1 }%
		}{}%
		%
		\@scoreitem[##1]{##2}{##3}%
	}

	\newcommand{\penaltyitem}[3][1]{%
		\ifthenelse{##2 < 0}{%
			\@scoreitem[##1]{\absval{##2}}{##3}%
		}{%
			\@scoreitem[##1]{\absval{-##2}}{##3}%
		}%
	}

	\newcommand{\bonusitem}[3][1]{%
		\@scoreitem[##1]{##2}{##3}%
	}

	%%%%%%%%%%%%%%%%%%%%%%%%%%%%%%%%%%%%%%%%%%%%%%%%%%%%%%%%%%%%%%%
	% Commands for overriding internal calculations %%%%%%%%%%%%%%%
	%%%%%%%%%%%%%%%%%%%%%%%%%%%%%%%%%%%%%%%%%%%%%%%%%%%%%%%%%%%%%%%

	% set score counter to arbitrary value %%%%%%%%%%%%%%%%%%%%%%%%
	\newcommand{\setTotalScore}[1]%
	{%
		\setcounter{currTestScore}{##1}%
	}

	% set outstanding bonus counter to arbitrary value %%%%%%%%%%%%
	\newcommand{\setOutstandingBonus}[1]%
	{%
		\setcounter{currOutstandingBonus}{##1}%
	}

	%%%%%%%%%%%%%%%%%%%%%%%%%%%%%%%%%%%%%%%%%%%%%%%%%%%%%%%%%%%%%%%
	% environment internal commands %%%%%%%%%%%%%%%%%%%%%%%%%%%%%%%
	%%%%%%%%%%%%%%%%%%%%%%%%%%%%%%%%%%%%%%%%%%%%%%%%%%%%%%%%%%%%%%%

	% table entry %%%%%%%%%%%%%%%%%%%%%%%%%%%%%%%%%%%%%%%%%%%%%%%%%
	\newcommand{\@scoreitem}[3][1]{%
		##3\vspace{0.1em} &%
		\textit{%
			\ifthenelse{ \equal{##2}{0} }{~}{% else
				\ifthenelse{ \equal{##1}{1} }{}{##1$\times$}%
			##2}%
		}%
		\ifShortScoresheet{}{&\attemptScoreLines{\scorelistAttempts}}\\[0pt]%
	}

	% [INTERNAL] writes down the line for the final total score %%%
	\newcommand{\scoreTotal}{%
		\\%
		\textbf{Total score~}%
		\ifShortScoresheet{%
			(excluding penalties and standard bonuses) &%
			\textit{\thecurrTestScoreTotalWithoutBonus}%
		}{%
			&\textit{\thecurrTestScoreTotal}
			&\multicolumn{\scorelistAttempts}{c}{\scoreline[0.20]}%
		}\\[0pt]%
	}

	% [INTERNAL] writes down the lines for the total score per try
	\newcommand{\scorePerTry}{%
		\\%
		\ifShortScoresheet{}{%
			\ifthenelse{ \attempts > 1 }{%
				\textbi{Score per try} &%
				\textit{\thecurrTestScoreTotalWithoutBonus} &%
				\attemptScoreLines{\scorelistAttempts}%
				\\[0pt]%
			}{}%
		}%
	}

	% [INTERNAL] draws a line for referee scoring %%%%%%%%%%%%%%%%%
	\gdef\@marklinewidth{0.06}
	\newcommand{\markline}{\rule{\@marklinewidth\linewidth}{.2pt}}

	% [INTERNAL] draws all the line for referee scoring %%%%%%%%%%%
	\newcommand{\attemptScoreLines}[1]{%
		\protected@xdef\@scorelines{\markline}%
		\ifthenelse{##1 > 1}{%
			\foreach \i in {2,...,##1}{%
				\protected@xdef\@scorelines{\@scorelines & \markline}%
			}%
		}{}%
		\@scorelines%
	}

	% [INTERNAL] writes down the headings for referee scoring %%%%%
	\newcommand{\attemptHeadings}[1]{%
		\ifthenelse{\equal{##1}{1}}{%
			\gdef\@attemptheadings{\textbf{\scorelistSingleTryCC}}%
			\gdef\@marklinewidth{0.1}%
		}{}%
		\ifthenelse{\equal{##1}{2}}{%
			\gdef\@attemptheadings{\textbf{\scorelistFirstTryCC} & \textbf{\scorelistSecondTryCC}}%
			\gdef\@marklinewidth{0.08}%
		}{}%
		\ifthenelse{##1 > 2}{
			\protected@xdef\@attemptheadings{%
				\textbf{\small$1^{st}$~try} &%
				\textbf{\small$2^{nd}$~try} &%
				\textbf{\small$3^{rd}$~try}}%
		}{}%
		\ifthenelse{##1 > 3}{
			\foreach \i in {4,...,##1}{% chktex11
				\protected@xdef\@attemptheadings{%
					\@attemptheadings &%
					\textbf{\small$\i^{th}$~try}%
				}%
			}%
			\gdef\@marklinewidth{0.06}%
		}{}%
		\@attemptheadings%
	}

	%%%%%%%%%%%%%%%%%%%%%%%%%%%%%%%%%%%%%%%%%%%%%%%%%%%%%%%%%%%%%%%
	% setup table %%%%%%%%%%%%%%%%%%%%%%%%%%%%%%%%%%%%%%%%%%%%%%%%%
	%%%%%%%%%%%%%%%%%%%%%%%%%%%%%%%%%%%%%%%%%%%%%%%%%%%%%%%%%%%%%%%
	\vspace{0.8 em}%
	\noindent%
	\begin{tabularx}{\textwidth}{ @{}X @{}r *{\scorelistAttempts}{c}}
		\textbf{\scorelistFirstColCaption} &%
		\textbf{\scorelistSecondColCaption}%
		\ifShortScoresheet{}{&\attemptHeadings{\scorelistAttempts}}
	\\\hline



\BODY



	%%%%%%%%%%%%%%%%%%%%%%%%%%%%%%%%%%%%%%%%%%%%%%%%%%%%%%%%%%%%%%%
	% calculate max. score, and bonuses %%%%%%%%%%%%%%%%%%%%%%%%%%%
	%%%%%%%%%%%%%%%%%%%%%%%%%%%%%%%%%%%%%%%%%%%%%%%%%%%%%%%%%%%%%%%
	% base total score (accumulative) %%%%%%%%%%%%%%%%%%%%%%%%%%%%%
	\setcounter{currTestScoreTotal}{\thecurrTestScore}
	% outstanding performance bonus %%%%%%%%%%%%%%%%%%%%%%%%%%%%%%%
	\ifthenelse{\equal{\scorelistOutstanding}{true}}{%
		\ifthenelse{ \equal{\thecurrOutstandingBonus}{0} }{%
			\setcounter{currOutstandingBonus}{ \thecurrTestScore*\scorelistOutstandingPercentage/100 }
		}{}%
		\setcounter{currTestScoreTotal}{%
			\thecurrTestScoreTotal + \thecurrOutstandingBonus}%
	}{}%
	% data recording bonus %%%%%%%%%%%%%%%%%%%%%%%%%%%%%%%%%%%%%%%%
	\ifthenelse{\equal{\scorelistDataRecording}{true}}{%
		\ifthenelse{ \equal{\thecurrDataRecordingBonus}{0} }{%
			\setcounter{currDataRecordingBonus}{ \thecurrTestScore*\scorelistDataRecordingPercentage/100 }%
		}{}%
		\setcounter{currTestScoreTotal}{%
			\thecurrTestScoreTotal + \thecurrOutstandingBonus}%
	}{}%
	\setcounter{currTestScoreTotalWithoutBonus}{ \thecurrTestScore }

	%%%%%%%%%%%%%%%%%%%%%%%%%%%%%%%%%%%%%%%%%%%%%%%%%%%%%%%%%%%%%%%
	% Special penalties & bonuses %%%%%%%%%%%%%%%%%%%%%%%%%%%%%%%%%
	%%%%%%%%%%%%%%%%%%%%%%%%%%%%%%%%%%%%%%%%%%%%%%%%%%%%%%%%%%%%%%%
	\scoreheading{Special penalties \& standard bonuses}

	% not showing up penalty %%%%%%%%%%%%%%%%%%%%%%%%%%%%%%%%%%%%%%
	\penaltyitem{\notattendingpenalty}{Not attending \ifShortScoresheet{(see sec.~\ref{rule:not_attending})}{}}

	% require signal for door opening %%%%%%%%%%%%%%%%%%%%%%%%%%%%%
	\ifthenelse{ \equal{\scorelistStartButton}{true} }{
	  \penaltyitem{\scorelistStartButtonPenalty}{Using start button \ifShortScoresheet{(see sec.~\ref{rule:start_button})}{}}
	}{}

	% data recording bonus %%%%%%%%%%%%%%%%%%%%%%%%%%%%%%%%%%%%%%%%
	\ifthenelse{%
		\thecurrDataRecordingBonus>0 \AND %
		\equal{\scorelistDataRecording}{true}%
	}{%
		\bonusitem{\thecurrDataRecordingBonus}{Contributing with recorded data ($\frac{\sum gathered~points}{max~points} \times$) \ifShortScoresheet{(see sec.~\ref{rule:datarecording})}{}}%
	}{}%

	% outstanding performance bonus %%%%%%%%%%%%%%%%%%%%%%%%%%%%%%%
	\ifthenelse{%
		\value{currOutstandingBonus}>0 \AND %
		\equal{\scorelistOutstanding}{true}%
	}{%
		\bonusitem{\thecurrOutstandingBonus}{Outstanding performance~\ifShortScoresheet{(see sec.~\ref{rule:outstanding_performance})}{}}%
	}{}%
	%
	% Total score %%%%%%%%%%%%%%%%%%%%%%%%%%%%%%%%%%%%%%%%%%%%%%%%%
	\\[-1em]\hline
	\scorePerTry
	\scoreTotal

	\end{tabularx}
}
\makeatother%

% Local Variables:
% TeX-master: "../Rulebook"
% End:
%
