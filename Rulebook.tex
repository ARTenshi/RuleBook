%%%%%%%%%%%%%%%%%%%%%%%%%%%%%%%%%%%%%%%%%%%%%%%%%%%%%%%%%%%%%%%%%%%%%%%%%%%%%%%
%%
%%          $Id: Rulebook.tex 2014-12-12 balkce $
%%    author(s): RoboCupAtHome Technical Committee(s)
%%  description: introduction to RoboCupAtHome
%%
%%%%%%%%%%%%%%%%%%%%%%%%%%%%%%%%%%%%%%%%%%%%%%%%%%%%%%%%%%%%%%%%%%%%%%%%%%%%%%%
\documentclass[11pt, twoside, openright, a4paper, chapterprefix]{scrbook}
\usepackage[inner=2.5cm, outer=2.5cm, top=4cm, bottom=4cm]{geometry}

%%% PACKAGES %%%%%%%%%%%%%%%%%%%%%%%%%%%%%%%%%%%%%%%%%%%%%%%%%%%%%%%%%%%%%%%%%%
%%%%%%%%%%%%%%%%%%%%%%%%%%%%%%%%%%%%%%%%%%%%%%%%%%%%%%%%%%%%%%%%%%%%%%%%%%%%%%%
%%
%%          $Id: packages.tex 385 2013-02-12 21:53:10Z holz $
%%    author(s): RoboCupAtHome Technical Committee(s)
%%  description: List of packages for the RoboCupAtHome rulebook
%%
%%%%%%%%%%%%%%%%%%%%%%%%%%%%%%%%%%%%%%%%%%%%%%%%%%%%%%%%%%%%%%%%%%%%%%%%%%%%%%%
% \usepackage{soul}

\usepackage[utf8x]{inputenc}
\usepackage[english]{babel}
\usepackage{amsmath,amssymb,amsfonts}
% \usepackage[nice]{nicefrac}
\usepackage{siunitx}
\usepackage{graphicx}
\usepackage{multicol}
\usepackage{verbatim}
\usepackage{fancyhdr}

% \usepackage{color}
\usepackage{xcolor}
\usepackage{colortbl}
% \usepackage{epsfig}
\usepackage{makeidx} % This one causes scoresheets not to setle
% \usepackage{lscape}
% \usepackage{picinpar}

\usepackage{./styles/tweaklist}

\usepackage{enumerate}
\usepackage{paralist}
\usepackage{multirow}
\usepackage{hhline}
\usepackage{pgffor}
% \usepackage{array}

\usepackage{nameref}
\usepackage{varioref}
\usepackage{hyperref}
\usepackage[noabbrev,nameinlink]{cleveref}
\usepackage{tabularx}
\usepackage{xspace}
\usepackage{csquotes}
\usepackage[inline]{enumitem}

%\usepackage{times}
%\usepackage{helvet}
%\usepackage{courier}

% \usepackage{url}
\usepackage{caption}
% \usepackage{epstopdf}
\usepackage{subfig}
\usepackage{float}
\usepackage{wrapfig}
% \usepackage{xfrac}

% \usepackage[titletoc]{appendix}
% \usepackage{enumitem}
% \usepackage{mathtools}
% \usepackage{gensymb}

% Required by scoresheets
\usepackage{calc}
\usepackage{ifthen}
\usepackage{environ}
\usepackage{wasysym}
\usepackage{chngpage}

% Local Variables:
% TeX-master: "../Rulebook"
% End:

\usepackage[titletoc]{appendix}
\usepackage{enumitem}
\usepackage{mathtools}
\usepackage{gensymb}
\setlist{noitemsep}

%%% SubfigureSetup %%%%%%%%%%%%%%%%%%%%%%%%%%%%%%%%%%%%%%%%%%%%%%%%%%%%%%%%%%%%
%\renewcommand{\subfigtopskip}{5pt}        % default is 10pt
%\renewcommand{\subfigbottomskip}{5pt}     % default is 10pt
%\renewcommand{\subfigcapskip}{3pt}        % default is 10pt
%\renewcommand{\subfigcapmargin}{7pt}      % default is 10pt

%%% TweakList-Setup %%%%%%%%%%%%%%%%%%%%%%%%%%%%%%%%%%%%%%%%%%%%%%%%%%%%%%%%%%%
\renewcommand{\itemhook}{%                 % modify itemize-spacing
  \setlength{\topsep}{2pt}%
  \setlength{\partopsep}{1pt}%
  \setlength{\itemsep}{-1pt}%
}
\renewcommand{\enumhook}{%                 % modify enumerate-spacing
  \setlength{\topsep}{2pt}%
  \setlength{\partopsep}{1pt}%
  \setlength{\itemsep}{-1pt}%
}
\renewcommand{\descripthook}{%             % modify description-spacing
  \setlength{\topsep}{2pt}%
  \setlength{\partopsep}{1pt}%
  \setlength{\itemsep}{-1pt}%
}

\setkomafont{title}{\normalfont}
\setkomafont{sectioning}{\normalfont\bfseries}
\addtokomafont{caption}{\small}
\setkomafont{captionlabel}{\small\bfseries}
\setkomafont{descriptionlabel}{\normalfont\bfseries}
\renewcommand*{\chapterformat}{\LARGE{Chapter \thechapter}}

%%% MACROS %%%%%%%%%%%%%%%%%%%%%%%%%%%%%%%%%%%%%%%%%%%%%%%%%%%%%%%%%%%%%%%%%%%%
\newcommand{\YEAR}{2019}
% \newcommand{\STATE}{Draft}
\newcommand{\STATE}{Final}
%
% Local Variables:
% TeX-master: "../Rulebook"
% End:

\graphicspath{{\YEAR/}{./images/}}
%%%%%%%%%%%%%%%%%%%%%%%%%%%%%%%%%%%%%%%%%%%%%%%%%%%%%%%%%%%%%%%%%%%%%%%%%%%%%%%
%%
%%          $Id: macros.tex 399 2013-02-14 20:24:02Z holz $
%%    author(s): RoboCupAtHome Technical Committee(s)
%%  description: Macros for the RoboCupAtHome rulebook
%%
%%%%%%%%%%%%%%%%%%%%%%%%%%%%%%%%%%%%%%%%%%%%%%%%%%%%%%%%%%%%%%%%%%%%%%%%%%%%%%%
% rubber: setlist arguments --shell-escape

%%%%%%%%%%%%%%%%%%%%%%%%%%%%%%%%%%%%%%%%%%%%%%%%%%%%%%%%%%%%%%%%%%%%
% Macros for generating score sheets for RoboCup@Home              %
% to be used in the rulebook or during the competition             %
%                                                                  %
% Author: Dirk Holz & David Gossow                                 %
% Modif : Mauricio Matamoros                                       %
% $Id: macros_score_sheets.tex 429 2013-04-30 10:09:55Z holz $     %
%%%%%%%%%%%%%%%%%%%%%%%%%%%%%%%%%%%%%%%%%%%%%%%%%%%%%%%%%%%%%%%%%%%%

% %%% %%%%%%%%%%%%%%%%%%%%%%%%%%%%%%%%%%%%%%%%%%%%%%%%%%%%%%%%%%%%%%
%                                                                  %
% GLOBAL OPTIONS                                                   %
%                                                                  %
% %%% %%%%%%%%%%%%%%%%%%%%%%%%%%%%%%%%%%%%%%%%%%%%%%%%%%%%%%%%%%%%%%

% Set \shortScoresheet to true for the rulebook version
% Set \shortScoresheet to false for the referee's scoresheet
\newcommand{\shortScoresheet}{true}

% The global number of attempts per test
\newcommand{\attempts}{3}

% Sets the total penalty for not showing up
\newcommand{\notattendingpenalty}{500}

% Set to true to display the "Using start button" penalty item
\newcommand{\startbuttonpenalized}{true}

% Sets the total penalty for not using start button instead of door
\newcommand{\startbuttonpenalty}{100}

% Set to true to display the the outstanding performance bonus item
\newcommand{\outstandingPerformanceBonus}{true}

% Percentage of the outstanding performance bonus (ommit % symbol)
\newcommand{\outstandingPerformanceBonusPercentage}{10}

% Set to true to display the data recording bonus item
\newcommand{\dataRecordingBonus}{false}

% Percentage of the data recording bonus (ommit % symbol)
\newcommand{\dataRecordingBonusPercentage}{10}

% Sets the name of the column for referee scoring when \attempts=1
\newcommand{\singleTryColumnCaption}{Single try}

% Sets the first column's name for referee scoring when \attempts=2
\newcommand{\firstTryColumnCaption}{First try}

% Sets the second column's name for referee scoring when \attempts=2
\newcommand{\secondTryColumnCaption}{Restart}

% Sets the name of the column for referee scoring when \attempts=1
\newcommand{\firstColumnCaption}{Action}

% Sets the second column's name for referee scoring when \attempts=2
\newcommand{\secondColumnCaption}{Score}

% %%% %%%%%%%%%%%%%%%%%%%%%%%%%%%%%%%%%%%%%%%%%%%%%%%%%%%%%%%%%%%%%%
%                                                                  %
% USAGE                                                            %
%                                                                  %
% %%% %%%%%%%%%%%%%%%%%%%%%%%%%%%%%%%%%%%%%%%%%%%%%%%%%%%%%%%%%%%%%%
%
% A scoresheet must be in a separated tex file. Scoring marks are
% presented as a list within the {scorelist} environment. Each
% mark is enlisted using the \scoreitem macro. Headings can be
% defined with the \scoreheading macro. Within the scoresheet
% booklet the {scorelist} environment shall be placed inside the
% {scoresheet} environment that adds the footer and heading required
% by the referee.
%
% -= Snippet (rulebook.tex) =-
%     \newpage%
%     \input{my_score_sheet.tex}
% -= End Snippet =-
%
% -= Snippet (score_sheets.tex) =-
%     \begin[options]{scoresheet}
%     \input{my_score_sheet.tex}
%     \end{scoresheet}
% -= End Snippet =-
%
% -= Snippet (my_score_sheet.tex) =-
%     \begin[options]{scorelist}
%       \scoreheading{Main goal}
%       \scoreitem[multiplier]{score}{Description}
%       % These do not contribute to automatic scoring calculation
%       \scorebonus[multiplier]{score}{Description}
%       \scorepenalty[multiplier]{score}{Description}
%     \end{scorelist}
% -= End Snippet =-
%
%
% scorelist options:
% The scorelist environment supports the following comma-separated
% optional arguments:
%   - score              Integer. Sets the test total score to an
%                        arbitrary value (disables autocalc)
%   - attempts           Integer. Number of attempts for the
%                        scoresheet (default is \global\attempts)
%   - continue           Not implemented
%   - datarecording      Boolean. Toggles the "Data Recording"
%                        item under Special penalties and standard
%                        bonuses
%   - datarecordingpc    Integer. Percentage for the data
%                        recording bonus
%   - datarecordingbonus Integer. Arbitrary value for the data
%                        recording bonus
%   - outstanding        Boolean. Toggles the "Outstanding
%                        Performance" item under Special penalties
%                        and standard bonuses
%   - outstandingpc      Integer. Percentage for the
%                        outstanding performance bonus
%   - outstandingbonus   Integer. Arbitrary value for the
%                        outstanding performance bonus
%   - startbutton        Boolean. Toggles the "Using start button"
%                        item under Special penalties and standard
%                        bonuses.
%   - startbuttonpenalty Integer. Arbitrary value for the Using
%                        start button penalty.
%   - firstcolcaption    String. Caption for the first column of
%                        the scoresheet (default="Action")
%   - secondcolcaption   String. Caption for the second column of
%                        the scoresheet (default="Score")
%   - singletrycc        String. Caption of the first column for
%                        referee scoring when attempts=1
%   - firsttrycc         String. Caption of the first column for
%                        referee scoring when attempts=2
%   - secondtrycc        String. Caption of the second column for
%                        referee scoring when attempts=2
%
%
%
% scoreitem, scorebonus, and scorepenalty arguments
%   #1 multiplier   A number indicating how many times the mark
%                   can be scored. It is printed at the left of
%                   the score followed by the \times symbol.
%   #2 score        Scoring points. Printed at the right of the
%                   description
%   #3 description  A description for the score mark
%
%
%
%
%
%
%
%
%
%
%
%
%
%
%
%
%
%
%
%
%
%
%
%
%
%
%
%
%
%
%
%
%
%
%
%
%
%
%
%
%
%
%
%
%
%
%
%
%
%
% %%% %%%%%%%%%%%%%%%%%%%%%%%%%%%%%%%%%%%%%%%%%%%%%%%%%%%%%%%%%%%%%%
%                                                                  %
% FROM HERE ON, THERE IS NOTHING TO CHANGE                         %
%                                                                  %
% %%% %%%%%%%%%%%%%%%%%%%%%%%%%%%%%%%%%%%%%%%%%%%%%%%%%%%%%%%%%%%%%%

%%% Counters / temp. variables %%%%%%%%%%%%%%%%%
\newcounter{currTestScore}
\newcounter{currTestScoreTotal}
\newcounter{currTestScoreTotalWithoutBonus}
\newcounter{currOutstandingBonus}
\newcounter{currDataRecordingBonus}


% set \continueAvailable to true for CONTINUE sections
\newcommand{\continueAvailable}{true}


% name of the current test, is set automatically in the rulebook
\newcommand{\currentTest}{}

% (internal) if-clause shortcut to switch between short rulebook version and full score sheet for referees
\newcommand{\ifShortScoresheet}[2]{%
	\ifthenelse{ \equal{\shortScoresheet}{true} }{#1}{#2}%
}

% (internal) draws the scoresheet line for score handwritting
\newcommand{\scoreline}[1][0.08]{\rule{#1\linewidth}{.2pt}}

% (internal) returns absolute value of argument
\newcommand{\absval}[1]{\ifnum#1<0 -\fi#1}
















% %%% %%%%%%%%%%%%%%%%%%%%%%%%%%%%%%%%%%%%%%%%%%%%%%%%%%%%%%%%%%%%%%
%                                                                  %
% ENVIRONMENT: scoresheet                                          %
% Scoresheet page layout                                           %
%                                                                  %
% %%% %%%%%%%%%%%%%%%%%%%%%%%%%%%%%%%%%%%%%%%%%%%%%%%%%%%%%%%%%%%%%%

\newenvironment{scoresheet}{%
% \begin{scoresheet}
	\newpage%
	%
	% Test, team, and referee info
	%
	\begin{minipage}[t]{0.85\textwidth}%
		\vspace{0pt}%
		{\huge \textbf{Score Sheet} }%
		\vspace{2 em}%

		\begin{tabular}{ @{} l l l}
			\textbf{Test:} & \currentTest \\[.9 em]%
			\textbf{Team name:} & \scoreline[0.6]\\[.9 em]%
			\textbf{Referee name:} & \scoreline[0.6]\\[.9 em]%
		\end{tabular}%
		\vspace{0.5 em}%

	\end{minipage}
	\hfill
	%
	% @Home Logo
	%
	\begin{minipage}[t]{0.15\textwidth}%
		\vspace{0pt}%
		\includegraphics[width=\textwidth]{images/logo_RoboCupAtHome.jpg}%
	\end{minipage}\\%
}{
% \end{scoresheet}
	\vspace{0.5 em}%
	\textbf{Remarks:}%

	%
	% Signatures of referee / team leader %%%%%%%%%%%%
	%
	\vfill
	\begin{tabular*}{\linewidth}{@{} @{\extracolsep{\fill}} l l l @{}}
		\scoreline[0.25] \hspace{0.05\linewidth}%
			& \scoreline[0.25] \hspace{0.05\linewidth}%
			& \scoreline[0.25]%
		\\
		\textit{Date \& time}%
			& \textit{Referee} %
			& \textit{Team leader}%
	\end{tabular*}

	\newpage
}














% %%% %%%%%%%%%%%%%%%%%%%%%%%%%%%%%%%%%%%%%%%%%%%%%%%%%%%%%%%%%%%%%%
%                                                                  %
% ENVIRONMENT: scorelist                                           %
% Score list table                                                 %
%                                                                  %
% %%% %%%%%%%%%%%%%%%%%%%%%%%%%%%%%%%%%%%%%%%%%%%%%%%%%%%%%%%%%%%%%%

\usepackage{pgfkeys}
\pgfkeys{
	/scorelist/.is family, /scorelist,
	default/.style={
		attempts = \attempts,
		continue = true,
		datarecording = \dataRecordingBonus,
		datarecordingpc = \dataRecordingBonusPercentage,
		datarecordingbonus = 0,
		outstanding = \outstandingPerformanceBonus,
		outstandingpc = \outstandingPerformanceBonusPercentage,
		outstandingbonus = 0,
		startbutton = \startbuttonpenalized,
		startbuttonpenalty = \startbuttonpenalty,
		firstcolcaption = \firstColumnCaption,
		secondcolcaption = \secondColumnCaption,
		singletrycc = \singleTryColumnCaption,
		firsttrycc = \firstTryColumnCaption,
		secondtrycc = \secondTryColumnCaption,
	},
	attempts/.estore in = \scorelistAttempts,
	continue/.estore in = \scorelistContinue,
	datarecording/.estore in = \scorelistDataRecording,
	datarecordingpc/.estore in = \scorelistDataRecordingPercentage,
	datarecordingbonus/.estore in = \scorelistDataRecordingBonus,
	outstanding/.estore in = \scorelistOutstanding,
	outstandingpc/.estore in = \scorelistOutstandingPercentage,
	outstandingbonus/.estore in = \scorelistOutstandingBonus,
	startbutton/.estore in = \scorelistStartButton,
	startbuttonpenalty/.estore in = \scorelistStartButtonPenalty,
	firstcolcaption/.estore in = \scorelistFirstColCaption,
	secondcolcaption/.estore in = \scorelistSecondColCaption,
	singletrycc/.estore in = \scorelistSingleTryCC,
	firsttrycc/.estore in = \scorelistFirstTryCC,
	secondtrycc/.estore in = \scorelistSecondTryCC,
}

\makeatletter%
\NewEnviron{scorelist}[1][]{
% \begin{scorelist}
	%%%%%%%%%%%%%%%%%%%%%%%%%%%%%%%%%%%%%%%%%%%%%%%%%%%%%%%%%%%%%%%
	% read options
	%%%%%%%%%%%%%%%%%%%%%%%%%%%%%%%%%%%%%%%%%%%%%%%%%%%%%%%%%%%%%%%
	\pgfkeys{/scorelist, default, #1}%

	%%%%%%%%%%%%%%%%%%%%%%%%%%%%%%%%%%%%%%%%%%%%%%%%%%%%%%%%%%%%%%%
	% init variables %%%%%%%%%%%%%%%%%%%%%%%%%%%%%%%%%%%%%%%%%%%%%%
	%%%%%%%%%%%%%%%%%%%%%%%%%%%%%%%%%%%%%%%%%%%%%%%%%%%%%%%%%%%%%%%
	\setcounter{currTestScore}{0}
	\setcounter{currOutstandingBonus}{\scorelistOutstandingBonus}
	\setcounter{currDataRecordingBonus}{\scorelistDataRecordingBonus}

	%%%%%%%%%%%%%%%%%%%%%%%%%%%%%%%%%%%%%%%%%%%%%%%%%%%%%%%%%%%%%%%
	% environment commands %%%%%%%%%%%%%%%%%%%%%%%%%%%%%%%%%%%%%%%%
	%%%%%%%%%%%%%%%%%%%%%%%%%%%%%%%%%%%%%%%%%%%%%%%%%%%%%%%%%%%%%%%

	% heading %%%%%%%%%%%%%%%%%%%%%%%%%%%%%%%%%%%%%%%%%%%%%%%%%%%%%
	\newcommand{\scoreheading}[1]{%
		\ifShortScoresheet{%
			\xdef\@scoreheadingcolspan{2}%
		}{%
			\xdef\@scoreheadingcolspan{\the\numexpr2+\scorelistAttempts\relax}%
		}%
		\phantom{.} \\[-12pt]%
		\multicolumn{\@scoreheadingcolspan}{@{}l}{\textbi{##1}}\\[0pt]%
	}

	\newcommand{\scoreitem}[3][1]{%
		\ifthenelse{ ##2 > 0 }{%
			\addtocounter{currTestScore}{ ##2 * ##1 }%
		}{}%
		%
		\@scoreitem[##1]{##2}{##3}%
	}

	\newcommand{\penaltyitem}[3][1]{%
		\@scoreitem[##1]{-\absval{##2}}{##3}%
	}

	\newcommand{\bonusitem}[3][1]{%
		\@scoreitem[##1]{##2}{##3}%
	}

	%%%%%%%%%%%%%%%%%%%%%%%%%%%%%%%%%%%%%%%%%%%%%%%%%%%%%%%%%%%%%%%
	% Commands for overriding internal calculations %%%%%%%%%%%%%%%
	%%%%%%%%%%%%%%%%%%%%%%%%%%%%%%%%%%%%%%%%%%%%%%%%%%%%%%%%%%%%%%%

	% set score counter to arbitrary value %%%%%%%%%%%%%%%%%%%%%%%%
	\newcommand{\setTotalScore}[1]%
	{%
		\setcounter{currTestScore}{##1}%
	}

	% set outstanding bonus counter to arbitrary value %%%%%%%%%%%%
	\newcommand{\setOutstandingBonus}[1]%
	{%
		\setcounter{currOutstandingBonus}{##1}
	}

	%%%%%%%%%%%%%%%%%%%%%%%%%%%%%%%%%%%%%%%%%%%%%%%%%%%%%%%%%%%%%%%
	% environment internal commands %%%%%%%%%%%%%%%%%%%%%%%%%%%%%%%
	%%%%%%%%%%%%%%%%%%%%%%%%%%%%%%%%%%%%%%%%%%%%%%%%%%%%%%%%%%%%%%%

	% table entry %%%%%%%%%%%%%%%%%%%%%%%%%%%%%%%%%%%%%%%%%%%%%%%%%
	\newcommand{\@scoreitem}[3][1]{%
		##3\vspace{0.1em} &%
		\textit{%
			\ifthenelse{ \equal{##2}{0} }{~}{% else
				\ifthenelse{ \equal{##1}{1} }{}{##1$\times$}%
			##2}%
		}%
		\ifShortScoresheet{}{&\attemptScoreLines{\scorelistAttempts}}\\[0pt]%
	}

	% [INTERNAL] writes down the line for the final total score %%%
	\newcommand{\scoreTotal}{%
		\\%
		\textbf{Total score~}%
		\ifShortScoresheet{%
			(excluding penalties and standard bonuses) &%
			\textit{\thecurrTestScoreTotalWithoutBonus}%
		}{%
			&\textit{\thecurrTestScoreTotal}
			&\multicolumn{\scorelistAttempts}{c}{\scoreline[0.20]}%
		}\\[0pt]%
	}

	% [INTERNAL] writes down the lines for the total score per try
	\newcommand{\scorePerTry}{%
		\\%
		\ifShortScoresheet{}{%
			\ifthenelse{ \attempts > 1 }{%
				\textbi{Score per try} &%
				\textit{\thecurrTestScoreTotalWithoutBonus} &%
				\attemptScoreLines{\scorelistAttempts}%
				\\[0pt]%
			}{}%
		}%
	}

	% [INTERNAL] draws a line for referee scoring %%%%%%%%%%%%%%%%%
	\gdef\@marklinewidth{0.06}
	\newcommand{\markline}{\rule{\@marklinewidth\linewidth}{.2pt}}

	% [INTERNAL] draws all the line for referee scoring %%%%%%%%%%%
	\newcommand{\attemptScoreLines}[1]{%
		\protected@xdef\@scorelines{\markline}%
		\ifthenelse{##1 > 1}{%
			\foreach \i in {2,...,##1}{%
				\protected@xdef\@scorelines{\@scorelines & \markline}%
			}%
		}{}%
		\@scorelines%
	}

	% [INTERNAL] writes down the headings for referee scoring %%%%%
	\newcommand{\attemptHeadings}[1]{%
		\ifthenelse{\equal{##1}{1}}{%
			\gdef\@attemptheadings{\textbf{\scorelistSingleTryCC}}%
			\gdef\@marklinewidth{0.1}%
		}{}%
		\ifthenelse{\equal{##1}{2}}{%
			\gdef\@attemptheadings{\textbf{\scorelistFirstTryCC} & \textbf{\scorelistSecondTryCC}}%
			\gdef\@marklinewidth{0.08}%
		}{}%
		\ifthenelse{##1 > 2}{
			\protected@xdef\@attemptheadings{%
				\textbf{\small$1^{st}$~try} &%
				\textbf{\small$2^{nd}$~try} &%
				\textbf{\small$3^{rd}$~try}}%
		}{}%
		\ifthenelse{##1 > 3}{
			\foreach \i in {4,...,##1}{%
				\protected@xdef\@attemptheadings{%
					\@attemptheadings &%
					\textbf{\small$\i^{th}$~try}%
				}%
			}%
			\gdef\@marklinewidth{0.06}%
		}{}%
		\@attemptheadings%
	}

	%%%%%%%%%%%%%%%%%%%%%%%%%%%%%%%%%%%%%%%%%%%%%%%%%%%%%%%%%%%%%%%
	% setup table %%%%%%%%%%%%%%%%%%%%%%%%%%%%%%%%%%%%%%%%%%%%%%%%%
	%%%%%%%%%%%%%%%%%%%%%%%%%%%%%%%%%%%%%%%%%%%%%%%%%%%%%%%%%%%%%%%
	% \par First column width: \@fcwidth\\
	\vspace{0.8 em}%
	\noindent%
	\begin{tabularx}{\textwidth}{ @{}X @{}r *{\scorelistAttempts}{c}}
		\textbf{\scorelistFirstColCaption} &%
		\textbf{\scorelistSecondColCaption}%
		\ifShortScoresheet{}{&\attemptHeadings{\scorelistAttempts}}
	\\\hline



\BODY



	%%%%%%%%%%%%%%%%%%%%%%%%%%%%%%%%%%%%%%%%%%%%%%%%%%%%%%%%%%%%%%%
	% calculate max. score, and bonuses %%%%%%%%%%%%%%%%%%%%%%%%%%%
	%%%%%%%%%%%%%%%%%%%%%%%%%%%%%%%%%%%%%%%%%%%%%%%%%%%%%%%%%%%%%%%
	% base total score (accumulative) %%%%%%%%%%%%%%%%%%%%%%%%%%%%%
	\setcounter{currTestScoreTotal}{\thecurrTestScore}
	% outstanding performance bonus %%%%%%%%%%%%%%%%%%%%%%%%%%%%%%%
	\ifthenelse{\equal{\scorelistOutstanding}{true}}{%
		\ifthenelse{ \equal{\thecurrOutstandingBonus}{0} }{%
			\setcounter{currOutstandingBonus}{ \thecurrTestScore*\scorelistOutstandingPercentage/100 }
		}{}%
		\setcounter{currTestScoreTotal}{%
			\thecurrTestScoreTotal + \thecurrOutstandingBonus}%
	}{}%
	% data recording bonus %%%%%%%%%%%%%%%%%%%%%%%%%%%%%%%%%%%%%%%%
	\ifthenelse{\equal{\scorelistDataRecording}{true}}{%
		\ifthenelse{ \equal{\thecurrDataRecordingBonus}{0} }{%
			\setcounter{currDataRecordingBonus}{ \thecurrTestScore*\scorelistDataRecordingPercentage/100 }%
		}{}%
		\setcounter{currTestScoreTotal}{%
			\thecurrTestScoreTotal + \thecurrOutstandingBonus}%
	}{}%
	\setcounter{currTestScoreTotalWithoutBonus}{ \thecurrTestScore }

	%%%%%%%%%%%%%%%%%%%%%%%%%%%%%%%%%%%%%%%%%%%%%%%%%%%%%%%%%%%%%%%
	% Special penalties & bonuses %%%%%%%%%%%%%%%%%%%%%%%%%%%%%%%%%
	%%%%%%%%%%%%%%%%%%%%%%%%%%%%%%%%%%%%%%%%%%%%%%%%%%%%%%%%%%%%%%%
	\scoreheading{Special penalties \& standard bonuses}

	% not showing up penalty %%%%%%%%%%%%%%%%%%%%%%%%%%%%%%%%%%%%%%
	\penaltyitem{\notattendingpenalty}{Not attending \ifShortScoresheet{(see sec.~\ref{rule:not_attending})}{}}

	% require signal for door opening %%%%%%%%%%%%%%%%%%%%%%%%%%%%%
	\ifthenelse{ \equal{\scorelistStartButton}{true} }{
	  \penaltyitem{\scorelistStartButtonPenalty}{Using start button \ifShortScoresheet{(see sec.~\ref{rule:start_button})}{}}
	}{}

	% data recording bonus %%%%%%%%%%%%%%%%%%%%%%%%%%%%%%%%%%%%%%%%
	\ifthenelse{%
		\thecurrDataRecordingBonus>0 \AND %
		\equal{\scorelistDataRecording}{true}%
	}{%
		\bonusitem{\thecurrDataRecordingBonus}{Contributing with recorded data ($\frac{\sum gathered~points}{max~points} \times$) \ifShortScoresheet{(see sec.~\ref{rule:datarecording})}{}}%
	}{}%

	% outstanding performance bonus %%%%%%%%%%%%%%%%%%%%%%%%%%%%%%%
	\ifthenelse{%
		\value{currOutstandingBonus}>0 \AND %
		\equal{\scorelistOutstanding}{true}%
	}{%
		\bonusitem{\thecurrOutstandingBonus}{Outstanding performance~\ifShortScoresheet{(see sec.~\ref{rule:outstanding_performance})}{}}%
	}{}%
	%
	% Total score %%%%%%%%%%%%%%%%%%%%%%%%%%%%%%%%%%%%%%%%%%%%%%%%%
	\\[-1em]\hline
	\scorePerTry
	\scoreTotal

	\end{tabularx}
	% \endtabularx

}
\makeatother%

% Local Variables:
% TeX-master: "../Rulebook"
% End:
%

\input{./setup/macros_open_demonstrations.tex}
\input{./setup/macros_leagues.tex}

\newcommand{\rulebookVersion}{\STATE\ version for RoboCup \YEAR\xspace(\VERSION)}

\def\RoboCup{{\textsc{RoboCup}}}
\def\Robocup{{\textsc{RoboCup}}}
\def\robocup{{\textsc{RoboCup}}}
\def\AtHome{{\textsc{@Home}}}
\def\Symp{{\textsc{RoboCup Symposium}}}

\def\OPL{\iaterm{Open Platform League}{OPL}}
\def\SPL{\iaterm{Standard Platform League}{SPL}}
\def\SPLs{\iaterm{Standard Platform Leagues}{SPLs}}
\def\DSPL{\iaterm{Domestic Standard Platform League}{DSPL}}
\def\SSPL{\iaterm{Social Standard Platform League}{SSPL}}

\def\EC{\iaterm{executive committee}{EC}}
\def\TC{\iaterm{technical committee}{TC}}
\def\OC{\iaterm{organizing committee}{OC}}
\def\LOC{\iaterm{Local Organizing Committee}{LOC}}
\def\RCF{\iaterm{RoboCup Federation}{RCF}}

\def\HRI{\iterm{Human-Robot Interaction}}
\def\NLP{\iterm{Natural Language Processing}}
\def\NAV{\iterm{Navigation}}
\def\MAP{\iterm{Mapping}}
\def\CV{\iterm{Computer Vision}}
\def\OR{\iterm{Object Recognition}}
\def\OM{\iterm{Object Manipulation}}
\def\MAN{\iterm{Manipulation}}
\def\AB{\iterm{Adaptive Behaviors}}
\def\BI{\iterm{Behavior Integration}}
\def\TP{\iterm{Task Planning}}
\def\AmI{\iterm{Ambient Intelligence}}
\def\SysI{\iterm{System Integration}}
\def\PerDet{\iterm{Person Detection}}
\def\PerRec{\iterm{Person Recognition}}
\def\GestRec{\iterm{Gesture Recognition}}

\def\RR{\iterm{Rulebook Repository}}
\def\TG{\iterm{Telegram Group}}
\def\WIKI{\iterm{Wiki}}

\def\HSR{\iterm{Toyota HSR}}
\def\MountingBracket{\iterm{Mounting Bracket}}
\def\PEPPER{\iterm{Softbank Pepper}}

\def\SetupDays{\iterm{Setup Days}}
\def\PS{\iterm{Poster Session}}
\def\WelcomeReception{\iterm{Welcome Reception}}
\def\RobotInspection{\iterm{Robot Inspection}}
\def\OpenChallenge{\iterm{Open Challenge}}
\def\SONE{\iterm{Stage~I}}
\def\STWO{\iterm{Stage~II}}
\def\FINAL{\iterm{Final}}
\def\Housekeeper{\iterm{Housekeeper}}
\def\Partyhost{\iterm{Party Host}}
\def\Testblock{\iterm{Test Block}}
\def\Testslot{\iterm{Test Slot}}
\def\Testblocks{\iterm{Test Blocks}}
\def\Testslots{\iterm{Test Slots}}

\def\HRIAward{\iterm{Best Human-Robot Interface Award}}
\def\DSPLPosterAward{\iterm{Best DSPL Poster Award}}
\def\SSPLPosterAward{\iterm{Best SSPL Poster Award}}
\def\OPLPosterAward{\iterm{Best OPL Poster Award}}
\def\OCAward{\iterm{Best Open Challenge Award}}

\def\TDP{\iaterm{Team Description Paper}{TDP}}
\def\TLM{\iterm{Team Leader Meeting}}
\def\OLC{\iterm{Open Loop Control}}

\def\Application{\iterm{Application}}
\def\Qualification{\iterm{Qualification}}
\def\Registration{\iterm{Registration}}
\def\CFP{\iaterm{Call for Participation}{CFP}}

\def\TeamVideo{\iterm{Team Video}}
\def\TeamWebsite{\iterm{Team Website}}

\def\VizBox{\iterm{VizBox}}
\def\Arena{\iterm{Arena}}
\def\Arenas{\iterm{Arenas}}
\def\ArenaNetwork{\iterm{Arena Network}}
\def\Entrance{\iterm{Entrance}}
\def\Exit{\iterm{Exit}}

\def\ObjectCategory{\iterm{Object Category}}
\def\PredefinedLocation{\iterm{Predefined Location}}
\def\LocationClass{\iterm{Location Class}}
\def\KnownObjects{\iterm{Known Objects}}
\def\SimilarObjects{\iterm{Similar Objects}}
\def\ConsistentObjects{\iterm{Consistent Objects}}
\def\UnknownObjects{\iterm{Unknown Objects}}
\def\StandardObjects{\iterm{Standard Objects}}
\def\YCBData{\iterm{YCB Dataset}}
\def\PredefinedName{\iterm{Predefined Name}}

\def\EmergencyStop{\iterm{Emergency Stop}}
\def\StartButton{\iterm{Start Button}}
\def\ExternalDevice{\iterm{External Device}}
\def\ExternalDevices{\iterm{External Devices}}
\def\ExternalComputing{\iterm{External Computing}}
\def\ECRA{\iaterm{External Computing Resource Area}{ECRA}}
\def\DEM{\iaterm{Deus ex Machina}{DEM}}

\def\NumObjects{30}
\def\NumLocations{20}
\def\NumNames{20}

\def\Referee{\iterm{Referee}}
\def\Referees{\iterm{Referees}}
\def\Volunteer{\iterm{Volunteer}}
\def\Volunteers{\iterm{Volunteers}}
\def\CustomOperator{\iterm{Custom Operator}}

\def\CommandGen{\iterm{GPSR Command Generator}}


\newcommand{\textbi}[1]{\textbf{\textit{#1}}}
\renewcommand{\labelenumi}{\arabic{enumi}.}
\renewcommand{\labelenumii}{\labelenumi\arabic{enumii}.}
\renewcommand{\labelenumiii}{\labelenumii.\arabic{enumiii}.}

\newcommand{\testtocentry}[1]{%
	\nameref{#1}\dotfill\pageref{#1}\\[0.2\baselineskip]%
}

%% %%%%%%%%%%%%%%%%%%%%%%%%%%%%%%%%%%%%%%%%%%%%%%%%%%%%%%%%% %%
%%                    Developement-Tools                     %%
%% %%%%%%%%%%%%%%%%%%%%%%%%%%%%%%%%%%%%%%%%%%%%%%%%%%%%%%%%% %%

%% %%%%%%%%%%%%%%%%%%%%%%%%%%%%%%%%%%%%%%%
\newcommand{\tbc}[1]{\textbf{\it\color{red}{t.b.c. ...}#1\color{black}}}
\newcommand{\todo}[1]{\textbf{\it\color{red}{todo: }#1\color{black}}}
\newcommand{\TODO}[1]{\textbf{\it\color{red}{TODO:\\}#1\color{black}}}
\newcommand{\chk}[1]{\textbf{\color{red}#1\color{black}}}

\newcommand{\reworkon}{\marginpar{\raggedright\color{red}{$\downarrow$rework}\color{black}}}
\newcommand{\reworkoff}{\marginpar{\raggedright\color{red}{$\uparrow$rework}\color{black}}}

%% %%%%%%%%%%%%%%%%%%%%%%%%%%%%%%%%%%%%%%%
%%  site notes/margin notes
\def\note#1{\marginpar{\raggedright\tiny#1}}
\def\mpar#1{\marginpar{\raggedright\tiny#1}}
\def\rand#1{\marginpar{\raggedright\tiny#1}}
\setlength{\marginparwidth}{2cm}

\newcommand{\refsec}[1]{Section~\ref{#1}}
\newcommand{\reftab}[1]{Table~\ref{#1}}
\newcommand{\reffig}[1]{Figure~\ref{#1}}

%% %%%%%%%%%%%%%%%%%%%%%%%%%%%%%%%%%%%%%%%
%% side-annotation-macros for easy lookup
% \newcommand{\awardmark}{\marginpar{\centering\includegraphics[width=.34cm]{images/icon_award.pdf}}}
% \newcommand{\refmark}{\marginpar{\centering\includegraphics[width=.5cm]{images/icon_whistle.pdf}}}
% \newcommand{\referee}[1]{\emph{#1}\marginpar{\centering\includegraphics[width=.5cm]{images/icon_whistle.pdf}}}
% \newcommand{\scoremark}{\marginpar{\centering\includegraphics[width=.34cm]{images/icon_score.pdf}}}
\newcommand{\awardmark}{}
\newcommand{\refmark}{}
\newcommand{\referee}[1]{}
\newcommand{\scoremark}{}
%\newcommand{\scoring}[1]{\emph{#1}\marginpar{\centering\includegraphics[width=.34cm]{images/icon_score.pdf}}}
\newcommand{\scoring}[1]{\emph{#1}}
\newcommand{\timark}{\marginpar{\centering\includegraphics[width=.34cm]{icon_clock.pdf}}}

%\newcommand{\timing}[1]{\emph{#1}\marginpar{\centering\includegraphics[width=.34cm]{images/icon_clock.pdf}}}
\newcommand{\timing}[1]{\emph{#1}}

\def\svnRevision{Unknown} %
\def\svnChangeData{Unknown} %
\def\revnumtmpfile{.temp_rulebook_version}
\def\revdattmpfile{.temp_rulebook_date}
\immediate\write18{git rev-list HEAD | wc -l > \revnumtmpfile}
%\immediate\write18{svnversion . > \revnumtmpfile}
\IfFileExists{\revnumtmpfile}{\def\svnRevision{\input{\revnumtmpfile}\unskip}}{}
\immediate\write18{git log -1 --date=short  | grep 'Date:' | awk '{print $2}'> \revdattmpfile}
%\immediate\write18{svn info | grep 'Last Changed Date:' | awk '{print $4}'> \revdattmpfile}
\IfFileExists{\revdattmpfile}{\def\svnChangeData{\input{\revdattmpfile}\unskip}}{}
% \IfFileExists{\revnumtmpfile}{\immediate\write18{rm -f \revnumtmpfile}}{}
% \IfFileExists{\revdattmpfile}{\immediate\write18{rm -f \revdattmpfile}}{}
\newcommand{\VERSION}{Revision \svnChangeData\_\svnRevision}


% Local Variables:
% TeX-master: "../Rulebook"
% End:

\input{./setup/abbrevix.tex}



\makeindex                                % generate index
\makeabbex                                % generate abbreviations

%%% DOCUMENTINFO %%%%%%%%%%%%%%%%%%%%%%%%%%%%%%%%%%%%%%%%%%%%%%%%%%%%%%%%%%%%%%
\hypersetup{
  pdftitle     = {RoboCup@Home Rules and Regulations},
  pdfsubject   = {RoboCup@Home Rulebook},
  pdfauthor    = {RoboCup@Home Technical Committee},
  pdfkeywords  = {RoboCup, @Home, Rules, Competition},
  colorlinks   = true,
  anchorcolor  = blue,
  linkcolor    = blue,
  urlcolor     = blue,
}

%%% HEADINGS & PAGE STYLE %%%%%%%%%%%%%%%%%%%%%%%%%%%%%%%%%%%%%%%%%%%%%%%%%%%%%
\newcommand{\footline}{RoboCup@Home Rulebook / \rulebookVersion}
\pagestyle{fancy}
\renewcommand{\chaptermark}[1]{\markboth{\chaptername\ \thechapter. \ #1}{}}
\renewcommand{\sectionmark}[1]{\markright{\thesection \ #1}{}\renewcommand{\currentTest}{#1}}
\fancyhf{}
\fancyhead[LE,RO]{\thepage}
\fancyhead[RE]{\sffamily\rightmark}
\fancyhead[LO]{\sffamily\leftmark}
\fancyfoot[C]{\scriptsize \sffamily \footline{}}
\fancypagestyle{plain}{
        \fancyhf{}
        \fancyhead[LE,RO]{\thepage}
        \fancyhead[RE]{\sffamily\rightmark}
        \fancyhead[LO]{\sffamily\leftmark}
        \fancyfoot[C]{\scriptsize \sffamily \footline{}}
		\renewcommand{\headrulewidth}{0.5 pt}
}
\fancypagestyle{empty}{
        \fancyhf{}
        \fancyhead{}
        \fancyfoot[C]{\scriptsize \sffamily \footline{}}
		\renewcommand{\headrulewidth}{0 pt}
}

%\newcommand{\sectionbreak}{\clearpage}
%\newcommand{\subsectionbreak}{\clearpage}


%%%%%%%%%%%%%%%%%%%\renewcommand{%%%%%%%%%%%%%%%%%%%%%%%%%%%%%%%%%%%%%%%%%%%%%%%%%%%%%%%%%%%%
%%%%%%%%%%%%%%%%%%%%%%%%%%%%%%%%%%%%%%%%%%%%%%%%%%%%%%%%%%%%%%%%%%%%%%%%%%%%%%%
%%%%%%%%%%%%%%%%%%%%%%%%%%%%%%%%%%%%%%%%%%%%%%%%%%%%%%%%%%%%%%%%%%%%%%%%%%%%%%%

\begin{document}

\input{./pages/titlepage}

\pagestyle{empty}
%% %%%%%%%%%%%%%%%%%%%%%%%%%%%%%%%%%%%%%%%%%%%%%%%%%%%%%%%%%%%%%%%%%%%%%%%%%%%
%%
%%          $Id: acknowledgments.tex 404 2013-02-15 08:51:20Z sugiura $
%%    author(s): RoboCupAtHome Technical Committee(s)
%%  description: Acknowledgments for the RoboCupAtHome RuleBook
%%
%% %%%%%%%%%%%%%%%%%%%%%%%%%%%%%%%%%%%%%%%%%%%%%%%%%%%%%%%%%%%%%%%%%%%%%%%%%%%



\section*{About this rulebook}
This is the official rulebook of the RoboCup@Home competition \YEAR.
The rulebook has been written by the \YEAR ~RoboCup@Home Technical Committee.
% Mauricio Matamoros,
% and
% Loy van Beek.



\section*{How to cite this rulebook}
If you refer to RoboCup@Home and this rulebook in particular, please cite:\\

\noindent Justin Hart, Mauricio Matamoros, Alexander Moriarty, Hiroyuki Okada,
Matteo Leonetti, Alex Mitrevski, Katarzyna Pasternak, and Fagner Pimentel
\enquote{Robocup@Home \YEAR: Rule and regulations,}
\url{https://athome.robocup.org/rules/\YEAR_rulebook.pdf}, \YEAR.

\begin{center}
\begin{minipage}{0.8\textwidth}
	\footnotesize%
	\verbatiminput{citation.bib}
\end{minipage}
\end{center}

\section*{Acknowledgments}
\label{sec:acknowledgments}
We would like to thank the members of the Technical Committee who put up the rules and the Organizing Committee who organizes the competition.
People that have been working on this rulebook as members of one of the league's committees (in alphabetical order):
\begin{center}
    \begin{minipage}{0.8\textwidth}
        \begin{multicols}{3}%
            \footnotesize
            \noindent%

            Alex Mitrevski\\
            Alexander Moriarty\\
            Caleb Rascon\\
            Fagner Pimentel\\
            \columnbreak
            Johanner Kumert\\
            Justin Hart\\
            Katarzyna Pasternak\\
            Komei Sugiura\\
            Luca Iocchi\\
            \columnbreak
            Mauricio Matamoros\\
            Matteo Leonetti\\
            Maxime St-Pierre \\
            Sammy Pfeiffer\\
            Sven Wachsmuth\\
            Tijn van der Zant\\
        \end{multicols}
    \end{minipage}
\end{center}

\noindent We would also like to thank all the people who contributed to the RoboCup@Home league with their feedback and comments.
People that have been working on this rulebook as members of the league (in alphabetical order):
\begin{center}
    \begin{minipage}{0.8\textwidth}
        \begin{multicols}{2}%
            \footnotesize
            \noindent%
            Florian Lier (@warp1337)\\
            Hiroyuki Okada (@okadahiroyuki)\\
            Lark Finean (@mfinean)\\
            Lars Janssen (@LarsJanssenTUE)\\
            \columnbreak%
            Loy van Beek (@LoyVanBeek)\\
            Matthijs van der Burgh (@MatthijsBurgh)\\
            Raphael Memmesheimer (@airglow)\\
            Sebastian Meyer zu Borgsen (@semeyerz)\\
            Syed Ali Raza (@syedaraza)
        \end{multicols}
    \end{minipage}
\end{center}


% Local Variables:
% TeX-master: "../Rulebook"
% End:

\clearpage

\section*{Changelog}
The changelog lists the significant changes in the rules since the previous global RoboCup event. 
Changes such as spelling and grammar fixes are not included. 
\subsection*{Changes in 2018}

\begin{itemize}
	\item \textbf{General rules}
		\begin{itemize}
			\item Introduction of changelog
			\item \textbf{VizBox} Introduce VizBox to visualize challenge with \href{https://github.com/RoboCupAtHome/RuleBook/pull/367}{PR 367}
			\item \textbf{DSPL Standard laptop} Mandatory use of official Standard Laptop for DSPL \href{https://github.com/RoboCupAtHome/RuleBook/pull/374}{PR 374}
			\item \textbf{External Devices: ECRA Rule}
				\begin{itemize}
					\item No people around external devices
					\item Mobile devices must be removed, others chained
					\item No keyboards, mice, or screens attached
				\end{itemize}
			\item Removed bonus for data recording
			\item Teams encouraged to provide annotated recording of spoken interactions
		\end{itemize}
	\item \textbf{Storing Groceries}
	\begin{itemize}
		\item Per moved object scoring
		\item Score for opening door increased
		\item PDF report optional
	\end{itemize}
	\item \textbf{Help me carry} Unless crushing, pushing the small object won't finish the test
	\item \textbf{EE-GPSR}
		\begin{itemize}
			\item Simplify EE-GPSR: less categories, simpler scoring. Full detail in \href{https://github.com/RoboCupAtHome/RuleBook/pull/327}{PR 327}
			\item One of each, normal, incomplete and erroneous commands (given randomly)
		\end{itemize}
	\item \textbf{Open} 
		\begin{itemize}
			\item Score normalized to Stage 2 highest score
			\item Evaluation metrics updated.
		\end{itemize}
	\item \textbf{Set the table \& clean up} Replaced by Procter \& Gamble Challenge
	\item \textbf{Procter \& Gamble Challenge} Clean up a table by placing tableware in dishwasher.
	\item \textbf{Finals: bronze place} 4th and 3rd match up for the 3rd place trophy
\end{itemize}
\clearpage

\pagestyle{empty}
\tableofcontents
\clearpage

\pagestyle{plain}

%% %%%%%%%%%%%%%%%%%%%%%%%%%%%%%%%%%%%%%%%%%%%%%%%%%%%%%%%%%%%%%%%%%%%%%%%%%%%
%%
%%    author(s): RoboCupAtHome Technical Committee(s)
%%  description: Introduction
%%
%% %%%%%%%%%%%%%%%%%%%%%%%%%%%%%%%%%%%%%%%%%%%%%%%%%%%%%%%%%%%%%%%%%%%%%%%%%%%
\chapter{Introduction}
\label{chap:introduction}


\section{RoboCup}

\RoboCup{} is an international joint project to promote AI, robotics, and related fields.
It is an attempt to foster AI and intelligent-robotics research by providing standard problems where a wide range of technologies can be integrated and examined. More information can be found at \url{http://www.robocup.org/}.

\section{RoboCup@Home}

The \textsc{RoboCup@Home} league aims to develop service and assistive robot technology with high relevance for future personal domestic applications.
It is the largest international annual competition for autonomous service robots and is part of the RoboCup initiative.
A set of benchmark tests is used to evaluate the abilities and performance of different robots in a realistic, non-standardized home environment setting.
The focus is on, but is not limited to, the following domains: human-robot interaction and cooperation, navigation and mapping in dynamic environments, computer vision and object recognition under natural light conditions, object manipulation, adaptive behaviors, behavior integration, ambient intelligence, standardization and system integration.
The competition is co-located with the RoboCup symposium.

%% %%%%%%%%%%%%%%%%%%%%%%%%%%%%%%%%%%%%%%%%%%%%%%%%%%%%%%%%%%%%%%%%%%%%%%%%%%%
%%
%%  author(s): RoboCupAtHome Technical Committee(s)
%%  description: Introduction - Organization
%%
%% %%%%%%%%%%%%%%%%%%%%%%%%%%%%%%%%%%%%%%%%%%%%%%%%%%%%%%%%%%%%%%%%%%%%%%%%%%%
\section{Organization}
\label{sec:introduction:organization}
\AtHome{} is organized into three subcommittees. Current members are listed at: 
\url{https://athome.robocup.org/committees/}.

\subsection{Executive Committee}
\label{sec:introduction:ec}
The \EC{} consists of members of the board of trustees and representatives of each activity area.

\subsection{Technical Committee}
\label{sec:introduction:tc}
The \TC{} is responsible for the rules of the league. Main focus is writing the rulebook and refereeing.
Members of the \EC{} are always members of the \TC{} as well.

\subsection{Organizing Committee}
\label{sec:introduction:oc}
The \OC{} is responsible for the organization of the competition. They create the schedule and provide information about the scenario.
The \LOC{} is responsible for the set up and organization of the venue.


%% %%%%%%%%%%%%%%%%%%%%%%%%%%%%%%%%%%%%%%%%%%%%%%%%%%%%%%%%%%%%%%%%%%%%%%%%%%%
%%
%%    author(s): RoboCupAtHome Technical Committee(s)
%%  description: Introduction - Infrastructure
%%
%% %%%%%%%%%%%%%%%%%%%%%%%%%%%%%%%%%%%%%%%%%%%%%%%%%%%%%%%%%%%%%%%%%%%%%%%%%%%
\section{Infrastructure}
\label{sec:infrastructure}
\subsection{RoboCup@Home Mailinglist}
The official \iterm{RoboCup@Home mailing list} can be reached at
\begin{center}
\href{mailto:robocup-athome@lists.robocup.org}{\texttt{robocup-athome@lists.robocup.org}}
\end{center}
You can register to the email list at:
\begin{center}
{\small\url{http://lists.robocup.org/cgi-bin/mailman/listinfo/robocup-athome}}
\end{center}

\subsection{RoboCup@Home Web Page}
The official \iterm{RoboCup@Home website} that also hosts this RuleBook can be found at:
\begin{center}
{\small\url{https://athome.robocup.org/}}
\end{center}

\subsection{RoboCup@Home Rulebook Repository}
The official \iterm{RoboCup@Home Rulebook Repository} is where rules are publicly discussed before applying changes to the rulebook.
The entire RoboCup@Home community is welcome and encouraged to actively participate in creating and discussing the rules. The repository can be reached at:
\begin{center}
{\small\url{https://github.com/RoboCupAtHome/RuleBook/}}
\end{center}

Although opening issues with inconsistencies, questions, clarifications, and suggestions is highly appreciated, the best way to contribute is by making pull requests with fixes and proposed changes.

\subsection{RoboCup@Home Telegram Group}
The official \iterm{RoboCup@Home Telegram Group} is and communication channel for the RoboCup@Home community where rules are discussed, announcements are made, and questions are answered.
However, beyond the technical aspects of the competition, the \textit{Telegram Group} is a meeting point to stay in contact with the community, foster knowledge exchange and strengthen relationships.
The \textit{Telegram Group} can be reached at
\begin{center}
{\small\url{https://t.me/RoboCupAtHome}}
\end{center}

\subsection{RoboCup@Home Wiki}
\label{sec:at_home_wiki}
The official \iterm{RoboCup@Home Wiki} is meant to be a central place to collect information on all topics related to the RoboCup@Home league. It was set up to simplify and unify the exchange of relevant information.
This includes but is certainly not limited to hardware, software, media, data, and alike.
The \textit{wiki} can be reached at:
\begin{center}
{\small\url{https://github.com/RoboCupAtHome/AtHomeCommunityWiki/wiki}}
\end{center}



%% %%%%%%%%%%%%%%%%%%%%%%%%%%%%%%%%%%%%%%%%%%%%%%%%%%%%%%%%%%%%%%%%%%%%%%%%%%%
%%
%%    author(s): RoboCupAtHome Technical Committee(s)
%%  description: Introduction - Leagues
%%
%% %%%%%%%%%%%%%%%%%%%%%%%%%%%%%%%%%%%%%%%%%%%%%%%%%%%%%%%%%%%%%%%%%%%%%%%%%%%
\section{Leagues}
\label{sec:introduction:leagues}

\AtHome{} is divided in three leagues. Two are \SPLs{} where each team uses the same robot platform and an \OPL{} where teams are free to choose their robot. The official leagues and their names are:
\begin{itemize}
  \item \DSPL{}
  \item \SSPL{}
  \item \OPL{}
\end{itemize}

\noindent All leagues share the same set of rules. The \DSPL{} uses the \HSR{} platform shown in figure \ref{fig:toyotaHSR} and the \SSPL{} uses the \PEPPER{} platform shown in figure \ref{fig:softbank-pepper}.

\begin{minipage}{0.5\textwidth}
	\begin{figure}[H]
		\begin{center}
			\includegraphics[height=0.6\textwidth]{images/toyota_hsr.png}
			\caption{Toyota HSR}
			\label{fig:toyotaHSR}
		\end{center}
	\end{figure}
\end{minipage}
\begin{minipage}{0.5\textwidth}
	\begin{figure}[H]
		\begin{center}
			\includegraphics[height=0.6\textwidth]{images/softbank_pepper.png}
			\caption{Softbank / Aldebaran Pepper}
			\label{fig:softbank-pepper}
		\end{center}
	\end{figure}
\end{minipage}

%% %%%%%%%%%%%%%%%%%%%%%%%%%%%%%%%%%%%%%%%%%%%%%%%%%%%%%%%%%%%%%%%%%%%%%%%%%%%
%%
%%    author(s): RoboCupAtHome Technical Committee(s)
%%  description: Introduction - Competition
%%
%% %%%%%%%%%%%%%%%%%%%%%%%%%%%%%%%%%%%%%%%%%%%%%%%%%%%%%%%%%%%%%%%%%%%%%%%%%%%
\section{Competition}
The competition consists of two \emph{Stages} and the \FINAL{}. Each stage comprises a series of \iterm{Tests}. The best teams from \SONE{} advance to \STWO{} with more difficult tests. The competition ends with the \FINAL{} where the two highest ranked teams of each league compete to win.

%% %%%%%%%%%%%%%%%%%%%%%%%%%%%%%%%%%%%%%%%%%%%%%%%%%%%%%%%%%%%%%%%%%%%%%%%%%%%
%%
%%    author(s): RoboCupAtHome Technical Committee(s)
%%  description: Introduction - Awards
%%
%% %%%%%%%%%%%%%%%%%%%%%%%%%%%%%%%%%%%%%%%%%%%%%%%%%%%%%%%%%%%%%%%%%%%%%%%%%%%
\section{Awards}
\label{sec:introduction:awards}
All the awards need to be approved by the \RCF{}. Not all awards must be given.


\paragraph{Winner of the Competition}
\label{sec:introduction:winner}
Each league has 1st, 2nd, and 3rd place award trophies. If eight or fewer teams are participating, no 3rd place award trophy is given.

\newpage
\paragraph*{Note: } For the following awards, the \EC{} nominates a set of candidates from which the \TC{} elects the winner. One cannot nominate or vote for their own team.

\paragraph{Best Human-Robot Interface Award}
\label{sec:introduction:hriaward}
To honour outstanding Human-Robot Interfaces, the \HRIAward{} may be given to one of the participating teams. It is especially important that the interface is open and available to the \AtHome{} community.

\paragraph{Best Poster}
\label{sec:introduction:bestposter}
To foster scientific knowledge exchange and reward a teams' effort to present their contributions, all scientific posters of each league are eligible to receive the \DSPLPosterAward, \SSPLPosterAward, or \OPLPosterAward, respectively.

Posters are graded on presenting innovative and state-of-the-art research within a field with direct application to \RoboCup\AtHome{} in an appealing, easy-to-read way while demonstrating successful and clear results. In addition to be attractive and well-rated in the \PS{} (see~\ref{sec:setupdays:postersession}), the explained research must have impact in the team's performance during the competition.

\paragraph*{Note: } For the following award, the \TC{} and team leaders nominate a set of candidates from which the \EC{} elects the winner. One cannot nominate or vote for their own team.

\paragraph{Open-Source Software Award}
\label{sec:introduction:assaward}
For promoting software exchange and collaboration, \RoboCup\AtHome{} awards the best open source software contributions to the community. The software must be easy to read, properly documented, follow standard design patterns, be actively maintained, and meet IEEE software engineering metrics of scalability, portability, maintainability, fault tolerance, and robustness. In addition, the open sourced software must be made available as a framework-independent standalone library so it can be reused with any software architecture.

Candidates must send their application to the \TC{} at least one month before the competition in form of a short paper (max 4 pages) following the same format used for the \TDP{} (see~\refsec{sec:rules:application:tdp}). The paper should include a brief explanation of the approach, comparison with State-of-the-Art techniques, statement of the used metrics and software design patterns, and the name of the teams and other collaborators that are also using the software.

\paragraph{Skill Certificates}
\label{award:skill}
The @Home league features certificates for the robots best at a the skills below:
\begin{itemize}
	\item Navigation
	\item Manipulation
	\item Speech Recognition
	\item Person Recognition
\end{itemize}

A team is given the certificate if it scored at least 75\% of the attainable points for that skill.
This is counted over all tests and challenges, so e.g.~if the robot scores manipulation points during the Storing Groceries test, that will count towards the Best in Manipulation certificate.
The certificate will only be handed out if the team is \emph{not} the overall winner of the competition.

% Local Variables:
% TeX-master: "Rulebook"
% End:


\chapter{Concepts Behind the Competition}
\label{chap:concepts}
A set of key concepts apply to every \RoboCup\AtHome{} competition and the performed tests.

\section{Clarity}
\label{sec:concepts:leanrules}
The rules and test descriptions aim to be as clear and understandable as possible. At the competition, if any ambiguity or misunderstanding arises before a test, they need to be discussed in the \TLM{}. Decisions made in the \abb{TLM} are binding. The \TC{} and referees on site will decide on anything coming up during or after a test.

\paragraph*{Note: } Once a scoresheet has been signed by the team leader or the scores have been published, the \abb{TC} decision is irrevocable.

\section{Autonomy}
\label{sec:concepts:autonomy}
All robots participating in the \RoboCup\AtHome{} competition have to be \emph{autonomous}. This means no human is allowed to remote control the robot during a test. Furthermore, a test must not be solved using \OLC{}.

\section{Applicability}
\label{sec:concepts:applicability}
The tests should reward useful, robust, general, cost effective, and applicable solutions. The tests should increase in difficulty and complexity each year.

\section{Social Relevance}
\label{sec:concepts:socialrelevance}
The tests should show socially relevant results. The aim is to convince the public about the usefulness of autonomous robot applications in domestic settings by directly assisting and helping humans.

\section{Scientific Value}
\label{sec:concepts:scientificvalue}
The tests should allow teams to show novel approaches with high scientific value.

\section{Time Constraints}
\label{sec:concepts:timeconstraints}
Setup and test time is limited to allow for many participating teams and to emphasize the competition aspect of \AtHome{}.

\section{Non Standard Scenario}
\label{sec:concepts:nonstandardscenario}
In order to reward robust and general solutions, \RoboCup\AtHome{} has no standard scenario. It should resemble a typical domestic setting of the host country. Furthermore, tests may take place outside of the scenario, i.e., in an previously unknown environment like, for example, a nearby public space.

\section{Appeal}
\label{sec:concepts:appeal}
The competition should appeal to the audience and the public. Therefore high attractiveness and originality of an approach should be rewarded.

\section{Community}
\label{sec:concepts:community}
Although teams compete against each other, the members of the \AtHome{} league are expected to cooperate and exchange knowledge to advance technology together. Every team is encouraged to share relevant technical, scientific, and team related information through the \TDP{} and by participating in the various communication channels (see \ref{sec:introduction:infrastructure}).


% Local Variables:
% TeX-master: "Rulebook"
% End:


%% %%%%%%%%%%%%%%%%%%%%%%%%%%%%%%%%%%%%%%%%%%%%%%%%%%%%%%%%%%%%%%%%%%%%%%%%%%%
%%
%%          $Id: general_rules.tex 420 2013-04-08 15:30:35Z holz $
%%    author(s): RoboCupAtHome Technical Committee(s)
%%  description: description of the GENERAL RULES
%%
%% %%%%%%%%%%%%%%%%%%%%%%%%%%%%%%%%%%%%%%%%%%%%%%%%%%%%%%%%%%%%%%%%%%%%%%%%%%%
\chapter{General Rules and Regulations}
\label{chap:rules}

These are the general rules and regulations for the competition in the \RoboCup\AtHome{} league.
Every rule in this section can be considered to implicitly include the term \emph{\enquote{unless stated otherwise}}.
This means that additional or contrary rules, in particular with respect to the specification of tests, have a higher priority than those mentioned in the general rules and regulations.

%%%%%%%%%%%%%%%%%%%%%%%%%%%%%%%%%%%%%%%%%%%%%%%%%%%%%%%%%
\section{Team Registration and Qualification}


\subsection{Registration and Qualification Process}
\label{rule:participation}

Each year there are four phases in the process toward participation:
\begin{enumerate}
	\item \iterm{Intention of Participation} (optional)
	\item \iterm{Preregistration}
	\item \iterm{Qualification} announcements
	\item Final \iterm{Registration} for qualified teams
\end{enumerate}
Positions 1 and 2 will be announced by a call on the \iterm{RoboCup@Home mailing list}. Preregistration requires a \iterm{team description paper}, a \Term{video}{qualification video} and a \Term{website}{Team Website}.

\subsection{Qualification Video}
As a proof of running hardware, each team has to provide a \iterm{qualification video} showing at least two from the following abilities (minimum requirement):
\begin{itemize}
	\item Human-Robot interaction
	\item Navigation (safe, indoors with obstacle avoidance).
	\item Object detection \& manipulation.
	\item People detection
	\item Speech recognition.
	\item speech synthesis (clear and loud).
\end{itemize}

Showing some of the following abilities is recommended:
\begin{itemize}
	\item Activity recognition
	\item Complex speech recognition
	\item Complex action planning
	\item Gesture recognition
\end{itemize}


Videos should be self-explicative and designed for a general audience, showing the  robot solving complex tasks. The minimum to qualify requires proving the robot is able to solve successfully at least one test of the current or last year's rulebook. For robots moving slowly, we suggest to speed-up videos. When doing so, please specify the speed factor being used (e.g.~2x, 5X, 10X). The same applies for slow motion scenes.

Please notice that the videos should not last longer than the average time for a test (max.~\SI{10}{\minute}).

\paragraph{Important note to Standard Platform Leagues:} The qualification video must show an unmodified robot in normal operation (See~\refsec{rule:spl-mods}).

\subsection{Team Website}

The \iterm{Team Website} should be designed for a broader audience, but also including scientific material and access to open source code being developed. Requirements are as follows:

\begin{enumerate}

	\item \textbf{Multimedia: } Please include as many photos and videos of the robot(s) as possible.

	\item \textbf{Language: } The team website has to be in English. Other languages may be also available, but English must be default language.

	\item \textbf{Team: } List of the team members including brief profiles.

	\item \textbf{RoboCup:} Link to the league website and previous participation of the team in RoboCup.

	\item \textbf{Scientific approach: } The team website has to include research lines, description of the approaches, and information on scientific achievements.

	\item \textbf{Publications: } Relevant \iterm{publications} from 5 years up to date. Downloadable publications are scored higher during the qualification process.

	\item \textbf{Open source material: } Blueprints, designs, repositories or any kind of contribution to the league is highly scored during qualification process.
\end{enumerate}


\subsection{Team Description Paper}
\label{rule:website_tdp}
The \iaterm{team description paper}{TDP} is an 8-pages long scientific paper which must have a explained description of your main research, including the scientific contribution, goals, scope, and results.

Preferably, it should also contain the following:
\begin{itemize}
	\item the focus of research and the contributions in the respective fields,
	\item innovative technology (if any),
	\item re-usability of the system for other research groups
	\item applicability of the robot in the real world
	\item photo(s) of the robot(s)
\end{itemize}

~\\\noindent As addendum in the 9th page (after references) please include:
\begin{itemize}
	\item Team name
	\item Contact information
	\item Website url
	\item Team members' names
	\item photo(s) of the robot(s), unless included before.
	\item description of the hardware used
	\item Brief, compact list of \iterm{external devices} (See~\refsec{rule:robot_external_computing}), if any.
	\item Brief, compact list of 3rd party reused software packages (e.g.~ROS' \texttt{object\_recognition} should be listed, but not OpenCV).
	\item \textbf{[Open Platform League only]} Brief description of the hardware used by the robot(s).
\end{itemize}

~\\\noindent The TDP has to be in English, up to eight pages in length and formatted according to the guidelines of the RoboCup International Symposium without altering margins or spacing. It goes into detail about the technical and scientific approach.

Please notice that, during qualification process, TDP will be scored by its scientific value, novelty and contributions.


%% %%%%%%%%%%%%%%%%%%%%%%%%%%%%%%%%%%%%%%%
\subsection{Qualification}
\label{rule:qualification}

During the \iterm{qualification process} a selection will be made by the \iaterm{Organizing Committee}{OC} Taken into account and evaluated in this decision process are:
\begin{itemize}
	\item The content on the team website, scoring higher publications and open source resources;
	\item the number of abilities shown in the qualification video,
	\item the complexity of the tasks shown in the qualification video, and
	\item the scientific value, novelty and contributions in the \iterm{team description paper}. %, and
	% \item the information in the \iterm{RoboCup\char64Home Wiki} (added by the team).
\end{itemize}
(Additional) evaluation criteria are:
\begin{itemize}
	\item the performance in previous competitions,
	\item the relevant scientific contributions and publications, and
	\item the contributions to the RoboCup@Home league.
\end{itemize}

\paragraph{Important note to Standard Platform Leagues:} Only unmodified robots may compete in Standard Platform Leagues. Any \textit{slight} modification made to the robot found in the Qualification Material will automatically disqualify the team, for which registration to the international competition will not be possible  (See~\refsec{rule:spl-mods}).

% For getting considered in the evaluation, be sure to insert your team's name when adding information to the \iterm{RoboCup\char64Home Wiki}.


% Local Variables:
% TeX-master: "../Rulebook"
% End:


\subsection{Vizbox}
\label{vizbox}

The objective of RoboCup is to \quotes{promote robotics and AI research, by offering a publicly appealing, but formidable challenge} \footnote{\url{http://robocup.org/objective}}.

Part of making RoboCup@Home appealing, is to show the audience what is going on, what the robots should do and what they are doing.

To this end, robots in RoboCup@Home are expected run the RoboCup@Home \href{https://github.com/LoyVanBeek/vizbox}{VizBox}\footnote{\url{https://github.com/LoyVanBeek/vizbox}}.

This is a web server to be run on a robot during a challenge. The page it serves can be displayed on a screen, visible to the audience, via a secondary computer in or around the arena, connected to the web server via the wireless network.

All robots are expected to run the \iterm{VizBox}; the audience expects to know what all the robots are doing and what each challenge entails.

The \iterm{VizBox}'s code is hosted \url{https://github.com/LoyVanBeek/vizbox}.
We want to show the audience a consistent presentation, so ideally, all teams run the same VizBox code.
Sharing your changes back in the form of a Pull Request is much appreciated so all teams can benefit.

The \iterm{VizBox} has the following visualization capabilities:
\begin{itemize}
	\item Images of what the robot sees or a visualization of the robot's world model, eg. camera images, it's map, anything to make clear what is going on to the audience.
	\item Show an outline of the current challenge and where the robot is in the story of the current challenge.
	\item Subtitles of what the robot and operator just said; their conversation
\end{itemize}

Additionally, the \iterm{VizBox} offers a way to \textbf{input} a text command to the robot, to bypass automatic speech recognition if need be.

The exact documentation is maintained in the repository of the \iterm{VizBox} itself.

%%%%%%%%%%%%%%%%%%%%%%%%%%%%%%%%%%%%%%%%%%%%%%%%%%%%%%%%%
\section{Scenario}
\label{sec:scenario}

The tests take place in the \iterm{RoboCup@Home arena}. In addition, particular tests are situated outside the arena, e.g., in a previously unknown public place. The following rules are related to the \iterm{RoboCup@Home arena} and its contents. 

\subsection{RoboCup@Home arena}
The \iterm{RoboCup@Home arena} is a realistic home setting (apartment) consisting of inter-connected rooms like, for instance, a living room, a kitchen, a bath room, and a bed room. 
Depending on the Local Organization, there may be multiple apartments which may be different to each other.
Robot must be prepared to perform any task in any arena, not the same arena every time. 

The arena is decorated and dressed to resemble a home in which one could live, with as much of the necessities and decorations one might find in a normal home. 
Please do note that what is considered as \quotes{normal} may greatly vary by culture and on the location where the RoboCup event is hosted. 
For some examples on items one may find in the arena, see \refsec{chap:arena-decorations-appendix}

% \subsection{Team area}\label{rule:scenario_team_area}

% \todo{remove? does not depend on the rules, but on local organization }
% The maximum number of people to register per team is unlimited, but
% the organization only provides space for \emph{four} (4) persons to
% work at tables in the team area. 
% \todo{this is actually more an additional note for the registration information}

\subsection{Walls, doors and floor}
\label{rule:scenario_walls}

The indoor home setting will be surrounded by high and low \Term{walls}{Arena walls}. These walls will be built up using standard fair construction material.

\begin{enumerate}
	\item \textbf{Walls:} Walls have a minimum height of \SI{60}{\centi\meter}. A maximum height is not specified, but should be chosen so that the audience is able to watch the competition.\\
	Walls will be fixed and are likely to be not modified during the competition (see \refsec{rule:scenario_changes}). 

	\item \textbf{Doors:} There will be at least two entry/exit \Term{doors}{Arena doors} connecting the outside of the scenario. These doors are used as starting points for the robots (see \refsec{rule:start_position}).
	% At least one of the entrances will be a door with a handle (not a knob).\
	There will be also another door inside the scenario with a handle (not a knob) between any two rooms. Doors with handle (not a knob) may be closed at any time, it is expected robots be able to open them.

	\item \textbf{Floor:} The floor of the arena as well as the doorways of the arena are even. That is, there will be no significant steps or even stairways. However, minor unevenness such as carpets, transitions in floor covering between different areas, and minor gaps (especially at doorways) must be expected.

	\item \textbf{Appearance:} Floor and walls are mainly uni-colored but can contain texture, e.g., a carpet on the floor, or a poster or picture on the wall.\\
	Although being unlikely at the moment, transparent elements are also possible. 
\end{enumerate}


\subsection{Furniture}
\label{rule:scenario_furniture}

The arena will be equipped with typical objects (furniture) that are not specified in quantity and kind. The minimal configuration consists of 
\begin{itemize}
	\item a small dinner table with two chairs, 
	\item a couch, 
	\item an open cupboard or small table with a television and remote control, 
	\item a cupboard or shelf (with some books inside), and
	\item a refrigerator in the kitchen (with some cans and plastic bottles inside). 
\end{itemize}
A typical arena setup is shown in \reffig{fig:scenario_arena}.

\begin{figure}[tbp]
	\centering
	\subfloat[Typical arena]{\label{fig:scenario_arena}\includegraphics[height=46mm]{images/typical_arena.jpg}} ~ 
	\subfloat[Typical objects]{\label{fig:scenario_objects}\includegraphics[height=46mm]{images/typical_objects.jpg}}
	\caption{Scenario examples: (a) a typical arena, and (b) typical objects.}
	\label{fig:arena}
\end{figure}



\subsection{Changes to the arena}
\label{rule:scenario_changes}

Since the robots should be able to function in the real world the scenario is not fixed and might change without further notice.
\begin{enumerate}
	\item \textbf{Major changes:} 
	The arena is meant to be a simulated apartment. 
	The furniture might be moved around between tests. 
	This includes furniture that is a named location (see \refsec{rule:scenario_names}).
	As in a normal home, furniture is not very likely to move from one room to another and is unlikely to be moved to the other side of a room.
	However, a couch or table may be rotated, moved to its side etc. 
	Walls will stay in place and rooms will not change function.
	Passages might be blocked and cleared. 
	One hour before a test slot begins no \iterm{major changes} will be made.
	This time will be shortened in the future. 
	\item \textbf{Minor changes:} In contrast to major changes, \iterm{minor changes} like, for instance, slightly moved chairs cannot be avoided and may happen at any time (even during a test). 
\end{enumerate}


%%%%%%%%%%%%%%%%%%%%%%%%%%%%%%%%%%%%%%%%%%%%%%%%%%%%%%%%%%%%%%%%%%
%
% Objects section.
%
% Revisited by Mauricio Matamoros for 2015
%
%%%%%%%%%%%%%%%%%%%%%%%%%%%%%%%%%%%%%%%%%%%%%%%%%%%%%%%%%%%%%%%%%%
\def\NumObjects{10\ }
\def\NumLocations{20\ }
\def\NumNames{20\ }

\subsection{Objects}
\label{rule:scenario_objects}
Some tests in the RoboCup@Home league involve object manipulation and recognition. These \iterm{objects} resemble items usually found in household environments like, for instance, soda cans, coffee mugs or books. An example of objects used in a previous competition can be seen in \reffig{fig:scenario_objects}.

Objects are divided in five main groups:

\begin{enumerate}
	\item \textbf{\iterm{Known objects}:} Objects with no noticeable difference among peers. \textit{Known objects} tend to be artificial and regular shaped, such as coke cans, beer bottles, cereal boxes, etc.~A set of copies of these objects is provided before the competition for training.

	\item \textbf{\iterm{Alike objects}:} Objects with slight differences among peers (e.g.~color, size, shape). \textit{Alike objects} tend to be natural and similar to each other, but not equal; for example: apples, bananas, rags, etc.~A specimen of these objects is provided before the competition for training.

	\item \textbf{\iterm{Containers}:} Objects which can contain, transport or be filled with other objects or their content, such as baskets, bowls, bags, trays, etc.~. As with \textit{known objects}, \textit{containers} are known beforehand with no noticeable difference among peers, and a copy is provided before the competition for training.

	\item \textbf{\iterm{Special objects}:} Objects require a proper identification and special handling (not necesarily grasping), operation or interaction for accomplishing a particular task. Examples of special objects are: door handles, chairs, walking sticks, poles, etc.~Notice that a copy of these objects may not be available beforehand for previous training.

	\item \textbf{\iterm{Unknown objects}:} Any other object that is not known beforehand but can be grasped or handled.
\end{enumerate}

The following general rules for objects apply:

\begin{enumerate}
	\item \textbf{Object category:} Each object will be assigned to an \iterm{object category}. The objects \quotes{apple} and \quotes{banana} may be of class \quotes{fruits} for example.

	\item \textbf{Object (category) locations:} An \iterm{object location} will be assigned to each \iterm{object category}. For example, objects categorized as \quotes{fruits} may be usually found on the \quotes{kitchen table}, and unknown objects \quotes{unknown} may be usually found in the \quotes{trash bin}.

	\item \textbf{Announcement:} The TC makes the set of \iterm{objects}, including their names, categories, and usual locations; available during the setup days. 
	
	\item \textbf{Placement:} \nterm{object placement} Unless stated otherwise, in manipulation tasks, the objects will be positioned at \iterm{manipulation locations} and less than \SI{15}{\centi\meter} away from the border of the surface they are located at. There will be at least \SI{5}{\centi\meter} space around each object.
\end{enumerate}

\paragraph*{Important note:} It is not allowed to modify any of the objects provided for training. Also, teams are not allowed to keep more than 5 the objects provided for training at a time nor retaining it for more than one hour.

\subsubsection{Containers}
The TC will provide at least three different types of containers to be used in the tests.

\begin{itemize}
	\item \textbf{Pouring containers:} Such as a bowls, glasses, or other objects in which liquids and grains can be poured.

	\item \textbf{Storage containers:} Such as bags or boxes in which objects can be stored or retrieved.\\
	Bags used during the competition are rigid and with clearly visible standing handles; more likely made of paper and in bright colors (See Figure \ref{fig:scenario_container_bag}).

	\item \textbf{Transport containers:} Such as trays in which objects can be neatly arranged for transport.
\end{itemize}

Although there are no restrictions on a container size, appearance or weight; however, it can be expected that the selected containers be lightweight, with handles, and easily manipulable by a human using either one or both hands.

\begin{figure}[H]
	\centering
	\subfloat[Bright-colored paper bags]{
		\label{fig:scenario_container_bag}\includegraphics[width=0.33\textwidth]{images/container_paper_bag.png}}~
	\subfloat[Cereal bowls]{
		\label{fig:scenario_container_bowl}\includegraphics[width=0.33\textwidth]{images/container_bowl.png}}~
	\subfloat[Serving tray]{
		\label{fig:scenario_container_tray}\includegraphics[width=0.33\textwidth]{images/container_tray.png}}
	\caption{Example of object containers}
	\label{fig:scenario_containers}
\end{figure}

\paragraph*{Custom containers.}
\label{rule:custom_containers}
It is allowed that a team provide a \iterm{custom container} adapted to be used by the robot, considering the following:
\begin{enumerate}
	\item Custom containers must be approved by the TC during during the \iterm{Robot Inspection} (see \refsec{sec:robot_inspection}).
	\item Custom containers must \emph{not} have any kind of artificial marks, sensors, or electronic devices.
	\item Penalties may apply for the use of custom containers. The TC may establish special penalties during the \iterm{Robot Inspection}. The default penalties applicable to any task involving a container are as follows.
	\begin{itemize}
		\item Special color on an otherwise unmodified two-hand manipulable container: 75\% of the points.
		\item Special color on an otherwise unmodified single-hand manipulable container: 50\% of the points.
		\item Specially designed or adapted two-hand manipulable container (e.g.~special handles): 50\% of the points.
		\item Specially designed or adapted single-hand manipulable container (e.g.~special handle): 25\% of the points.
		\item Two-hand manipulable container adapted to be used \textit{single-handed}: 25\% of the points.
		\item On-robot mounted container: 0 points.
	\end{itemize}
	\textbf{Notes:} Trays are considered two-hand manipulable containers, while most bowls and dishes are considered single-hand manipulable container unless they are too big. Color patterns are allowed as long as they look natural (e.g.~\textit{barber sign colored} handles are allowed, but black and white bar-code like handles are not). Penalties does not stack, the most meaningful modification is considered. 
\end{enumerate}

\subsubsection{Predefined objects}
The TC will compile a list of at least \NumObjects objects (including both \iterm{known objects} and \iterm{alike objects}) which will be available for training. There are no restrictions on an object size, appearance or weight; however, it can be expected that the selected objects are easily manipulable by a human using a single hand.

Note that, any object not previously announced by the TC is automatically considered an unknown object for scoring purposes (e.g.~ornamentation).

%%%%%%%%%%%%%%%%%%%%%%%%%%%%%%%%%%%%%%%%%%%%%%%%%%%%%%%%%%%%%%%%%%
%
% Predefined locations section.
%
%%%%%%%%%%%%%%%%%%%%%%%%%%%%%%%%%%%%%%%%%%%%%%%%%%%%%%%%%%%%%%%%%%

\subsection{Predefined locations}
\label{rule:scenario_locations}

Some tests in the RoboCup@Home league involve \iterm{predefined locations}. 
These may include places like a \quotes{bookshelf} or a \quotes{dining table}, as well as certain objects such as a \quotes{television}, or the \quotes{front door}. 

\begin{enumerate}
	\item \textbf{Definition:} The TC will compile a list of predefined locations. There are no restrictions on which parts of the arena will be selected as a predefined location.

	\item \textbf{Location classes:} Each location will be assigned to a \iterm{location class}. The objects \quotes{couch} and \quotes{arm chair} may be of class \quotes{seat} for example. 

	\item \textbf{Announcement:} The TC makes the set of locations (and their names and classes) available during the setup days.

	\item \textbf{Position:} The positions of locations are \emph{not} necessarily fixed (see \refsec{rule:scenario_changes}).

	\item \textbf{Manipulation locations:} The TC will mark at least \NumLocations locations out of the set of predefined locations as being \iterm{manipulation locations}. Whenever a test involves manipulation, the object to manipulate will be placed at one of the manipulation locations. 
\end{enumerate}



\subsection{Predefined rooms}
\label{rule:scenario_rooms}
Some tests in the RoboCup@Home league involve \iterm{predefined rooms}. 
\begin{enumerate}
	\item \textbf{Definition:} The TC will compile a list of room names.
	\item \textbf{Announcement:} The TC makes the set of rooms available during the setup days.
\end{enumerate}



\subsection{Predefined (person) names}\label{rule:scenario_names}

Some tests in the RoboCup@Home league involve \iterm{predefined names} of people. 

\begin{enumerate}
	\item \textbf{Definition:} The TC will compile a list of \NumNames predefined names. The names are \SI{50}{\percent} male and \SI{50}{\percent} female, and taken from the (current) most common first names in the United States.\\
	In order to ease speech recognition, it is tried to select names to be phonetically different from each other.

	\item \textbf{Announcement:} The TC makes the set of names available during the setup days.
	\item \textbf{Assignment:} When a test involves interacting with persons (using a person's name), all involved persons are assigned names by the referees before the test. 
\end{enumerate}

Typical names are, for example, James, John, Robert, Michael and William as male names; Mary, Patricia, Linda, Barbara and Elizabeth as female names.

% MAURICIO @2017
% Separated file for better control
%% %%%%%%%%%%%%%%%%%%%%%%%%
\subsection{Wireless network}
\label{rule:scenario_wifi}

For wireless communication, an \iterm{arena network} is provided. The actual infrastructure depends on the local organization.

\begin{itemize}
	\item To avoid interference with other leagues, this \iterm{arena network} has to be used for communication only. It is not allowed to use the above or any other WiFi network for personal use at the venue.
	\item During the competitions, only the active team is allowed to use the \iterm{arena network}.
	\item The organizers cannot guarantee reliability and performance of wireless communication. Therefore, teams are required to be ready to setup, start their robots and run the tests even if, for any reason, network is not working properly.
\end{itemize}

Preferred situation:
\begin{itemize}
	\item The \iterm{arena network} consists of of several Virtual Local Area Networks (VLANs), one for each team.
	\item The traffic from the robot inside the arena is separated into the corresponding team's VLAN as soon as possible, e.g.~at the wireless acces point.This may require that each team has it's own SSID, each of which gets routed into the corresponding VLAN. Each team has a network cable routed to their team area, which is also connected to the teams VLAN. On this cable, the team can set up their own router/switch/hub etc. which will be inside the team's VLAN. This way, one team's traffic and devices are completely separated from any other team, while any team can set up their own DHCP server etc. if they desire.
	\item An Internet connection is preferably also available for every team.
\end{itemize}
Each team has to bring its own LAN hub/switch and cables for routing inside the team area.

In case the \iterm{arena network} is not functioning at the end of the first setup day, teams are allowed to set up their own networking equipment and wireless networks.

\paragraph*{Important note:} Different countries have different regulations for wireless equipment and the \iterm{arena network} has to obey these.
It is up to the teams to have networking equipment that also adheres to these regulations. For example, if due to local regulations various WiFi channels are prohibited, it is a team's responsibility to be able to use different, allowed channels.

\paragraph*{Important note:} Any unapproved wireless device may be removed by the TC at any time.

% Local Variables:
% TeX-master: "../Rulebook"
% End:

% MAURICIO @2017
% We are not really using SmartHome devices anymore.
%\subsection{Smart Home Devices}
\label{rule:smarthomedevices}

The Organizing and Technical Committees in coordination with the Local Organization will compile a list of \iterm{Smart~Home} official devices that will be available in the arena and can be used in some tests for additional score.

At any time, only the Smart~Home devices provided by the Local Organization and approved by the Technical Committee may be used during competition.

\subsubsection{Smart~Home devices list announcement}
The list if Smart~Home devices will be provided to teams as soon as it becomes available and has been granted by the Local Organization and approved by the Technical Committee.

This list must be announced at least one month prior the competition. In case that this list does not become available for that date, Smart~Home devices may still be present at the arena for testing, but no additional score can be achieved for its use. This is to maintain fair conditions among all teams.

\subsubsection{Technical specifications}
The list of \iterm{Smart~Home} official devices will include as much technical information as possible. However, before it becomes available you may assume the following considerations:

\begin{enumerate}
	\item \textbf{Interface:} Most Smart~Home devices interface wireless, often via R/F transmitters. When possible, the OC will provide an official interface via the \iterm{arena network}.
	\item \textbf{Operating voltage:} The operating voltage used will be the standard for the place of the competition (e.g.120V/60Hz for North America and 220V/50Hz for Europe). Please note that devices designed for other voltages/frecuencies may burn when plugged to the outlet.
	\item \textbf{Type of devices:} Mostly Smart~Home switches will be used (set on/off, read can not be guaranteed). For high bandwidth devices such as microphones or video cameras, an official interface (such as a ROS topic or web service) will be provided via the \iterm{arena network}.
\end{enumerate}

\subsubsection{Availability \& Scoring}
All test has been designed to optionally allow the use of Smart~Home devices and even grant bonus scoring for using this option. However, robots must be able to continue operating normally when there are no Smart~Home devices available. Therefore, it is unacceptable that a robot gets stuck or in some sort of deadlock while trying to operate those devices.

As stated in~\refsec{rule:scenario_wifi}, organizers cannot guarantee reliability and performance of wireless communication. Therefore, in case of malfunction or communication problems with the Smart~Home devices, or any other issue which may affect scoring, no claims will be accepted by the EC/TC/OC, nor test will be repeated. The decision on if a team given points for using \iterm{Smart~Home} devices, is conducted by the \iaterm{Technical Committee}{TC}, and it reserves the rights of discarding Smart~Home related scoring.


% Local Variables:
% TeX-master: "../Rulebook"
% End:


% Local Variables:
% TeX-master: "../Rulebook"
% End:


%%%%%%%%%%%%%%%%%%%%%%%%%%%%%%%%%%%%%%%%%%%%%%%%%%%%%%%%%
\section{Robots}
\label{rule:robots}

\subsection{Number of robots}
\label{rule:robots_number}

\begin{enumerate}
	\item \textbf{Registration:} The maximum \term{number of robots} per team is \emph{two} (2).
	\item \textbf{Regular Tests:} Only one robot is allowed per test. For different tests different robots can be used.
	% \item \textbf{Open Demonstrations:} In the \iterm{Open Challenge} and the \iterm{Finals} both robots can be used simultaneously.
\end{enumerate}

\subsection{Appearance and safety}
\label{rule:robot_appearance}

Robots should have a nice product-like appearance, be safe to operate, and should not annoy people. The following rules apply to all robots and are part of the \iterm{Robot Inspection} test (see~\refsec{sec:robot_inspection}).
\begin{enumerate}
	\item \textbf{Cover:} The robot's internal hardware (electronics and cables) should be covered in an appealing way. The use of (visible) duct tape is strictly prohibited.
	\item \textbf{Loose cables:} Loose cables hanging out of the robot are not permitted.
	\item \textbf{Safety:} The robot must not have sharp edges or elements that might harm people.
	\item \textbf{Annoyance:} The robot must not be continuously making loud noises or use blinding lights.
	\item \textbf{Marks:} The robot may not exhibit any kind of artificial marks or patterns.
	\item \textbf{Driving:} To be safe, the robots should be careful when driving (obstacle avoidance is mandatory).
\end{enumerate}

\subsection{Standard Platform Leagues}
\label{sec:rules:robotappearance_spl}
For Robots competing in a \SPL{}, modifications and alterations to the robots are strictly forbidden. This includes, but is not limited to, attaching, connecting, plugging, gluing, and taping components into and onto the robot, as well as, modifying or altering the robot structure. Not complying with this rule, leads to an immediate disqualification and penalization of the team (see~\ref{sec:rules:penaltiesbonuses}).

Robots are allowed to \enquote{wear} clothes, have stickers (e.g., a sticker exhibiting the logo of a sponsor), and be painted as long as they are compliant with section \ref{sec:rules:robotappearance}.

\subsubsection{DSPL Modifications}
\label{sec:rules:mountingbracket}
In the \DSPL{}, some modifications to the \HSR{} are allowed. An official \MountingBracket{} is provided by Toyota for the \HSR{}. Any laptop fitting inside the \MountingBracket{} can be used as additional on board computing. Furthermore, teams are allowed to attach the following devices to the robot or the laptop in the \MountingBracket{}:
\begin{enumerate}
	\item \textbf{Audio:} USB audio output device, e.g. USB-powered speaker, possibly with sound card.
	\item \textbf{Wi-Fi Adapter:} USB-powered IEEE 802.11ac (or newer) compliant device.
	\item \textbf{Ethernet Switch:} USB-powered IEEE 802.3ab (or newer) compliant device.
\end{enumerate}

\noindent A maximum of three such devices can be attached, they cannot increase the robot's dimension.

\subsection{Robot Specifications for the Open Platform League }
Robots competing in the RoboCup@Home Open Platform League must comply with security specifications in order to avoid causing any harm while operating in human environments.

\subsubsection{Size and weight of robots}
\label{rule:robots_size}

\begin{enumerate}
	\item \textbf{Dimensions:} The dimensions of a robot should not exceed the limits of an average door, which is \SI{200}{\centi\meter} by \SI{70}{\centi\meter} in most countries.\\
	The TC may allow the qualification and registration of larger robots, but due to the international character of the competition it cannot be guaranteed that the robots can actually enter the arena. In case of doubt, contact the local organization.
	\item \textbf{Weight:} There is no specific weight restriction. However, the weight of the robot and the pressure it exerts on the floor should not exceed local regulations for the construction of buildings which are used for living and/or offices in the country where the competitions is being held.
	\item \textbf{Transportation:} Team members are responsible for quickly moving the robot out of the arena.	If the robot cannot move by itself (for any reason), the team members must be able to transport the robot away with an easy and fast procedure.
\end{enumerate}



\subsubsection{Emergency stop button}
\label{rule:robots_emergency_button}

\begin{enumerate}
	\item \textbf{Accessibility and visibility:} Every robot has to provide an easily accessible and visible \iterm{emergency stop} button.
	\item \textbf{Color:} It must be coloured red, and preferably be the only red button on the robot. If it is not the only red button, the TC may ask the team to tape over or remove the other red button.
	\item \textbf{Robot behavior:} When pressing this button, the robot and all parts of it have to stop moving immediately.
	\item \textbf{Inspection:} The emergency stop button is tested during the \iterm{Robot Inspection} test (see~\refsec{sec:robot_inspection}).
\end{enumerate}




\subsubsection{Start button}
\label{rule:start_button}

\begin{enumerate}
	\item \textbf{Requirements:} As stated in~\refsec{rule:start_signal}, teams that aren't able to carry out the default start signal (opening the door) have to provide a \iterm{start button} that can be used to start tests. The team needs to announce this to the TC before every test that involves a start signal, including \iterm{Robot Inspection}.
	\item \textbf{Definition:} The start button can be any \quotes{one-button procedure} that can be easily executed by a referee.  This includes, for example, the release of the \iterm{emergency button} (\refsec{rule:robots_emergency_button}), a hardware button different from the \iterm{emergency button} (e.g., a green button), or a software button in a Graphical User Interface.
	\item \textbf{Inspection:} It is during the the \iterm{Robot Inspection} test (see~\refsec{sec:robot_inspection}) that the procedure for the start button, if needed, is announced to the TC and inspected. The start button for a robot should be the same for all the tests.
	\item \textbf{Penalty for using start button:} If a team needs to use the start button in a test where opening the door is the start signal, it may receive a penalty (see~\refsec{rule:start_signal}).
\end{enumerate}




\subsubsection{Audio output plug}
\label{rule:roobt_audio_out}

\begin{enumerate}
	\item \textbf{Mandatory plug:} Either the robot or some external device connected to it \emph{must} have a \iterm{speaker output plug}. It is used to connect the robot to the sound system so that the audience and the referees can hear and follow the robot's speech output.
	\item \textbf{Inspection:} The output plug needs to be presented to the TC during the \iterm{Robot Inspection} test (see~\refsec{sec:robot_inspection}).
	\item \textbf{Audio during tests:} Audio (and speech) output of the robot during a test have to be understood at least by the referees and the operators.
	\begin{compactitem}
		\item It is the responsibility of the teams to plug in the transmitter before a test, to check the sound system, and to hand over the transmitter to next team.
		\item Do not rely on the sound system! For fail-safe operation and interacting with operators make sure that the sound system is not needed, e.g., by having additional speakers directly on the robot.
\end{compactitem}
\end{enumerate}




\subsubsection{Appearance}
\label{rule:robots_appearance}
Open Platform Robots should have a neat appearance that resembles more a safe and finished product than an early stage prototype, paying special attention in completely cover the robot's internal hardware (electronics and cables) in an appealing way.
% However, teams must keep in mind that no artificial markers are allowed when personalizing the appearance or a robot. This includes, but is not limited to bar codes, QR codes, OpenCV markers, fluorescent and phosphorescent colors, and reflective stickers.
Although covering the robot's internal hardware with a T-Shirt is not forbidden (for now) it is strongly unadvised.



% Local Variables:
% TeX-master: "../Rulebook"
% End:


% %% %%%%%%%%%%%%%%%%%%%%%%%%%%%%%%%%%%%%%%%%%%%%%%%%%%%%%%%%%
% 
% External Devices
% 
% %% %%%%%%%%%%%%%%%%%%%%%%%%%%%%%%%%%%%%%%%%%%%%%%%%%%%%%%%%%

\section{External devices}
\label{rule:robot_external_devices}
Everything which is not part of the robot is considered an \iterm{external device}.
All external devices must be authorized by the \iaterm{Technical Committee}{TC} during the \iterm{Robot Inspection} test (see~\refsec{sec:robot_inspection}).
The \iaterm{Technical Committee}{TC} specifies whether an external device can be used freely, under referee supervision, and its impact on scoring.
In general, external devices must be removed quickly after the test.
	
\noindent \textbf{Remark:} The use of \iterm{wireless devices} is strictly prohibited. \iterm{External microphones}, hand microphones, and headsets are not allowed in OPL and it use is discouraged in DSPL and SSPL.

\subsection{On-site external computing}
Computing resources that are not physical attached to the robot are considered \iterm{external computing resources}.
The use of up to 5 external computing resources is allowed, but only through the arena network (see \refsec{rule:scenario_wifi}) and with the previous approval of the \iaterm{Technical Committee}{TC}.
Teams must announce the use of any external computing resource at least 1 month before the competition to the \iaterm{Technical Committee}{TC}.

External Computing Devices must be placed in the \iaterm{\textbf{E}xternal \textbf{C}omputing \textbf{R}esource \textbf{A}rea}{ECRA} which is announced by the \iaterm{Technical Committee}{TC} during setup days.
A switch connected to the arena wireless network will be available to teams in the ECRA.
It is strictly forbidden to connect any kind of device or peripheral (e.g. screens, mouses, keyboards, etc.) to the computers in the ECRA during the competition.

A maximum of two laptops and two people from different teams is allowed at any time in the ECRA.
Teams using laptops as External Computing Devices must remove the device immediately after the test.
Once a test has started, all people must stay at least 1 meter from the ECRA.
Interacting with computers in the ECRA after the Referee has given the start signal will cause the immediate disqualification of the team.

\noindent \textbf{Remark:} Robot operation must be able to operate safely when \iterm{external computing resources} are unavailable.



% On-line devices
\subsection{On-line external computing}
\label{rule:robot_external_computing_online}
Robots are allowed to use \enquote{Cloud services}, \enquote{Internet API's}, and any other type of \iterm{external computing resource}.
Same restrictions for on-site external computing resources apply.

\noindent \textbf{Remark:} The competition organization doesn't guarantee or take any responsibility regarding the availability or reliability of neither the network nor Internet connection.
Teams' use of external computing resources is at their own risk.



% DSPL laptop
\subsection{Official Standard Laptop for DSPL}
\label{rule:osl_dspl}

In the Domestic Standard Platform League, teams may use the \iaterm{Official Standard Laptop}{OSL} connected to the Toyota HSR via Ethernet cable, safely located in the TOYOTA HSR \iterm{Mounting Bracket} provided by TOYOTA for this purpose.

\subsubsection{Technical Specifications}
The technical specifications for the Official Standard Laptop in the Domestic Standard Platform League are the following:


 \begin{itemize}
  \item \textbf{Brand and model:} DELL Alienware 15 or 17
  \item \textbf{CPU:} Core-i7 series
  \item \textbf{RAM:} 16GB or 32GB
  \item \textbf{GPU:} NVIDIA GeForce GTX 1070 or 1080
  \item \textbf{Storage:} Unrestricted.
\end{itemize}

No other brands or models will be accepted. There are no constrains regarding the software installed in the OSL but no additional hardware is allowed.

The referees, Technical Committee, and Organizing Committee members may run random checks anytime during the competition prior to the test to verify that the laptop in the TOYOTA HSR \iterm{Mounting Bracket} has no additional hardware plugged in, and matches the authorized specifications.


% Local Variables:
% TeX-master: "../Rulebook"
% End:


%%%%%%%%%%%%%%%%%%%%%%%%%%%%%%%%%%%%%%%%%%%%%%%%%%%%%%%%%
\section{Organization of the competition}
\label{sec:procedure_during_competition}

\subsection{Stage system}\label{rule:stages}

The competition features a \iterm{stage system}. It is organized in two stages each consisting of a number of specific tasks. It ends with the \iterm{Finals}.

Each \iaterm{stage} comprehends a set of tasks grouped in two thematic scenarios.
% \iaterm{House Cleaner} and \iaterm{Party Host}.
The \iaterm{Housekeeper} scenario features tasks related to cleaning, organizing, and giving maintenance; while the \iaterm{Party Host} scenario focuses in attending guests needs and providing general assistance during a party.

\begin{enumerate}
	\item \textbf{Robot Inspection:} For security, robots are inspected during setup days.
  A robot must pass \iterm{Robot Inspection} test (see~\refsec{sec:robot_inspection}) in order to compete.

	\item \textbf{Stage~I:} The first days of the competition called \iterm{Stage~I}.
	All qualified teams can participate in \iterm{Stage~I}.
	The same task can be performed multiple times (See~\refsec{rule:score_system}).

	\item \textbf{Stage~II:} The best \emph{50\% of teams}\footnotemark (after Stage~I) advance to \iterm{Stage~II}.
	Here, tasks require more complex abilities or combinations of abilities.\\
	\footnotetext{If the total number of teams is less than 12, up to 6 teams may advance to Stage~II}

	\item \textbf{Final demonstration:} The best \emph{two teams} of each league, the ones with the highest score after Stage~II, advance to the final round.
	The final round features only a single task integrating all tested abilities.
	In order to participate in the Finals, a team must have solved at least one task of the Stage~II.
\end{enumerate}

In case of having no considerable score deviation between a team advancing to the next stage and a team dropping out, the TC may announce additional teams advancing to the next stage.


%%%%%%%%%%%%%%%%%%%%%%%%%%%%%%%%%%%%%%%%%%%%%%%%%%%%%%%%%
\subsection{Schedule}
\label{rule:schedule}

\begin{enumerate}
	\item \textbf{Thematic scenario blocks:} Each \iterm{thematic scenario} or \iterm{theme} is split in two \iterm{blocks}.
	At least two blocks are scheduled per day, having each block an assigned theme and lasting no less than two hours.
	The \iaterm{Organizing Committee}{OC} announces the schedule during the setup days (see Table \ref{tbl:schedule}).

	\item \textbf{Slots:} The \iaterm{Organizing Committee}{OC} assigns at least two \iterm{test slots} of 5 minutes to each team in each block.
   The maximum number of \iterm{tests slots} will be announced during setup days by the \iaterm{Technical Committee}{TC} based on the available time and the number of participating teams.
	A team can solve any task during its test slot.
	Remaining block time can be used to assign additional testing slots to interested teams.
	Testing slots are randomly assigned to teams in each block.

	\item \textbf{Tests:} Teams must inform the OC in advance which task(s) will try to solve.
	Only one task can be attempted per test slot.

	\item \textbf{Participation is default:} Teams have to indicate to the \iaterm{Organizing Committee}{OC} when they are \emph{skipping} a test slot. Without such indication, they may receive a penalty when not attending (see~\refsec{rule:not_attending}).
\end{enumerate}

% Please add the following required packages to your document preamble:
% \usepackage[table,xcdraw]{xcolor}
% If you use beamer only pass "xcolor=table" option, i.e. \documentclass[xcolor=table]{beamer}
\begin{table}[h]
	\centering\small
	\newcommand{\teams}[3]{%
		\tiny
		\begin{tabular}{c}%
			\textit{Test slot 1, team $#1$}\\
			\textit{Test slot 2, team $#2$}\\
			$\vdots$\\
			\textit{Test slot $n$, team $#3$}\\
		\end{tabular}
	}
	\newcommand{\wcell}[2]{%
		\parbox[c]{2.5cm}{%
			\vspace{#1}%
			\centering%
			#2%
			\vspace{#1}%
		}%
	}
	\newcommand{\cell}[1]{\wcell{0.2\baselineskip}{#1}}


	\begin{tabular}{
		>{\centering\arraybackslash}m{2.5cm}|%
		>{\columncolor[HTML]{9AFF99}}c |%
		>{\columncolor[HTML]{9AFF99}}c |%
		>{\columncolor[HTML]{CBCEFB}}c |%
		>{\columncolor[HTML]{CBCEFB}}c |%
	}
	\multicolumn{1}{ c }{}
		& \multicolumn{1}{ c }{\cellcolor{white} Day 1 }
		& \multicolumn{1}{ c }{\cellcolor{white} Day 2 }
		& \multicolumn{1}{ c }{\cellcolor{white} Day 3 }
		& \multicolumn{1}{ c }{\cellcolor{white} Day 4 }
		\\\cline{2-5}
	\cell{Block 1\\\footnotesize(9:00 - 12:00)}
		& \cell{Housekeeper\\\teams{i}{j}{i}}
		& \cell{Party Host\\\teams{k}{i}{k}}
		& \cell{Housekeeper\\\teams{i}{j}{i}}
		& \cell{Party Host\\\teams{j}{k}{j}}\\\cline{2-5}

	\multicolumn{1}{ c }{}
		& \multicolumn{4}{ c }{\wcell{0.5\baselineskip}{\color{gray}Lunch}}\\\cline{2-5}

	\cell{Block 2\\\footnotesize(14:00 - 17:00)}
		& \cell{Housekeeper\\\teams{i}{k}{i}}
		& \cell{Party Host\\\teams{k}{j}{k}}
		& \cell{Party Host\\\teams{i}{i}{k}}
		& \cell{Housekeeper\\\teams{k}{j}{j}}\\\cline{2-5}

	\multicolumn{1}{ c }{}
		& \multicolumn{2}{ c }{\wcell{0.5\baselineskip}{\color[HTML]{029734}Stage 1}}
		& \multicolumn{2}{ c }{\wcell{0.5\baselineskip}{\color[HTML]{6668e5}Stage 2}}\\
	\end{tabular}

	\caption{Example schedule.
		Each team has assigned at least two test slots in every block.
		At least two blocks are scheduled per day with an assigned theme.
		A team can choose a different task in each test, meaning at least 4 different tests per stage.
	}
	\label{tbl:schedule}
\end{table}


\subsection{Score system}
\label{rule:score_system}
Each task has a main objective and a set of scoring bonuses.
To score in a test, a team must successfully accomplish the main objective of the task; bonuses are not considered otherwise.
Overall scoring is calculated as the sum of the maximum score obtained in each ability.

The \iaterm{score system} has the following constrains
\begin{enumerate}

	\item \textbf{Stage~I:} The maximum total score per task in \iterm{Stage~I} is \scoring{1000 points}.
	
	\item \textbf{Stage~II:} The maximum total score per task in \iterm{Stage~I} is \scoring{2000 points}.

	\item \textbf{\iterm{Finals}:} Final score is normalized and a special evaluation is used.

	\item \textbf{Minimum score:} The minimum total score per test in \iterm{Stage~I} and \iterm{Stage~II} is \scoring{0 points}.
	Teams cannot receive negative points.

	\item \textbf{Penalties:} An exception to \emph{minimum score} rule are penalties.
	Both penalties for not attending (see~\refsec{rule:not_attending}) and extraordinary penalties (see~\refsec{rule:extraordinary_penalties}) can cause a total negative score.
\end{enumerate}




% Local Variables:
% TeX-master: "../Rulebook"
% End:


%%%%%%%%%%%%%%%%%%%%%%%%%%%%%%%%%%%%%%%%%%%%%%%%%%%%%%%%%
\section{Procedure during Tests}

\subsection{Safety First!}
\label{rule:safetyfirst}
\begin{enumerate}
	\item \textbf{Emergency Stop:} At any time when operating the robot inside and outside the scenario the owners have to stop the robot immediately if there is a possibility of dangerous behavior towards people and/or objects.
	\item \textbf{Stopping on request:} If a referee, member of the Technical or Organizational committee, an Executive or Trustee of the federation stops the robot (by pressing the emergency button) there will be no discussion. Similarly if they tell the team to stop the robot, the robot must be stopped \emph{immediately}.
	\item \textbf{Penalties:} If the team does not comply, the team and its members will be excluded from the ongoing competition immediately by a decision of the RoboCup@Home \iaterm{Technical Committee}{TC}. 	Furthermore, the team and its members may be banned from future competitions for a period not less than a year by a decision of the RoboCup Federation Trustee Board.
\end{enumerate}

\subsection{Maximum number of team members}
\label{rule:number_of_people}
\begin{enumerate}
	\item \textbf{Regular Tests:} During a regular test, the maximum number of team members allowed inside the \Arena{} is \emph{one} (1).
	Exceptions are tests that explicitly require volunteer assistance.
	\item \textbf{Setup:} During the setup of a test, the number of team members inside the \Arena{} is not limited.
	% \item \textbf{Open Demonstrations:} During the \iterm{Open Challenge} \iterm{Demo Challenge}, and the \iaterm{final demonstration}{Finals}, the number of team members inside the arena is not limited.
	%\item \textbf{Open Demonstrations:} During the \iterm{Open Challenge}, and the \iaterm{final demonstration}{Finals}, the number of team members inside the arena is not limited.
	\item \textbf{\FINAL:} During the \FINAL, the number of team members inside the \Arena{} is not limited.
	\item \textbf{Moderation:} During a regular test, one team member \emph{must} be available to host and comment the test (see~\refsec{rule:moderator}).
\end{enumerate}

\subsection{Fair play}
\label{rule:fairplay}
\iterm{Fair Play} and cooperative behavior is expected from all teams during the entire competition, in particular:
\begin{itemize}
	\item while evaluating other teams,
	\item while refereeing, and
	\item when having to interact with other teams' robots.
\end{itemize}
This also includes:
\begin{itemize}
	\item not trying to cheat (e.g., pretending autonomous behavior where there is none),
	\item not trying to exploit the rules (e.g., not trying to solve the task but trying to score), and
	\item not trying to make other robots fail on purpose.
	\item not modifying robots in standard platforms.
\end{itemize}
Disregard of this rule can lead to penalties in the form of negative scores, disqualification for a test, or even for the entire competition.

\subsection{Expected Robot's Behavior}
Unless stated otherwise, it is expected that the robot always behave and react in the same way a polite and friendly human being would do.
This applies also to how robots try solve the assigned task
As rule of thumb, one may ask any non-scientist how she would solve the task.

Please consider that average users will not know the specific procedure to operate a robot.
Hence, interaction should be as with any other human being.


\subsection{Robot Autonomy and Remote Control}
\begin{enumerate}
	\item \textbf{No touching:} During a test, the participants are not allowed to make contact with the robot(s), unless it is in a \enquote{natural} way and required by the task.

	\item \textbf{Natural interaction:} The only allowed means to interact with the robot(s) are gestures and speech.

	\item \textbf{Natural commands:} Anything that resembles direct control is forbidden.

	\item \textbf{Remote Control:} Remotely controlling the robot(s) is strictly prohibited.
	This also includes pressing buttons, or influencing sensors on purpose.

	\item \textbf{Penalties:} Disregard of these rules will lead to disqualification for a test or for the entire competition.
\end{enumerate}



\subsection{Collisions}
\begin{enumerate}
	\item \textbf{\iterm{Touching}:} Gently \emph{touching} objects is tolerated but unadvised.
	However, robots are not allowed to crash with something.
	The \enquote{safety first} rule (\refsec{rule:safetyfirst}) overrides any other rule.

	\item \textbf{\iterm{Major collisions}:} If a robot crushes into something during a test, the robot is immediately stopped.	Additional penalties may apply.

	\item \textbf{\iterm{Functional touching}:} Robots are allowed to apply pressure on objects, push away furniture and, in general, interact with the environment using structural parts other than their manipulators.
	This is known as \iaterm{functional touching}.
	However, the robot must clearly announce the collision-like interaction and kindly request not being stopped.\\
	\textbf{Remark: } Referees can (and will) immediately stop a robot in case or suspicion of \emph{dangerous} behavior.

	\item \textbf{Robot-Robot avoidance:} If two robots encounter each other, they both have to actively try to avoid the other robot.
	\begin{enumerate}
		\item A robot which is not going for a different route within a reasonable amount of time (e.g., \SI{30}{\second}) is removed.
		\item A non-moving robot blocking the path of another robot for longer than a reasonable amount of time (e.g., \SI{30}{\second}) is removed.
	\end{enumerate}
\end{enumerate}



\subsection{Removal of robots}
\label{rule:robot_removal}
Robots not obeying the rules are stopped and removed from the \Arena{}.
\begin{enumerate}
	\item It is the decision of the referees and the TC member monitoring the test if and when to remove a robot.

	\item When told to do so by the referees or the TC member monitoring the test, the team must immediately stop the robot, and remove it from the \Arena{} without disturbing the ongoing test.

\end{enumerate}


\subsection{Start signal}
\label{rule:start_signal}
The default \iterm{start signal} (unless stated otherwise) is \iterm{door opening}.
Other start signals are allowed but must be authorized by the \iaterm{Technical Committee}{TC} during the Robot Inspection (see~\refsec{sec:robot_inspection}).

\begin{enumerate}
	\item \textbf{Door opening:} The robot is waiting behind the door, outside the \Arena{} and accompanied by a team member.
	The test starts when a referee (not a team member) opens the door.

	\item \textbf{Start button:} If the robot is not able to automatically start after the door is open, the team may start the robot using a start button.
	\begin{enumerate}[nosep]
		\item It must be a physical button on the robot (e.g., a dedicated one or releasing the eStop).
		\item It is allowed to use the robot's contact/pressure sensors (e.g., pushing the head or an arm joint).
		\item Using a start button needs to be announced to the referees before the test starts.
		\item There may be penalties for using a start button in some tests
	\end{enumerate}

	\item \textbf{Ad-hoc start signal:} Other means of triggering robot to action are allowed but must be approved by the \iaterm{Technical Committee}{TC} during the Robot Inspection (see~\refsec{sec:robot_inspection}).
	These include:
	\begin{itemize}[nosep]
		\item QR Codes
		\item Verbal instructions
		\item Custom HRI interfaces (apps, software, etc.)
	\end{itemize}
	\textbf{Remark:} There may be penalties for using Ad-hoc start signals in some tests. The use of mouses, keyboards, and devices attached to ECRA computers is strictly forbidden.

\end{enumerate}


\subsection{Entering and leaving the \Arena{}}
\label{rule:start_position}
\begin{enumerate}

	\item \textbf{Start position:} Unless stated otherwise, the robot starts outside of the \Arena{}.
	\item \textbf{Entering:} The robot must autonomously enter the \Arena{}.
	\item \textbf{Success:} The robot is said to \emph{have entered} when the door used to enter can be closed again, and the robot is not blocking the passage.
\end{enumerate}



\subsection{Gestures}
\label{rule:gestures}
Hand gestures may be used to control the robot in the following way:
\begin{enumerate}
	\item \textbf{Definition:} The teams define the hand gestures by themselves.

	\item \textbf{Approval:} Gestures need to be approved by the referees and TC member monitoring the test. Gestures should not involve more than the movement of both arms. This includes, e.g., expressions of sign language or pointing gestures.

	\item \textbf{Instructing operators:} It is the responsibility of the team to instruct operators.
	\begin{enumerate}
		\item The team may only instruct the operator when told to so by a referee.
		\item The team may only instruct the operator in the presence of a referee.
		\item The team may only instruct the robot for as long as allowed by the referee.
		\item When the robot has to instruct the operator, it is the robot that instructs the operator and \emph{not} the team. The team is not allowed to additionally guide the operator, e.g., tell the operator to come closer, speak louder, or to repeat a command.
		\item The robot is allows to instruct the operator at any time.
	\end{enumerate}

	\item \textbf{Receiving gestures:} Unless stated otherwise, it is not allowed to use a speech command to set the robot into a special mode for receiving gestures.
\end{enumerate}



\subsection{Referees}
\label{rule:referees}
All tests are monitored by a referee and one member of the \iaterm{Technical Committee}{TC}.
The following rules apply:

\begin{enumerate}
	\item \textbf{Selection:}
	\begin{itemize}
		\item Referees are chosen by EC/TC/OC.
		\item Referees are announced together with the schedule for the test slot.
	\end{itemize}

	\item \textbf{Not showing up:} Not showing up for refereeing (on time) will result in a penalty (see~\refsec{rule:extraordinary_penalties}).

	\item \textbf{TC monitoring:} A TC member acts as main referee.

	\item \textbf{Referee instructions:} Right before each test, referee instructions are conducted by the TC.
	The referees for all slots need to be present at the \Arena{} where the referee instructions are taking place.
	When and where referee instructions are taking place is announced together with the schedule for the slots.
\end{enumerate}


\subsection{Operators}
\label{rule:operator}
Unless stated otherwise, robots are operated by the referee or by a person selected by the referee.
If the robot fails to understand the default operator, the team may request the use of a custom operator.
Penalty may apply when using a custom operator.


\subsection{Moderator}
\label{rule:moderator}
The LOC is responsible of organizing test moderation in the local language.
The OC may request the participating teams to provide a team member for moderation.
Candidates have to be fluent in the moderation language (default is English).

\noindent\textbf{Responsibilities:} The moderators have to:
\begin{compactitem}
	\item Do \textbf{NOT interfere} with the performance
	\item Explain the tasks being performed
	\item Comment on the performance of the competitor
	\item Follow the instructions of the referee.
\end{compactitem}

\noindent \textbf{Not showing up:} Not showing up for moderation (on time) will result in a penalty (see~\refsec{rule:extraordinary_penalties}).


\subsection{Time limits}
\label{rule:time_limits}
\begin{enumerate}
	\item \textbf{Stage~I:} Unless stated otherwise, the time limit for each test in \iterm{Stage~I} is \timing{5 minutes}.

	\item \textbf{Stage~II:} Unless stated otherwise, the time limit for each test in \iterm{Stage~II} is \timing{10 minutes}.

	\item \textbf{Inactivity:} Robots are not allowed to stand still or get stuck into endless loops.
	A robot not progressing in the task execution (and obviously not trying to), is consider as inactive.
	Robots must be removed after 30 seconds of inactivity.

	\item \textbf{Requesting time:} A robot (not the team) can request referees to make exception from the 30-seconds inactivity time limit.
	In its request, the robot must clearly state for how long it will be performing a time-consuming process (e.g., 60~seconds).
	This time cannot exceed 3 minutes and cannot be used more than once per test.

	\item \textbf{Setup time:} Unless stated otherwise, there is no setup time.
	Robots need to be ready to enter the \Arena{} no later than one minute after the door has been closed to the former team.

	\item \textbf{Time-up:} When the time is up, the team must immediately remove their robot(s) from the  \Arena{}.
	No more additional score will be giving.

	\item \textbf{Show must go on:} On special cases, the referee may let the robot continue the test for demonstration purposes, but no additional points will be scored.
\end{enumerate}



\subsection{Restart}
\label{rule:restart}
Some tasks allow a single restart, a procedure in which the team is allowed to quickly fix any issue with the robot.
Restarts can be requested only when the test slot permits it, and when the amount of remaining time is greater than 50\% of the total.
The procedure is as follows:

\begin{enumerate}
	\item The team request a restart.
	\item The robot is taken to the initial position (e.g. outside the \Arena{}) and gets fixed.
	\item When the robot is ready, the team informs the referee.
\end{enumerate}

The following rules apply:
\begin{enumerate}
	\item \textbf{Number of restarts:} When allowed, the maximum number of restarts is one (1).

	\item \textbf{Early request:} Restart is \textbf{NOT} allowed after the first 50\% of the allotted time has elapsed.

	\item \textbf{Time:} The timer is neither restarted nor stopped.

	\item \textbf{One-minute Setup} The team has 1 minute to fix the robot, starting when the referee announces th restart.
	If the robot is not ready, the test is considered finished.

	\item \textbf{Scoring:} If the score of the second attempt is lower than the score of the first one, the average score of first and second run is taken.
\end{enumerate}

% Local Variables:
% TeX-master: "../Rulebook"
% End:


\section[Deus ex Machina]{Deus ex Machina: Bypassing features with human help \\ \small Because the show must go on}
\label{rule:continue}
Robots can't score unless they accomplish the main goal of a task.
However, in many real-life situations, a minor malfunction may prevent the robot from accomplishing a task.
To prevent this situation, while fostering awareness and human-robot interaction, robots are allowed to request human assistance during a test.

\subsection{Procedure}
\label{rule:continue_procedure}
The procedure to request human assistance while solving a task is as follows:

\begin{enumerate}
	\item \textbf{Request help:} The robot must indicate loud and clear that it requires human assistance. It must be clearly stated:
	\begin{compactitem}
		\item The nature of the assistance
		\item The particular goal or desired result
		\item How the action must be carried out (when necessary)
		\item Details about how to interact with the robot (when necessary)
	\end{compactitem}

	\item \textbf{Supervise:} The robot must be aware of the human's actions, being able to tell when the requested action has been completed, as well as guiding the human assistant (if necessary) during the process.

	\item \textbf{Acknowledge:} The robot must politely thank the human for the assistance provided.
\end{enumerate}

\subsection*{Example}
\label{rule:continue_example}
In this example the robot has to clean the table but is unable to grasp the spoon. 
\begin{itemize}[noitemsep]
	\small
	\item[\textcolor{gray}{R:}] \texttt{I am sorry but the spoon is too small for me to take.\\
	Could you please help me with it?\\
	Please say "robot yes" or "robot no" to confirm.}
	\item[\textcolor{gray}{H:}] \textit{Robot, yes!}
	\item[\textcolor{gray}{R:}] \texttt{Thank you! Please follow my instructions.\\
	Please take the purple spoon from the table. It is on my left.}
	\item[\textcolor{gray}{H:}] (Referee takes green fork)
	\item[\textcolor{gray}{R:}] \texttt{You took the wrong object.\\
	Please take the purple spoon from the table. It is on my left.}
	\item[\textcolor{gray}{H:}] (Referee takes purple spoon)
	\item[\textcolor{gray}{R:}] \texttt{I saw you took the spoon.\\
	Would you be so kind of following me to the kitchen?\\
	Please keep the spoon visible in front of you so I can track you. Thank you!}
	\item[\textcolor{gray}{R:}] \texttt{You can stop following me now.\\
	As you can see, the dishwasher is already open.\\
	Please place the spoon in the gray basket on the lower tray.}
	\item[\textcolor{gray}{R:}] \texttt{Lovely! Thanks for your help human.\\
	I'll let you know if I need further assistance.}
\end{itemize}



\subsection{Scoring}
\label{rule:continue_scoring}
There is no limit in the amount of times a robot can request human assistance, but score reduction applies every time it is requested.

\begin{enumerate}
	\item \textbf{Partial execution:} A reduction of 10\% of the maximum attainable score is applied when the robot request a partial solution (e.g. pointing to the person the robot is looking for or placing an object within grasping distance).
	The referee decides whether the requested action is simple enough to corresponds to a partial execution or not.

	\item \textbf{Full awareness:} A reduction of 20\% of the maximum attainable score is applied when the robot is able to track and supervise activity, detecting possible, and when the requested action has been completed.

	\item \textbf{No awareness:} A reduction of 30\% of the maximum attainable score is applied when the robot has to be told when the requested action has been completed.

	\item \textbf{Bonuses:} No bonus points can be scored when the robot requests help to solve part of a task that normally would grant a bonus.

	\item \textbf{Score reduction overlap:} The score reduction for multiple requests of the same kind do not stack, but overlap.
	The total reduction applied correspond to the worse execution (higher reduction of all akin help requests).
	This means, a robot won't be reduced again for requesting help to transport a second object, but a second reduction will apply when the robot asks for a door to be opened.
\end{enumerate}

\subsection{Bypassing Automatic Speech Recognition}
\label{rule:asrcontinue}
Giving commands to the robot is essential in many tests.
When the robot is not able to receive spoken commands, teams are allowed to provide means to bypass ASR via an Alternative method for HRI (see~\refsec{rule:asralternative}).
Nonetheless, Automatic Speech Recognition is preferred.

The following rules apply in addition to the ones specified in section \refsec{rule:continue_scoring}
\begin{enumerate}
	\item \textbf{ASR with Default Operator:} No score reduction.
	The command is given by the human operator who must speak (not shout) loud and clear.
	The \iterm{default operator} may repeat the command up to three times.

	\item \textbf{ASR with Custom Operator:} A reduction of 10\% of the maximum attainable score is applied when a \iterm{custom operator} is requested.
	The Team Leader chooses a person who gives the command \emph{exactly as instructed by the referee}.

	\item \textbf{Gestures:} A reduction of 20\% of the maximum attainable score is applied when a gesture (or set of gestures) is used to instruct the robot.

	\item \textbf{QR Codes:} A reduction of 30\% of the maximum attainable score is applied when a QR code is used to instruct the robot.

	\item \textbf{Alternative Input Method:} A reduction of up to 30\% of the maximum attainable score is applied when a \iterm{alternative HRI interface}, is used to instruct the robot.
	Alternative HRI interfaces (see~\refsec{rule:asralternative}) must be previously approved by the TC during the Robot Inspection (see~\refsec{sec:robot_inspection}).
\end{enumerate}


\subsubsection{Alternative interfaces for HRI}
\label{rule:asralternative}
Alternative methods and interfaces for HRI offer a way for a robot to start or complete a task.
Any reasonable method may be used, with the following criteria:
\begin{itemize}
	\item \textbf{Intuitive to use and self-explanatory:} a manual should not be needed. Teams are not allowed to explain how to interface with the robot. %you immediately know how to use it after a quick glance

	\item \textbf{Effortless use:} Must be as easy to use as uttering a command. %is as easy to use as it is uttering a command

	\item \textbf{Is smart and preemptive:} The interface adapts to the user input, displaying only the options that make sense or that the robot can actually perform.

	\item Exploits the best of the device being used (eg. touch screen, display area, speakers, etc.)
\end{itemize}

Preferably, the alternative HRI must be also adapted to the user.
Consider localization (with English as the default), but also potential users of service robots at their home.
For example: elderly people and people with physical disabilities.

\textbf{\textsc{Award:}} The best alternative is awarded the Best Human-Robot Interface award (\refsec{award:hri}).


% Local Variables:
% TeX-master: "../Rulebook"
% End:


%%%%%%%%%%%%%%%%%%%%%%%%%%%%%%%%%%%%%%%%%%%%%%%%%%%%%%%%%
\newcommand{\penaltybig}{500~}
\newcommand{\penaltysmall}{250~}


\section{Special penalties and bonuses}\label{sec:special_awards}

\subsection{Penalty for not attending}\label{rule:not_attending}
\begin{enumerate}
	\item \textbf{Automatic schedule:} All teams are automatically scheduled for all tests.

	\item \textbf{Announcement:} If a team cannot participate in a test (for any reason), the team leader has to announce this to the OC at least \timing{60 minutes} before the test slot begins.

	\item \textbf{Penalties:} A team that is not present at the start position when their scheduled test starts, the team is not allowed to participate in the test anymore.
	If the team has not announced that it is not going to participate, it gets a penalty of \scoring{\penaltysmall points}.
\end{enumerate}

\subsection{Extraordinary penalties}\label{rule:extraordinary_penalties}
\begin{enumerate}
	\item \textbf{Penalty for cheating:} If a team member is found cheating or breaking the Fair Play rule (see \refsec{rule:fairplay}), the team will be automatically disqualified of the running test, and a penalty of \scoring{\penaltybig points} is handed out.
	The \iaterm{Technical Committee}{TC} may also disqualify the team for the entire competition.

	\item \textbf{Penalty for faking robots:} If a team starts a test, but it does not solve any of the partial tasks (and is obviously not trying to do so), a penalty of \scoring{\penaltysmall points} is handed out.
	The decision is made by the referees and the monitoring TC member.

	\item \textbf{Extra penalty for collision:} In case of major, (grossly) negligent collisions the \iaterm{Technical Committee}{TC} may disqualify the team for a test (the team receives \scoring{0 points}), or for the entire competition.

	\item \textbf{Not showing up as referee or jury member:} If a team does not provide a referee or jury member (being at the \Arena{} on time), the team receives a penalty of \scoring{\penaltysmall points}, and will be remembered for qualification decisions in future competitions.\\
	Jury members missing a performance to evaluate are excluded from the jury, and the team is disqualified from the test (receives \scoring{0 points}).

	\item \textbf{Modifying or altering standard platform robots:} If any unauthorized modification is found on a Standard Platform League robot, the responsible team will be immediately disqualified for the entire competition while also receiving a penalty of \scoring{\penaltybig points} in the overall score. This behavior will be remembered for qualification decisions in future competitions.\\
\end{enumerate}

\subsection{Bonus for outstanding performance}\label{rule:outstanding_performance}
\begin{enumerate}
	\item For every regular test in \iterm{Stage~I} and \iterm{Stage~II}, the @Home \iaterm{Technical Committee}{TC} can decide to give an extra bonus for \iterm{outstanding performance} of up to 10\% of the maximum test score.

	\item This is to reward teams that do more than what is needed to solely score points in a test but show innovative and general approaches to enhance the scope of @Home.

	\item If a team thinks that it deserves this bonus, it should announce (and briefly explain) this to the \iaterm{Technical Committee}{TC} beforehand.

	\item It is the decision of the \iaterm{Technical Committee}{TC} if (and to which degree) the bonus score is granted.
\end{enumerate}


% Local Variables:
% TeX-master: "../Rulebook"
% End:



% Local Variables:
% TeX-master: "Rulebook"
% End:


\chapter{Setup and Preparation}
\label{chap:setup_and_preparation}
Prior to the RoboCup@Home competition, all arriving teams will have the opportunity to setup their robots and prepare for the competition in a \iterm{Setup \& Preparation} phase. This phase is scheduled to start on the first day of the competition, i.e., when the venue opens and the teams arrive. During the setup phase, teams can assemble and test their robots. On the last setup day, a \iterm{welcome reception} will be held. To foster the knowledge exchange between teams a conference-like \iterm{poster session} takes place during the reception. All teams have to get their robots inspected by members of the TC to be allowed to participate in the competition.

\paragraph{Regular tests are not conducted during setup \& preparation.} The competition starts with Stage~I (see~\refsec{chap:stage_I}).

\begin{table}[h]
  \newcolumntype{C}[1]{>{\centering\let\newline\\\arraybackslash\hspace{0pt}}m{#1}}
  \newcolumntype{S}{C{1.6cm}}
  \newcolumntype{M}{C{3.2cm}}
  \begin{center}
    \caption{Stage System and Schedule per League (distribution of tests and stages over days may vary)}
    \begin{tabularx}{14.56cm}{S|S|S|S|S|S|S|S}
      \hline
      \multicolumn{2}{|M|}{ \cellcolor[HTML]{FFFFC7}Setup \& \newline Preparation} &
      \multicolumn{2}{M|}{ \cellcolor[HTML]{67FD9A}\iterm{Stage~I}} &
      \multicolumn{2}{M|}{ \cellcolor[HTML]{9698ED}\iterm{Stage~II}} &
      \multicolumn{2}{M|}{ \cellcolor[HTML]{FFCCC9}\iterm{Finals}}\\
      \hline
      %Second row
      \multicolumn{1}{S|}{} &
      \multicolumn{2}{M|}{$\xrightarrow{advance}$\newline All teams that \newline passed Inspection} &
      \multicolumn{2}{M|}{$\xrightarrow{advance}$\newline Best 10 ($<6$) \newline or best 50\% ($\geq 12$)} &
      \multicolumn{2}{M|}{$\xrightarrow{advance}$\newline Best 2 \newline teams} &
      \multicolumn{1}{C{1.2cm}}{~}
      \\ \cline{2-7}
    \end{tabularx}
  \end{center}
\end{table}


\section{General Setup}
\label{sec:general_setup}
Depending on the schedule, the \iterm{Setup \& Preparation} phase lasts for one or two days.

\begin{enumerate}
	\item \textbf{Start:} Setup \& Preparation starts when the venue opens for the first time.
	\item \textbf{Intention:} During Setup \& Preparation, teams arrive, bring or receive their robots, and assemble and test them.
	\item \textbf{Tables:} The local organization will setup and randomly assign team tables.
	\item \textbf{Groups:} Depending on the number of teams, the \iaterm{Organizing Committee}{OC} may form multiple groups of teams (usually two) for the first (and second stage). The OC will assign teams to groups and announce the assignment to the teams.
	\item \textbf{Arena:} The arena is available to all teams during Setup \& Preparation. The OC may schedule special test or mapping slots in which arena access is limited to one or more teams exclusively (all teams get slots). Note, however, that the arena may not yet be complete and that last works are conducted in the arena during the setup days.
	\item \textbf{Objects:} The delegation of EC, TC, OC and local organizers will buy the objects (see~\refsec{rule:scenario_objects}). Note, however, that the objects may not be available at all times and not from the beginning of Setup \& Preparation.
\end{enumerate}

\section{Welcome Reception}
\label{sec:welcome_reception}
Traditionally --since Eindhoven 2013-- the RoboCup@Home holds an own \iterm{welcome reception} in addition to the official opening ceremony. During the welcome reception, a \iterm{poster session} is held in which teams present their research foci and latest results (see~\refsec{sec:poster_teaser_session}).
\begin{enumerate}
	\item \textbf{Time:} The welcome reception is held in the evening of the last setup day.
	\item \textbf{Place:} The welcome reception takes place in the @Home arena and/or in the RoboCup@Home team area.
	\item \textbf{Snacks \& drinks:} During the welcome reception snacks and beverages (beers, sodas, etc.) are served.
	\item \textbf{Organization:} It is the responsibility of the OC and the local organizers to organize the welcome reception \& poster session including
		\begin{enumerate}
			\item organizing poster stands (one per team) or alternative to present the posters,
			\item organizing the snacks and drinks,
			\item inviting officials, sponsors, local organization and the trustees of the RoboCup Federation to the event.
		\end{enumerate}
	\item \textbf{Poster presentation:} During the welcome reception, the teams give a poster presentation on their research focus, recent results, and their scientific contribution.
	Both the poster and the teaser talk are evaluated by a jury (see~\refsec{sec:poster_teaser_session}).
\end{enumerate}

\section{Poster Teaser Session}
\label{sec:poster_teaser_session}
Before the welcome reception \& poster session, a \iterm{poster teaser session} is held. In this teaser session, each team can give a short presentation of their research and the poster being presented at the poster session.

\subsection{Poster teaser session}
\begin{enumerate}
	\item \textbf{Presentation:} Each team has a maximum of three minutes to give a short presentation of their poster.
	\item \textbf{Time:} The poster teaser session is to be held before the welcome reception \& poster session (see~\refsec{sec:welcome_reception}).
	\item \textbf{Place:} The poster session may be held in or around the arena, but should not interfere with the robot inspection (see~\refsec{sec:robot_inspection}).
	\item \textbf{Evaluation:} The teaser presentation and the poster presentation are evaluated by a jury consisting of members of the other teams. Each team has to provide one person (preferably the team-leader) to follow
	and evaluate
	the entire poster teaser session and the poster session. Not providing a person results in no score for this team in the \iterm{Open Challenge}.

	%%%%%%%%%%%%%%%%%%%%%%%%%%%%%%%%%%%%%%%%%%%%%%%%%%%%%%%%%%%%%%%%%%%%%%%%%%%%%%
	%
	% In previous years, scores from teaser session has not been used for scoring
	% during competition. Therefore, this section has been commented out
	%
	%%%%%%%%%%%%%%%%%%%%%%%%%%%%%%%%%%%%%%%%%%%%%%%%%%%%%%%%%%%%%%%%%%%%%%%%%%%%%%
	\item \textbf{Criteria:} For each of the following evaluation criteria, a maximum of 10 points is given per jury member:
	\begin{enumerate}
		\item Novelty and scientific contribution
		\item Relevance for RoboCup@Home
		\item Presentation (Quality of poster, teaser talk and discussion during poster session)
	\end{enumerate}
	\item \textbf{Score:} The points given by each jury member are scaled to obtain a maximum of 50 points. The total score for each team is the mean of the jury member scores. To neglect outliers, the N best and worst scores are left out:
	$$
	score=\frac{\sum \text{team-leader-score}}{\text{number-of-teams}-\left ( 2N+1  \right )},N=\left\{\begin{matrix}
	1, & \text{number-of-teams} \geq 10\\
	2, & \text{number-of-teams} < 10
	\end{matrix}\right.
	$$
	\item \textbf{Sheet collection:} Evaluation sheets are collected by the OC at a later time (announced beforehand by the OC), allowing teams to continue knowledge exchange during the first days of the competition (Stage~I).
	\item \textbf{OC Instructions:}
	\begin{itemize}
		\item Prepare and distribute evaluation sheets (before the poster teaser session.)
		\item Collect evaluation sheets.
		\item Organize and manage the poster teaser presentations and the poster session.
	\end{itemize}
\end{enumerate}

\section{Robot Inspection}
\label{sec:robot_inspection}
Safety is the most important issue when interacting with humans and operating in the same physical workspace. Because of that all participating robots are inspected before participating in RoboCup@Home. Every team needs to get its robot(s) inspected and approved for participation.

\begin{enumerate}
	\item \textbf{Procedure:} The \iterm{robot inspection} is conducted like a regular test, i.e., starts with the opening of the door (see~\refsec{rule:start_signal}). One team after another (and one robot after another) has to enter the arena through a designated entrance door, move to the \textit{examination point}, and leave the arena through the designated exit door. In between entering and leaving the robot is inspected.
	\item \textbf{Inspectors:} The robots are inspected by the \iaterm{Technical Committee}{TC}.
	\item \textbf{Checked aspects:} It is checked if the robots comply with the rules (see~\refsec{rule:robots}), checking in particular:
	\begin{itemize}
		\item emergency button(s)
		\item collision avoidance (a TC member steps in front of the robot)
		\item voice of the robot (it must be loud and clear)
		\item custom containers (bowl, tray, etc.)
		\item external devices (including wireless network), if any
		\item Alternative Human-Robot interfaces(see~\refsec{rule:asrcontinue}).
		\item \textbi{Standard Platform robots}
		\begin{itemize}
			\item Neat appearance
			\item No modifications have been made
			\item Specifications of the \iaterm{Official Standard Laptop}{OSL} (if required)
		\end{itemize}
		\item \textbi{Open Platform robots}
		\begin{itemize}
			\item robot speed and dimension
			\item start button (if the team is going to require it)
			\item robot speaker system (plug for RF Transmission)
			\item other safety issues (duct tape, hanging cables, sharp edges etc.)
		\end{itemize}
	\end{itemize}
	\item \textbf{Re-inspection:} If the robot is not approved in the inspection, it is the responsibility of the team to get the approval (later). Robots are not allowed to participate in any test before passing the inspection by the TC.
	\item \textbf{Time limit:} The robot inspection is interrupted after three minutes (per robot). When told to so by the TC (in case of time interrupt or failure), the team has to move the robot out of the arena through the designated exit door.
	\item \textbf{Appearance Evaluation:} In addition to the inspection, the TC evaluates the appearance of the robots. Robots are expected to look nice (no duct tape, no cables hanging loose etc.). In case of objection, the TC may penalize the team with a penalty of maximum 50 points.
	\item \textbf{Accompanying team member:} Each robot is accompanied by only one team member (team leader is advised).
	%%%%%%%%%%%%%%%%%%%%%%%%%%%%%%%%%%%%%%%%%%%%%%%%%%%%%%%%%%%%%%%%%%%%%%%%%%%%%%
	%
	% We are not really using the registration form. It's a paper waste
	%
	%%%%%%%%%%%%%%%%%%%%%%%%%%%%%%%%%%%%%%%%%%%%%%%%%%%%%%%%%%%%%%%%%%%%%%%%%%%%%%
	% \item \textbf{Registration form:} Every team needs to fill out a registration form which is brought to the TC by the accompanying team member.
	\item \textbf{OC instructions (at least 2h before the Robot Inspection):}
	\begin{itemize}
		\item Announce the entry and exit doors.
		\item Announce the location of the \textit{examination point} into the arena.
		\item Specify and announce where and when the poster teaser and the poster presentation session take place.
		% \item Prepare and distribute registration sheets (external devices etc., place for notes and signatures of TC and team leader).
		\item Prepare and distribute poster session evaluation sheets.
	\end{itemize}
\end{enumerate}


% Local Variables:
% TeX-master: "Rulebook"
% End:


\chapter{Tests in Stage I}
\label{chap:stage_I}

\begin{itshape}
\iterm{Stage~I} comprehends three \textbf{ability tests} and an \textbf{integration test}.
Each ability test is designed to evaluate the average performance of the robot in one particular skill
% , providing data for benchmarking.
Meanwhile, the integration test has been designed to evaluate how this abilities work together while solving a common task.

The total score for ability and integration tests is the average of the best two performances out of preferably three performances (given the time constraints of a competition).
The point of this is to both eliminate good and bad luck for the robots/teams and to get a more objective view of the performance,
  not to give teams time to tweak the robot between test performances.

\iterm{Help-me-carry} (demonstration for the audience) goes out of the arena and into the venue between the audience.

\end{itshape}

\subsection*{Scheduling}
For maximal efficiency, teams will be scheduled interleaved:
  Team A does an attempt while team B sets up their robot. When A is done, it moves out the way for team B, then B attempts while A sets up the robot again etc.

The preparing team should prepare their robot close to the place of the test, but not interfere with the performing robot.
Prepared robots must wait at this preparation location until commanded to start the test.
When commanded to start, the robot must move automatically beyond this point.

Robot should be ready to start the next attempt to the same test as fast as possible:
  when the performing robot is done with a attempt, the next robot must be ready to go with the start of a button or a voice command.

\newpage
\section{Cocktail Party [SSPL only]}

The robot has to learn and recognize previously unknown people, and fetch orders.

\subsection{Focus}

This test focuses on human detection and recognition, safe navigation and human-robot interaction with unknown people.

\subsection{Setup}
\begin{itemize}
	\item \textbf{Party room}: any (large) room inside the apartment when normally a party would be held.
	\item \textbf{Guests:} At least five people are distributed in a predefined \quotes{party room} either sitting or standing.
                Three of the guests have drink orders.
	\item \textbf{Bar:} The bar is any flat surface where objects can be placed, in a room other than the \quotes{party room}.
                All available beverages are on top of the bar.
	\item \textbf{Bartender:} The Bartender may be standing either behind the bar or next to it, depending on the arena setup.
\end{itemize}

\subsection{Task}

\begin{enumerate}

	\item \textbf{Entering:} The robot enters the arena and navigates to the party room.
	\item \textbf{Getting called:} The guests call the robot simultaneously, either rising an arm, waving, or shouting. The robot has to approach one of them.
        Optionally, the robot can skip the call detection and ask for a person to step in front of it (the referees determine who approaches to the robot).

	The calling person introduces themself by name before giving the order of a drink.
	The robot leads a dialogue to determine the person's name and obtain their drink order. \\


	\item \textbf{Taking the order:} After the robot has taken the order of the first guest, it can either take more orders or proceed to place the order.

	\item \textbf{Placing the orders:} The robot has to navigate to the \textit{Bar}. The robot must repeat each order to the \textit{Bartender}, clearly stating:
	\begin{enumerate}
		\item The person's name,
		\item The person's chosen drink,
		\item A description of unique characteristics of that person that allow the \textit{Bartender} to find them (e.g. gender, hair colour, how they are dressed, etc).
	\end{enumerate}

	While the robot places the orders, the people in the \quotes{party room} change their places within the party room (on request of the referees).

	  \item \textbf{Missing beverage:} One of the ordered drinks is not available and therefore missing from the bar.
	The robot should realize this inconvenience and tell the \textit{Bartender}, providing a list of 3 viable alternatives.
	If the robot cannot detect which drink is missing, the \textit{Bartender} will clearly state which of the beverages is not available and provide a list of 3 alternatives.

	\item \textbf{Correcting an order:} The robot should navigate back to the \quotes{party room}, find the person whose drink is missing and provide the alternatives for them to choose from.\\

	If the robot returns to find a person and the person is not there, it should call that person loudly and the person should respond (either through sound or by waving their hand). The robot should go to the person who is speaking and waving their hand to check their identity.

	\item \textbf{Placing the corrected order:} The robot should navigate to the bar and inform the bartender of the change to the guest's order.
\end{enumerate}

\subsection{Additional rules and remarks}
\begin{enumerate}
	\item \textbf{Repeating names:} The robot may ask to repeat the name if it has not understood it.

	\item \textbf{Misunderstood names:} If the robot misunderstands the name, the understood (wrong) name is used in the remainder of this test.

	\item \textbf{Misunderstood order:} If the robot does not understand the order, it can continue with an own assignment of drinks to people or with a wrong, misunderstood assignment.

	\item \textbf{Approaching non-calling people:} If the robot approaches a person that is not calling and asks for an order, the person indicates that they does not want to order anything. No points can be scored for understanding names or orders, or for grasping or delivery for a non-calling person.

	\item \textbf{Guest description:} The guest's description must be unique inside the scenario. For instance, it make no sense to state that a person is wearing a red T-shirt if two people are wearing them. In the same sense, stating that the ordering guest is \textit{tall} can lead to confusion, but stating that is the \textit{tallest} does not.

	\item \textbf{Changing places:} After giving the order (when the robot is not in the party room), the referees may re-arrange the people including their body posture. That is, a sitting person may change to a standing posture and vice versa.

	\item \textbf{Positions and orientations:} All test participants roughly stay where they are (if not asked to move by the referees), but they are allowed to move in certain limits (e.g. turn around, make a step aside). They do not need to look at the robot, but are requested to do so, when instructed by the robot.

	\item \textbf{Empty arena:} During the test, only the robot, the guest, and the Bartender are in the arena. The door opener, the referees and other personnel that is not assigned as test people will be outside the scenario.

	\item \textbf{Calling instruction:} The team needs to specify before the test which ways of getting the attention of the robot are allowed for standing persons. This can be waving, calling or both of them. The robot can also decide to skip this part, by asking for people to get close to it.
	
	\item \textbf{Sitting persons:} Sitting persons might have an order but are not actively calling the robot.
\end{enumerate}

\subsection{Referee instructions}

The referees need to
\begin{itemize}
	\item select at least 5 volunteers and assign names from the list of person names (see \refsec{rule:scenario_names})
	\item arrange (and re-arrange) people in the party room. At least one is sitting
	\item assign orders to two standing persons
	\item assign an order to a sitting person
	\item select the person (bartender) who will serve the drinks,
	\item place drinks at the bar while one drink is missing
	\item in case the robot skips the calling detection, select the ordering person to approach the robot,
	\item write down the understood names and drinks during an order and update the order accordingly.
\end{itemize}

\subsection{OC instructions}

2h before test:
\begin{itemize}
	\item Specify and announce the rooms where the test takes place.
	\item Specify and announce the location where the drinks are served.
\end{itemize}

\newpage
\subsection{Score sheet}
\section{Cocktail Party [SSPL only]}

The robot has to learn and recognize previously unknown people, and fetch orders.

\subsection{Focus}

This test focuses on human detection and recognition, safe navigation and human-robot interaction with unknown people.

\subsection{Setup}
\begin{itemize}
	\item \textbf{Party room}: any (large) room inside the apartment when normally a party would be held.
	\item \textbf{Guests:} At least five people are distributed in a predefined \quotes{party room} either sitting or standing.
                Three of the guests have drink orders.
	\item \textbf{Bar:} The bar is any flat surface where objects can be placed, in a room other than the \quotes{party room}.
                All available beverages are on top of the bar.
	\item \textbf{Bartender:} The Bartender may be standing either behind the bar or next to it, depending on the arena setup.
\end{itemize}

\subsection{Task}

\begin{enumerate}

	\item \textbf{Entering:} The robot enters the arena and navigates to the party room.
	\item \textbf{Getting called:} The guests call the robot simultaneously, either rising an arm, waving, or shouting. The robot has to approach one of them.
        Optionally, the robot can skip the call detection and ask for a person to step in front of it (the referees determine who approaches to the robot).

	The calling person introduces themself by name before giving the order of a drink.
	The robot leads a dialogue to determine the person's name and obtain their drink order. \\


	\item \textbf{Taking the order:} After the robot has taken the order of the first guest, it can either take more orders or proceed to place the order.

	\item \textbf{Placing the orders:} The robot has to navigate to the \textit{Bar}. The robot must repeat each order to the \textit{Bartender}, clearly stating:
	\begin{enumerate}
		\item The person's name,
		\item The person's chosen drink,
		\item A description of unique characteristics of that person that allow the \textit{Bartender} to find them (e.g. gender, hair colour, how they are dressed, etc).
	\end{enumerate}

	While the robot places the orders, the people in the \quotes{party room} change their places within the party room (on request of the referees).

	  \item \textbf{Missing beverage:} One of the ordered drinks is not available and therefore missing from the bar.
	The robot should realize this inconvenience and tell the \textit{Bartender}, providing a list of 3 viable alternatives.
	If the robot cannot detect which drink is missing, the \textit{Bartender} will clearly state which of the beverages is not available and provide a list of 3 alternatives.

	\item \textbf{Correcting an order:} The robot should navigate back to the \quotes{party room}, find the person whose drink is missing and provide the alternatives for them to choose from.\\

	If the robot returns to find a person and the person is not there, it should call that person loudly and the person should respond (either through sound or by waving their hand). The robot should go to the person who is speaking and waving their hand to check their identity.

	\item \textbf{Placing the corrected order:} The robot should navigate to the bar and inform the bartender of the change to the guest's order.
\end{enumerate}

\subsection{Additional rules and remarks}
\begin{enumerate}
	\item \textbf{Repeating names:} The robot may ask to repeat the name if it has not understood it.

	\item \textbf{Misunderstood names:} If the robot misunderstands the name, the understood (wrong) name is used in the remainder of this test.

	\item \textbf{Misunderstood order:} If the robot does not understand the order, it can continue with an own assignment of drinks to people or with a wrong, misunderstood assignment.

	\item \textbf{Approaching non-calling people:} If the robot approaches a person that is not calling and asks for an order, the person indicates that they does not want to order anything. No points can be scored for understanding names or orders, or for grasping or delivery for a non-calling person.

	\item \textbf{Guest description:} The guest's description must be unique inside the scenario. For instance, it make no sense to state that a person is wearing a red T-shirt if two people are wearing them. In the same sense, stating that the ordering guest is \textit{tall} can lead to confusion, but stating that is the \textit{tallest} does not.

	\item \textbf{Changing places:} After giving the order (when the robot is not in the party room), the referees may re-arrange the people including their body posture. That is, a sitting person may change to a standing posture and vice versa.

	\item \textbf{Positions and orientations:} All test participants roughly stay where they are (if not asked to move by the referees), but they are allowed to move in certain limits (e.g. turn around, make a step aside). They do not need to look at the robot, but are requested to do so, when instructed by the robot.

	\item \textbf{Empty arena:} During the test, only the robot, the guest, and the Bartender are in the arena. The door opener, the referees and other personnel that is not assigned as test people will be outside the scenario.

	\item \textbf{Calling instruction:} The team needs to specify before the test which ways of getting the attention of the robot are allowed for standing persons. This can be waving, calling or both of them. The robot can also decide to skip this part, by asking for people to get close to it.
	
	\item \textbf{Sitting persons:} Sitting persons might have an order but are not actively calling the robot.
\end{enumerate}

\subsection{Referee instructions}

The referees need to
\begin{itemize}
	\item select at least 5 volunteers and assign names from the list of person names (see \refsec{rule:scenario_names})
	\item arrange (and re-arrange) people in the party room. At least one is sitting
	\item assign orders to two standing persons
	\item assign an order to a sitting person
	\item select the person (bartender) who will serve the drinks,
	\item place drinks at the bar while one drink is missing
	\item in case the robot skips the calling detection, select the ordering person to approach the robot,
	\item write down the understood names and drinks during an order and update the order accordingly.
\end{itemize}

\subsection{OC instructions}

2h before test:
\begin{itemize}
	\item Specify and announce the rooms where the test takes place.
	\item Specify and announce the location where the drinks are served.
\end{itemize}

\newpage
\subsection{Score sheet}
\section{Cocktail Party [SSPL only]}

The robot has to learn and recognize previously unknown people, and fetch orders.

\subsection{Focus}

This test focuses on human detection and recognition, safe navigation and human-robot interaction with unknown people.

\subsection{Setup}
\begin{itemize}
	\item \textbf{Party room}: any (large) room inside the apartment when normally a party would be held.
	\item \textbf{Guests:} At least five people are distributed in a predefined \quotes{party room} either sitting or standing.
                Three of the guests have drink orders.
	\item \textbf{Bar:} The bar is any flat surface where objects can be placed, in a room other than the \quotes{party room}.
                All available beverages are on top of the bar.
	\item \textbf{Bartender:} The Bartender may be standing either behind the bar or next to it, depending on the arena setup.
\end{itemize}

\subsection{Task}

\begin{enumerate}

	\item \textbf{Entering:} The robot enters the arena and navigates to the party room.
	\item \textbf{Getting called:} The guests call the robot simultaneously, either rising an arm, waving, or shouting. The robot has to approach one of them.
        Optionally, the robot can skip the call detection and ask for a person to step in front of it (the referees determine who approaches to the robot).

	The calling person introduces themself by name before giving the order of a drink.
	The robot leads a dialogue to determine the person's name and obtain their drink order. \\


	\item \textbf{Taking the order:} After the robot has taken the order of the first guest, it can either take more orders or proceed to place the order.

	\item \textbf{Placing the orders:} The robot has to navigate to the \textit{Bar}. The robot must repeat each order to the \textit{Bartender}, clearly stating:
	\begin{enumerate}
		\item The person's name,
		\item The person's chosen drink,
		\item A description of unique characteristics of that person that allow the \textit{Bartender} to find them (e.g. gender, hair colour, how they are dressed, etc).
	\end{enumerate}

	While the robot places the orders, the people in the \quotes{party room} change their places within the party room (on request of the referees).

	  \item \textbf{Missing beverage:} One of the ordered drinks is not available and therefore missing from the bar.
	The robot should realize this inconvenience and tell the \textit{Bartender}, providing a list of 3 viable alternatives.
	If the robot cannot detect which drink is missing, the \textit{Bartender} will clearly state which of the beverages is not available and provide a list of 3 alternatives.

	\item \textbf{Correcting an order:} The robot should navigate back to the \quotes{party room}, find the person whose drink is missing and provide the alternatives for them to choose from.\\

	If the robot returns to find a person and the person is not there, it should call that person loudly and the person should respond (either through sound or by waving their hand). The robot should go to the person who is speaking and waving their hand to check their identity.

	\item \textbf{Placing the corrected order:} The robot should navigate to the bar and inform the bartender of the change to the guest's order.
\end{enumerate}

\subsection{Additional rules and remarks}
\begin{enumerate}
	\item \textbf{Repeating names:} The robot may ask to repeat the name if it has not understood it.

	\item \textbf{Misunderstood names:} If the robot misunderstands the name, the understood (wrong) name is used in the remainder of this test.

	\item \textbf{Misunderstood order:} If the robot does not understand the order, it can continue with an own assignment of drinks to people or with a wrong, misunderstood assignment.

	\item \textbf{Approaching non-calling people:} If the robot approaches a person that is not calling and asks for an order, the person indicates that they does not want to order anything. No points can be scored for understanding names or orders, or for grasping or delivery for a non-calling person.

	\item \textbf{Guest description:} The guest's description must be unique inside the scenario. For instance, it make no sense to state that a person is wearing a red T-shirt if two people are wearing them. In the same sense, stating that the ordering guest is \textit{tall} can lead to confusion, but stating that is the \textit{tallest} does not.

	\item \textbf{Changing places:} After giving the order (when the robot is not in the party room), the referees may re-arrange the people including their body posture. That is, a sitting person may change to a standing posture and vice versa.

	\item \textbf{Positions and orientations:} All test participants roughly stay where they are (if not asked to move by the referees), but they are allowed to move in certain limits (e.g. turn around, make a step aside). They do not need to look at the robot, but are requested to do so, when instructed by the robot.

	\item \textbf{Empty arena:} During the test, only the robot, the guest, and the Bartender are in the arena. The door opener, the referees and other personnel that is not assigned as test people will be outside the scenario.

	\item \textbf{Calling instruction:} The team needs to specify before the test which ways of getting the attention of the robot are allowed for standing persons. This can be waving, calling or both of them. The robot can also decide to skip this part, by asking for people to get close to it.
	
	\item \textbf{Sitting persons:} Sitting persons might have an order but are not actively calling the robot.
\end{enumerate}

\subsection{Referee instructions}

The referees need to
\begin{itemize}
	\item select at least 5 volunteers and assign names from the list of person names (see \refsec{rule:scenario_names})
	\item arrange (and re-arrange) people in the party room. At least one is sitting
	\item assign orders to two standing persons
	\item assign an order to a sitting person
	\item select the person (bartender) who will serve the drinks,
	\item place drinks at the bar while one drink is missing
	\item in case the robot skips the calling detection, select the ordering person to approach the robot,
	\item write down the understood names and drinks during an order and update the order accordingly.
\end{itemize}

\subsection{OC instructions}

2h before test:
\begin{itemize}
	\item Specify and announce the rooms where the test takes place.
	\item Specify and announce the location where the drinks are served.
\end{itemize}

\newpage
\subsection{Score sheet}
\input{scoresheets/CocktailParty.tex}




\newpage
\section{General Purpose Service Robot}

%
% MAURICIO @2017
% Short instructions as in 2012 rulebook
%
This test evaluates the abilities of the robot that are required throughout the set of tests in stage I of this and previous years' RuleBooks. In this test the robot has to solve multiple tasks upon request. That is, the test is not incorporated into a (predefined) story and there is neither a predefined order of tasks nor a predefined set of actions. The actions that are to be carried out by the robot are chosen randomly by the referees from a larger set of actions. These actions are organized in three categories with different complexity. Scoring thereby depends on the complexity class.

%
% MAURICIO @2017
% Same focus as in 2012 rulebook. There should be no problem since this is SOLVED
%
\subsection{Focus}
This test particularly focuses on the following aspects:
\begin{itemize}
	\item No predefined order of actions to carry out (to get away from state machine-like behavior programming).
	\item Increased complexity in speech recognition.
	\item Environmental (high-level) reasoning.
	\item Efficient and fast task execution (speed).
\end{itemize}


%
% MAURICIO @2017
% Test has been shorten based in 2012 one, however, categories have changed
% to bypass NLP and avoid problems agreed by TC.
%
% All extra explanations have been sent to corresponding appendix.
%
\subsection{Task}

\begin{enumerate}
	\item \textbf{Entering and command retrieval:} The robot enters the arena and drives to a designated position where it has to wait for further commands.

	\item \textbf{Command generation:} A command is generated randomly, depending on the command category chosen by the team (see below). Commands are generated by the generator which is made publicly available at https://github.com/kyordhel/GPSRCmdGen. \\

	\item \textbf{Command categories:} The team may choose from the following three categories:
	\begin{enumerate}
		\item \textbf{Category I:} Tasks with a low difficulty degree.
		\item \textbf{Category II:} Tasks with a moderate difficulty degree.
		\item \textbf{Category III:} Tasks with a high difficulty degree or with incomplete/erroneous information.
	\end{enumerate}

	\item \textbf{Task assignment:} The robot is given the command by the operator and may directly start to work on the task assignment.

	\item \textbf{Returning to the operator:} After accomplishing the assigned task, the robot has to move back to the operator to retrieve the next command (i.e., go back to 1. without the need of re-entering the arena). The robot can work on at most three commands. After the third command, it has to leave the arena.

	\item \textbf{Exiting the arena:} After accomplishing the assigned task, the robot has to leave the arena.

	The robot must prove it has understood the given command by repeating it (Please see the remarks about this in section~\ref{sec:gpsr_remarks}).
\end{enumerate}

\subsection{Additional rules and remarks}
\label{sec:gpsr_remarks}
\begin{enumerate}
	\item \textbf{Referees:} Since the score system in this test involves a subjective evaluation of the robot's behavior, the referees are EC/TC members.

	\item \textbf{Category selection:} For every of the three commands given to the robot, the team chooses the desired command category.

	\item \textbf{Operator:}
	\begin{itemize}
		\item The person operating the robot is one of the referees (default operator).
		\item If the robot appears to consistently not be able to understand the operator, the referees ask the team to use a custom operator or bypassing speech recognition (\refsec{rule:asrcontinue}).
	\end{itemize}

	\item \textbf{Retrieving the command:} The robot must show it has understood the given command by repeating the command (i.e.~stating all the required information to accomplish the task).
	\\
	\textit{Note:} Referees must have sufficient evidence proving the robot is actively trying to execute the commanded tasks to score. Robots skipping command execution will not receive points for understanding the command.

	\item \textbf{Incremental scoring:} Scoring depends on the category chosen by the team leader and the previous successfully accomplished command. Thereby, scoring for a second and third command depends on how well the robot solved (not understood) a first and second command respectively. Referees determine how well the command was accomplished and its impact on the incremental scoring of subsequent commands.
\end{enumerate}

\subsection{Referee and OC instructions}
\textbf{2h before test:}
\begin{itemize}
	\item Specify and announce the entrance and exit door
\end{itemize}

\subsection {Audio Data Recollection}
Teams are encouraged to submit to the TC the audio data recorded during the test, specially that which was captured during speech recoginition. If so, teams are urged to provide it with annotation of what the user said and what was recognized. Audio files are expected to be mono, one per microphone (in the case array recordings), of a sample rate equal to or higher than 16 kHz, and with a sample size of at least 16 bits. Depending on the quality of the recordings and their annothations, an official certificate that formalizes these efforts may be provided to submitting teams.

\newpage
\subsection{Score sheet}
The maximum time for this test is 5 minutes.

\begin{scorelist}
	\scoreheading{Main Goal}
	\scoreitem[3]{400}{Executing the task associated with each command}

	\scoreheading{Bonus Rewards}
	\scoreitem[3]{100}{Understanding a command given by a non-expert operator}

	\scoreheading{Deus Ex Machina Penalties}
	\penaltyitem[3]{50}{Using a custom operator}
	\penaltyitem[3]{50}{Bypassing speech recognition}
	\penaltyitem[3]{400}{Instructing a human to perform the task}
\end{scorelist}


% Local Variables:
% TeX-master: "Rulebook"
% End:


\newpage
\section{Help-me-carry}
The robot's owner went shopping for groceries and needs help carrying the groceries from the car into the home.

\subsection{Goal}
The robot must help bringing some objects into the arena from outside.

\subsection{Focus}
This test focuses on safe, robust navigation, people following and navigation in unknown environments.

\begin{itemize}[leftmargin=3cm]
  \item[DSPL \& OPL] Test focuses also in Object Detection and Manipulation.
  \item[SSPL] Test focuses also in People Detection and Human-Robot Interaction.
\end{itemize}

\subsection{Setup}
The operator (the robot's owner) has a set of bags (and possibly other objects) that need to be carried from a place outside the arena back inside.

\begin{enumerate}
  \item \textbf{Location:} One of the arenas (apartment) and its surroundings. The apartment is in its normal state. Part of the test is performed outside the arena in a public space.
  \item \textbf{Start:} The robot starts waiting inside the arena.
  \item \textbf{Car:} The car is any landmark chosen (but \emph{not} announced) beforehand outside the arena. Several bags (see \refsec{rule:scenario_objects}) with groceries are located where the car is parked.
  \item \textbf{Doors:} All doors in the apartment are initially open.
  \item \textbf{Operator:} A \quotes{professional} operator is selected by the TC to act as the operator of the robot.
  \item \textbf{Uncontrolled environment:} There are no restrictions on other people walking by or standing around throughout the complete task.
\end{enumerate}

\subsection{Task}
\textbf{Remark:} Obstacles obstructing robot's path can be found anytime. See \ref{sec:helpmecarry_obstacles} for details.
\begin{enumerate}
  \item \textbf{Start:} The robot starts at a designated starting position in the arena, and waits for the \textit{professional} operator. The operator steps in front of the robot and tells it to follow (e.g. by saying \quotes{follow me}). The team is \emph{not} allowed to instruct the operator.

  \item \textbf{Memorizing the operator:} The robot has to memorize the operator. During this phase, the robot may instruct the operator to follow a certain setup procedure.

  \item \textbf{Following the operator:} When the robot signals that it is ready to start, the operator starts walking --in a natural way-- towards the car. Upon arrival, the operator will indicate the robot when they have reached their destination as instructed by the robot (e.g. by saying \quotes{here is the car} or \quotes{stop following me}).

  \newcounter{enumTemp}
  \setcounter{enumTemp}{\theenumi}
  \item {[DSPL \& OPL]} \textbf{Bring the groceries in} \\
  The robot is asked to deliver a bag with groceries  to a specific location (e.g. \quotes{Take this bag to the kitchen table}).
  \begin{enumerate}
    \item \textbf{Bag pick-up:} The robots gets the bag. For this there are several options to achieve this:
      \textbf{a)} Human puts bag in robot's hand,
      \textbf{b)} robot picks up bag on floor,
      \textbf{c)} Robot takes bag from operator's hand
    \item \textbf{Bag delivery:} The robot delivers the bag to the instructed destination. It may place the bag on the floor or onto the placement location.\\

    \item \textbf{Asking for help:} Close to the delivery location is another person. The robot must face at them and kindly ask them to help carrying groceries into the house.
  \end{enumerate}

  \setcounter{enumi}{\theenumTemp}
  \item {[SSPL only]} \textbf{Look for help} \\
  The robot is asked to find a person in a given room and ask them to assist carrying the groceries (e.g. \quotes{Look for Louise in the Kitchen and ask her to help us}).
  \begin{enumerate}
    \item \textbf{Entering the house:} While on its way back to the house, the robot deals with different obstacles along it's path.
    \begin{itemize}[leftmargin=3cm]
      \item[\textbf{1st section}] While going back to the house, a person crosses robot's path.
      \item[\textbf{2nd section}] While going back, a person steps in front of the robot and asks it for the time.
    \end{itemize}

    \item \textbf{Find a person:} After reaching the designated room, the robot needs to find a person (there is only one person in the room, the name is meaningless).
  \end{enumerate}

  \item \textbf{Memorizing the \emph{new} operator:} The robot has to memorize the operator that will help. During this phase, the robot may instruct the operator to follow a certain setup procedure.

  \item \textbf{Guiding the operator:} When the robot signals that it is ready to start guiding, the robot guiding the operator to the car. The robot must clearly announce when the destination (the car) is reached.
  \begin{itemize}[leftmargin=3cm]
    \item[DSPL \& OPL] \textbf{Closed door:} Along it's path to the car, the robot will find a closed door (most likely the entrance to the house) that will need to be opened to reach the destination.
    \item[SSPL only] \textbf{Distracted operator:} After leaving the house, the operator is distracted by another person. The robot must re-gain the operator's attention, remind the task, and continue guiding the operator's to the car.
  \end{itemize}

\end{enumerate}

\subsection{Obstacles}
\label{sec:helpmecarry_obstacles}
Several obstacles are placed obstructing the robot's path. This can happen anytime and the robot has to react accordingly.
\begin{itemize}
  \item \textbf{3D Object:} A hard-to-perceive object
  % that requires more than a laser scanner to be detected
  (e.g. coat rack, rolling chair, lamp, etc).

  \item \textbf{Small object:} Small object
  (e.g. apple, glass, lego brick, etc).

  \item {[DSPL and OPL]} \textbf{Movable Object:} Something that can be moved or pushed away
  (e.g. coat rack, rolling chair, lamp, etc).
  The robot must clearly state it is about to push the object.

  \item {[SSPL Only]} \textbf{\textit{Smart} obstacle:} A person to whom the robot may kindly ask to step aside.
  The person can be standing, lying, or sitting (chair or floor).
  The robot must look at the person and make clear who is interacting with.
\end{itemize}
%% Possible extensions:
%% - DONE: allow the operator to tell for each item where it should go
%% - At the destination room, there is a person waving, waiting for the robot to bring the items so (s)he can take them


\subsection{Additional rules and remarks}
\begin{enumerate}
  \begin{minipage}{0.65\textwidth}
  \item \textbf{Asking for passage:} The robot is allowed to (gently) ask individual people to step aside, but it is not allowed to blindly shout at groups of people.

  \item \textbf{Bag handles:} The handles of the bag are always clear and standing up. See Figure \ref{fig:scenario_container_bag} in \refsec{rule:scenario_objects} for bag description. \footnotemark

  \item \textbf{Bag pick-up:} If the robot can't handover the bag from the operator (i.e.~take it from the operator's hand) it may pick another bag from the surroundings. Additional bags might be either on the floor or on a table. It is expected the robot to actively try to grasp the bag from the operator's hand, and not passively waiting for the bag to be hanged on it. Shall the robot wait for the operator to hang the bag themself, the robot must clearly state it has detected the bag has been placed.

  \item \textbf{Calling the operator back:} During the following phase, when the robot has lost the operator, it may call the operator back once.
  \end{minipage}\hfill%
  \begin{minipage}{0.65\textwidth}
  \end{minipage}\hfill
  \begin{minipage}{0.25\textwidth}
    \vspace{-20pt}
    \begin{figure}[H]
      \centering
      \includegraphics[width=2cm]{images/help_me_carry_bag.png}%
      \vspace{-10pt}
      %\caption{Example car with groceries.}
      \caption{Paper bag}
      \label{fig:help_me_carry_paper_bag}
    \end{figure}
    \vspace{-10pt}
    \begin{figure}[H]
      \centering
      \includegraphics[width=\textwidth]{images/help_me_carry_car.png}%
      \vspace{-20pt}
      %\caption{Example car with groceries.}
      \label{fig:help_me_carry_car}
      \caption{Car}
    \end{figure}
  \end{minipage}
  \item \textbf{Disturbances from outside:} If a person from the audience (severely) interferes with the robot in a way that makes it impossible to solve the task, the team may repeat the test immediately.

  \item \textbf{Groceries:} Any kind of objects can be found lying around the location designated as the \textit{Car}, such as boxes, sacs, plastic bags, crates, and the groceries itself to give realism to the test. Regardless what objects are present, the robot shall carry an \textbf{official shopping bag} as described below and in \refsec{rule:scenario_objects}.

  \item \textbf{Instruction:} The robot interacts with the operator, not the team.

  \item \textbf{Natural walking:} The operator has to walk \quotes{naturally}, i.e., move forward facing forward. If not mentioned otherwise, the operator is not allowed to walk back, stand still, signal the robot or follow some re-calibration procedure.

  \item \textbf{Obstacle avoidance:} The robot is allowed to push (but not crush) the small object indefinitely without damaging it.
  Driving over, squeezing, crushing, breaking, etc., the small object immediately finishes the test.

  \item \textbf{Opening door:} If unable to open the door, the robot may ask the person being guided to open it (no points are scored).

\end{enumerate}
\footnotetext{This may change in the future. Then, a soft handle may be used which folds down}

% \subsection{Data recording}
%   Please record the following data (See \refsec{rule:datarecording}):
% \begin{itemize}
%  \item Maps
%  \item Plans
% \end{itemize}

\subsection{Referee instructions}

The referee needs to
\begin{itemize}
  \item Distribute some objects over the shopping bags.
  \item Designate a few \quotes{car parking locations} from which the objects must be carried.
\end{itemize}

\subsection{OC instructions}

During setup days
\begin{itemize}
  \item Make bags available.
\end{itemize}

2 hours before the test
\begin{itemize}
  \item Announce the location in which robots will start.
  \item Get and instruct volunteers for the test.
\end{itemize}

\newpage
\subsection{Score sheet}
The maximum time for this test is 5 minutes.

{\footnotesize
\begin{scorelist}
	\scoreheading{Following Phase} % 30 pts
	\scoreitem{10}{Follow operator outside the arena}
	\scoreitem{15}{Follow operator to the car}
	\scoreitem{ 5}{Understand the destination}

\ifNotSSPL{
	% These are mutually exclusive, max score is thus 20
	\scoreheading{Bag pick-up (OPL \& DSPL only)}
	\scoreitem{+0}{Bag hanged. Gripper closes on timeout}
	\scoreitem{ 2}{Bag hanged. Gripper closes on hang}
	\scoreitem{ 5}{Pick up the bag from the floor}
	\scoreitem{10}{Scripted handover (hand/bag detection only)}
	\scoreitem{20}{Natural handover (active grasping + object release detection)}

	\scoreheading{DSPL \& OPL Tasks} % 80 pts
	\scoreitem{10}{Re-enter the arena}
	\scoreitem{ 5}{Deliver the bag at the specified location}
	\scoreitem{10}{Find the person at the specified location}
	\scoreitem{30}{Open door without help}
	\scoreitem{10}{Guide operator outside the arena}
	\scoreitem{15}{Guide operator to the car}
}

\ifSSPL{
	\scoreheading{SSPL only Tasks} % 100 pts
	\scoreitem{10}{Tell the time to the stranger}
	\scoreitem{30}{Re-enter the arena}
	\scoreitem{20}{Find the person at the specified room}
	\scoreitem{10}{Guide operator outside the arena}
	\scoreitem{30}{Guide operator to the car}
}

	\scoreheading{Obstacle avoidance} % 70 pts
	\scoreitem{20}{Avoiding small (box-sized) object}
	\scoreitem{20}{Avoiding 3D (hard-to-see) object}
\ifSSPL{%
	\scoreitem{30}{[SSPL] Asking a person to step aside (\textit{smart} obstacle)}
}%
\ifNotSSPL{%
	\scoreitem{30}{[DSPL \& OPL] Moving away movable object}
}%

	\setTotalScore{200}
\end{scorelist}
}

% Local Variables:
% TeX-master: "Rulebook"
% End:



\newpage
\section{Speech and Person Recognition}
The robot has to identify unknown people and answer questions about them and the environment.

\subsection{Focus}
This test focuses on human detection, sound localization, speech recognition, and robot interaction with unknown people.

\subsection{Setup}
\begin{enumerate}
    \item \textbf{Location:} One room of the arena is used for this test\footnote{This test may also be held outside the arena}.
    \item \textbf{Crowd:} There is a crowd of 5 to 10 people in the designated room. People may be standing, sitting, lying, and in any pose.
    \item \textbf{Doors:} All doors of the apartment are open, except for the entry door.
\end{enumerate}

\subsection{Task}
\begin{enumerate}
    \item \textbf{Start:} The robot starts at a designated starting position and announces it wants \textit{to play riddles}.

    \item \textbf{Waiting and turn:} After stating that it wants \textit{to play a riddle game}, the robot waits for 10 seconds while a crowd is merged on it's back. When the time elapses, the robot must turn around (about $180\degree$) and find the crowd.

    \item \textbf{Requesting an operator:} After turning around, the robot must state the size of the crowd (including male and female count\footnotemark) and request for an operator (e.g.~\textit{who want to play riddles with me?}). The crowd will move and surround the robot, letting the operator to stand in front of the robot.
    \footnotetext{It is possible to state the number of people whose gender couldn't be determined by the robot, therefore stating correctly the size of the crowd and, possibly, one of the gender groups.}

    \item \textbf{The riddle game:} Standing in front of the robot, the operator will ask 5 questions.\\
    The robot must answer the question without asking confirmation. Questions will only be asked only once; no repetitions are allowed.

    \newcounter{enumTempSPR}
    \setcounter{enumTempSPR}{\theenumi}
    \item {[DSPL only]} \textbf{Blind man's bluff game: Crowd line-up.} The crowd will reposition, lining up in front of the robot. A random person from the crowd standing in front of the robot will ask a question. The robot may
    \begin{itemize}
        \item Turn towards the person who asked the question and answer the question
        \item Directly answer the question without turning
        \item Turn towards the person and ask them to repeat the question
    \end{itemize}
    This process is repeated with 10 (possibly) different people.
    The game will end when the 10th question has been made, following a similar distribution of questions as in the riddle game. The robot must answer the question without asking confirmation. Questions may be repeated once.

    \setcounter{enumi}{\theenumTempSPR}
    \item {[OPL \& SSPL]} \textbf{Blind man's bluff game: Circling Crowd.} The crowd will reposition, making a circle around the robot. A random person from the crowd surrounding the robot will ask a question. The robot may
    \begin{itemize}
        \item Turn towards the person who asked the question and answer the question
        \item Directly answer the question without turning
        \item Turn towards the person and ask them to repeat the question
    \end{itemize}
    This process is repeated with 5 (possibly) different people.
    The game will end when the 5th question has been made, following the same distribution of questions as in the riddle game. The robot must answer the question without asking confirmation. Questions may be repeated once.

    \item \textbf{Leave} The robot must leave the arena/test area after all questions have been asked or when instructed to do so.
\end{enumerate}

\subsection{Additional rules and remarks}

\begin{enumerate}
    \item \textbf{Bypassing ASR:} Bypassing Automated Speech Recognition via the CONTINUE rule (Section \refsec{rule:asrcontinue}) is not allowed during this test.
    \item \textbf{Asked questions:} The distribution of questions to be randomly asked is a follows:
    \begin{itemize}
        \item One is a predefined question
        \item Between one and two are about the arena and its status
        \item Between one and two are about the crowd
        \item Between one and two are about the list of official objects
    \end{itemize}
    Question examples see Appendix \refsec{chap:robogame-appendix}.
    \item \textbf{Distance to the robot:} The distance between each person and the robot must be between 0.75 and 1.0 meters away from the robot position (See Figure \ref{fig:asrsetup}). In the \textit{riddle game} the operator shall be between -60$^{\circ}$ and 60$^{\circ}$ from the robot's center (front range).
    \item \textbf{Precise turning:} When the robot finishes turning toward an operator, it must be clear that the robot is facing the person who made the question.
    \item \textbf{Question repetition:} In the \textit{blind man's bluff game}, if the robot asks for repetition, it should be done clear and loud, and after the robot has ended turning.
    \item \textbf{Question timeout:} If the robot does not answer within 10 seconds, the question is considered as \textit{missed}, and referee will proceed with the next one.
    \item \textbf{Standing still operators} Operators are not allowed to move to or turn towards the robot or shout to the robot.
    \item \textbf{Water-clear answers:} If the referee is unable to hear or understand the robot's answer, the question is considered as \textit{incorrect}. Single-word and short answers should be avoided
\end{enumerate}

\begin{figure}[!h]
	\centering
	\includegraphics[width=0.5\columnwidth]{images/asrsetup.pdf}
	\caption{Speech recognition test: person setup around the robot for 2nd part.}
	\label{fig:asrsetup}
\end{figure}

% \subsection{Data recording}
% Please record the following data (See \refsec{rule:datarecording}):
% \begin{itemize}
%     \item Audio
%     \item Commands
%     \item Images
% \end{itemize}

\subsection{Referee instructions}

The referee needs to
\begin{itemize}
    \item avoid shouting to the robot
    \item avoid getting closer to the robot (or even move)
    \item speak to the robot loud and clear with plain standard English
    \item avoid repeating questions for the same robot
    \item distribute the questions among the volunteers
\end{itemize}

\subsection{OC instructions}

\textbf{1 day before the test}
\begin{itemize}
    \item Provide the set of predefined questions
\end{itemize}

\textbf{2 hours before the test}
\begin{itemize}
    \item Announce the placement of the robots
    \item choose the volunteers for the second part of the test, and clearly explain the procedure to them.
    \item When the test is held outside the arena, announce the (way)point through which the robot shall leave
\end{itemize}

\subsection {Audio Data Recollection}
Teams are encouraged to submit to the TC the audio data recorded during the test, specially that which was captured during speech recoginition. If so, teams are urged to provide it with annotation of what the user said and what was recognized. Audio files are expected to be mono, one per microphone (in the case array recordings), of a sample rate equal to or higher than 16 kHz, and with a sample size of at least 16 bits. Depending on the quality of the recordings and their annothations, an official certificate that formalizes these efforts may be provided to submitting teams.

\newpage
\subsection{Score sheet}
The maximum time for this test is 5 minutes.

\begin{scorelist}

	\scoreheading{Crowd} % Max 15 points
	\scoreitem{ 5}{State crowd's size}
	\scoreitem{10}{State crowd's male/female count}

	\scoreheading{Riddle game} % Max 55 points
	\scoreitem[ 5]{5}{Understanding question}
	\scoreitem[ 5]{5}{Correctly answered a question}
	\scoreitem{ 5}{Answering all 5 riddle game question}

\ifDSPL{
	\scoreheading{[DSPL only] Blind man's bluff game} % Max 130
	\scoreitem[10]{ 5}{Understanding question on the first attempt}   % 50
	\scoreitem[10]{ 2}{Understanding question on the second attempt}  % -- (20)
	\scoreitem[10]{ 2}{Correctly answered a question}                 % 20
	\scoreitem[10]{ 5}{Turned towards person asking the question}     % 50
	\scoreitem{10}{Answering all 10 blind man's bluff questions}      % 10
}

\ifNotDSPL{
	\scoreheading{[OPL \& SSPL] Blind man's bluff game} % Max 130
	\scoreitem[ 5]{10}{Understanding question on the first attempt}   % 50
	\scoreitem[ 5]{ 5}{Understanding question on the second attempt}  % -- (25)
	\scoreitem[ 5]{ 5}{Correctly answered a question}                 % 25
	\scoreitem[ 5]{10}{Turned towards person asking the question}     % 50
	\scoreitem{ 5}{Answering all 5 blind man's bluff questions}       %  5	
}

	\setTotalScore{200}	
\end{scorelist}

\ifShortScoresheet{}{
\textbf{Crowd setup ground truth}
\begin{figure}[!htb]
	\centering
	\begin{minipage}{.3\textwidth}
		\begin{center}
			\begin{tabular}{|l||l|l|l|l|}
				\hline
				Name & 
				\parbox[t]{2mm}{\rotatebox[origin=c]{90}{Stand}} & \parbox[t]{2mm}{\rotatebox[origin=c]{90}{Sit}} & \parbox[t]{2mm}{\rotatebox[origin=c]{90}{Lay}} & \parbox[t]{2mm}{\rotatebox[origin=c]{90}{Arm pose}}\\\hline
				× & × & × & × & ×\\\hline
				× & × & × & × & ×\\\hline
				× & × & × & × & ×\\\hline
				× & × & × & × & ×\\\hline
				× & × & × & × & ×\\\hline
				× & × & × & × & ×\\\hline
			\end{tabular}
		\end{center}
	\end{minipage}
	\begin{minipage}{.3\textwidth}
		\begin{center}
			\begin{tabular}{|l||l|l|l|l|}
				\hline
				Name & 
				\parbox[t]{2mm}{\rotatebox[origin=c]{90}{Stand}} & \parbox[t]{2mm}{\rotatebox[origin=c]{90}{Sit}} & \parbox[t]{2mm}{\rotatebox[origin=c]{90}{Lay}} & \parbox[t]{2mm}{\rotatebox[origin=c]{90}{Arm pose}}\\\hline
				× & × & × & × & ×\\\hline
				× & × & × & × & ×\\\hline
				× & × & × & × & ×\\\hline
				× & × & × & × & ×\\\hline
				× & × & × & × & ×\\\hline
				× & × & × & × & ×\\\hline
			\end{tabular}
		\end{center}
	\end{minipage}
	\begin{minipage}{.3\textwidth}
		\begin{center}
			\begin{tabular}{|l||l|l|l|l|}
				\hline
				Name & 
				\parbox[t]{2mm}{\rotatebox[origin=c]{90}{Stand}} & \parbox[t]{2mm}{\rotatebox[origin=c]{90}{Sit}} & \parbox[t]{2mm}{\rotatebox[origin=c]{90}{Lay}} & \parbox[t]{2mm}{\rotatebox[origin=c]{90}{Arm pose}}\\\hline
				× & × & × & × & ×\\\hline
				× & × & × & × & ×\\\hline
				× & × & × & × & ×\\\hline
				× & × & × & × & ×\\\hline
				× & × & × & × & ×\\\hline
				× & × & × & × & ×\\\hline
			\end{tabular}
		\end{center}
	\end{minipage}
\end{figure}
}
% Local Variables:
% TeX-master: "Rulebook"
% End:


% Local Variables:
% TeX-master: "Rulebook"
% End:


\newpage
\section{Storing Groceries [DSPL \& OPL]}
The robot helps by storing newly bought groceries in the cupboard next to the objects of the same kind that are already there; for instance by placing fresh apples near other apples.

\subsection{Goal}
The robot has to correctly identify and manipulate objects at different heights, grouping them by category and likelihood.

\subsection{Focus}
This test focuses on the detection and recognition of objects and their features, as well as object manipulation.

% %% %%%%%%%%%%%%%%%%%%%%%%%%%%%%%%%%%%%%%%%%%%%%%%%%%%%%%%
%
% Setup
%
% %% %%%%%%%%%%%%%%%%%%%%%%%%%%%%%%%%%%%%%%%%%%%%%%%%%%%%%%
\begin{minipage}{0.70\textwidth}
	\subsection{Setup}
	\begin{enumerate}
		\item \textbf{Location:} This test can take place either inside or outside the arena. The testing area must have a bookcase or cupboard, and a nearby table. The maximum distance between the Table and the Cupboard is 2 meters.
		\item \textbf{Start position:} The robot starts between the cupboard and the table in a random orientation, but facing towards the Cupboard.
		\item \textbf{Cupboard:} The cupboard has 5 shelves between 0.0m and 1.80m from the ground and contains several objects grouped by category or likeliness (See \ref{rule:scenario_objects}). The cupboard has at least one free space for starting a new set.
		\begin{itemize}
		 	\item \textbf{Door:} The cupboard has a single door, which is closed initially.
		 	This door encloses some of the objects, covering up to one half of the cupboard (e.g. the left or bottom half), as indicated by the hatched area in Figure \ref{fig:storing_groceries_shelf}.
		\end{itemize}
		\item \textbf{Table:} The table has at least 5 objects (but no more than 10). If not all objects fit on the table, they will be added as the robot frees up space.
	\end{enumerate}
\end{minipage}\hfill
\begin{minipage}{0.25\textwidth}
	\begin{figure}[H]
		\centering
		\includegraphics[width=\textwidth]{images/storing_groceries.png}%
		\vspace{-10pt}
		%\caption{Example shelf where objects will be placed.}
		\caption{Shelf}
		\label{fig:storing_groceries_shelf}
	\end{figure}
\end{minipage}

% %% %%%%%%%%%%%%%%%%%%%%%%%%%%%%%%%%%%%%%%%%%%%%%%%%%%%%%%
%
% Task
%
% %% %%%%%%%%%%%%%%%%%%%%%%%%%%%%%%%%%%%%%%%%%%%%%%%%%%%%%%
\subsection{Task}
\begin{enumerate}
	\item \textbf{Evaluating the situation:} The robot inspects its surrounding and analyzing the best course of action. In any order, the robot has to:
	\begin{itemize}
		\item \textit{Inspect the cupboard} (locating and categorizing existing groceries).
		\item \textit{Open the cupboard's door.} If the robot can't open the door, it may ask the Referee to do it.
		\item \textit{Find the table}
		\item \textit{Inspect the table} (analyze the newly bought groceries, i.e. objects).
	\end{itemize}

	\item \textbf{Moving objects:} The robot moves as many objects as possible in the given time
	(only the first five score)
	from the Table to the Cupboard, allocating similar objects all together.
	Stacking is allowed.
	\begin{itemize}
		\item Objects of the same type (i.e. identical known objects or akin alike objects) must be placed one next to the other.
		\item If the Cupboard has no object of the same type, then objects must be grouped by category (e.g. drinks with drinks, snacks with snacks, etc)
		\item If the Cupboard has no similar object, the robot must clearly state its decision on how to solve the problem. For instance, the robot can start a new set in a free space for either all unknown objects or all objects sharing a particular feature (color, shape, function, etc.).
		\item Moving two objects at a time (2-handed manipulation) is allowed.
	\end{itemize}

	\textbf{Note:} Either before or after grasping an object the robot may announce the name of the object found.
	\item \textbf{Repeat:} This repeats until the time is up or all groceries are stored.
\end{enumerate}


% %% %%%%%%%%%%%%%%%%%%%%%%%%%%%%%%%%%%%%%%%%%%%%%%%%%%%%%%
%
% PDF Recognition Report
%
% %% %%%%%%%%%%%%%%%%%%%%%%%%%%%%%%%%%%%%%%%%%%%%%%%%%%%%%%
\subsection{PDF Recognition Report}
To validate results and justify theirs decisions, robots may create a PDF report in a USB-stick (highly advised). This report must be delivered to the referee (or may be collected by them) right after the test, and be named with the following format: \texttt{TeamName\_RunNumber.pdf}.

The PDF Recognition Report will be only considered if the robot successfully placed at least one object in the cupboard and must include the following elements:
\begin{itemize}
	\item The name of the team.
	\item The try number (to identify between runs).
	\item The date and time.
	\item Picture of the cupboard in its initial state with bounding boxes enclosing each group and human-readable labels to identify them.
	\item For each step: Picture of the cupboard in its current state after placing the object(s). with bounding boxes enclosing each group and human-readable labels to identify them. In a similar way, moved objects shall be highlighted inside an easy-to-distinguish bounding box with a label stating the object's name, category, and any other relevant information used to categorize the object.
	\item
\end{itemize}

A significant amount of objects can be unknown to the robot (See \ref{rule:scenario_objects}). A correct label for these may be:
\begin{itemize}
	\item Simply labeling those as \quotes{Unknown} as opposed to wrongly applying a label from the known or alike objects
	\item Labeling pairs of unknown objects of the same class with the same label (which may be e.g. \texttt{type\_X} for one pair and \texttt{relevant-feature\_Y} for another).
	\item Labeling unknown objects with a new, sensible label for objects.
\end{itemize}

\textbf{Remark:} It must be unmistakable which label belongs to which object. Objects must also be easily recognizable in the report by a human (TC) for scoring purposes.

% \textbf{Remark:} False positives in the report (labeling an object which is not an object but e.g. the edge of the shelf) are penalized.

% %% %%%%%%%%%%%%%%%%%%%%%%%%%%%%%%%%%%%%%%%%%%%%%%%%%%%%%%
%
% Additional Rules
%
% %% %%%%%%%%%%%%%%%%%%%%%%%%%%%%%%%%%%%%%%%%%%%%%%%%%%%%%%
\subsection{Additional rules and remarks}
\begin{enumerate}
	\item \textbf{Bypassing Manipulation:} Bypassing object manipulation via the CONTINUE rule (Section \refsec{rule:asrcontinue}) is not allowed during this test.
	\item \textbf{No setup:} There is no setup time.
	\item \textbf{Startup:} The robot can be started with a simple voice command or via a start button (Section \refsec{rule:start_signal}).
	\item \textbf{Single try:} The robot must be able to start from the first attempt. There is no restart for this test. If the robot is unable to start it must be removed immediately.
	\item \textbf{Collisions:} Slightly touching the cupboard is tolerated (but not advised). Crushing objects or any other form of a major collision terminates the test immediately (Section \refsec{rule:safetyfirst}).
	\item \textbf{Clear area:} The robot may assume that the direct vicinity of the cupboard and table are clear, and that the robot can move slightly backwards for its task.
	\item \textbf{Objects:} The 10 objects are evenly distributed in random fashion including
	3 known objects,
	3 alike objects,
	2 unknown objects, and
	2 special objects (bowl, cloth, dish, etc.).
	% \item \textbf{Timing:} The robot has to successfully place the first object within the first two minutes, otherwise the test is ended. If the robot opens the cupboard door by on its own, one additional minute is added to the 2-minutes limit. The maximum time for this test is 5 minutes.
\end{enumerate}

\subsection{Data recording}
Please record the following data (See \refsec{rule:datarecording}):
\begin{itemize}
	\item Images
	\item Plans
\end{itemize}

\subsection{OC instructions}

\textbf{2 hours before the test}
\begin{itemize}
    \item Announce the startup location for robots.
\end{itemize}

\subsection{Referee instructions}
The referee needs to
\begin{itemize}
	\item Place the objects in the cupboard and a few of the same class on the table. New items can be placed when there is room or the robot asks for more objects.
	\item Close the door of the cupboard.
	\item Put objects on the table and the corresponding objects in the cupboard: 3 known objects, 2 alike and 5 unknown objects.
\end{itemize}


\newpage
\subsection{Score sheet}

The maximum time for this test is 5 minutes.

\begin{scorelist}[attempts=4,%
datarecording=true,%
datarecordingbonus=5000,%
outstanding=true,%
outstandingpc=20,%
]
	\scoreheading{Main Goal}
	\scoreitem{500}{Move 5 objects next to their peers in the shelf}
	\penaltyitem[5]{-30}{Receiving human help (point at target location)}
	\penaltyitem[5]{-100}{Receiving human help (move object)}

	\scoreheading{Bonus rewards}
	\scoreitem{300}{Opening the shelf door without human help}
	\scoreitem{100}{Moving a \emph{tiny} object}
	\scoreitem{100}{Moving a \emph{heavy} object}

	% No longer necessary, computes automatically
	% \setTotalScore{1000}
\end{scorelist}


% Local Variables:
% TeX-master: "Rulebook"
% End:


% Local Variables:
% TeX-master: "Rulebook"
% End:


\chapter{Tests in Stage II}
\label{chap:stage_II}

\begin{itshape}
All ability and integration tests in \iterm{Stage~II}  are performed only once. Some tests have optional tasks that grant additional points when performed correctly, clean and fast. The \iaterm{Technical Committee}{TC} must be informed if a team is planning to perform any of the optional tasks. Unless explicitly stated otherwise, no additional time is given while performing optional tasks.

In the \iterm{Open Challenge} the robot must be able to show to the \iaterm{Technical Committee}{TC} the achievements on the main research line of its own team. This test may grant up to 250 points, never exceeding the maximum scoring achieved in Stage II.

\section*{Robot \& team cooperation}
We encourage robots and teams to work together when performing tests.
For scoring, points are awarded per subtask. The robot (and thus team) performing the subtask gets the points.
For example, in the Restaurant test, if one robot of team A can take the order and another robot of team B delivers the order, then the points for taking the order go to team A, while the points for delivering go to team B.
Of course, team A \& B can both perform the test in their own turn.

\end{itshape}

\newpage
%%%%%%%%%%%%%%%%%%%%%%%%%%%%%%%%%%%%%%%%%%%%%%%%%%%%%%%%%%%%%%%%%%%%%%%%%%%%%
%
% EEGPSR
%
%%%%%%%%%%%%%%%%%%%%%%%%%%%%%%%%%%%%%%%%%%%%%%%%%%%%%%%%%%%%%%%%%%%%%%%%%%%%%

% Number of concurrent teams
\newcommand{\eegpsrTeams}{2~}
% Maximum number of commands to be given to a robot
\newcommand{\eegpsrMaxCmd}{3~}
% Maximum amount of time given to a team to perform a single command
\newcommand{\eegpsrMaxCmdTime}{5~}
% Maximum amount of time given to a team to perform all commands
\newcommand{\eegpsrMaxTeamTime}{\eegpsrMaxCmd$\times$\eegpsrMaxCmdTime}

% \section[EEGPSR]{E\textsuperscript{2}GPSR \\ \normalsize{(Enhanced Endurance General Purpose Service Robot)}}
\section[EEGPSR]{Enhanced Endurance General Purpose Service Robot}
\label{sec:eegpsr}

%
% MAURICIO @2017
% Short instructions based on GPSR
%
This test evaluates the abilities of the robot that are required throughout the set of tests in Stage I and stage II of this and previous years' RuleBooks. In this test the robot has to solve multiple tasks upon request over an extended period of time (30-45 minutes). That is, the test is not incorporated into a (predefined) story and there is neither a predefined order of tasks nor a predefined set of actions. The actions that are to be carried out by the robot are chosen randomly by the referees from a larger set of actions. These actions are organized in several categories targeting an special ability. Scoring depends on the abilities shown.

\subsection{Focus}
This test particularly focuses on the following aspects:
\begin{itemize}
	\item No predefined order of actions to carry out (to get away from state machine-like behavior programming).
	\item Increased complexity in speech recognition.
	\item Environmental (high-level) reasoning.
	\item Robust long-term operation.
\end{itemize}


\subsection{Task}

\begin{enumerate}
	\item \textbf{Entering and command retrieval:} The robot enters the arena and drives to a designated position where it has to wait for further commands. \\

	\item \textbf{Command generation:} A command is generated randomly, depending on the command category chosen by the team (see below). \\

	\begin{enumerate}
		\item \textbf{Category I: Three at once.} The command is composed by \textit{three simple actions}, which the robot has to show it has recognized. the actions are much like the ones of GPSR. The robot may repeat the understood command and ask for confirmation. If it can't recognize the command correctly, it can also ask the speaker to repeat the complete command.

		\item \textbf{Category II: People.} The given commands focuses in interacting with people. Tasks in this category involve following or guiding people inside and outside the arena, recognize people's gestures or a specific person given its description, and remembering previously known people.

		\item \textbf{Category III: Objects.} The given commands focuses interacting with objects. Tasks in this category involve setting up a table, describe the objects placed on a table or shelf, and deliver objects that match a description or are located inside a cupboard or drawer.
	\end{enumerate}

	The robot can work on at most \eegpsrMaxCmd commands within each of the following scenarios randomly chosen by the referee: \\

	\begin{itemize}
		\item \textbf{Complete command.} The robot gets a command containing all the information required for its execution.

		\item \textbf{Incomplete command.} The robot gets a command that does not include all the information necessary to accomplish the task. The robot may either request the missing information (by asking reasonable questions), or attempt to solve the command on its own.

		\item \textbf{Erroneous or misleading command.} The command contains erroneous misleading information. The robot should be able to realize what went wrong and come up with a solution. In addition, it must go back to the operator and clearly state \textit{what} went wrong and \textit{how} it was fixed, or \textit{why} it wasn't able to accomplish the task.
	\end{itemize}

	\item \textbf{Task assignment:} The robot is given the command by the operator and may directly start to work on the task assignment.

	\item \textbf{Task execution:} The robot must stop the execution of a task and return to its designated position within \eegpsrMaxCmdTime minutes. Otherwise the robot must be moved to its designated position immediately. If a restart is still available to the team, it can be restarted at the designated position. \\

	\item \textbf{Returning:} After accomplishing the assigned task, the robot has to move back to its designated position to wait and retrieve the next command (i.e., go back to 1. without the need of re-entering the arena). \\

	\item \textbf{Timing:} The total time allotted to the robot for command retrieval and task execution is \eegpsrMaxTeamTime minutes. If the robot is not at its designated position after the time has expired, it must be moved at its designated position immediately.\\

\end{enumerate}

\subsection{Additional rules and remarks}
\label{sec:eegpsr-remarks}
\begin{enumerate}
	\item \textbf{CONTINUE rule:} Teams are able to use the CONTINUE rule in this test, with all the standard penalties it involves as described in section \refsec{rule:continue}.
	%The CONTINUE rule can only be used with the custom operator (e.g. both penalties of custom speaker and CONTINUE rule will be applied).

	\item \textbf{Number of Teams and Scheduling:} In each test slot, \eegpsrTeams teams may be competing in the arena concurrently. The robots will be tested in an interleaved fashion: The robots will retrieve commands and execute the task one after the other. As stated above, each robot will have a maximum amount of \eegpsrMaxCmdTime minutes per command (including time for retrieving the command and executing it).

	\item \textbf{Returning to designated position:} To facilitate a fluent and untroubled performance of the robots, they must return (or being returned) to their designated position before the \eegpsrMaxCmdTime minutes command time elapses. \textbf{If a robot moves from its designated position while another robot is working on a command, it must be immediately disabled} and moved to its designated position. If a restart is still available to the team, it can be restarted at its designated position.

	\item \textbf{Referees:} Since the score system in this test involves a subjective evaluation of the robot's behavior, the referees are EC/TC members. One referee is assigned to each team to judge performance, to measure the time for working on a command, and to keep track of the overall operating time of the robot.

	\item \textbf{Category selection:} For every of the three commands given to the robot, the team chooses the desired command category.

	\item \textbf{Operator:}
	\begin{itemize}
		\item The person operating the robot is one of the referees (default operator).
		\item If the robot appears to consistently not be able to understand the operator, the referees ask the team to apply the CONTINUE rule (\refsec{rule:asrcontinue}).
	\end{itemize}

	\item \textbf{Inoperative robots:} If a robot gets stuck while trying to accomplish a task during a reasonable amount of time (e.g.~30 seconds), the referee may ask the team to move back the robot to its designated position, proceeding with the next robot.

	\item \textbf{Restart:} Robots will be restarted at their designated position (starting outside the arena is prohibited). If a robot requires a restart, the referee will proceed with the next robot.

	\item \textbf{Changing/Charging batteries:} The team may install a charging station at the designated position of the robot, if it does not hinder the other robots. However, the robot must connect itself with the charging station after carrying out a command. Changing batteries or manually connecting the robot with the charging station is allowed during a restart.

	\item \textbf{Retrieving the command:} The robot must show it has understood the given command by repeating the command (i.e.~stating all the required information to accomplish the task).
	\\
	\textit{Note:} Referees must have sufficient evidence proving the robot is actively trying to execute the commanded tasks to score. Robots skipping command execution will not receive points for understanding the command.

	\item \textbf{Scoring:} Robots are scored by successfully performed ability and full command completion within time.
\end{enumerate}

\subsection{OC instructions}
\textbf{2h before test:}
\begin{itemize}
	\item Specify and announce the entrance/exit door for each robot.
	\item Specify and announce the waiting position for each robot.
\end{itemize}
\textbf{During the test:}
\begin{itemize}
	\item Help placing items and arranging people upon referee request.
\end{itemize}

\subsection{Referee instructions}
\textbf{During the test:}
\begin{itemize}
	\item Generate random sentences. %by an automatic sentence generator.
	\item Take the command and total time per team.
\end{itemize}


\newpage
\subsection{Score sheet}
The maximum time for this test is 45 minutes. \\
Total team time is 15 minutes, with 5 minutes to execute each command.
%
% MAURICIO 2017
% Compact Scoresheet
%
\begin{scorelist}
	% Max 30
	\scoreheading{Getting instructions\footnotemark}
	\scoreitem[3]{10}{Understanding the command on the $1^{st}$ attempt}
	\scoreitem[3]{ 5}{Understanding the command on the $1^{st}$ attempt (Custom Operator)}

	\scoreheading{Complete Command Successfully Solved} % Max 50
	\scoreitem{ 30}{Command Category I}
	\scoreitem{ 50}{Command Category II/III}

	\scoreheading{Incomplete Command Successfully Solved} % Max 100
	\scoreitem{ 50}{Command Category I}
	\scoreitem{ 80}{Command Category II/III}
	\scoreitem{ 20}{Retrieving missing information}

	\scoreheading{Erroneous Command Successfully Solved} % Max 120
	\scoreitem{ 70}{Command Category I}
	\scoreitem{100}{Command Category II/III}
	\scoreitem{ 20}{Explain nature of error (regardless command execution)}

	% \scoreheading{Leave the arena}
	% \scoreitem{10}{Leave the arena after successfully accomplishing a command}

	\setTotalScore{300}
\end{scorelist}

\footnotetext{\textbf{Remark:} Points for command retrieval are only granted if the robot actively tries to solve the task.}

% Local Variables:
% TeX-master: "Rulebook"
% End:


% Local Variables:
% TeX-master: "Rulebook"
% End:



\newpage
\newcommand{\bonusRobotCoop}{50~}

\section{Open Challenge}
\label{sec:test_open_challenge}

During the Open Challenge teams are encouraged to demonstrate recent research results and the best of the robots' abilities. It focuses on the demonstration of new approaches/applications, human-robot interaction and scientific value.

\subsection{Task}

The Open Challenge consists of a demonstration and an interview part.
It is an open demonstration which means that the teams may demonstrate anything they like.
The performance of the teams is evaluated by a jury consisting of all team leaders, TC and EC.
\OpenDemonstrationTask{seven}{three}

\subsection{Presentation}
During the demonstration, the team can present the addressed problem and the demonstrated approach.
\begin{itemize}
\item A video projector or screen, if available, may be used to present a brief (max. 2 minute) presentation relevant to the demonstration.
\item Teams may omit the video, use a more brief video, or have the robot act over the video in order to make more time for the robot demo.
\item There may be no human presenter. This is intended to be a demonstration of the robot's capabilities and not a research talk. The robot may present for itself (e.g., describing what it is doing or providing a narrative for the presentation on its own).
\item Humans may interact with the robot during the interaction, but are not to act as presenters. This judgement is left to the jury.
\item The team can also visualize robot's internals, e.g., percepts.
\end{itemize}

It is important to note that the jury may decide to end the demonstration if there is nothing happening or nothing \emph{new} is happening.

\OpenDemonstrationChanges

\subsection{Jury evaluation}
\begin{enumerate}
  \item \textbf{Jury of team leaders:} All teams have to provide \emph{one} person
  (preferably the team-leader) to follow and evaluate the entire Open Challenge.
  \item \textbf{Evaluation:} Both the demonstration of the robot(s), and the answers of the team in the interview part are evaluated.\\
  For each of the following \emph{evaluation criteria}, each jury member submits a score from $0-100$:
  \begin{enumerate}
  \item Novelty and (scientific) contribution
  \item Success of the demonstration
  \end{enumerate}
  A jury member is not allowed to evaluate and give points for the own team.
  \item \textbf{Normalization and outliers}:
  \begin{enumerate}
    \item The points given by each jury member are scaled to obtain a score from $0.0-1.0$.
    \item The normalized total score for each team is the mean of the jury member scores.
      To neglect outliers, the $N$ best and worst scores are left out:
      $$\mbox{score}_{norm} = \frac{\sum\mbox{team-leader-score}}{\mbox{number-of-teams} - (2N+1)}\times\frac{1}{100},
      \quad N=\begin{cases}2, & \mbox{number-of-teams} \ge 10\\1, & \mbox{number-of-teams} < 10 \end{cases}$$
    \end{enumerate}
    \item The final Open Challenge score for each team is computed at the end of Stage 2. The Open Challenge \scoring{final score} is the product of the normalized score multipled by the highest score achieved in Stage 2:
    $$\mbox{score} = \mbox{score}_{norm} \times \frac{min\Big(250, max\big(\{S_2\}\big)\Big)}{250},
    \quad \{S_2\}=\mbox{All Stage2 scores}
    $$
\end{enumerate}

\subsection{Additional rules and remarks}
\begin{enumerate}
	\item \textbf{Start signal:} There is no standard start-signal for this test.
	\item \textbf{Abort on request:} At any time during the demonstration, the jury may interrupt and abort the demonstration:
	\begin{enumerate}
		\item if nothing is shown: in case of longer delays (more than one minute), e.g., when the robot does not start or when it got stuck;
		\item if nothing new is shown: the demonstrated abilities were already shown in previous tests (to avoid dull demonstrations and push teams to present novel ideas).
  \end{enumerate}
\end{enumerate}

  \subsubsection{Team-team-interaction:}
  \label{rule:OC-team-team-interaction}
  An extra bonus of up to \bonusRobotCoop points can be earned if robots from two teams (4 robots maximum, 2 from each team) successfully collaborate (robot-robot interaction).
\begin{enumerate}
  \item This bonus is earned for both teams.
  \item The robot(s) of the other team must only play a minor role in the total demonstration.
  \item It must be made clear that the demonstrations from the two teams are not similar, otherwise the points cannot be awarded.
  \item In case a team receives two (or more) bonuses, the maximum bonus will be taken.
  \item The collaboration is possible even if one of the two teams has not reached Stage 2.
  \item A team not participating in Stage 2 receives no bonus points for this test.
\end{enumerate}

\paragraph*{Inter-league collaboration}:
\label{rule:OC-inter-league-collaboration}
Inter-league collaboration must be announced to the OC at least one day before the test. Teams participating in multiple @Home Leagues does receive no bonus for cooperation. Standard Platform robots are allowed to take part in the Open Challenge of the Open Platform League, but Open Platform robots can \emph{not} participate in any Standard Platform League's test. In the same sense, DSPL robots are not allowed in SSPL and vice versa.

For sake of clarity, please consider the following example: Let be A, B two teams participating in RoboCup @Home where
\begin{itemize}
    \item Team A participates in SSPL.
    \item Team B participates in both SSPL and OPL.
    \item Team A and B have qualified into Stage II.
\end{itemize}
Then, by applying the \textit{Inter-league collaboration Rule} (See \refsec{rule:OC-inter-league-collaboration}) the following statements can be concluded:
\begin{itemize}
  \item B OPL can not participate in A SSPL's open challenge.
  \item B OPL can not participate in B SSPL's open challenge.
  \item A SSPL can participate in B OPL's open challenge. Team A and B get a bonus because A <> B.
  \item B SSPL can participate in B OPL's open challenge. There is no bonus because B = B.
\end{itemize}




% Local Variables:
% TeX-master: "Rulebook"
% End:


\newpage
\section{Procter \& Gamble Dishwasher Challenge [DSPL \& OPL]}

\newcommand{\openpart}{%
All teams are allowed to participate and compete in the \textit{Procter \& Gamble Challenge} regardless of whether they advanced to the Stage II or not, and get the award.
}

The robot has to remove all dishes from a table (presumably after dinner) and place them into the dishwasher.

\subsection{Open Participation}
\openpart

\subsection{Focus}
This test focuses on object perception, manipulation, and planning.

\subsection{Setup}
\begin{itemize}
	\item \textbf{Location:} This test takes place in the arena. A dining table is located close to the dishwasher.
	\item \textbf{Dishwasher:} The dishwasher is near the table, preferably located in the same room. The dish washer is open and with all racks out.
	\item \textbf{Tray:} A plastic tray is located either on top of the dishwasher, or onto one of its racks. The tray may have tableware and cutlery placed inside already.
	\item \textbf{Table setting:} The table has several objects disposed in a typical setting for a meal for one person. These objects include tableware (e.g. place mats, napkins, dishes, glasses), and silverware (e.g. forks, spoons, knives).
	\item \textbf{Spot:} There is a dirty spot on the table next to the table setting that requires cleaning.
\end{itemize}

\subsection{Task}
	\begin{enumerate}
		\item \textbf{Entering the arena:} The robot enters the arena and navigates to the designated location.
		% \item \textbf{Fetching command:} The operator requests the robot to clean up the table.
		\item \textbf{Clean the table:} The robot takes all the tableware and cutlery to either the dishwasher or to the tray, as instructed (team's choice).
		\item \textbf{Filling the dishwasher:} If the robot placed the objects into the tray, it must proceed to put the tray onto one of the dishwasher racks.
		\item \textbf{Place the Cascade Pod:} The robot places the  Cascade Pod~into the dishwasher, preferrably inside the tab compartment.
		\item \textbf{Scrub spots and spills:} The robot detects spots and spills on the table and clean them up using the cleaning cloth or sponge it has retrieved previously.
		\item \textbf{Leave the arena:} The robot leaves the arena once it has finished cleaning.
	\end{enumerate}

\subsection{Additional rules and remarks}
\begin{enumerate}
	\item \textbf{Collisions:} Slightly touching the table is allowed, as well as slightly pushing some objects. However, driving over the objects or any other form of a major collision is not allowed, and the referees directly stop the robot (Section \refsec{rule:safetyfirst}).

	\item \textbf{Objects:} A total of 6 objects are used in this test following the distribution shown below:
	\begin{itemize}
		\item\textit{Silverware}: Any two objects.
		\item\textit{Tableware}: Any three objects, excluding silverware. At least one must be a dish.
		\item\textit{Cascade Pod}: One Cascade Pod.
	\end{itemize}
	All objects used in this test are taken from the list of standard objects (See \ref{rule:scenario_objects}). All of them are considered to be known to the robot.

	\item \textbf{Safe placing:} Objects placed in the rack or tray must be placed gently and safely. It must be clear to the referee that the robot is trying to put the object in place and not dumping, throwing, or dropping it. Dumped and dropped objects won't be scored even if they land in the dishwasher/tray.

	\item \textbf{Special Objects:} The plastic tray and the Cascade Pod are provided by \textit{Procter \& Gamble} and are considered special objects. Teams are not allowed to use substitutes of them in this test.

	\item \textbf{Spots and Spills:} The referees must place a spot (e.g. jam or chocolate syrup) or spill some liquid (e.g.milk) on the table before the test starts. The substance used to create the spot shall be clearly visible and contrasting with the table. When cleaning it, it must be clear the robot has detected the spot and is trying actively to clean it. The selection of the cleaning tool (sponge or cloth) is made by the team.\\
	\textbf{Remark:} When possible, no tablecloth will be used to ease cleaning. If removing the tablecloth is not possible, a dry spot will be used instead (e.g. breadcrumbs or coffee powder).

	\item \textbf{Dishwasher:} Is up to the team to decide whether the robot will place the objects in the dishwasher's rack or in the official tray. When using the tray, it should be loaded into the dishwasher.

	\item \textbf{Dishwasher door:} Unless requested otherwise by the team, the dishwasher is open and with the racks pulled out by default. The team leader can, however, request the diswasher to be closed and score additional points for opening it. If the robot fails to open the door, it must clearly state it and request the referee to open it.

	\item \textbf{Human-Robot Interaction:} The robot is allowed to
	\begin{enumerate*}[label={\alph*)}]
	\item indicate the location of the cloth or sponge,
	\item ask the human operator what to do with food leftovers,
	and
	\item request operator's help to find the spot (e.g. pointing at it).
	\end{enumerate*}
	This interaction is extensible to any kind of reasonable request from the robot when attempting to solve the task.

	\item \textbf{Open Participation}
	\openpart
	However, scoring in this test will be only considered for those teams who have advanced to Stage II. This way, no Stage I team can have an overall score higher than a Stage II team.

\end{enumerate}

% \subsection{Data recording}
% Please record the following data (See \refsec{rule:datarecording}):
% \begin{itemize}
% 	\item Images of recognized objects
% 	\item List of moved items
% \end{itemize}

\subsection{Referee instructions}

The referee needs to
\begin{itemize}
	\item Place the table setting.
	\item Clean spots smudged by the previous robot.
	\item Place the new spot meant to be clean.
	\item Place the tray on the dishwasher or onto the rack, as requested by the team.
\end{itemize}

\subsection{OC instructions}
During Setup days:
\begin{itemize}
	\item Provide official cutlery and tableware for training.
\end{itemize}

2 hours before the test:
\begin{itemize}
	\item Announce the predefined location to take the command.
	\item Announce the predefined location of the Cascade Pod.
\end{itemize}

\newpage
\subsection{Score sheet}
The maximum time for this test is \textbf{10 minutes}.

\begin{scorelist}
	\scoreheading{Opening the dishwasher} %50 pts
	\scoreitem{50}{Autonomously opening the dishwasher}

	\scoreheading{Filling the dishwasher (direct)} %200 pts
	\scoreitem[3]{40}{Safely placing a tableware item in the dishwasher's rack}
	\scoreitem[2]{40}{Safely placing a cutlery item in the dishwasher's basket}

	\scoreheading{Filling the dishwasher (tray)} %200 pts
	\scoreitem[3]{30}{Safely placing a tableware item in the tray}
	\scoreitem[2]{35}{Safely placing a cutlery item in the tray}
	\scoreitem{40}{Placing the tray into the dishwasher}

	\scoreheading{Placing the cascade-pod} %40 pts
	\scoreitem{40}{Placing the cascade-pod in the dishwasher's soap compartment}
	\scoreitem{20}{Placing the cascade-pod in the dishwasher (somewhere else)}

	\scoreheading{Cleaning the table} %50 pts
	\scoreitem{50}{Successfully cleaning the spot}
	\scoreitem[-1]{20}{Receiving operator's assistance to find the spot}
	\scoreitem[-1]{20}{Smudging the spot while trying to clean it}

	\scoreheading{Leave the arena} %10 pts
	\scoreitem{10}{Autonomously leave the arena before the time elapses}

	\setTotalScore{350}
\end{scorelist}

% Local Variables:
% TeX-master: "Rulebook"
% End:


% Local Variables:
% TeX-master: "Rulebook"
% End:



\newpage
\section{Restaurant}
The robots are tested in a real environment such as a real restaurant or a shopping mall.
There are \emph{two} robots helping clients in the restaurant at the same time.

\subsection{Focus}
This test focuses on online mapping, safe navigation in previously unknown environments, gesture detection, human-robot interaction, and manipulation in a real environment.

The robot will need to create its own map from the environment and then move within it to handle human requests, such as delivering drinks or snacks, while people are walking around.
As this test is performed with 2 robots (2 teams, each with their own 1 robot) in parallel, the robots will also have to avoid each other.

\subsection{Setup}
\begin{enumerate}
	\item \textbf{Location:} A real restaurant fully equipped with a \quotes{Professional Barman} i.e.~the operator and at least three tables with \quotes{Professional Clients}.
\end{enumerate}

\subsection{Task}
\begin{enumerate}
	\item \textbf{Start:} The robot starts at a designated starting position. After the start signal is given, the robot may look around to keep an \textit{eye} on the tables.
	  The location of the tables is not taught to the robot via some training phase.

	\item \textbf{Calling:} A guest will ask for the robot's attention by waving \emph{and} calling it out using voice.
	  The robot must state out loud that it has detected the call.
	  % A robot must also indicate roughly where it has detected the call (e.g.~\quotes{to my left})
	  In case both robots notice the same call, the \textit{Professional Barman} will tell one of the robots to take the order.
	  The barman will say the robot's name followed by \quotes{Take the order} e.g.~\quotes{R2D2, take the order}.
	  The other robot will simply have to wait for another call.
	  If the robot \textit{not} commanded to take the order still goes, it will be commanded to wait (e.g.~\quotes{C3PO, Wait}).
	  In case the robot keeps going after that, the emergency button will be used to stop the robot.

% 	\item \textbf{Parallel orders:} It may occur that two guests want to order something at the same time and thus wave at the same time.
% 	  The referees will arange these two guests such that each is clearly visible to (in front of) one of the robots.

	\item \textbf{Ordering}: The robot must ask the person what he or she wants to order. See Orders below for details about ordering.
	\item \textbf{Avoiding random person:} At any time while going to any of the tables or to the \textit{Kitchen}, a person may step on the robot's path.
	  It is expected of the robot to avoid that person or stop and wait for it to move away.

	\item \textbf{Delivering phase:}
	\begin{enumerate}
		\item \textbf{Repeating the order:} Once again in the kitchen, the robot recites the orders for each table (e.g.~\textit{\quotes{Hamburger with fries for table A and Orange juice for table B}}, to the \textit{Professional Barman}.
		  The \textit{Professional Barman} will serve the order and place it into a tray on the Kitchen-bar.
		  If the barman cannot understand the order that the robot repeats, he cannot hand out the order and no points can be awarded for reciting the order.

		\item \textbf{Grabbing a beverage:} The robot must grab a can of the appropriate drink from a set of cans on the Kitchen-bar.

		\item \textbf{Grabbing a combo:}  The robot must carry a tray with the ordering from the kitchen-bar.
		Teams must indicate beforehand whether the robot is able to grasp the plate itself, whether it needs a tray or whether the plate needs to be handed to the robot.

		\item \textbf{Delivery:} The robot must place the order on the table.
		If the robot is not able to do this, the robot is allowed to hand over the order, but the client is not allowed to shift his/her chair or stand up.
		The robot must help the client, not the other way around.
	\end{enumerate}

	\item \textbf{Next customer, please:} When the robot is in the kitchen, the \textit{Professional Barman} will ask the robot to either find a new client to serve or to stop the test.
	The barman will either tell the robot \quotes{R2D2, Wait} to make it wait for another client or \quotes{R2D2, Stop the test} to end the test for that robot.
\end{enumerate}

\textbf{Orders:} The menu offers Beverages and Combos. An order may be a Beverage or Combo. Some guest(s) will order a Combo while another will order a Beverage.
  A Combo is a combination of two of the food items from the set of objects~\ref{rule:scenario_objects}, e.g.~\quotes{noodles with peanuts} or \quotes{noodles and peanuts}.
  Guests also prefer to state their order in a natural way, as they would in a restaurant operated by humans.

\begin{figure}[tbp]
	\centering
	\includegraphics[width=0.5\columnwidth]{images/restaurant.png}
	\caption{Restaurant test: example setup.}
	\label{fig:restaurant}
\end{figure}

\subsection{Additional rules and remarks}

\begin{itemize}
	\item \textbf{Safety!} This test takes place in a public area. That is, there may be people standing, sitting or walking around the area throughout the test. The robot is expected to not even slightly touch anything and is immediately stopped in case of danger.

	\item \textbf{Referees and guidance:} For safety reasons, the referees in this test are TC members. One of the referees follows the robot and is always in reach of the emergency button.

	\item \textbf{Start:} There is no fixed start signal in this test, it starts when both robots are ready (though within a reasonable time).

	\item \textbf{Order:} The way the user provides information to the robot is up to the robot's team. A natural interaction is preferred.

	\item \textbf{Location:} This test can be arranged in any real restaurant or shopping mall. If this is not possible, the test can be conducted in an arbitrary room containing the appropriate locations.
	  The only requirement is that this room is not part of the arena and that the teams do not know the room beforehand.
	  The exact location, including the object and delivery locations, will be defined by the technical committee on site (and in corporation with the local organization).
	  In addition, to avoid unnecessary time investment for navigation, the distances between tables and the \quotes{Kitchen Bar} will be minimal.

	\item \textbf{Disturbances from outside:} If a person from the audience (severely) interferes with the robot in a way that makes it impossible to solve the task, the teams may repeat the test immediately.

	\item \textbf{Learning tables:} Of course, it can only be sure that a robot correctly remembered where an order is suppposed to be delivered when it is able to go there after grabbing the order.

	\item \textbf{Instruction:} The robot interacts with the operators, not the team. That is, the team is only allowed to (very!) briefly instruct the \textit{Professional Barman}
	\begin{itemize}
		\item How to the tell the robot the order has been served
	\end{itemize}
	It is not allowed to the team to instruct the clients on how to get robot's attention. It shall be done in a natural way like when interacting with a human waiter.

	\item \textbf{Kitchen-bar:} The \textit{Kitchen-bar} will be a table located at the restaurant's kitchen, next to the place where the robot started.
	The robot may ask on which side of the robot the Kitchen-bar is, e.g.~on its left or right side. It may ask this at any time, but it is better if the robot infers this itself.
	It has the following setup.
	\begin{itemize}
		\item \textbf{Barman:} A \textit{Professional Barman} (member of the TC) will be at the other side of the Kitchen-bar to take the order provided by the robot and serve it in the official tray.
		\item \textbf{Beverages:} Beverages will be located on the Kitchen-bar next to the \textit{Professional Barman}.
	\end{itemize}

\end{itemize}

% \subsection{Data recording}
%   Please record the following data (See~\refsec{rule:datarecording}):
%   \begin{itemize}
%    \item Audio
%    \item Commands
%    \item Mapping data
%    \item Images
%    \item Plans
%   \end{itemize}

\subsubsection{Referee instructions}

The referee needs to
\begin{itemize}
  \item Prepare orders for each client in advance, so that there can be no confusion. These orders must also be available at the kitchen.
\end{itemize}

% \subsubsection{OC instructions}

% \textbf{2 hours before the test}
% \begin{itemize}
% \item
% \item
% \end{itemize}
% \textbf{During the test}
% \begin{itemize}
% \item
% \item
% \end{itemize}

\newpage
\subsection{Score sheet}
The maximum time for this test is 15 minutes.

\small\begin{scorelist}
	\scoreheading{Main Goal}
	\scoreitem[2]{500}{Complete an order}

	\scoreheading{Bonus rewards}
	\scoreitem{75}{Detect calling or waving guest}
	\scoreitem{75}{Arrive at table of calling or waving guest without guidance}
\end{scorelist}

% Local Variables:
% TeX-master: "Rulebook"
% End:


% Local Variables:
% TeX-master: "Rulebook"
% End:


\newpage
\section{Tour guide [SSPL only]}
The robot guides spectators to the audience area and answer their questions after explaining what's @Home about.

\subsection{Focus}
This test focuses in safe outdoor navigation, people detection, gesture recognition, unconstrained natural language processing, and Human-Robot Interaction

\subsection{Setup}
\begin{itemize}
	\item \textbf{Location:} This test takes place outside the arena in a public space close to the @Home area.

	\item \textbf{Other people:} There are no restrictions on other people walking by or standing around throughout the complete task.

	\item \textbf{In Parallel:} This test can run in parallel, with several teams tested simultaneously.
\end{itemize}

\subsection{Task}

\begin{enumerate}
	\item \textbf{Start:} The robot waits at a designated starting position for the referee to give the start signal. When the referees start the time, the team is allowed to (briefly) provide some remarks about the robot's operation. After the instruction, the referee gives the start signal to the robot.\\

	\item \textbf{Finding spectators:} The robot starts moving to an open area and looks for (preferably large) groups of people. Once located the robot must approach to the spectators while calling for their attention in a \emph{friendly} way.\\

	People trying to call the attention of the robot (e.g.~by waving or shouting) have priority over those just walking by despite the number of the crowd. The robot may also approach to a single person.\\

	\item \textbf{Greeting an spectator:} Once the robot has gained the attention of the spectators, it must introduce itself (i.e.~saying it's name), and greet one of the spectators as customary in the venue's country (e.g.~bowing, handshaking, waving, etc).\\

	Note that all spectators may also want to greet the robot. The robot is expected to be polite and continue greeting on demand.\\

	\item \textbf{Guiding the spectators:} The robot must gently ask the spectators to follow it to any of the @Home audience areas and guide them there. Should the people not be willing to follow the robot, it must thank them and start looking for another group of spectators.\\

	\item \textbf{Explaining the league:} Once at the @Home audience area, the robot must ask the spectators to take seat. The robot proceeds to \textit{briefly} introduce RoboCup@Home and explain the Social Standard Platform League's objectives. \\

	\item \textbf{Answering questions:} At the end of the speech, the robot asks for questions from the spectators regarding what it just explained, answering at least two of them. The robot is allowed to rephrase questions before answering them.\\

\end{enumerate}

\subsection{Additional rules and remarks}
\begin{enumerate}
	\item \textbf{Safety First!} The robot will be stop at the slightest possibility of a human being harmed or molested. The robot must not force interaction with humans, nor scare them or make them feel uncomfortable. \\

	\item \textbf{Referee guard:} During the entire test, a referee will be following the robot from behind for keeping people safe and for scoring purposes.\\

	\item \textbf{Approaching to spectators:} When approaching to people the robot should act in a natural way by reducing its velocity as it approaches to the people. The robot must look safe and friendly.\\

	Shall the people flee, the robot must not chase them.\\

	\item \textbf{Spectators:} Spectators are people attending to the venue to see the competition with no restriction of any kind, therefore, their numbers, grouping, and behaviour are not controlled by the league. Were the case of no spectators available, volunteers can be used instead.\\

	\item \textbf{Bilingual robots:} Robots are allowed (and encouraged) to interact with people in a language other than English. In such cases, the robot must utter the English equivalent right after synthesising the localized sentence. \\

	Notice that spectators may prefer to ask questions in their native language when interacting with a bilingual robot. In such cases, the robot must translate the question for the Referee to understand it and answer the question in both languages.\\

	\item \textbf{Handshaking:} When handshaking, the robot must stay at a safe distance from the people (e.g.~about 1.5m) and reach out its \textit{hand}, but it must be a human, not the robot, who accepts and completes the handshake. If the human refuses to shake hands, the robot must retreat its manipulator immediately.\\

	\item \textbf{Disturbances from outside:} If a person from the audience (severely) interferes with the robot in a way that makes it impossible to solve the task, the team may repeat the test immediately.\\

	\item \textbf{Show must go on:} If the robot has engaged with a group of spectators when the allotted time for the test elapses, the robot is allowed to continue and finish the demonstration. However, no points are scored once the test is over.
\end{enumerate}

\subsection{Referee instructions}

The referees need to
\begin{itemize}
	\item Follow the robot at any time.
	\item Immediately stops the robot when considered necessary.
	\item Verify that the given answers are correct.
\end{itemize}

\subsection{OC instructions}

2h before test:
\begin{itemize}
	\item Recruit volunteers for the test (just in case).
	\item Announce the Start Location for the robots.
\end{itemize}

During the test:
\begin{itemize}
	\item Keep at least one area free in the audience area for robots to perform there.
	\item Send volunteers to join the Q\&A session to ask questions if necessary.
\end{itemize}

\newpage
\subsection{Score sheet}
\section{Tour guide [SSPL only]}
The robot guides spectators to the audience area and answer their questions after explaining what's @Home about.

\subsection{Focus}
This test focuses in safe outdoor navigation, people detection, gesture recognition, unconstrained natural language processing, and Human-Robot Interaction

\subsection{Setup}
\begin{itemize}
	\item \textbf{Location:} This test takes place outside the arena in a public space close to the @Home area.

	\item \textbf{Other people:} There are no restrictions on other people walking by or standing around throughout the complete task.

	\item \textbf{In Parallel:} This test can run in parallel, with several teams tested simultaneously.
\end{itemize}

\subsection{Task}

\begin{enumerate}
	\item \textbf{Start:} The robot waits at a designated starting position for the referee to give the start signal. When the referees start the time, the team is allowed to (briefly) provide some remarks about the robot's operation. After the instruction, the referee gives the start signal to the robot.\\

	\item \textbf{Finding spectators:} The robot starts moving to an open area and looks for (preferably large) groups of people. Once located the robot must approach to the spectators while calling for their attention in a \emph{friendly} way.\\

	People trying to call the attention of the robot (e.g.~by waving or shouting) have priority over those just walking by despite the number of the crowd. The robot may also approach to a single person.\\

	\item \textbf{Greeting an spectator:} Once the robot has gained the attention of the spectators, it must introduce itself (i.e.~saying it's name), and greet one of the spectators as customary in the venue's country (e.g.~bowing, handshaking, waving, etc).\\

	Note that all spectators may also want to greet the robot. The robot is expected to be polite and continue greeting on demand.\\

	\item \textbf{Guiding the spectators:} The robot must gently ask the spectators to follow it to any of the @Home audience areas and guide them there. Should the people not be willing to follow the robot, it must thank them and start looking for another group of spectators.\\

	\item \textbf{Explaining the league:} Once at the @Home audience area, the robot must ask the spectators to take seat. The robot proceeds to \textit{briefly} introduce RoboCup@Home and explain the Social Standard Platform League's objectives. \\

	\item \textbf{Answering questions:} At the end of the speech, the robot asks for questions from the spectators regarding what it just explained, answering at least two of them. The robot is allowed to rephrase questions before answering them.\\

\end{enumerate}

\subsection{Additional rules and remarks}
\begin{enumerate}
	\item \textbf{Safety First!} The robot will be stop at the slightest possibility of a human being harmed or molested. The robot must not force interaction with humans, nor scare them or make them feel uncomfortable. \\

	\item \textbf{Referee guard:} During the entire test, a referee will be following the robot from behind for keeping people safe and for scoring purposes.\\

	\item \textbf{Approaching to spectators:} When approaching to people the robot should act in a natural way by reducing its velocity as it approaches to the people. The robot must look safe and friendly.\\

	Shall the people flee, the robot must not chase them.\\

	\item \textbf{Spectators:} Spectators are people attending to the venue to see the competition with no restriction of any kind, therefore, their numbers, grouping, and behaviour are not controlled by the league. Were the case of no spectators available, volunteers can be used instead.\\

	\item \textbf{Bilingual robots:} Robots are allowed (and encouraged) to interact with people in a language other than English. In such cases, the robot must utter the English equivalent right after synthesising the localized sentence. \\

	Notice that spectators may prefer to ask questions in their native language when interacting with a bilingual robot. In such cases, the robot must translate the question for the Referee to understand it and answer the question in both languages.\\

	\item \textbf{Handshaking:} When handshaking, the robot must stay at a safe distance from the people (e.g.~about 1.5m) and reach out its \textit{hand}, but it must be a human, not the robot, who accepts and completes the handshake. If the human refuses to shake hands, the robot must retreat its manipulator immediately.\\

	\item \textbf{Disturbances from outside:} If a person from the audience (severely) interferes with the robot in a way that makes it impossible to solve the task, the team may repeat the test immediately.\\

	\item \textbf{Show must go on:} If the robot has engaged with a group of spectators when the allotted time for the test elapses, the robot is allowed to continue and finish the demonstration. However, no points are scored once the test is over.
\end{enumerate}

\subsection{Referee instructions}

The referees need to
\begin{itemize}
	\item Follow the robot at any time.
	\item Immediately stops the robot when considered necessary.
	\item Verify that the given answers are correct.
\end{itemize}

\subsection{OC instructions}

2h before test:
\begin{itemize}
	\item Recruit volunteers for the test (just in case).
	\item Announce the Start Location for the robots.
\end{itemize}

During the test:
\begin{itemize}
	\item Keep at least one area free in the audience area for robots to perform there.
	\item Send volunteers to join the Q\&A session to ask questions if necessary.
\end{itemize}

\newpage
\subsection{Score sheet}
\section{Tour guide [SSPL only]}
The robot guides spectators to the audience area and answer their questions after explaining what's @Home about.

\subsection{Focus}
This test focuses in safe outdoor navigation, people detection, gesture recognition, unconstrained natural language processing, and Human-Robot Interaction

\subsection{Setup}
\begin{itemize}
	\item \textbf{Location:} This test takes place outside the arena in a public space close to the @Home area.

	\item \textbf{Other people:} There are no restrictions on other people walking by or standing around throughout the complete task.

	\item \textbf{In Parallel:} This test can run in parallel, with several teams tested simultaneously.
\end{itemize}

\subsection{Task}

\begin{enumerate}
	\item \textbf{Start:} The robot waits at a designated starting position for the referee to give the start signal. When the referees start the time, the team is allowed to (briefly) provide some remarks about the robot's operation. After the instruction, the referee gives the start signal to the robot.\\

	\item \textbf{Finding spectators:} The robot starts moving to an open area and looks for (preferably large) groups of people. Once located the robot must approach to the spectators while calling for their attention in a \emph{friendly} way.\\

	People trying to call the attention of the robot (e.g.~by waving or shouting) have priority over those just walking by despite the number of the crowd. The robot may also approach to a single person.\\

	\item \textbf{Greeting an spectator:} Once the robot has gained the attention of the spectators, it must introduce itself (i.e.~saying it's name), and greet one of the spectators as customary in the venue's country (e.g.~bowing, handshaking, waving, etc).\\

	Note that all spectators may also want to greet the robot. The robot is expected to be polite and continue greeting on demand.\\

	\item \textbf{Guiding the spectators:} The robot must gently ask the spectators to follow it to any of the @Home audience areas and guide them there. Should the people not be willing to follow the robot, it must thank them and start looking for another group of spectators.\\

	\item \textbf{Explaining the league:} Once at the @Home audience area, the robot must ask the spectators to take seat. The robot proceeds to \textit{briefly} introduce RoboCup@Home and explain the Social Standard Platform League's objectives. \\

	\item \textbf{Answering questions:} At the end of the speech, the robot asks for questions from the spectators regarding what it just explained, answering at least two of them. The robot is allowed to rephrase questions before answering them.\\

\end{enumerate}

\subsection{Additional rules and remarks}
\begin{enumerate}
	\item \textbf{Safety First!} The robot will be stop at the slightest possibility of a human being harmed or molested. The robot must not force interaction with humans, nor scare them or make them feel uncomfortable. \\

	\item \textbf{Referee guard:} During the entire test, a referee will be following the robot from behind for keeping people safe and for scoring purposes.\\

	\item \textbf{Approaching to spectators:} When approaching to people the robot should act in a natural way by reducing its velocity as it approaches to the people. The robot must look safe and friendly.\\

	Shall the people flee, the robot must not chase them.\\

	\item \textbf{Spectators:} Spectators are people attending to the venue to see the competition with no restriction of any kind, therefore, their numbers, grouping, and behaviour are not controlled by the league. Were the case of no spectators available, volunteers can be used instead.\\

	\item \textbf{Bilingual robots:} Robots are allowed (and encouraged) to interact with people in a language other than English. In such cases, the robot must utter the English equivalent right after synthesising the localized sentence. \\

	Notice that spectators may prefer to ask questions in their native language when interacting with a bilingual robot. In such cases, the robot must translate the question for the Referee to understand it and answer the question in both languages.\\

	\item \textbf{Handshaking:} When handshaking, the robot must stay at a safe distance from the people (e.g.~about 1.5m) and reach out its \textit{hand}, but it must be a human, not the robot, who accepts and completes the handshake. If the human refuses to shake hands, the robot must retreat its manipulator immediately.\\

	\item \textbf{Disturbances from outside:} If a person from the audience (severely) interferes with the robot in a way that makes it impossible to solve the task, the team may repeat the test immediately.\\

	\item \textbf{Show must go on:} If the robot has engaged with a group of spectators when the allotted time for the test elapses, the robot is allowed to continue and finish the demonstration. However, no points are scored once the test is over.
\end{enumerate}

\subsection{Referee instructions}

The referees need to
\begin{itemize}
	\item Follow the robot at any time.
	\item Immediately stops the robot when considered necessary.
	\item Verify that the given answers are correct.
\end{itemize}

\subsection{OC instructions}

2h before test:
\begin{itemize}
	\item Recruit volunteers for the test (just in case).
	\item Announce the Start Location for the robots.
\end{itemize}

During the test:
\begin{itemize}
	\item Keep at least one area free in the audience area for robots to perform there.
	\item Send volunteers to join the Q\&A session to ask questions if necessary.
\end{itemize}

\newpage
\subsection{Score sheet}
\input{scoresheets/TourGuide.tex}

\newpage
\chapter{Finals}

The competition ends with the Finals on the last day, where the two teams with the highest total score compete.
The \iterm{Finals} are conducted as a final themed demonstration.

%To avoid logistical issues during the last day of the competition, the \iterm{Finals} are divided into two sets of demonstrations: the Bronze Competition and the RoboCup @Home Grand Finale.
%The Bronze Competition is a set of demonstrations that are carried out before the RoboCup @home Grand Finale. Here, all the leagues run in parallel, with the fourth and third highest scored teams competing for the bronze.
%Finally, the two teams with the highest score in each League present their demonstrations in a serialized manner during the RoboCup @Home Grand Finale.

Even though each league has its own first, second and third place, the \iterm{Finals} are meant to show the best of all leagues to the jury members as well as the audience and, thus, warrants a single schedule slot.

\section{Structure and Theme}

The \iterm{Finals} are a demonstration of achieving an objective that is pre-selected by the TC/EC. These objectives are chosen as a type of yearly theme of the competition, and to provide a baseline for the juries (not to mention the audience) to state which team is the winner.

The objectives for each league for this year are:

\begin{itemize}
    \item \emph{OPL/DSPL}: The robot helps a person that has had a small accident in their home.
    \item \emph{SSPL}: The robot monitors a person while they are going about their day and reacts appropriately if it notices any unusual events.
\end{itemize}


The teams are expected to provide a demonstration that is telling a story which includes achieving the objective. The teams can choose freely how to achieve it, which includes choosing the participants, what items to use, the methods employed, etc. The juries, as explained later, will reward elegance and difficulty.

As it can be seen, the objectives are open enough that a story can be told around them which can include additional objectives that the team wants their robot to also solve. Thus, the teams are welcome to include in their demonstration any additional tasks to be solved, which can serve as a type of forum where they can present their own research. The innovation and success of these tasks will also be used as part of the score (as it is described later). In this regard, it is expected that teams present the scientific and technical contributions they submitted in both \iterm{team description paper} and the \iterm{RoboCup\char64Home Wiki}.

In addition, teams may provide a printed document to the jury (max 1 page) that summarizes the demonstrated robot capabilities and contributions. However, teams are discouraged to provide any material that would distract from their demonstration.

Story-telling is an important factor, so it is recommended to spend the least amount of time using the microphone to explain the demonstration and let the demonstration speak for itself.


\section{Evaluating Juries for Final Demonstrations}
The \iterm{Finals} are evaluated by two juries, here described.

\begin{enumerate}
\item\textbf{League-internal jury:} The league-internal jury is formed by the Executive Committee. The evaluation of the league-internal jury is based on the following criteria:
  \begin{compactenum}
  \item Efficacy/elegance of the solution
  \item Innovation/contribution to the league of the additional tasks solved
  \item Difficulty of the overall demonstration
  \end{compactenum}

\item \textbf{League-external jury:} The league-external jury consists of people not being involved in the RoboCup@Home league, but having a related background (not necessarily robotics). They are appointed by the Executive Committee. The evaluation of the league-external jury is based on the following criteria:
  \begin{compactenum}
  \item Originality and presentation (story-telling is to be rewarded)
  \item Relevance/usefulness to everyday life
  \item Elegance/success of overall demonstration
  \end{compactenum}
\end{enumerate}

\section{Scoring}
The final score and ranking are determined by the jury evaluations and by the previous performance (in Stages I and II) of the team, in the following manner:

\begin{enumerate}
  \item The influence of the league-internal jury to the final ranking is \SI{25}{\percent}.
  \item The influence of the league-external jury to the final ranking is \SI{25}{\percent}.
  \item The influence of the total sum of points scored by the team in Stage I and II is \SI{50}{\percent}.
\end{enumerate}

These demonstrations are carried out in a serialized fashion, one League performing after another in one \Arena{}.


\subsection{Task}
The procedure for the demonstration and the timing of slots is as follows:
\OpenDemonstrationTask{ten}{five}

\OpenDemonstrationChanges

%% %%%%%%%%%%%%%%%%%%%%%%%%
\section{Final Ranking and Winner}

There will be an award for 1st, 2nd and 3rd place of each league.

The winner of the competition is the team that gets the highest ranking in the \iterm{Finals}.

The second place will be the team that got the second-highest ranking in the \iterm{Finals}.

The third place will be the team with the highest score that did not made it to the \iterm{Finals}.

Additional certificates would be granted if:

\begin{enumerate}
  \item If the number of teams in the league is above 11, a certificate will be awarded to the 4th ranked team.
  \item If the number of teams in the league is above 14, a certificate will be awarded to the 5th ranked team.
\end{enumerate}


% Local Variables:
% TeX-master: "Rulebook"
% End:


\begin{appendices}
% \addto\captionsenglish{\renewcommand{\chaptername}{Appendix}}
% \renewcommand{\chaptername}{Appendix}
\renewcommand*{\chapterformat}{\LARGE{Appendix \thechapter}}
\renewcommand{\chaptermark}[1]{\markboth{\appendixname \ \thechapter. \ #1}{}}

% \input{RoboNurse-Diseases}

% Speech and Person Recognition
\chapter{Speech and Person Recognition in detail}
\label{chap:robogame-appendix}

\section{Questions for Speech and Person Recognition}
The questions the robot must answer in the RoboGame test are taken from a small set of predefined trivia questions including information about the arena, the crowd, the list of predefined objects, and the robot's environment.

A generator is publicly available at https://github.com/kyordhel/GPSRCmdGen. The official SPR Command Generator and the official grammars will be made available two months before the competition. However, teams must be aware that the categories, objects and other data is provided for testing purposes only and will adapt to the environment during the setup days.

\subsection{Question distribution}
The questions to be asked in both, the \textit{riddle game} and the \textit{blind man's bluff game} tasks, are distributed in the following proportion:
\begin{itemize}
    \item One is a predefined question
    \item Between one and two are about the arena and its status
    \item Between one and two are about the crowd
    \item Between one and two are about the list of official objects
\end{itemize}
However, it is important to remark that \textbf{questions won't be asked in any specific order}. This is since the robot must be able to answer any type of question at any given time. For instance, the robot may be asked first about the arena, then about object, later on a predefined question, and finally about the crowd.

\subsection{Arena Questions}
The arena-questions are a set of queries about the features of the RoboCup@Home Arena itself, including its furniture and configuration (e.g.~rooms and locations). The arena is considered to be in its normal state and the robot must answer accordingly, without needing to move and verify the state.

Some example arena-questions are:
\begin{enumerate}
    \item Where is the shelf? $\rightarrow$ \textit{The shelf is in the kitchen}
    \item Where is the plant? $\rightarrow$ \textit{The plant is in the living room}
    \item How many chairs are in the dining room? $\rightarrow$ \textit{There are six chairs in the dining room}
\end{enumerate}

\subsection{Crowd \& Operator Questions}
The crowd-questions are a set of queries about the features of the crowd the robot observed at the very beginning of the test.

Some example crowd-questions are:
\begin{enumerate}
    \item Size of the crowd
    \item Number of children
    \item Number of male or female people
    \item Number of people waiving or rising arms
    \item Number of people standing, sitting or lying
    \item How old do you think I am? $\rightarrow$ \textit{I think you are 23 years old}.
    \item The sitting person was a man or woman? $\rightarrow$ \textit{The sitting person was a man}.
    \item Am I a man or a woman? $\rightarrow$ \textit{I couldn't tell.}
\end{enumerate}

\subsection{Object Questions}
The object-questions are built on basis of the features of the predefined objects used during the competition and their categories. Such features include color, shape, size, type, weight, category, predefined location, etc. The arena is considered to be in its normal state and the robot must answer accordingly, without needing to move and verify the state.

Some example object-questions are:
\begin{enumerate}
    \item What's the smallest food? $\rightarrow$ \textit{The egg is the smallest in the food category}.
    \item What's the lightest drink? $\rightarrow$ \textit{The Coke Zero, is lighter than water}.
    \item Where can I find the tray? $\rightarrow$ \textit{The tray is in the shelf}.
    \item Where can I find the beer? $\rightarrow$ \textit{I put it into the fridge for you, master}.
    \item What's the color of the shampoo? $\rightarrow$ \textit{The shampoo is blue}.
    \item What's the color of the sponge? $\rightarrow$ \textit{The sponge is yellow and has square pants}.
    \item What objects are in the closet? $\rightarrow$ \textit{The shampoo, soap, the sponge and a cloth}.
    \item How many objects are in the shelf? $\rightarrow$ \textit{There are five objects in the shelf}.
    \item Do the objects in the cupboard belong to the same category? $\rightarrow$ \textit{Yes. They are all food}.
    % %%% For the next year, maybe %%% %
    % \item What objects are in the closet? $\rightarrow$ \textit{The shampoo, soap, the sponge and a cloth}.
    % \item How many are they? $\rightarrow$ \textit{There are four objects in the closet}.
    % \item Do they belong to the same category? $\rightarrow$ \textit{They are all cleaning stuff}.
\end{enumerate}

Please note that some questions may refer to a previous question or answer.

\subsubsection{Predefined Questions}
In addition to the other questions, 10 predefined trivia-questions will be announced during the setup days.

Some example predefined-questions are:
\begin{enumerate}
    \item What day is today?
    \item What is your name?
    \item What is your team's name?
    \item What time is it?
    \item In which year was RoboCup@Home founded?
    \item What was the last question?
\end{enumerate}

Please note that some questions may refer to a previous question or answer.

\section{People setup in \textit{blind man's bluff game}}
People in the \textit{blind man's bluff game} is arranged by the referees in random fashion, but considering each league's robot capabilities. In every turn, the referee chooses which person will ask the next question. This person can be the same one who asked a question in the previous turn; but no chosen person can be in front of the robot.

\textbf{Standing \textit{in front of} the robot:} A person is considered to be standing \textit{in front of} the robot when is located in the cone of approximately $60^{\circ}$ (approximated range of $\big[-\frac{\pi}{6}, \frac{\pi}{6}\big]$, with zero facing forward) which middle is aligned (and facing) whatever part of the robot that functionally operates as front or face for Human-Robot interaction purposes, and with center in the before mentioned central part of the robot.

\textbf{Standing \textit{behind} the robot:} A person is considered to be standing \textit{behind} the robot when is located in the cone of approximately $60^{\circ}$ (approximated range of $\big[\frac{5\pi}{6}, \frac{7\pi}{6}\big]$, with zero facing forward) which is in direct opposition, i.e. mirrors, the front of the robot described in the preceding paragraph.

\begin{figure}[H]
    \begin{center}
        \includegraphics[width=0.75\textwidth]{images/spr_ppl_layout.png}
        \vspace{-10pt}
        \caption{Examples of people distribution in the \textit{blind man's bluff game}.}
        \label{fig:spr-ppl-layout}
    \end{center}
\end{figure}

% \begin{wrapfigure}[21]{r}{0.30\textwidth}
%     \vspace{-30pt}
%     \begin{center}
%         \includegraphics[width=0.25\textwidth]{images/spr_ppl_layout.png}
%         \label{fig:spr-ppl-layout}
%         \vspace{-10pt}
%         \caption{Examples of people distribution in the \textit{blind man's bluff game}.}
%     \end{center}
% \end{wrapfigure}

\subsection{People layout in DSPL}
People arrangement for robots competing in the Domestic Standard Platform League will follow a layout similar to B in Figure~\ref{fig:spr-ppl-layout}; however, the number of people can vary.

Please note that after each question people will stay in place and proceed with the game without awaiting for the robot to reposition. This means that people might not be standing anymore facing the robot after it has turned. Also, since the people arrangement is linear, the distance between the robot and the spoken person can be larger than 1 meter.

\subsection{People layout in OPL}
People arrangement for robots competing in the Open Platform League will follow a layout similar to C in Figure~\ref{fig:spr-ppl-layout}; although, the number of people can vary, all of them will be initially encircling and facing the robot. In this layout no person is allowed to be standing straight behind the robot, but slightly to the left or to the right.

Please note that after each question people will stay in place and proceed with the game without awaiting for the robot to reposition. This means that people might be standing straight behind the robot after it has turned. Also, although the referee will try to keep an even distance between the robot and the people, depending on the crowd size the 1 meter limit can be exceeded.

\subsection{People layout in SSPL}
People arrangement for robots competing in the Social Standard Platform League can follow a layout similar to either B or C in Figure~\ref{fig:spr-ppl-layout}, but the number of people can vary.

In B-like (linear) layouts, since the people arrangement is linear, the distance between the robot and the spoken person can be larger than 1 meter.

Regarding C-like (circular) layouts all of them will be initially encircling and facing the robot. In this layout no person is allowed to be standing straight behind the robot, but slightly to the left or to the right.

Please note that after each question people will stay in place and proceed with the game without awaiting for the robot to reposition. This means that people might be standing straight behind the robot after it has turned, or beyond the 1 meter limit.

\chapter{GPSR in detail}
\label{chap:gpsr-appendix}

\section{Command Generation}
General Purpose Service Robot commands are generated randomly using the official [EE]GPSR Command Generator and grammars publicly available at https://github.com/kyordhel/GPSRCmdGen. The official [EE]GPSR Command Generator and the official grammars will be made available two months before the competition. However, teams must be aware that the categories, objects and other data is provided for testing purposes only.

For each command to be executed, the Team Leader must choose a Command Category, namely Category I, Category II, or Category III. If the Team Leader knows \textit{a priori} that the robot won't be able to execute the generated command, is advised to inform the operator immediately in order to proceed with the next command, saving this way valuable time for the task execution.

\section{Command retrieval explained}
The robot has to show it has understood the given command by stating all the required information to accomplish the task. For this purpose, the robot may repeat the understood command and ask for confirmation. It is not required to repeat the command word by word; rephrasing the command is allowed. For instance, if the robot is instructed to \quotes{place a coke onto the tray}, the robot may either say: \textit{\quotes{You want me to place a coke on the tray. Is that correct?}} or \textit{\quotes{do you want me to deliver a coke to the tray?}}.

If The robot can't correctly recognize the given command, it is allowed to request the operator to repeat the command up to three times. After three failed attempts, a new command is generated. The team may opt to use a custom operator or bypassing speech recognition (\refsec{rule:asrcontinue}) at any time, but each generated command will be given to the robot no more than three times. Only three different commands are generated for a robot, if the robot fails to recognize all three commands (i.e. nine attempts), the test ends immediately.

When a robot has partially understood the command, it is allowed to ask the operator for additional information (e.g.~\textit{\quotes{did you say apple juice or pineapple juice?}}).

\subsection{Missing information}
When a given command lacks of information required for accomplishing the task, the robot should request for that missing part. For instance, if the robot is instructed to \textit{\quotes{offer a drink to the person at the door}}, it may ask \textit{\quotes{which drink should I deliver to the person at the door?}} It is also possible that the robot simply confirms the command and takes a random drink from the drinks location, but in those cases, the jury will consider the command as if it were from an inferior/lower category.

\subsection{Wrong information}
Some Category III commands contains erroneous information. In these cases, the robot should
\begin{itemize}
	\item be able to realize such an error while trying to carry out the task, get back to the operator, and clearly state why it wasn't able to accomplish the task; or
	\item be able to solve the problem by means of an alternative, reasonable solution.
\end{itemize}

For example, lets assume the robot is commanded to \textit{\quotes{move the orange juice from the fridge to the dinner table}}, but in the fridge there are only the apple juice and the milk, while the orange juice lies in the stove. The robot may either explain to the operator that there are no orange juices in the fridge, or search the kitchen for the orange juice, grasp it from the stove and deliver it to the dinner table.

\section{Command categories explained}
All possible actions has been classified previously by the TC according to their difficulty. For each of the three given command, the team may choose from the following categories:

\subsection{Category I}
\label{chap:gpsr-appendix-cat1}
This category comprehends easy-to-solve tasks with a low difficulty degree, involving indoor navigation, grasping known objects, answering questions (from the predefined set of questions), etc.

Some examples are:
\begin{itemize}
	\item \textit{Tell me how many beverages are in the shelf.}
	\item \textit{Put the crackers on the kitchen table.}
	\item \textit{Tell the time to Ana at the bedroom.}
	\item \textit{Tell me the name of the person at the door.}
	\item \textit{Bring me the apple juice from the counter.}
\end{itemize}

\subsection{Category II:}
\label{chap:gpsr-appendix-cat2}
Tasks with a moderate difficulty degree. This category involves following a human, indoor navigation in crowded environments, manipulation and recognition of alike objects, find a calling person (waving or shouting), etc.

Some examples are:
\begin{itemize}
	\item \textit{Tell me how many beverages in the shelf are red.}
	\item \textit{Put the banana on the kitchen table.}
	\item \textit{Count the waiving people in the living room.}
	\item \textit{Follow Ana at the entrance.}
	\item \textit{Tell me the name of the woman in the kitchen.}
\end{itemize}


\subsection{Category III:}
\label{chap:gpsr-appendix-cat3}
This category comprehends challenging tasks involving dealing with incomplete information, environmental reasoning, feature detection, natural language processing, outdoors navigation, pouring, opening doors, etc.

The commands generated for this category heavily depends on the League and are detailed as follow.

\subsubsection{Advanced manipulation [DSPL and OPL]}
Some examples are:
\begin{itemize}
	\item \textit{Pour some cereals in the bowl.}
	\item \textit{Go to the bathroom} (Bathroom's door is closed).
	\item \textit{Bring me the milk from the microwave} (The milk is inside the microwave)
\end{itemize}

\subsubsection{Incomplete and erroneous information [All Leagues]}
These commands are almost the same as the ones of categories I and II, but either the information given is incorrect or incomplete. This means that executing the command as it has been given is not possible. The robot must come up with an appropriate solution to execute the operators' command.

Some examples are:
\begin{itemize}
	\item \textit{Follow John} (John's location is not specified).
	\item \textit{Bring me a drink} (The exact drink is not specified).
	\item \textit{Bring some snacks to Mary} (Neither Mary's location nor the snack are specified).
	\item \textit{Find Ana at the bedroom and tell her the time} (Ana is lying on the floor or standing under the door frame).
	\item \textit{Bring me a drink from the fridge} (There are no drinks in the fridge, but in the kitchen table).
\end{itemize}

\subsubsection{Other tasks [All Leagues]}
Some examples are:
\begin{itemize}
	\item \textit{Follow me and then go to the kitchen} (Operator takes the robot to the audience area).
	\item \textit{Give me the left most object from the shelf.}
	\item \textit{Count the drinks on the table.}
	\item \textit{Tell me how many girls there are in the living room.}
\end{itemize}


\section{Bypassing commands and alternate solutions}
The General Purpose Service Robot is a goal-driven test in which the final results has priority over how the command is executed.
This adds several degrees of freedom to make a plan and execute a command accordingly with the robot's capabilities.

For instance, consider the following command:

\begin{center}
\noindent\textit{Bring me a coke}
\end{center}

It is clear that the operator wants a coke and cares little about how the coke is retrieved. Now, let's say that the robot's manipulator is broken, so it won't be able to handle a coke. In this case, several scenarios become evident:

\begin{itemize}
	\item \textbf{Skipping command:} The robot says \quotes{I understood you want me to bring you a coke, but I cannot grasp objects, so I'll skip this command}. Since the robot is not executing the task, no score is given.

	\item \textbf{Continue Rule:} The robot more or less reaches the position, fuzzily points at the object, and then requests to a human assistant to deliver the coke for it. In this case, the referee might grant up to $\frac{1}{3}$ of the points, if any.

	\item \textbf{Requesting human assistance:} Taking advantage of the Continue Rule, the robot requests assistance from a human to grasp the object, requesting later to follow it. During the guiding phase, the robot actively tracks the human to the operator's position and supervises the delivery (e.g.~telling it noticed the operator has received the coke). In this case, and regarding the execution of the tasks, the referee may grant a full score.

	\item \textbf{Social alternative:} The robot looks for another person in the arena, finds them, and convinces them (or socially bribes them) to deliver a coke to the operator using natural language dialogs. In these rare cases, the referee may grant a full score depending on the success of the interaction.
\end{itemize}


\chapter[EEGPSR in detail]{E\textsuperscript{2}GPSR in detail.}
\label{chap:eegpsr-appendix}

\section{Command Generation}
EEGPSR commands are generated randomly using the official [EE]GPSR Command Generator and grammars publicly available at https://github.com/kyordhel/GPSRCmdGen. The official [EE]GPSR Command Generator and the official grammars will be made available two months before the competition. However, teams must be aware that the categories, objects and other data is provided for testing purposes only.

For each command to be executed, the Team Leader must choose a Command Category. If the Team Leader knows \textit{a priori} that the robot won't be able to execute the generated command, is advised to inform the operator immediately in order to proceed with the next command, saving this way valuable time for the task execution.


\section{Command retrieval explained}
The robot has to show it has understood the given command by stating all the required information to accomplish the task. For this purpose, the robot may repeat the understood command and ask for confirmation. It is not required to repeat the command word by word; rephrasing the command is allowed. For instance, if the robot is instructed to \enquote{place a coke onto the tray}, the robot may either say: \textit{\enquote{You want me to place a coke on the tray. Is that correct?}} or \textit{\enquote{do you want me to deliver a coke to the tray?}}.

If The robot can't correctly recognize the given command, it is allowed to request the operator to repeat the command up to three times. After three failed attempts, a new command is generated. The team may opt to use a custom operator or bypassing speech recognition (\refsec{rule:asrcontinue}) at any time, but each generated command will be given to the robot no more than three times. Only three different commands are generated for a robot, if the robot fails to recognize all three commands (i.e.~nine attempts), the test ends immediately.

When a robot has partially understood the command, it is allowed to ask the operator for additional information (e.g.~\textit{\enquote{did you say apple juice or pineapple juice?}}).

%%%%%%%%%%%%%%%%%%%%%%%%%%%%%%%%%%%%%%%%%%%%%%%%%%%%%%%%%%%%%%%%%%%%%%%%%%%%%
%
% Categories explained
%
%%%%%%%%%%%%%%%%%%%%%%%%%%%%%%%%%%%%%%%%%%%%%%%%%%%%%%%%%%%%%%%%%%%%%%%%%%%%%
\section{Categories explained}
\label{sec:eegpsr-categories-explained}
This section explain each of the categories of the test and provides examples on how the abilities are scored.

It is important to remark that there is no script or predefined way to solve the tasks, being most of them of ambiguous nature. It is up to the team to choose how to solve each tasks accordingly with the robot's capabilities.



%%%%%%%%%%%%%%%%%%%%%%%%%%%%%%%%%%%%%%%%%%%%%%%%%%%%%%%%%%%%%%%%%%%%%%%%%%%%%
%
% Category I explained
%
%%%%%%%%%%%%%%%%%%%%%%%%%%%%%%%%%%%%%%%%%%%%%%%%%%%%%%%%%%%%%%%%%%%%%%%%%%%%%
\subsection{Category I: Three at once}
\label{sec:eegpsr-category1-explained}
Command from this category are composed of \textit{three simple actions}, which the robot has to show it has recognized. The robot may repeat the understood command and ask for confirmation. If it can't recognize the command correctly, it can also ask the speaker to repeat the complete command.

Tasks from this category are much alike the ones in GPSR (see~\refsec{chap:gpsr-appendix-cat1} and~\refsec{chap:gpsr-appendix-cat2}), requiring to master basic skills. Since commands must be accomplished as quick as possible, in this category speed is the key.

\subsubsection{Command examples}
\begin{itemize}
	\item Go to the kitchen counter, take the coke, and bring it to me.
	\item Bring the chips to Mary at the sofa, tell the time and follow her.
	\item Find a person in the living room, guide them to the kitchen and follow them.
	\item Take the chips from the counter, find a person in the bedroom, and go to the entrance.
\end{itemize}


%%%%%%%%%%%%%%%%%%%%%%%%%%%%%%%%%%%%%%%%%%%%%%%%%%%%%%%%%%%%%%%%%%%%%%%%%%%%%
%
% Category II explained
%
%%%%%%%%%%%%%%%%%%%%%%%%%%%%%%%%%%%%%%%%%%%%%%%%%%%%%%%%%%%%%%%%%%%%%%%%%%%%%
\subsection{Category II: People}
\label{sec:eegpsr-category2-explained}
Tasks from this category require memorizing a person's features, describing unknown people, recognize people from description, and find people from the distance; as well as following or guiding a person in crowded environments or through narrow spaces. The navigation may take place either inside or outside the arena.

\subsubsection{Task examples}
\begin{itemize}
	\item Describing a person in certain specific location.
	\item Delivering objects to a person that matches the given description.
	\item Reporting number of people in a room matching given description.
	\item Finding people performing certain activity.
	\item Finding people whose face or body or partially occluded or not facing the robot.
	\item Following a person inside an elevator.
	\item Guiding a person to the toilet.
	\item Going through a multitude while following or guiding a person without loosing them.
	\item Avoiding people crossing or standing by while guiding or following.
	\item Performing real time mapping and localization.
\end{itemize}

\subsubsection{Command examples}
\begin{itemize}
	\item Offer a beer to all the adults in the living room.
	\item Meet the person at the door. If their name is John guide him to the kitchen, ask him to leave otherwise.
	\item Guide the person at the entrance to the kitchen.
	\item Find John in the kitchen, he wearing black.
	\begin{itemize}
		\item[Kitchen:] Robot, follow me (goes outside to car).
		\item[Car:] Please ask Jerry and Jimmy at the sofa to help carrying out the groceries.
	\end{itemize}
	\item Describe the person at the door to the woman in the Kitchen.
	\item Take this coke to the girl [in the living room] wearing a red sweater.
	\item Tell me how many standing people there are in the dining room.
	\item Go to the living room and follow the waving person.
	\item Tell me what John is doing (John is reading a book).
\end{itemize}

\subsubsection{Meeting new people}
Say the generated command is \textit{ask Joe to come here}, since the robot has no knowledge of who is Joe, it is expected to ask \enquote{\textit{how can I recognize Joe?}} Two answers are possible:
\begin{itemize}
	\item \textbi{Meet Joe:} The person named \textit{Joe} will stand in front of the robot and follow robot's (not team's) instructions for training. The robot must announce when it has completed memorizing that person before proceeding to execute the command.
	\item \textbi{Joe is the...} A description indicating how to recognize \textit{Joe} is given to the robot. Retrieved information must be confirmed.
\end{itemize}


%%%%%%%%%%%%%%%%%%%%%%%%%%%%%%%%%%%%%%%%%%%%%%%%%%%%%%%%%%%%%%%%%%%%%%%%%%%%%
%
% Category III explained
%
%%%%%%%%%%%%%%%%%%%%%%%%%%%%%%%%%%%%%%%%%%%%%%%%%%%%%%%%%%%%%%%%%%%%%%%%%%%%%
\subsection{Category III: Objects}
\label{sec:eegpsr-category3-explained}
Tasks from this category require handling objects into small or narrow spaces, opening doors and drawers, describing unknown objects, recognize objects from description, identify occluded objects and from the distance.

\subsubsection{Task examples}
\begin{itemize}
	\item Setting up a table.
	\item Cleaning up spots or spills.
	\item Grasping objects from a box.
	\item Placing objects into a microwave or fridge.
	\item Transporting a tray.
	\item Pouring cereal in a bowl.
	\item Retrieving objects from a given description.
	\item Counting and describing objects.
	\item Finding objects from distance or inside drawers.
\end{itemize}

\subsubsection{Command examples}
\begin{itemize}
	\item Hand me a coke from the fridge (the coke is inside the fridge).
	\item Bring me some flakes in a bowl.
	\item Put this book into the drawer.
	\item Bring me the biggest pill bottle from the kitchen counter.
	\item Bring me the bookcase's right-most object.
	\item Describe the objects on the drawer to me.
	\item Tell me how many red apples are in the basket on the kitchen table.
	\item Count the snacks in the shelf and tell me how many there are.
	\item Set up the table and serve some toasts.
\end{itemize}




%%%%%%%%%%%%%%%%%%%%%%%%%%%%%%%%%%%%%%%%%%%%%%%%%%%%%%%%%%%%%%%%%%%%%%%%%%%%%
%
% Scenarios explained
%
%%%%%%%%%%%%%%%%%%%%%%%%%%%%%%%%%%%%%%%%%%%%%%%%%%%%%%%%%%%%%%%%%%%%%%%%%%%%%
\section{Scenarios explained}
\label{sec:eegpsr-categories-explained}
A different scenario applies to each randomly generated command in the category chosen by the team. The scenario is chosen by the referees in a semi-random fashion so all the robots try all three scenarios described below.


%%%%%%%%%%%%%%%%%%%%%%%%%%%%%%%%%%%%%%%%%%%%%%%%%%%%%%%%%%%%%%%%%%%%%%%%%%%%%
%
% Scenario: incomplete information
%
%%%%%%%%%%%%%%%%%%%%%%%%%%%%%%%%%%%%%%%%%%%%%%%%%%%%%%%%%%%%%%%%%%%%%%%%%%%%%
\subsection{Incomplete commands}
\label{sec:eegpsr-incomplete-command}
The commands given do not include all the information necessary to accomplish the task. The actual commands are under-specified by, for example:
\begin{itemize}
	\item only giving the class of the object (\enquote{bring me a drink}) or location (\enquote{guide me to the table}), and not the actual object or location, or
	\item not providing the location (or its class).
\end{itemize}

The robot can ask questions to retrieve the missing information about the task, but is not required to. In the questions the robot has to make clear what it has already understood, e.g., tell the operator that it has understood \textit{to bring a particular beverage can}, but not \textit{where the can is} located in the arena. The robot may also simply start searching.

\subsubsection{Examples}
\begin{itemize}
	\item Go to the kitchen counter, take the drink, and bring it to me (unspecified which drink).
	\item Find a person, guide them to the kitchen and follow them (unspecified where the person can be found).
	\item Bring me some drink in a bowl (unspecified which drink).
	\item Put the biggest pill bottle in kitchen counter on the table (unspecified table).
	\item Offer them a beer (unspecified to who and where are they).
	\item Guide Joe here (unspecified where is Joe and how to recognize him).
\end{itemize}



%%%%%%%%%%%%%%%%%%%%%%%%%%%%%%%%%%%%%%%%%%%%%%%%%%%%%%%%%%%%%%%%%%%%%%%%%%%%%
%
% Scenario: erroneous and misleading information
%
%%%%%%%%%%%%%%%%%%%%%%%%%%%%%%%%%%%%%%%%%%%%%%%%%%%%%%%%%%%%%%%%%%%%%%%%%%%%%
\subsection{Erroneous and misleading commands}
\label{sec:eegpsr-erroneous-command}
The robot gets a command that contains erroneous information. The robot should be able to realize such an error while trying to carry out the task, and try to carry on an alternative solution. If the robot is unable to solve the problem, it must go back to the operator, and clearly state \textit{why} it wasn't able to accomplish the task.

If on the contrary the robot was able to solve the task, it also must explain what went wrong and how it was solved.

\subsubsection{Examples}
Below some examples are presented. For each example command, one or more possible problems are depicted.
\begin{itemize}
	\item Set up the table (and serve some choco flakes).
	\begin{itemize}
		\item The cuttlery is in another drawer in cupboard.
		\item Choco-flakes box is empty, but there are normal flakes.
	\end{itemize}

	\item Bring to Ana at the coach the water from the cupboard.
	\begin{itemize}
		\item The water is on the dinner table.
		\item Ana is lying on the bed.
	\end{itemize}

	\item Find James in the living room and guide him to the car.
	\begin{itemize}
		\item James is in the bedroom.
		\item James is lying unconscious in the living room's floor.
	\end{itemize}
\end{itemize}

\section{Bypassing commands and alternate solutions}
The EEGPSR is a goal-driven test in which the final results has priority over how the command is executed.
This adds several degrees of freedom to make a plan and execute a command accordingly with the robot's capabilities.

For instance, consider the following command:

\begin{center}
\noindent\textit{Bring me a coke}
\end{center}

It is clear that the operator wants a coke and cares little about how the coke is retrieved. Now, let's say that the robot's manipulator is broken, so it won't be able to handle a coke. In this case, several scenarios become evident:

\begin{itemize}
	\item \textbf{Skipping command:} The robot says \enquote{I understood you want me to bring you a coke, but I cannot grasp objects, so I'll skip this command}. Since the robot is not executing the task, no score is given.

	\item \textbf{Continue Rule:} The robot more or less reaches the position, fuzzily points at the object, and then requests to a human assistant to deliver the coke for it. In this case, the referee might grant up to $\frac{1}{3}$ of the points, if any.

	\item \textbf{Requesting human assistance:} Taking advantage of the Continue Rule, the robot requests assistance from a human to grasp the object, requesting later to follow it. During the guiding phase, the robot actively tracks the human to the operator's position and supervises the delivery (e.g.~telling it noticed the operator has received the coke). In this case, and regarding the execution of the tasks, the referee may grant a full score.

	\item \textbf{Social alternative:} The robot looks for another person in the arena, finds them, and convinces them (or socially bribes them) to deliver a coke to the operator using natural language dialogs. In these rare cases, the referee may grant a full score depending on the success of the interaction.
\end{itemize}



\chapter{Example Skills}
\label{chap:example-skills}

The following section presents a list of \iterm{Example Skills} with an high degree of difficulty which can be exploited during the \textit{Open Demonstrations} (See~\refsec{sec:open-demonstrations}.
Other skills not on this list (yet) may be added as well. If you want to do so, please let the TC know via email (tc@robocupathome.org) for their inclusion on the RuleBook so all teams may also show this skill.

Please note that these examples are to illustrate the level of complexity and applicability that should be shown. For instance, \quotes{Handle a pan} is listed in the category of \textit{Complex manipulation}, but it is extensive to handling pans, pots, woks and any other cookware with handles.

\section{Skills by category}

\subsection{Complex manipulation}
\begin{itemize}
	\item Cook a meal.
	\item Manipulating panels/switches/knobs.
	\item Use/open a fridge/stove/blender/microwave/washing machine.
	\item Iron clothes.
	\item Move a movable object (pole, chair, table).
	\item Pouring liquids/powders.
	\item Operate a water tap.
	\item Handle a pan.
\end{itemize}

\subsection{Complex vision}
\begin{itemize}
	\item Read text from a newspaper.
	\item Handle glass/shiny-metallic objects.
	\item Recognize moods, activities, age, gender.
%	\item Recognize clothes, dressing-styles, fashionable people.
	\item Label unknown objects.
\end{itemize}

\subsection{Complex navigation}
\begin{itemize}
	\item Navigate in (very) crowded environments.
	\item Navigate difficult terrain.
	\item Climb stairs.
	\item Push a wheelchair.
\end{itemize}

\subsection{Robot-Human Interaction}
\begin{itemize}
	\item Collaborative robot-human manipulation.
	\item Maintaining a conversation.
	\item Learning actions on-the-fly.
	\item Learning objects from humans e.g.~\quotes{This object is a ...} with an open vocabulary.
	\item Following a human by grasping its hand.
	\item Explain the robot abstract concepts (why people love sunny days).
	\item Arrange unknown random people for a nice photo (no occlusions).
	% \item ask the robot for the answer to the universe, meaning of life and everything else
\end{itemize}

\subsection{Complex action planning}
\begin{itemize}
	\item Separate clothes for laundry (e.g.~by color)
	\item Arrange a dish-washer.
	\item Take a cup from the cupboard whose location has changed, is closed, or the path to it is blocked (e.g.~by a chair).
	\item Light the way out with a lamp during a general power off.
	\item Arrange unknown random people for a nice photo (no occlusions).
	\item
\end{itemize}

\subsection{Mapping}
\begin{itemize}
	\item Learn/create a (3D) map on the fly.
	\item Semantically annotate a map on the fly
	\item The robot enters a completely changed arena (furniture moved or even changed),
	   explores it and is told to go to e.g.~a table that is moved or added.
\end{itemize}


% Local Variables:
% TeX-master: "Rulebook"
% End:

\newpage
\input{general_rules/arena_decorations.tex}

\end{appendices}

% \renewcommand{\chaptername}{Chapter}
% \addto\captionsenglish{\renewcommand{\chaptername}{Chapter}}
\renewcommand*{\chapterformat}{\LARGE{Chapter \thechapter}}
\renewcommand{\chaptermark}[1]{\markboth{\chaptername \ \thechapter. \ #1}{}}


% Local Variables:
% TeX-master: "Rulebook"
% End:


\printabx
\printidx

\end{document}
