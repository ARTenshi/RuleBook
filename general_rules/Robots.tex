%%%%%%%%%%%%%%%%%%%%%%%%%%%%%%%%%%%%%%%%%%%%%%%%%%%%%%%%%
\section{Robots}
\label{rule:robots}

\subsection{Number of Robots}
\label{rule:robots_number}

\begin{enumerate}
	\item \textbf{Registration:} The maximum \term{number of robots} per team is \emph{two} (2).
	\item \textbf{Regular Tests:} Only one robot is allowed per test. For different test runs, different robots can be used.
	\item \textbf{\FINAL:} In the \FINAL, both robots can be used simultaneously.
\end{enumerate}

\subsection{Appearance and Safety}
\label{rule:robot_appearance}

Robots should have a product-like appearance and be safe to operate.
The following rules apply to all robots:
\begin{enumerate}
	\item \textbf{Cover:} The robot's internal hardware (electronics and cables) should be covered so that safety is ensured. The use of (visible) duct tape is strictly prohibited.
	\item \textbf{Loose cables:} Loose cables hanging out of the robot are not permitted.
	\item \textbf{Safety:} The robot must not have sharp edges or elements that might harm people.
	\item \textbf{Annoyance:} The robot must not be continuously making loud noises or use blinding lights.
	\item \textbf{Marks:} The robot may not exhibit any kind of artificial marks or patterns.
	\item \textbf{Driving:} To be safe, the robots should be careful when driving. Obstacle avoidance is mandatory.
\end{enumerate}
The compliance with these rules will be verified during \RobotInspection{} (see \ref{sec:robot_inspection}).

\subsection{Standard Platform Leagues}
\label{sec:rules:robotappearance_spl}
For Robots competing in a \SPL{}, modifications and alterations to the robots are strictly forbidden. This includes, but is not limited to, attaching, connecting, plugging, gluing, and taping components into and onto the robot, as well as, modifying or altering the robot structure. Not complying with this rule, leads to an immediate disqualification and penalization of the team (see~\ref{sec:rules:penaltiesbonuses}).

Robots are allowed to \enquote{wear} clothes, have stickers (e.g., a sticker exhibiting the logo of a sponsor), and be painted as long as they are compliant with section \ref{sec:rules:robotappearance}.

\subsubsection{DSPL Modifications}
\label{sec:rules:mountingbracket}
In the \DSPL{}, some modifications to the \HSR{} are allowed. An official \MountingBracket{} is provided by Toyota for the \HSR{}. Any laptop fitting inside the \MountingBracket{} can be used as additional on board computing. Furthermore, teams are allowed to attach the following devices to the robot or the laptop in the \MountingBracket{}:
\begin{enumerate}
	\item \textbf{Audio:} USB audio output device, e.g. USB-powered speaker, possibly with sound card.
	\item \textbf{Wi-Fi Adapter:} USB-powered IEEE 802.11ac (or newer) compliant device.
	\item \textbf{Ethernet Switch:} USB-powered IEEE 802.3ab (or newer) compliant device.
\end{enumerate}

\noindent A maximum of three such devices can be attached, they cannot increase the robot's dimension.

\subsection{Robot Specifications for the Open Platform League }
Robots competing in the RoboCup@Home Open Platform League must comply with security specifications in order to avoid causing any harm while operating in human environments.

\subsubsection{Size and weight of robots}
\label{rule:robots_size}

\begin{enumerate}
	\item \textbf{Dimensions:} The dimensions of a robot should not exceed the limits of an average door, which is \SI{200}{\centi\meter} by \SI{70}{\centi\meter} in most countries.\\
	The TC may allow the qualification and registration of larger robots, but due to the international character of the competition it cannot be guaranteed that the robots can actually enter the arena. In case of doubt, contact the local organization.
	\item \textbf{Weight:} There is no specific weight restriction. However, the weight of the robot and the pressure it exerts on the floor should not exceed local regulations for the construction of buildings which are used for living and/or offices in the country where the competitions is being held.
	\item \textbf{Transportation:} Team members are responsible for quickly moving the robot out of the arena.	If the robot cannot move by itself (for any reason), the team members must be able to transport the robot away with an easy and fast procedure.
\end{enumerate}



\subsubsection{Emergency stop button}
\label{rule:robots_emergency_button}

\begin{enumerate}
	\item \textbf{Accessibility and visibility:} Every robot has to provide an easily accessible and visible \iterm{emergency stop} button.
	\item \textbf{Color:} It must be coloured red, and preferably be the only red button on the robot. If it is not the only red button, the TC may ask the team to tape over or remove the other red button.
	\item \textbf{Robot behavior:} When pressing this button, the robot and all parts of it have to stop moving immediately.
	\item \textbf{Inspection:} The emergency stop button is tested during the \iterm{Robot Inspection} test (see~\refsec{sec:robot_inspection}).
\end{enumerate}




\subsubsection{Start button}
\label{rule:start_button}

\begin{enumerate}
	\item \textbf{Requirements:} As stated in~\refsec{rule:start_signal}, teams that aren't able to carry out the default start signal (opening the door) have to provide a \iterm{start button} that can be used to start tests. The team needs to announce this to the TC before every test that involves a start signal, including \iterm{Robot Inspection}.
	\item \textbf{Definition:} The start button can be any \quotes{one-button procedure} that can be easily executed by a referee.  This includes, for example, the release of the \iterm{emergency button} (\refsec{rule:robots_emergency_button}), a hardware button different from the \iterm{emergency button} (e.g., a green button), or a software button in a Graphical User Interface.
	\item \textbf{Inspection:} It is during the the \iterm{Robot Inspection} test (see~\refsec{sec:robot_inspection}) that the procedure for the start button, if needed, is announced to the TC and inspected. The start button for a robot should be the same for all the tests.
	\item \textbf{Penalty for using start button:} If a team needs to use the start button in a test where opening the door is the start signal, it may receive a penalty (see~\refsec{rule:start_signal}).
\end{enumerate}




\subsubsection{Audio output plug}
\label{rule:roobt_audio_out}

\begin{enumerate}
	\item \textbf{Mandatory plug:} Either the robot or some external device connected to it \emph{must} have a \iterm{speaker output plug}. It is used to connect the robot to the sound system so that the audience and the referees can hear and follow the robot's speech output.
	\item \textbf{Inspection:} The output plug needs to be presented to the TC during the \iterm{Robot Inspection} test (see~\refsec{sec:robot_inspection}).
	\item \textbf{Audio during tests:} Audio (and speech) output of the robot during a test have to be understood at least by the referees and the operators.
	\begin{compactitem}
		\item It is the responsibility of the teams to plug in the transmitter before a test, to check the sound system, and to hand over the transmitter to next team.
		\item Do not rely on the sound system! For fail-safe operation and interacting with operators make sure that the sound system is not needed, e.g., by having additional speakers directly on the robot.
\end{compactitem}
\end{enumerate}




\subsubsection{Appearance}
\label{rule:robots_appearance}
Open Platform Robots should have a neat appearance that resembles more a safe and finished product than an early stage prototype, paying special attention in completely cover the robot's internal hardware (electronics and cables) in an appealing way.
% However, teams must keep in mind that no artificial markers are allowed when personalizing the appearance or a robot. This includes, but is not limited to bar codes, QR codes, OpenCV markers, fluorescent and phosphorescent colors, and reflective stickers.
Although covering the robot's internal hardware with a T-Shirt is not forbidden (for now) it is strongly unadvised.



% Local Variables:
% TeX-master: "../Rulebook"
% End:
