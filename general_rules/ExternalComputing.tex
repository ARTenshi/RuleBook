%%%%%%%%%%%%%%%%%%%%%%%%%%%%%%%%%%%%%%%%%%%%%%%%%%%%%%%%%
\section{External computing}\label{rule:robot_external_computing}
Robots are allowed to use some form of external computing, for example in the form of so-called \enquote{Cloud services} and/or \enquote{Internet API's} etc.
\begin{enumerate}
	\item \textbf{Definition:} Computing resources that are not physical part of the robot are \iterm{external computing resources}.
	\item \textbf{Inspection:} In general, external computers are not allowed unless explained to and allowed by the \iaterm{Technical Committee}{TC}.
	  A team must announce to the TC at least 1 month in advance the external computing resources they want to use, for what purpose and how to reach the resources (e.g.~specify the URL or IP address and port). Inspected software must meet the following \textbf{requirements:}
	  \begin{itemize}
	  	\item The software must be open source (BSD/GPL/etc), or
        \item Detailed information about the propietary product must be provided (e.g.~vendor, patent number, licencing, pricing, etc.), as well as publishing the interface for scientific use.
	  \end{itemize}
	All relevant information must be specified in the team description paper.
	\item \textbf{Connection:} The robot may connect to \iterm{external computing resources} via a network connection, e.g.~the Internet.
	  The competition organisation cannot make any guarantees concerning availability, connectivity and performance of the connection.
	  The robot should still be functional (albeit limited perhaps) if the \iterm{external computing resources} cannot be used for some reason.
	  This is the team's responsibility.
	\item \textbf{Autonomy:} The robot has to maintain full autonomy when using \iterm{external computing resources},
	  meaning there may not be a human giving the robot any kind of instructions via \iterm{external computing resources}.
	  It is up to the team to prove to the \iaterm{Technical Committee}{TC} that there was no cheating introduced via the \iterm{external computing resources}.
	  For example, the use of Amazon Mechanical Turk to classify and recognize objects during a competition will be considered cheating, since effectively a human will do the classification.
	  Remote control or tele-operation is also considered cheating.
	\item \textbf{Availability:} The resources must be publicly available, for use by robots of other teams, well before and after the competition.
	\item \textbf{Recognition:} In case the resources are not developed by the team itself, the creators must be properly credited in the Team Description Paper (See~\refsec{rule:website_tdp}).
	\item \textbf{Limit:} A robot is limited to use up to 5 \iterm{external computing resources}.
\end{enumerate}

\textbf{Remark:} Teams are allowed to use their own software in the external computing devices (not only cloud services). This software must be publicly available to other teams for scientific purposes (evaluation, test, and benchmarking), as well as for TC for inspection. Although open-sourcing the software is not mandatory, this practice is advised and encouraged by the league.

\subsection{On-Site External Computing Devices}
\label{rule:robot_external_computing_devices}

% Second iteration
\begin{itemize}
  \item \textbf{Location:} All \iaterm{External Computing Devices}{ECDs} must be approved by the \iaterm{Technical Committee}{TC}
  		(see~\refsec{rule:robot_external_computing}) and presented during the robot inspection. Furthermore, ECDs
  		that exceed the form factor of a laptop (e.g.~a desktop PC), referred hereinafter as servers, must be placed and
  		setup in a designated area that is announced by the \iaterm{Technical Committee}{TC} during setup days.
  		A server must not have a screen, mouse, keyboard, bluetooth or any other peripheral device attached to it.
  		Servers cannot be touched by anyone (e.g.~for plugging/un-plugging cables) during the competition phase.
  \item \textbf{Procedure:} Before a teams' test run, ECDs (except servers) must be setup in time in the
		\iaterm{\textbf{E}xternal \textbf{C}omputing \textbf{R}esource \textbf{A}rea} {ECRA}. The OC will announce the
        ECRA during setup days. A switch connected to the arena wireless network will be available to teams in the ECRA.
  		Immediately after a teams' test slot, all equipment must be removed from the ECRA in order to give ensuing
        teams adequate space and time to setup their devices. As soon as the referee indicates the teams' test slot
        starts, team members are strictly forbidden to touch their external devices until their current test run is over.
        Breaking this \enquote{hands-off} rule is penalized with -25 points for the corresponding test run. Also, not removing
        equipment within a reasonable time frame is penalized with -25 points for the tests run. The ECRA is not to be
		occupied by more than three times at a time.
\end{itemize}


% Initial version as discussed in: https://github.com/RoboCupAtHome/RuleBook/pull/388
%\begin{itemize}
%  \item \textbf{Location:} All external computing devices, approved by the \iaterm{Technical Committee}{TC}
%  		(see~\refsec{rule:robot_external_computing}), must be placed at the designated location, e.g., a table close
%        to the arena, referred hereinafter as the \iaterm{\textbf{E}xternal \textbf{C}omputing \textbf{R}esource
%        \textbf{A}rea} {ECRA}. The OC will announce the ECRA during setup days. A switch connected to the arena
%        wireless network will be available to teams.
%  \item \textbf{Procedure:} Before a teams' test run, external devices must be setup in time.
%  		Immediately after a teams' test slot, all equipment must be removed from the ECRA in order to give ensuing
%        teams adequate space and time to setup their devices. As soon as the referee indicates the teams' test slot
%        starts, team members are strictly forbidden to touch their external devices until the current test run is over.
%        Breaking this \enquote{hands-off} rule is penalized with -25 points for the corresponding test run. Also, not removing
%        equipment within a reasonable time frame is penalized with -25 points for the tests run.
%  \item \textbf{Limitations:} External computing devices currently have no limitation concerning computation
%  		power and form factor. However, in order to maintain enough space and a	clear arrangement (e.g.~wrt
%        inspectability for referees), stand-alone screens are \textbf{not allowed}. Mice and keyboards connected
%        to external computing devices are allowed.
%\end{itemize}

\textbf{Remark:} Please keep the fair play rule~\ref{rule:fairplay} in mind. The referees, technical committee,
        and organizing committee members may run random checks anytime during the competition in order to check if a team
        occupies the ECRA continuously and whether devices outside the ECRA are used to, e.g., remote control a robot.
        If one of the aforementioned situations is discovered, disqualification from at least the current test --- or
        even the whole competition is considered an option.


\subsection{Official Standard Laptop for DSPL}
\label{rule:osl_dspl}
%
% Mandatory mounting bracket
%
% In the Domestic Standard Platform League, teams are required to have the \iaterm{Official Standard Laptop}{OSL} connected to the Toyota HSR via Ethernet cable, safely located in the TOYOTA HSR \iterm{Mounting Bracket} provided by TOYOTA for this purpose.

% The robot must wear the \iterm{Mounting Bracket} with the OSL in all the tests during the competition and the OSL must be connected to the robot whether it's used or not. Since the use of the \iterm{Mounting Bracket} and the presence of the OSL is compulsory, teams missing this requirement will not be allowed to compete.

%
% Optional mounting bracket
%
In the Domestic Standard Platform League, teams may use the \iaterm{Official Standard Laptop}{OSL} connected to the Toyota HSR via Ethernet cable, safely located in the TOYOTA HSR \iterm{Mounting Bracket} provided by TOYOTA for this purpose.

\subsubsection{Technical Specifications}
The technical specifications for the Official Standard Laptop in the Domestic Standard Platform League are the following:


 \begin{itemize}
  \item \textbf{Brand and model:} DELL Alienware 15 or 17
  \item \textbf{CPU:} Core-i7 series
  \item \textbf{RAM:} 16GB or 32GB
  \item \textbf{GPU:} NVIDIA GeForce GTX 1070 or 1080
  \item \textbf{Storage:} Unrestricted.
\end{itemize}

No other brands or models will be accepted. There are no constrains regarding the software installed in the OSL but no additional hardware is allowed.

The referees, Technical Committee, and Organizing Committee members may run random checks anytime during the competition prior to the test to verify that the laptop in the TOYOTA HSR \iterm{Mounting Bracket} has no additional hardware plugged in, and matches the authorized specifications.

% Local Variables:
% TeX-master: "../Rulebook"
% End:
