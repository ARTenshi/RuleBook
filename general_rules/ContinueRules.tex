\section{Deus ex Machina}
\label{sec:rules:deusexmachina}
To allow teams to participate in tests, where the robot is only missing one ability to complete the main goal, and to allow robots to continue in case of minor malfunctions, a robot can request human assistance during a test, called \DEM{}. Help can not be requested to complete bonus goals.

\subsection{Procedure}
\label{sec:rules:demprocedure}
To request human assistance while solving a task:

\begin{enumerate}
	\item \textbf{Request Help:} The robot has to indicate loud and clear that it requires human assistance. It must clearly state:
	\begin{enumerate}
		\item The nature of the assistance.
		\item The particular goal or desired result.
		\item How the action must be carried out (when necessary).
		\item Details about how to interact with the robot (when necessary).
	\end{enumerate}

	\item \textbf{Supervise:} The robot must be aware of the human's actions, being able to tell when the requested action has been completed, as well as guiding the human assistant (if necessary) during the process.

	\item \textbf{Acknowledge:} The robot must politely thank the human for the assistance provided.
\end{enumerate}


\subsection{Scoring}
\label{sec:rules:demscoring}
The amount of times a robot can request human assistance is not limited, but score reduction applies each time. The score for a goal or partial goal cannot be negative due to \abb{DEM} reductions. Possible \abb{DEM} requests are stated for each test with the corresponding score reductions. Other requests must be announced one day before the test in the \TLM{} where the \abb{TC} will decide on the scoring. In general, points will be deducted increasingly for:
\begin{enumerate}
 	\item \textbf{Partial Solutions:} The robot requests a partial solution (e.g. pointing to the person the robot is looking for or placing an object within grasping distance).
 	
 	\item \textbf{Full Awareness:} The robot requests a whole step of the test to be completed but is able to track and supervise activity. This means detecting when something goes wrong and when the request is done.
 	
 	\item \textbf{No Awareness:} The robot requests a whole step of the test to be completed and has to be told when the request is done.
\end{enumerate}

\subsection{Bypassing Speech Recognition}
\label{sec:rules:asrcontinue}
When the robot is not able to receive spoken commands, teams are allowed to provide alternatives.

\begin{enumerate}
	\item \textbf{Custom Operator:} A reduction of 20\% of the maximum attainable score is applied when a \CustomOperator{} is requested. The person choosen by the team gives the command \emph{exactly} as instructed by the referee.

	\item \textbf{Gestures:} A reduction of 20\% of the maximum attainable score is applied when a gesture (or set of gestures) is used to instruct the robot.

	\item \textbf{QR Codes:} A reduction of 30\% of the maximum attainable score is applied when a QR code is used to instruct the robot.

	\item \textbf{Alternative Input Method:} A reduction of up to 30\% of the maximum attainable score is applied when an alternative \abb{HRI} interface is used to instruct the robot.
\end{enumerate}


\subsubsection{Alternative HRI Interfaces}
\label{sec:rules:asrcontinue:hriinterface}
Alternative methods and interfaces for HRI offer a way for a robot to start or complete a task.
Any reasonable method may be used, with the following criteria:
\begin{itemize}
	\item \textbf{Intuitive:} A manual should not be needed. Teams are not allowed to explain how to interact with the robot.

	\item \textbf{Effortless:} Must be as easy to use as uttering a command.

	\item \textbf{Smart:} The interface adapts to the user input, displaying only the options that make sense or that the robot can actually perform.
\end{itemize}
\noindent\textbf{\textsc{Award:}} The best interface is awarded the Best Human-Robot Interface award (see~\ref{sec:introduction:hriaward}).
\noindent\textbf{Note:} All methods to bypass ASR need to be announced during \RobotInspection{} (see~\ref{sec:setupdays:inspection}).

% Local Variables:
% TeX-master: "../Rulebook"
% End:
