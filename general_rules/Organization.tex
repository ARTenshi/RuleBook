%%%%%%%%%%%%%%%%%%%%%%%%%%%%%%%%%%%%%%%%%%%%%%%%%%%%%%%%%
\section{Organization of the competition}
\label{sec:procedure_during_competition}

\subsection{Stage system}\label{rule:stages}

The competition features a \iterm{stage system}. It is organized in two stages each consisting of a number of specific tasks. It ends with the \iterm{Finals}.

Each \iaterm{stage} comprehends a set of tasks grouped in two thematic scenarios.
% \iaterm{House Cleaner} and \iaterm{Party Host}.
The \iaterm{House Holder} scenario features tasks related to cleaning, organizing, and giving maintenance; while the \iaterm{Party Host} scenario focuses in attending guests needs and providing general assistance during a party.

\begin{enumerate}
	\item \textbf{Robot Inspection:} For security, robots are inspected during setup days. A robot must pass \iterm{Robot Inspection} test (see~\refsec{sec:robot_inspection}) in order to compete.

	\item \textbf{Stage~I:} The first days of the competition called \iterm{Stage~I}.
	All qualified teams can participate in \iterm{Stage~I}.
	The same task can be performed multiple times (See~\refsec{rule:score_system}).

	\item \textbf{Stage~II:} The best \emph{50\% of teams}\footnotemark (after Stage~I) advance to \iterm{Stage~II}.
	Here, tasks require more complex abilities or combinations of abilities.\\
	\footnotetext{If the total number of teams is less than 12, up to 6 teams may advance to Stage~II}
	The \iterm{Open Challenge} is the open demonstration in Stage~II.

	\item \textbf{Final demonstration:} The best \emph{two teams} of each league, the ones with the highest score after Stage~II, advance to the final round.
	The final round features only a single task integrating all tested abilities.
	In order to participate in the Finals, a team must have solved at least one task of the Stage~II.
\end{enumerate}

In case of having no considerable score deviation between a team advancing to the next stage and a team dropping out, the TC may announce additional teams advancing to the next stage.


%%%%%%%%%%%%%%%%%%%%%%%%%%%%%%%%%%%%%%%%%%%%%%%%%%%%%%%%%
\subsection{Schedule}
\label{rule:schedule}

\begin{enumerate}
	\item \textbf{Scenario blocks:} Each scenario is split in two \iterm{blocks}: morning and afternoon.
	Each block lasts two hours.
	The \iaterm{Organizing Committee}{OC} announces the schedule during the setup days.

	\item \textbf{Slots:} The \iaterm{Organizing Committee}{OC} assigns at least two \iterm{test slots} of 5 minutes to each team in each block.
	A team is allowed to try to solve one task during its test slot.
	Remaining block time can be used to assign testing slots to interested teams.

	\item \textbf{Tests:} Teams must announce to the referees which task will they solve during their test slots.

	\item \textbf{Participation is default:} Teams have to indicate to the \iaterm{Organizing Committee}{OC} when they are \emph{skipping} a test slot. Without such indication, they may receive a penalty when not attending (see~\refsec{rule:not_attending}).
\end{enumerate}

% Please add the following required packages to your document preamble:
% \usepackage[table,xcdraw]{xcolor}
% If you use beamer only pass "xcolor=table" option, i.e. \documentclass[xcolor=table]{beamer}
\begin{table}[h]
	\centering\small
	\newcommand{\teams}{%
		\tiny
		\begin{tabular}{c}%
			\textit{Test slot 1}\\
			$\vdots$\\
			\textit{Test slot $n$}\\
		\end{tabular}
	}
	\begin{tabular}{r|%
		>{\columncolor[HTML]{9AFF99}}c |%
		>{\columncolor[HTML]{9AFF99}}c |%
		>{\columncolor[HTML]{CBCEFB}}c |%
		>{\columncolor[HTML]{CBCEFB}}c |%
	}
	 \multicolumn{1}{ c }{}
		& \multicolumn{1}{ c }{\cellcolor{white} Day 1 }
		& \multicolumn{1}{ c }{\cellcolor{white} Day 2 }
		& \multicolumn{1}{ c }{\cellcolor{white} Day 3 }
		& \multicolumn{1}{ c }{\cellcolor{white} Day 4 }
		\\ \hline 
	\multicolumn{1}{|c|}{Block 1}                      & House Holder & Party Host & House Holder & Party Host   \\
	\multicolumn{1}{|c|}{\footnotesize( 9:00 - 12:00)} & \teams       & \teams     & \teams       & \teams       \\ \hline
	\multicolumn{5}{ c }{}                                                                                       \\
	\multicolumn{1}{ c }{}                             & \multicolumn{4}{ c }{\color{gray}Lunch}                 \\
	\multicolumn{5}{ c }{}                                                                                       \\ \hline
	\multicolumn{1}{|c|}{Block 2}                      & House Holder & Party Host & Party Host   & House Holder \\
	\multicolumn{1}{|c|}{\footnotesize(15:00 - 18:00)} & \teams       & \teams     & \teams       & \teams       \\ \hline
	
	& \multicolumn{2}{c|}{\cellcolor[HTML]{9AFF99}Stage 1} & \multicolumn{2}{c|}{\cellcolor[HTML]{CBCEFB}Stage 2}\\ \cline{2-5}
	\end{tabular}
\end{table}


\subsection{Score system}
\label{rule:score_system}
Each task has a main objective and a set of scoring bonuses.
To score in a test, a team must successfully accomplish the main objective of the task; bonuses are not considered otherwise.
Overall scoring is calculated as the sum of the maximum score obtained in each ability.

The \iaterm{score system} has the following constrains
\begin{enumerate}
	\item \textbf{Stage~I:} The maximum total score per task in \iterm{Stage~I} is \scoring{1000 points}.
	
	\item \textbf{Stage~II:} The maximum total score per task in \iterm{Stage~I} is \scoring{2000 points}.

	\item \textbf{\iterm{Finals}:} Final score is normalized and a special evaluation is used.

	\item \textbf{Minimum score:} The minimum total score per test in \iterm{Stage~I} and \iterm{Stage~II} is \scoring{0 points}.
	Teams cannot receive negative points.

	\item \textbf{Penalties:} An exception to \emph{minimum score} rule are penalties.
	Both penalties for not attending (see~\refsec{rule:not_attending}) and extraordinary penalties (see~\refsec{rule:extraordinary_penalties}) can cause a total negative score.
\end{enumerate}




% Local Variables:
% TeX-master: "../Rulebook"
% End:
