\subsection{Open Platform League}
\label{sec:rules:robotappearance_opl}
Robots competing in the \OPL{} must comply with security specifications in order to avoid causing any harm while operating.

\subsubsection{Size and Weight}
\label{sec:rules:robotappearance_opl:size}

\begin{itemize}
	\item \textbf{Dimensions:} The dimensions of a robot should not exceed the limits of an average door (\SI{200}{\centi\meter} by \SI{70}{\centi\meter}). The \TC{} may allow the qualification and registration of larger robots, but it cannot be guaranteed that the robots can actually enter the arena. In doubt, contact the \LOC{}.
	\item \textbf{Weight:} There is no specific weight restriction. However, the weight of the robot and the pressure it exerts on the floor should not exceed local regulations for the construction of buildings which are used for living and/or offices in the country where the competition is being held.
	\item \textbf{Transportation:} Team members are responsible for quickly moving the robot out of the \Arena{}.	If the robot cannot move by itself (for any reason), the team members must be able to transport the robot away in an easy and fast manner.
\end{itemize}


\subsubsection{Emergency Stop Button}
\label{sec:rules:robotappearance_opl:esb}

\begin{itemize}
	\item \textbf{Accessibility and Visibility:} Every robot has to provide an easily accessible and visible \EmergencyStop{} button.
	\item \textbf{Color:} It must be coloured red and be the only red button on the robot.
	The TC may ask the team to tape over or remove any other red button present on the robot.
	\item \textbf{Robot behavior:} When the \EmergencyStop{} button is pressed, the robot and all its parts must stop moving immediately.
\end{itemize}


\noindent\textbf{Note:} All robot requirements will be tested during \RobotInspection{} (see~\ref{sec:setupdays:inspection}).