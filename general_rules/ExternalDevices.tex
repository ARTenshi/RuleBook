%%%%%%%%%%%%%%%%%%%%%%%%%%%%%%%%%%%%%%%%%%%%%%%%%%%%%%%%%
\section{External devices}\label{rule:roobt_external_devices}
\begin{enumerate}
	\item \textbf{Definition:} Everything which is not part of the robot is considered an \iterm{external device}.
	\item \textbf{Inspection:} In general, external devices are not allowed unless presented and explained to the \iaterm{Technical Committee}{TC} during the \iterm{Robot Inspection} test (see~\refsec{sec:robot_inspection}).
	\item \textbf{Supervision:} In regular tests, external devices may only be used under supervision by referees and after approval by the TC. The devices have to be brought to the arena for every test, and removed quickly after the test.
	\item \textbf{Open demonstrations:} For the \iterm{Open Challenge} and the \iterm{Finals}, external devices are allowed, still their use needs to be announced beforehand.
	\item \textbf{Wireless devices:} All \iterm{wireless devices} including bluetooth devices, walkie-talkies, and anything else that uses an RF signal to operate need to be announced to the \iaterm{Organizing Committee}{OC}. The use of any wireless device not approved by the TC is strictly prohibited.
	\item \textbf{Artificial landmarks:} \iterm{Artificial landmarks} and \iterm{markers} are not allowed.
	\item \textbf{Computing devices:} External computers for decentralized computations are allowed, please see~\refsec{rule:robot_external_computing}.
	\item \textbf{Wireless LAN:} The use of networks other than the \iterm{arena network} (see~\refsec{rule:scenario_wifi}) is strictly prohibited.
	\item \textbf{External device for audio processing: }An external speech processing device is allowed. The device is only allowed to connect the mixer's audio line out. The device can be used for sending the raw signal to the robot, processing it on the device, or sending the signal to a third-party's ASR service.
	\item \textbf{External microphones: }\iterm{External microphones}, hand microphones, and headsets are not allowed in OPL. Although using an \iterm{on-board microphone} is recommended in DSPL and SSPL, using the following \iterm{official microphone} is allowed as a backup.
	\begin{itemize}
	\item \textbf{DSPL/SSPL only: }In order to make the audience to catch what is spoken to the robot, the speaker is supposed to use the official microphone to speak to the robot. The official microphone is used for the tasks inside the arena except SSL-related tests (\iterm{Speech and Person Recognition}). Outside the arena, the official microphone is not used.
	\end{itemize}
\end{enumerate}


% Local Variables:
% TeX-master: "../Rulebook"
% End:
