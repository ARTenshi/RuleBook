% %% %%%%%%%%%%%%%%%%%%%%%%%%%%%%%%%%%%%%%%%%%%%%%%%%%%%%%%%%%
%
% External Devices
%
% %% %%%%%%%%%%%%%%%%%%%%%%%%%%%%%%%%%%%%%%%%%%%%%%%%%%%%%%%%%
\section{External Devices}
\label{sec:rules:externaldevices}
Everything a team uses in a test that is not part of the robot is considered an \ExternalDevice{}.
An \ExternalDevice{} must be authorized by the \TC{} during \RobotInspection{} (see~\ref{sec:setupdays:inspection}).
The \abb{TC} decides whether an \ExternalDevice{} can be used freely or under referee supervision and determines its impact on scoring.
Wireless devices, such as hand microphones and headsets, are not allowed with the exception of \ExternalComputing{}. 

\subsection{On-Site External Computing}
\label{sec:rules:onsiteexternalcomputing}
Computing resources that are not physically attached to the robot are considered \ExternalComputing{}.
They must be placed in the \ECRA{}, which is announced by the \TC{} during \SetupDays{} (see~\ref{chap:setupdays}), where a switch, connected to the \ArenaNetwork{} (see~\ref{sec:rules:scenario:wifi}), will be available.
During a \Testblock{} (see~\ref{sec:rules:schedule}), only two persons are allowed in the \abb{ECRA} at any time, one team member each of the two teams up next. No peripherals (e.g.~screens, mouses, keyboards) are allowed to be present. Laptops can only be placed if the team is up next and need to be removed as soon as the test finishes.
During a \Testslot{}, all people must stay at least \SI{1}{\meter} away from the \abb{ECRA}.
Interacting with anything in the \abb{ECRA} after the referee has given the start signal will cause the test to end with a score of zero.


% On-line devices
\subsection{On-Line External Computing}
\label{sec:rules:onlineexternalcomputing}
Teams can utilize \ExternalComputing{} through the internet connection of the \ArenaNetwork{} (e.g.~cloud services, online APIs). These must be announced to and approved by the \TC{} one month prior to the competition.

% Local Variables:
% TeX-master: "../Rulebook"
% End:
