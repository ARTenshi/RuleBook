%% %%%%%%%%%%%%%%%%%%%%%%%%%%%%%%%%%%%%%%%%%%%%%%%%%%%%%%%%%%%%%%%%%%%%%%%%%%%
%%
%%    author(s): RoboCupAtHome Technical Committee(s)
%%  description: Introduction
%%
%% %%%%%%%%%%%%%%%%%%%%%%%%%%%%%%%%%%%%%%%%%%%%%%%%%%%%%%%%%%%%%%%%%%%%%%%%%%%
\chapter{Introduction}
\label{chap:introduction}


\section{RoboCup}
\iterm{RoboCup} is an international joint project to promote AI, robotics, and related fields. It is an attempt to foster AI and intelligent robotics research by providing standard problems where a wide range of technologies can be integrated and examined. More information can be found at http://www.robocup.org/.

\section{RoboCup@Home}
The \iterm{RoboCup@Home} league aims to develop service and assistive robot technology with high relevance for future personal domestic applications. It is the largest international annual competition for autonomous service robots and is part of the RoboCup initiative. A set of benchmark tests is used to evaluate the robots abilities and performance in a realistic non-standardized home environment setting. Focus lies on the following domains but is not limited to: Human-Robot-Interaction and Cooperation, Navigation and Mapping in dynamic environments, Computer Vision and Object Recognition under natural light conditions, Object Manipulation, Adaptive Behaviors, Behavior Integration, Ambient Intelligence, Standardization and System Integration. It is collocated with the RoboCup symposium.

%% %%%%%%%%%%%%%%%%%%%%%%%%%%%%%%%%%%%%%%%%%%%%%%%%%%%%%%%%%%%%%%%%%%%%%%%%%%%
%%
%%  author(s): RoboCupAtHome Technical Committee(s)
%%  description: Introduction - Organization
%%
%% %%%%%%%%%%%%%%%%%%%%%%%%%%%%%%%%%%%%%%%%%%%%%%%%%%%%%%%%%%%%%%%%%%%%%%%%%%%
\section{Organization}
\label{sec:introduction:organization}
\AtHome{} is organized into three subcommittees. Current members are listed at: 
\url{https://athome.robocup.org/committees/}.

\subsection{Executive Committee}
\label{sec:introduction:ec}
The \EC{} consists of members of the board of trustees and representatives of each activity area.

\subsection{Technical Committee}
\label{sec:introduction:tc}
The \TC{} is responsible for the rules of the league. Main focus is writing the rulebook and refereeing.
Members of the \EC{} are always members of the \TC{} as well.

\subsection{Organizing Committee}
\label{sec:introduction:oc}
The \OC{} is responsible for the organization of the competition. They create the schedule and provide information about the scenario.
The \LOC{} is responsible for the set up and organization of the venue.


%% %%%%%%%%%%%%%%%%%%%%%%%%%%%%%%%%%%%%%%%%%%%%%%%%%%%%%%%%%%%%%%%%%%%%%%%%%%%
%%
%%    author(s): RoboCupAtHome Technical Committee(s)
%%  description: Introduction - Infrastructure
%%
%% %%%%%%%%%%%%%%%%%%%%%%%%%%%%%%%%%%%%%%%%%%%%%%%%%%%%%%%%%%%%%%%%%%%%%%%%%%%
\section{Infrastructure}
\label{sec:infrastructure}
\subsection{RoboCup@Home Mailinglist}
The official \iterm{RoboCup@Home mailing list} can be reached at
\begin{center}
\href{mailto:robocup-athome@lists.robocup.org}{\texttt{robocup-athome@lists.robocup.org}}
\end{center}
You can register to the email list at:
\begin{center}
{\small\url{http://lists.robocup.org/cgi-bin/mailman/listinfo/robocup-athome}}
\end{center}

\subsection{RoboCup@Home Web Page}
The official \iterm{RoboCup@Home website} that also hosts this RuleBook can be found at:
\begin{center}
{\small\url{https://athome.robocup.org/}}
\end{center}

\subsection{RoboCup@Home Rulebook Repository}
The official \iterm{RoboCup@Home Rulebook Repository} is where rules are publicly discussed before applying changes to the rulebook.
The entire RoboCup@Home community is welcome and encouraged to actively participate in creating and discussing the rules. The repository can be reached at:
\begin{center}
{\small\url{https://github.com/RoboCupAtHome/RuleBook/}}
\end{center}

Although opening issues with inconsistencies, questions, clarifications, and suggestions is highly appreciated, the best way to contribute is by making pull requests with fixes and proposed changes.

\subsection{RoboCup@Home Telegram Group}
The official \iterm{RoboCup@Home Telegram Group} is and communication channel for the RoboCup@Home community where rules are discussed, announcements are made, and questions are answered.
However, beyond the technical aspects of the competition, the \textit{Telegram Group} is a meeting point to stay in contact with the community, foster knowledge exchange and strengthen relationships.
The \textit{Telegram Group} can be reached at
\begin{center}
{\small\url{https://t.me/RoboCupAtHome}}
\end{center}

\subsection{RoboCup@Home Wiki}
\label{sec:at_home_wiki}
The official \iterm{RoboCup@Home Wiki} is meant to be a central place to collect information on all topics related to the RoboCup@Home league. It was set up to simplify and unify the exchange of relevant information.
This includes but is certainly not limited to hardware, software, media, data, and alike.
The \textit{wiki} can be reached at:
\begin{center}
{\small\url{https://github.com/RoboCupAtHome/AtHomeCommunityWiki/wiki}}
\end{center}



%% %%%%%%%%%%%%%%%%%%%%%%%%%%%%%%%%%%%%%%%%%%%%%%%%%%%%%%%%%%%%%%%%%%%%%%%%%%%
%%
%%    author(s): RoboCupAtHome Technical Committee(s)
%%  description: Introduction - Leagues
%%
%% %%%%%%%%%%%%%%%%%%%%%%%%%%%%%%%%%%%%%%%%%%%%%%%%%%%%%%%%%%%%%%%%%%%%%%%%%%%
\section{Leagues}
\label{sec:introduction:leagues}

\AtHome{} is divided in three leagues. Two are \SPLs{} where each team uses the same robot platform and an \OPL{} where teams are free to choose their robot. The official leagues and their names are:
\begin{itemize}
  \item \DSPL{}
  \item \SSPL{}
  \item \OPL{}
\end{itemize}

\noindent All leagues share the same set of rules. The \DSPL{} uses the \HSR{} platform shown in figure \ref{fig:toyotaHSR} and the \SSPL{} uses the \PEPPER{} platform shown in figure \ref{fig:softbank-pepper}.

\begin{minipage}{0.5\textwidth}
	\begin{figure}[H]
		\begin{center}
			\includegraphics[height=0.6\textwidth]{images/toyota_hsr.png}
			\caption{Toyota HSR}
			\label{fig:toyotaHSR}
		\end{center}
	\end{figure}
\end{minipage}
\begin{minipage}{0.5\textwidth}
	\begin{figure}[H]
		\begin{center}
			\includegraphics[height=0.6\textwidth]{images/softbank_pepper.png}
			\caption{Softbank / Aldebaran Pepper}
			\label{fig:softbank-pepper}
		\end{center}
	\end{figure}
\end{minipage}

%% %%%%%%%%%%%%%%%%%%%%%%%%%%%%%%%%%%%%%%%%%%%%%%%%%%%%%%%%%%%%%%%%%%%%%%%%%%%
%%
%%    author(s): RoboCupAtHome Technical Committee(s)
%%  description: Introduction - Competition
%%
%% %%%%%%%%%%%%%%%%%%%%%%%%%%%%%%%%%%%%%%%%%%%%%%%%%%%%%%%%%%%%%%%%%%%%%%%%%%%
\section{Competition}
The competition consists of two \emph{Stages} and the \FINAL{}. Each stage comprises a series of \iterm{Tests}. The best teams from \SONE{} advance to \STWO{} with more difficult tests. The competition ends with the \FINAL{} where the two highest ranked teams of each league compete to win.

%% %%%%%%%%%%%%%%%%%%%%%%%%%%%%%%%%%%%%%%%%%%%%%%%%%%%%%%%%%%%%%%%%%%%%%%%%%%%
%%
%%    author(s): RoboCupAtHome Technical Committee(s)
%%  description: Introduction - Awards
%%
%% %%%%%%%%%%%%%%%%%%%%%%%%%%%%%%%%%%%%%%%%%%%%%%%%%%%%%%%%%%%%%%%%%%%%%%%%%%%
\section{Awards}
\label{sec:introduction:awards}
All the awards need to be approved by the \RCF{}. Not all awards must be given.


\paragraph{Winner of the Competition}
\label{sec:introduction:winner}
Each league has 1st, 2nd, and 3rd place award trophies. If eight or fewer teams are participating, no 3rd place award trophy is given.

\newpage
\paragraph*{Note: } For the following awards, the \EC{} nominates a set of candidates from which the \TC{} elects the winner. One cannot nominate or vote for their own team.

\paragraph{Best Human-Robot Interface Award}
\label{sec:introduction:hriaward}
To honour outstanding Human-Robot Interfaces, the \HRIAward{} may be given to one of the participating teams. It is especially important that the interface is open and available to the \AtHome{} community.

\paragraph{Best Poster}
\label{sec:introduction:bestposter}
To foster scientific knowledge exchange and reward a teams' effort to present their contributions, all scientific posters of each league are eligible to receive the \DSPLPosterAward, \SSPLPosterAward, or \OPLPosterAward, respectively.

Posters are graded on presenting innovative and state-of-the-art research within a field with direct application to \RoboCup\AtHome{} in an appealing, easy-to-read way while demonstrating successful and clear results. In addition to be attractive and well-rated in the \PS{} (see~\ref{sec:setupdays:postersession}), the explained research must have impact in the team's performance during the competition.

\paragraph*{Note: } For the following award, the \TC{} and team leaders nominate a set of candidates from which the \EC{} elects the winner. One cannot nominate or vote for their own team.

\paragraph{Open-Source Software Award}
\label{sec:introduction:assaward}
For promoting software exchange and collaboration, \RoboCup\AtHome{} awards the best open source software contributions to the community. The software must be easy to read, properly documented, follow standard design patterns, be actively maintained, and meet IEEE software engineering metrics of scalability, portability, maintainability, fault tolerance, and robustness. In addition, the open sourced software must be made available as a framework-independent standalone library so it can be reused with any software architecture.

Candidates must send their application to the \TC{} at least one month before the competition in form of a short paper (max 4 pages) following the same format used for the \TDP{} (see~\refsec{sec:rules:application:tdp}). The paper should include a brief explanation of the approach, comparison with State-of-the-Art techniques, statement of the used metrics and software design patterns, and the name of the teams and other collaborators that are also using the software.

\paragraph{Skill Certificates}
\label{award:skill}
The @Home league features certificates for the robots best at a the skills below:
\begin{itemize}
	\item Navigation
	\item Manipulation
	\item Speech Recognition
	\item Person Recognition
\end{itemize}

A team is given the certificate if it scored at least 75\% of the attainable points for that skill.
This is counted over all tests and challenges, so e.g.~if the robot scores manipulation points during the Storing Groceries test, that will count towards the Best in Manipulation certificate.
The certificate will only be handed out if the team is \emph{not} the overall winner of the competition.

% Local Variables:
% TeX-master: "Rulebook"
% End:
