%% %%%%%%%%%%%%%%%%%%%%%%%%%%%%%%%%%%%%%%%%%%%%%%%%%%%%%%%%%%%%%%%%%%%%%%%%%%%
%%
%%    author(s): RoboCupAtHome Technical Committee(s)
%%  description: Introduction - Infrastructure
%%
%% %%%%%%%%%%%%%%%%%%%%%%%%%%%%%%%%%%%%%%%%%%%%%%%%%%%%%%%%%%%%%%%%%%%%%%%%%%%
\section{Infrastructure}
\label{sec:introduction:infrastructure}
\paragraph{RoboCup@Home Mailing List}
\label{sec:introduction:mailinglist}
The official \AtHome{} mailing list can be reached at:\\
\href{mailto:robocup-athome@lists.robocup.org}{\small\texttt{robocup-athome@lists.robocup.org}}. You can subscribe to the mailing list at: {\small\url{http://lists.robocup.org/cgi-bin/mailman/listinfo/robocup-athome}}

\paragraph{RoboCup@Home Web Page}
\label{sec:introduction:webpage}
The official \AtHome{} website that also hosts this rulebook can be found at: {\small\url{https://athome.robocup.org/}}

\paragraph{RoboCup@Home Rulebook Repository}
\label{sec:introduction:repo}
The official \AtHome{} \RR{} is where rules are publicly discussed before applying changes.
The entire \AtHome{} community is welcome and encouraged to actively participate in creating and discussing the rules. The \RR{} can be reached at: {\small\url{https://github.com/RoboCupAtHome/RuleBook/}}

\paragraph{RoboCup@Home Telegram Group}
\label{sec:introduction:telegramgroup}
The official \AtHome{} \TG{} is a communication channel for the \AtHome{} community where rules are discussed, announcements are made, and questions are answered.
Beyond supporting the technical aspects of the competition, the group is a meeting point to stay in contact with the community, foster knowledge exchange, and strengthen relationships.
The \TG{} can be reached at: {\small\url{https://t.me/RoboCupAtHome}}

\paragraph{RoboCup@Home Wiki}
\label{sec:introduction:wiki}
The official \AtHome{} \WIKI{} is meant to be a central place to collect information on all topics related to the \AtHome league. It was set up to simplify and unify the exchange of relevant information.
This includes but is certainly not limited to hardware, software, media, data, and more.
The \WIKI{} can be reached at: {\small\url{https://github.com/RoboCupAtHome/AtHomeCommunityWiki/wiki}}
