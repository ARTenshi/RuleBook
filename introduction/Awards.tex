%% %%%%%%%%%%%%%%%%%%%%%%%%%%%%%%%%%%%%%%%%%%%%%%%%%%%%%%%%%%%%%%%%%%%%%%%%%%%
%%
%%    author(s): RoboCupAtHome Technical Committee(s)
%%  description: Introduction - Awards
%%
%% %%%%%%%%%%%%%%%%%%%%%%%%%%%%%%%%%%%%%%%%%%%%%%%%%%%%%%%%%%%%%%%%%%%%%%%%%%%
\section{Awards}
\label{sec:awards}

The \AtHome{} league features the \iterm{awards} described below.
Note that all awards need to be approved by the \RCF; based on a decision by the RCF, some of them may not be given.

\subsection{Winner of the Competition}
\label{award:winner}

For each league, there will be 1st, 2nd, and 3rd place award trophies (or first and second place only if the number of teams in a league is eight or less).

%
% As of 2017, the Execs have decided to remove the Innovation Award since
% is rarely given and its discussion is time consuming.
%
% \subsection{Innovation award}
% \label{award:innovation}
% To honour outstanding technical and scientific achievements as well as applicable solutions in the @Home league, a special \iterm{innovation award} may be given to one of the participating teams. Special attention is being paid to making usable robot components and technology available to the @Home community.
%
% The \iaterm{Executive Committee}{EC} members from the RoboCup@Home league nominate a set of candidates for the award. The \iaterm{Technical Committee}{TC} elects the winner. A TC member whose team is among the nominees is not allowed to vote.
%
% There is no innovation award in case no outstanding innovation and no nominees, respectively.


%
% As of 2017, the TC have decided to add an award for the best alternate HRI method
% to bypass speech recognition.
%
\subsection{Best Human-Robot Interface Award}
\label{award:hri}

To honor outstanding human-robot interfaces developed for interacting with robots in \AtHome{}, a special \HRIAward{} may be given to one of the participating teams.
Special attention is paid to making the interface open and available to the \AtHome{} community.

The \AtHome{} EC members nominate a set of candidates for the award and the TC elects the winner.
A TC member whose team is among the nominees is not allowed to vote.
There is no \HRIAward{} in case the EC decides that there is no outstanding interface, and thus no nominees.

\subsection{Best Poster Award}
\label{award:poster}

To foster scientific knowledge exchange and reward the teams' efforts to present their research contributions, all scientific posters of each league are evaluated and have the chance of receiving the \DSPLPosterAward, the \OPLPosterAward, or the \SSPLPosterAward, respectively.

Candidate posters must present innovative and state-of-the-art research within a field with a direct application to \AtHome, and demonstrate successful and clear results in an easy-to-understand way.
In addition to being attractive and well-rated in the \PS{} (see~\refsec{sec:poster_teaser_session}), the described research must have impact in the team's performance during the competition.

The \AtHome{} EC members nominate a set of candidates for the award and the TC elects the winner.
A TC member whose team is among the nominees is not allowed to vote.

%
% As of 2013, the Execs have decided to remove the Innovation Award due to
% the lack of interest of the participants
%
% \subsection{Winner of the Technical Challenge}
% In parallel to the regular competition, the RoboCup@Home league features a \iterm{Technical Challenge}. The winner of the Technical Challenge is given a special \iterm{award for winning the Technical Challenge}.
%
% As with the innovation award, the award for winning the Technical Challenge is not given in case no team shows a \emph{sufficient performance}. The decision which team wins the Technical Challenge, and if the award is given at all, is conducted by the \iaterm{Technical Committee}{TC}.

\subsection{Open Challenge Award}
\label{award:oc}

To encourage teams to present their research to the other teams in the competition off-hours, \AtHome{} grants the \OCAward{} to the best open demonstration presented during the competition.
This award is granted only if a team has demonstrated innovative research that is related to the global objectives of \AtHome; thus, the award is not necessarily granted.

The \AtHome{} TC members, with a recommendation from the team leaders, nominate a set of candidates for the award; the EC decides if the award should be granted and elects the winner.
A TC member is not allowed to nominate their own team without a recommendation from the team leaders.

\subsection{Skill Certificates}
\label{award:skill}

The \AtHome{} league features certificates for best demonstrated skills in \NAV, \MAN, \PerRec, and \NLP.
A team is given the certificate if it scores at least 75\% of the attainable points for that skill.
This is counted over all tests and challenges, so, for example, if a robot scores manipulation points during the \emph{Help-me-Carry} test to open the door, that will count for the \MAN{} certificate.
Note that the certificate will only be handed out if the team is \emph{not} the overall winner of the competition.

\subsection{Open-source software award}
\label{award:oss}

Since Nagoya 2017, RoboCup@Home awards the best contribution to the community by means of an open source software solution.
To be eligible for the award, the software must be easy to read, have proper documentation, follow standard design patterns, be actively maintained, and meet the IEEE software engineering metrics of scalability, portability, maintainability, fault tolerance, and robustness.
In addition, the open sourced software must be made available as a framework-independent standalone library so it can be reused with any software architecture.

Candidates must send their application to the TC at least one month before the competition by means of a short paper (maximum 4 pages), following the same format used for the \TDP{} (see~\refsec{rule:website_tdp}), including a brief explanation of the approach, comparison with state-of-the-art techniques, statement of the used metrics and software design patterns, and the name of the teams and other collaborators that are also using the software being described.

The \AtHome{} TC members nominate a set of candidates for the award and the EC elects the winner.
An EC/TC member whose team is among the nominees is not allowed to vote.


% \subsection{Procter \& Gamble Dishwasher Challenge Award}
% \label{award:pandg}
% \textit{Procter \& Gamble} gives an special award to the winner of the \textit{Procter \& Gamble: Clean the Table} task (described in \refsec{test:clean-the-table}), typically to the team scoring higher in the task.
% All teams can participate and compete for this award, regardless of whether they advanced to the Stage II or not, and get the award.

% The award for winning the \textit{Procter \& Gamble Dishwasher Challenge Award} is not given in case no team shows a \emph{sufficient performance} in the aformentioned task. The decision on which team wins the \textit{Procter \& Gamble Dishwasher Challenge Award} task, and if the award is given at all, is conducted by \textit{Procter \& Gamble}.
