%% %%%%%%%%%%%%%%%%%%%%%%%%%%%%%%%%%%%%%%%%%%%%%%%%%%%%%%%%%%%%%%%%%%%%%%%%%%%
%%
%%    author(s): RoboCupAtHome Technical Committee(s)
%%  description: Introduction - Awards
%%
%% %%%%%%%%%%%%%%%%%%%%%%%%%%%%%%%%%%%%%%%%%%%%%%%%%%%%%%%%%%%%%%%%%%%%%%%%%%%
\section{Awards}
\label{sec:introduction:awards}
All the awards need to be approved by the \RCF{}. Not all awards must be given.


\paragraph{Winner of the Competition}
\label{sec:introduction:winner}
Each league has 1st, 2nd, and 3rd place award trophies. If eight or fewer teams are participating, no 3rd place award trophy is given.

\newpage
\paragraph*{Note: } For the following awards, the \EC{} nominates a set of candidates from which the \TC{} elects the winner. One cannot nominate or vote for their own team.

\paragraph{Best Human-Robot Interface Award}
\label{sec:introduction:hriaward}
To honour outstanding Human-Robot Interfaces, the \HRIAward{} may be given to one of the participating teams. It is especially important that the interface is open and available to the \AtHome{} community.

\paragraph{Best Poster}
\label{sec:introduction:bestposter}
To foster scientific knowledge exchange and reward a teams' effort to present their contributions, all scientific posters of each league are eligible to receive the \DSPLPosterAward, \SSPLPosterAward, or \OPLPosterAward, respectively.

Posters are graded on presenting innovative and state-of-the-art research within a field with direct application to \RoboCup\AtHome{} in an appealing, easy-to-read way while demonstrating successful and clear results. In addition to be attractive and well-rated in the \PS{} (see~\ref{sec:setupdays:postersession}), the explained research must have impact in the team's performance during the competition.

\paragraph*{Note: } For the following award, the \TC{} and team leaders nominate a set of candidates from which the \EC{} elects the winner. One cannot nominate or vote for their own team.

\paragraph{Open-Source Software Award}
\label{sec:introduction:assaward}
For promoting software exchange and collaboration, \RoboCup\AtHome{} awards the best open source software contributions to the community. The software must be easy to read, properly documented, follow standard design patterns, be actively maintained, and meet IEEE software engineering metrics of scalability, portability, maintainability, fault tolerance, and robustness. In addition, the open sourced software must be made available as a framework-independent standalone library so it can be reused with any software architecture.

Candidates must send their application to the \TC{} at least one month before the competition in form of a short paper (max 4 pages) following the same format used for the \TDP{} (see~\refsec{sec:rules:application:tdp}). The paper should include a brief explanation of the approach, comparison with State-of-the-Art techniques, statement of the used metrics and software design patterns, and the name of the teams and other collaborators that are also using the software.


\paragraph{Open Challenge Award}
\label{sec:introduction:ocaward}

To encourage teams to present their research to the rest of the league, \AtHome{} grants the \OCAward{} to the best open demonstration presented during the competition. This award is granted only if a team has demonstrated innovative research that is related to the global objectives of \AtHome{}. 


\paragraph{Skill Certificates}
\label{award:skill}
The @Home league features certificates for the robots best at a the skills below:
\begin{itemize}
	\item Navigation
	\item Manipulation
	\item Speech Recognition
	\item Person Recognition
\end{itemize}

A team is given the certificate if it scored at least 75\% of the attainable points for that skill.
This is counted over all tests and challenges, so e.g.~if the robot scores manipulation points during the Storing Groceries test, that will count towards the Best in Manipulation certificate.
The certificate will only be handed out if the team is \emph{not} the overall winner of the competition.