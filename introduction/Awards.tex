%% %%%%%%%%%%%%%%%%%%%%%%%%%%%%%%%%%%%%%%%%%%%%%%%%%%%%%%%%%%%%%%%%%%%%%%%%%%%
%%
%%    author(s): RoboCupAtHome Technical Committee(s)
%%  description: Introduction - Awards
%%
%% %%%%%%%%%%%%%%%%%%%%%%%%%%%%%%%%%%%%%%%%%%%%%%%%%%%%%%%%%%%%%%%%%%%%%%%%%%%
\section{Awards}
\label{sec:awards}
All the awards need to be approved by the RoboCup Federation (RCF). Based on RCF's decisions, some of them may not be given.

The RoboCup@Home league features the following \iterm{awards}.

\subsection{Winner of the competition}
\label{award:winner}
For each league, there will be a 1st, 2nd, and 3rd place award trophies (first and second place only when the number of teams is eight or less).

%
% As of 2017, the Execs have decided to remove the Innovation Award since
% is rarely given and its discussion is time consuming.
%
% \subsection{Innovation award}
% \label{award:innovation}
% To honour outstanding technical and scientific achievements as well as applicable solutions in the @Home league, a special \iterm{innovation award} may be given to one of the participating teams. Special attention is being paid to making usable robot components and technology available to the @Home community.
%
% The \iaterm{Executive Committee}{EC} members from the RoboCup@Home league nominate a set of candidates for the award. The \iaterm{Technical Committee}{TC} elects the winner. A TC member whose team is among the nominees is not allowed to vote.
%
% There is no innovation award in case no outstanding innovation and no nominees, respectively.


%
% As of 2017, the TC have decided to add an award for the best alternate HRI method
% to bypass speech recognition.
%
\subsection{Best Human-Robot Interface award}
\label{award:hri}
To honour outstanding Human-Robot Interfaces developed for interacting with robots in the @Home league, a special \iterm{Best HR Interface award} may be given to one of the participating teams. Special attention is being paid to making the interface open and available to the @Home community.

The \iaterm{Executive Committee}{EC} members from the RoboCup@Home league nominate a set of candidates for the award. The \iaterm{Technical Committee}{TC} elects the winner. A TC member whose team is among the nominees is not allowed to vote.

There is no Best HR Interface award in case no outstanding interface and no nominees, respectively.

\subsection{Best Poster}
\label{award:poster}
To foster scientific knowledge exchange and reward the teams' effort to present their contributions, as of 2017 all scientific posters of each League will be evaluated, having the chance of receiving the award for the \iterm{Best RoboCup @Home DSPL Poster}, the \iterm{Best RoboCup @Home OPL Poster}, or the \iterm{Best RoboCup @Home SSPL Poster}, respectively.

Candidate posters must present innovative and State-of-the-Art research within a field with direct application in RoboCup @Home in an appealing, easy-to-read way; demonstrating successful and clear results. In addition to be attractive and well-rated in the Poster Session (see~\refsec{sec:poster_teaser_session}), the explained research must have impact in the team's performance during the competition.

The \iaterm{Executive Committee}{EC} members from the RoboCup@Home league nominate a set of candidates for the award. The \iaterm{Technical Committee}{TC} elects the winner. A TC member whose team is among the nominees is not allowed to vote.

%
% As of 2013, the Execs have decided to remove the Innovation Award due to
% the lack of interest of the participants
%
% \subsection{Winner of the Technical Challenge}
% In parallel to the regular competition, the RoboCup@Home league features a \iterm{Technical Challenge}. The winner of the Technical Challenge is given a special \iterm{award for winning the Technical Challenge}.
%
% As with the innovation award, the award for winning the Technical Challenge is not given in case no team shows a \emph{sufficient performance}. The decision which team wins the Technical Challenge, and if the award is given at all, is conducted by the \iaterm{Technical Committee}{TC}.

\subsection{Open Challenge award}
\label{award:oc}

To encourage teams to present their research in off-hours of the competition to the rest of the teams, RoboCup@Home grants the \iterm{open challenge award} to the best open demonstration presented during the competition. This award is granted only if there a team has demonstrated innovative research that is related to the global objectives of RoboCup@Home. Thus, this award may be not be granted.

The \iaterm{Technical Committee}{TC} members from the RoboCup@Home league, with recommendations from team leaders, nominate a set of candidates for the award (a TC member whose team is among the nominees is not allowed to nominate). The \iaterm{Executive Committee}{EC} decides if the award should be granted and elects the winner.

\subsection{Skill Certificates}
\label{award:skill}
The @Home league features certificates for the robots best at a the skills below:
\begin{itemize}
   \item Navigation
   \item Manipulation
   \item Speech Recognition
   \item Person Recognition
  \end{itemize}

A team is given the certificate if it scored at least 75\% of the attainable points for that skill.
This is counted over all tests and challenges, so e.g.~if the robot scores manipulation points during the Help-me-Carry test to open the door, that will count for the Manipulation-certificate.
The certificate will only be handed out if the team is \emph{not} the overall winner of the competition.


\subsection{Open-source software award}
\label{award:oss}
Traditionally --since Nagoya 2017-- RoboCup@Home awards the best contribution to the community by means of open source software solutions. The software must be easy to read, properly documented, follow standard design patterns, be actively maintained, and meet IEEE software engineering metrics of scalability, portability, maintainability, fault tolerance, and robustness. In addition, the open sourced software must be made available as a framework-independent standalone library so it can be reused with any software architecture.

Candidates must send their application to the \iaterm{Technical Committee}{TC} at least one month before the competition by means of a short paper (max 4 pages) following the same format used for the \iterm{team description paper} (see~\refsec{rule:website_tdp}), including a brief explanation of the approach, comparison with State-of-the-Art techniques, statement of the used metrics and software design patterns, and the name of the teams and other collaborators that are also using the software being described.

The \iaterm{Technical Committee}{TC} members from the RoboCup@Home league nominate a set of candidates for the award. The \iaterm{Executive Committee}{EC} elects the winner. A EC/TC member whose team is among the nominees is not allowed to vote.


\subsection{Procter \& Gamble Dishwasher Challenge Award}
\label{award:skill}
\textit{Procter \& Gamble} gives an special award to the winner of the \textit{Procter \& Gamble Dishwasher Challenge}, typically to the team scoring higher in the challenge.
All teams can participate and compete for this award, regardless of whether they advanced to the Stage II or not, and get the award.

The award for winning the \textit{Procter \& Gamble Dishwasher Challenge} is not given in case no team shows a \emph{sufficient performance}. The decision on which team wins the \textit{Procter \& Gamble Dishwasher Challenge}, and if the award is given at all, is conducted by \textit{Procter \& Gamble}.
