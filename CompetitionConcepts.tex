\chapter{Concepts behind the competition}
\label{chap:concepts}
A set of conceptual key criteria builds the basis for the RoboCup@Home Competitions. These criteria are to be understood as a common agreement on the general concept of the competition. The concrete rules are listed in Chapter \refsec{chap:rules}.

\section{Lean set of rules}
\label{concept:lean_set_of_rules}
To allow for different, general and transmissible approaches in the RoboCup@Home competitions, the rule set should be as lean as possible. Still, to avoid rule discussions during the competition itself, it should be very concrete leaving no room for diverse interpretation.

If, during a competition, there are any discrepancies or multiple interpretations, a decision will be made by the \iaterm{Technical Committee}{TC} and the referees on site.

\paragraph*{Note: } Once the test scoresheet has been signed or the scores has been published, the TC decision is irrevocable.

\section{Autonomy \& Mobility}
\label{concept:autonomy_and_mobility}
All robots participating in the RoboCup@Home competition have to be \emph{autonomous} and \emph{mobile}.

An aim of RoboCup@Home is to foster mobile autonomous service robotics and natural human-robot interaction. As a consequence humans are not allowed to directly (remote) control the robot. This also includes verbally remote controlling the robot.

Furthermore, the specific tasks must not be solved using \emph{open loop control}.

\section{Aiming for applications}
\label{concept:aiming_for_applications}
To foster advance in technology and to keep the competition interesting, the scenario and the tests will steadily increase in complexity. While in the beginning necessary abilities are being tested, tests will focus more and more on real applications with a rising level of uncertainty. Useful, robust, general, cost effective, and applicable solutions are rewarded in RoboCup@Home.

\section{\iterm{Social relevance}}
\label{concept:social_relevance}
The competition and the included tests should produce socially relevant results. The aim is to convince the public about the usefulness of autonomous robotic applications. This should be done by showing applications where robots directly help or assist humans in everyday life situations. Examples are: Personal robot assistant, guide robot for the blind, robot care for elderly people, etc. Such socially relevant results are rewarded in RoboCup@Home.

\section{Scientific value}
\label{concept:scientific_value}
RoboCup@Home should not only show what can be put into practice today, but should also present new approaches, even if they are not yet fully applicable or demand a very special configuration or setup. Therefore high scientific value of an approach is rewarded.

\section{Time constraints}
\label{concept:time_constraints}
Setup time as well as time for the accomplishment of the tests is very limited, to allow for many participating teams and tests, and to foster simple setup procedures.

\section{No standardized scenario}
\label{concept:no_standardized_scenario}
The \iterm{scenario} for the competition should be simple but effective, available world-wide and low in costs. As uncertainty is part of the concept, no standard scenario will be provided in the RoboCup@Home League. One can expect that the scenario will look typical for the country where the games are hosted.

The scenario is something that people encounter in daily life. It can be a home environment, such as a living room and a kitchen, but also an office space, supermarket, restaurant etc. The scenario should change from year to year, as long as the desired tests can still be executed.

Furthermore, tests may take place outside of the scenario, i.e., in an previously unknown environment like, for example, a public space nearby.

\section{Attractiveness}
\label{concept:attractiveness}
The competition should be attractive for the audience and the public. Therefore high attractiveness and originality of an approach should be rewarded.

\section{\iterm{Community}}
\label{concept:community}
Though having to compete against each other during the competition, the members of the RoboCup@Home league are expected to cooperate and exchange knowledge to advance technology together. The \iterm{RoboCup@Home mailing list} can be used to get in contact with other teams and to discuss league specific issues such as rule changes, proposals for new tests, etc.
% Since 2007 there is also the \iterm{RoboCup@Home Wiki} (see \refsec{sec:at_home_wiki}) which serves as a central place to collect information relevant for the @Home league.
Every team is expected to share relevant technical, scientific (and team related) information there and in its \iterm{team description paper} (see \refsec{rule:website_tdp}) through the team's website.

All teams are invited to submit papers on related research to the RoboCup Symposium which accompanies the annual RoboCup World Championship.

\section{Desired abilities}
\label{concept:desired_abilities}
This is a list of the current desired technical abilities which the tests in RoboCup@Home will focus on.

\begin{itemize}
\item Navigation in dynamic environments
\item Fast and easy calibration and setup \\ The ultimate goal is to have a robot up and running out of the box.
\item Object recognition
\item Object manipulation
\item Detection and Recognition of Humans
\item Natural human-robot interaction
\item Speech recognition
\item Gesture recognition
\item Robot applications \\ RoboCup@Home is aiming for applications of robots in daily life.
\item Ambient intelligence, e.g., communicating with surrounding devices, getting information from the internet etc.
\end{itemize}


% Local Variables:
% TeX-master: "Rulebook"
% End:
