\chapter{Concepts Behind the Competition}
\label{chap:concepts}
A set of key concepts apply to every \RoboCup\AtHome{} competition and the performed tests.

\section{Clarity}
\label{sec:concepts:leanrules}
The rules and test descriptions aim to be as clear and understandable as possible. At the competition, if any ambiguity or misunderstanding arises before a test, they need to be discussed in the \TLM{}. Decisions made in the \abb{TLM} are binding. The \TC{} and referees on site will decide on anything coming up during or after a test.

\paragraph*{Note: } Once a scoresheet has been signed by the team leader or the scores have been published, the \abb{TC} decision is irrevocable.

\section{Autonomy}
\label{sec:concepts:autonomy}
All robots participating in the \RoboCup\AtHome{} competition have to be \emph{autonomous}. This means no human is allowed to remote control the robot during a test. Furthermore, a test must not be solved using \OLC{}.

\section{Applicability}
\label{sec:concepts:applicability}
The tests should reward useful, robust, general, cost effective, and applicable solutions. The tests should increase in difficulty and complexity each year.

\section{Social Relevance}
\label{sec:concepts:socialrelevance}
The tests should show socially relevant results. The aim is to convince the public about the usefulness of autonomous robot applications in domestic settings by directly assisting and helping humans.

\section{Scientific Value}
\label{sec:concepts:scientificvalue}
The tests should allow teams to show novel approaches with high scientific value.

\section{Time Constraints}
\label{sec:concepts:timeconstraints}
Setup and test time is limited to allow for many participating teams and to emphasize the competition aspect of \AtHome{}.

\section{Non Standard Scenario}
\label{sec:concepts:nonstandardscenario}
In order to reward robust and general solutions, \RoboCup\AtHome{} has no standard scenario. It should resemble a typical domestic setting of the host country. Furthermore, tests may take place outside of the scenario, i.e., in an previously unknown environment like, for example, a nearby public space.

\section{Appeal}
\label{sec:concepts:appel}
The competition should appeal to the audience and the public. Therefore high attractiveness and originality of an approach should be rewarded.

\section{Community}
\label{sec:concepts:community}
Although teams compete against each other, the members of the \AtHome{} league are expected to cooperate and exchange knowledge to advance technology together. Every team is encouraged to share relevant technical, scientific, and team related information through the \TDP{} and by participating in the various communication channels (see \ref{sec:introduction:infrastructure}).


% Local Variables:
% TeX-master: "Rulebook"
% End:
