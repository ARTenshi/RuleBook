\section{General Purpose Service Robot [Housekeeper]}
\label{test:gpsr}
Similar to a modern smart speaker, the robot can be asked to do anything involving functionalities from \SONE{} of this rulebook or any previous rulebook.\\

\noindent \textbf{Main Goal:} Execute 3 commands requested by the operator.\\

\noindent \textbf{Optional Goals:}
\begin{enumerate}[nosep]
	\item Understand a command given by a naive operator
	\item Provide an audio recording and transcript
	\item Autonomously leave the \Arena{}
\end{enumerate}

\subsection*{Focus}
\emph{Task planning}, \emph{object/people detection and recognition}, \emph{object feature recognition}, \emph{object manipulation}

% %% %%%%%%%%%%%%%%%%%%%%%%%%%%%%%%%%%%%%%%%%%%%%%%%%%%%%%%
%
% Setup
%
% %% %%%%%%%%%%%%%%%%%%%%%%%%%%%%%%%%%%%%%%%%%%%%%%%%%%%%%%
\subsection*{Setup}
\begin{enumerate}[nosep]
	\item \textbf{Location:} The task takes place inside the \Arena{}, but some commands may require the robot to go out. The \Arena{} is in its nominal state for this task.
	\item \textbf{Start location:} The robot starts outside the \Arena{}. When the door opens, the robot moves towards the \textit{Instruction Point}.
	\item \textbf{Operators:} A \emph{professional operator} (the referee) commands the robot to execute a task.
\end{enumerate}


% %% %%%%%%%%%%%%%%%%%%%%%%%%%%%%%%%%%%%%%%%%%%%%%%%%%%%%%%
%
% Additional Rules
%
% %% %%%%%%%%%%%%%%%%%%%%%%%%%%%%%%%%%%%%%%%%%%%%%%%%%%%%%%
\subsection*{Additional Rules and Remarks}
\begin{enumerate}[nosep]
	\item \textbf{Command generator:} Tasks will be generated using the official \emph{GPSR Command Generator}\footnote{\url{https://github.com/kyordhel/GPSRCmdGen}}, which will be available two months prior to the competition.

	\item \textbf{Naive Operators:} Optionally, commands can be issued by a \emph{naive operator}, i.e. a person from the audience with no robotics background.
	In this case, the referee gives the command to the naive operator, who will then issue it to the robot (rephrasing is allowed).
	If the robot consistently fails to understand the naive operator (e.g. after two retries), teams can default to a custom operator.
	\textbf{Remark:} Referees are not allowed to instruct naive operators on how to operate the robot.
	Teams attempting to instruct or bias the operator will be disqualified from the task.

	\item \textbf{Data recording:} Only when using naive operators, a team can get an additional bonus by providing the recording and transcript of the issued commands.

	\item \textbf{Instruction point:} At the beginning of the test, as well as after finishing the first and second command, the robot moves to the \textit{Instruction Point}.

	\item \textbf{Leaving the \Arena{}:} A bonus score can be earned if the robot autonomously leaves the \Arena{} after successfully executing all three given commands.

	\item \textbf{Deus ex Machina:} The scores are reduced if human assistance is received, in particular:
	\begin{itemize}
		\item using a custom operator
		\item bypassing speech recognition by using an alternative HRI
		\item receiving human assistance to accomplish a task
		\item instructing a human assistant to perform the whole task
	\end{itemize}
\end{enumerate}

\subsection*{Referee Instructions}
\begin{itemize}[nosep]
	\item Provide the commands to the operators.
\end{itemize}

\subsection*{OC Instructions}

2h before the test:
\begin{itemize}[nosep]
	\item Generate the robot commands (don't reveal them to the teams!).
	\item Announce the location of the instruction point.
	\item Recruit volunteers to assist during the test.
	\newline
\end{itemize}

\noindent During the test:
\begin{itemize}[nosep]
	\item Rearrange the arena so that it is in its nominal condition before each command.
\end{itemize}

\subsection*{Score Sheet}
The maximum time for this test is 5 minutes.

\begin{scorelist}
	\scoreheading{Main Goal}
	\scoreitem[3]{400}{Executing the task associated with each command}

	\scoreheading{Bonus Rewards}
	\scoreitem[3]{100}{Understanding a command given by a non-expert operator}

	\scoreheading{Deus Ex Machina Penalties}
	\penaltyitem[3]{50}{Using a custom operator}
	\penaltyitem[3]{50}{Bypassing speech recognition}
	\penaltyitem[3]{400}{Instructing a human to perform the task}
\end{scorelist}


% Local Variables:
% TeX-master: "Rulebook"
% End:
