\section{Hand Me That [Party Host]}
\label{test:hand-me-that}
A guest at the party speaks English, but with only a limited vocabulary. The robot will assist them in obtaining things that they gesture for.

%\subsection{Focus}
%Joint attention is a well-studied and important task in Human-Robot Interaction. The goal of this task is to really challenge the teams to perform a hard HRI task.

\subsection*{Main Goal}
The robot identifies (touching or naming) each object at which the operator is pointing at.

\noindent\textbf{Reward:} 2500pts (500pts per object)


% %% %%%%%%%%%%%%%%%%%%%%%%%%%%%%%%%%%%%%%%%%%%%%%%%%%%%%%%
%
% Setup
%
% %% %%%%%%%%%%%%%%%%%%%%%%%%%%%%%%%%%%%%%%%%%%%%%%%%%%%%%%
\subsection*{Setup}
\begin{enumerate}[nosep]
	\item \textbf{Location:} This takes place in a room in the arena.

	\item \textbf{Starting position:} The robot and the operator stand in a predefined location announced beforehand % (OC instructions: announce this 2 hours before the test).

	\item \textbf{Groups of Objects:} There are five groups of 2--5 objects randomly placed along the room

	\item \textbf{Deck:} The referee has a deck of objects to request, one per group, sorted by distance.

\end{enumerate}


% %% %%%%%%%%%%%%%%%%%%%%%%%%%%%%%%%%%%%%%%%%%%%%%%%%%%%%%%
%
% Procedure
%
% %% %%%%%%%%%%%%%%%%%%%%%%%%%%%%%%%%%%%%%%%%%%%%%%%%%%%%%%
\subsection*{Procedure}
\begin{enumerate}[nosep]
	\item \textbf{Pick an object} The robot asks the operator: \emph{what do you need?}. % We rule out natural language interaction
	Then
	\begin{enumerate}[nosep]
		\item The operator walks near to the object and points at it.
		\item The asks as many questions as necessary.
		\item The operator replies to each question (most likely with \emph{yes}, \emph{no}, \emph{I don't know}, etc).
		\\\textbf{Remark:} The operator does not know the name of the object.
	\end{enumerate}
  \item \textbf{Repeat} Repeat up to 5 times for the maximum score.
\end{enumerate}


% %% %%%%%%%%%%%%%%%%%%%%%%%%%%%%%%%%%%%%%%%%%%%%%%%%%%%%%%
%
% Additional Rules
%
% %% %%%%%%%%%%%%%%%%%%%%%%%%%%%%%%%%%%%%%%%%%%%%%%%%%%%%%%
\subsection*{Additional rules and remarks}
\begin{enumerate}[nosep]
	\item \textbf{Keep going:} The robot should keep trying to determine the referred to object until they score or run out of time.

	\item \textbf{Skipping groups:} The robot may say \emph{Pass} or \emph{I give up} to try with the next object.

	\item \textbf{Incorrect guesses:} Incorrect guesses reduce the value of the correct guess by 200 points, each, the first two times. Guessing correctly on the third or fourth attempt is worth 100 points. After the fourth guess is worth no points.

	\item\textbf{Colors and categories:} Asking for the color or category of a pointed object applies a penalty of 400 points for that particular object.

	\item\textbf{Uneducated operator:} The referee may instruct the operator to answer \emph{I don't understand} or \emph{I don't know} if the robot asks complex questions or is attempting blind guessing.

	\item \textbf{Groups of Objects:} A group consists of 2--5 random standard objects (see~\refsec{rule:scenario_objects}), separated one from another for about 2.5--10cm.
	The average distance between the starting position and each group ranges between 50cm and 150cm.

\end{enumerate}

\subsection*{Referee instructions}

The referee needs to
\begin{itemize}[nosep]
	\item Rearrange and mix groups between runs
	\item Verify that the operator is pointing at the right item
\end{itemize}

\subsection*{OC instructions}
During Setup days:
\begin{itemize}[nosep]
	\item Announce the starting position of the robot.
\end{itemize}


\subsection*{Score sheet}

The maximum time for this test is 10 minutes.

\begin{scorelist}
	\scoreheading{Communicating}
	\scoreitem[5]{500}{Correctly determine an item on the first attempt}
	\penaltyitem[5]{-200}{Correctly determine an item on the second attempt}
	\penaltyitem[5]{-400}{Correctly determine an item on a subsequent attempt}
	\penaltyitem[5]{-150}{Asking 1 clarifying question.}
	\penaltyitem[5]{-300}{Asking 2 clarifying questions.}
	\penaltyitem[5]{-450}{Asking 3 clarifying questions.}

	% No longer necessary, computes automatically
	% \setTotalScore{1000}
\end{scorelist}


% Local Variables:
% TeX-master: "Rulebook"
% End:


% Local Variables:
% TeX-master: "Rulebook"
% End:
