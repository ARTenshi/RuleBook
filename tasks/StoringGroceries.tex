\section{Storing Groceries [House Holder]}
The robot stores groceries into the shelf next to the objects of the same kind that are already there (e.g. for instance by placing apples near pears and bananas).

% \subsection{Focus}
% This test focuses on the detection and recognition of objects and their features, as well as object manipulation.

\subsection{Main Goal}
The robot has to move 5 out of 10 objects from a nearby table into the shelf. Placed objects must be placed next to objects of the same Category or grouped by similarity.

\noindent\textbf{Reward:} 500pts\\

\noindent\textbf{HINT:} The robot can ask the referee where to place the carried object (relative positions or pointing are both allowed).

\subsection{Bonus rewards}
\begin{enumerate}[nosep]
	\item Opening the shelf door (300pts)
	\item Moving a \emph{tiny} object (100pts)
	\item Moving a \emph{heavy} object (100pts)
\end{enumerate}

% %% %%%%%%%%%%%%%%%%%%%%%%%%%%%%%%%%%%%%%%%%%%%%%%%%%%%%%%
%
% Setup
%
% %% %%%%%%%%%%%%%%%%%%%%%%%%%%%%%%%%%%%%%%%%%%%%%%%%%%%%%%
\subsection{Setup}
\begin{enumerate}
	\item \textbf{Location:} The testing area has a shelf and a table.
	The distance between the Table and the Shelf cannot exceed 2 meters.

	\item \textbf{Shelf:} The shelf contains at least 10 objects arranged in groups of 2 or more, either by category or likeliness.
	The shelf has at least one free space for starting a new set.

	\item \textbf{Shelf door:} The shelf door is open by default.
	The team leader can, request the door to be closed and score additional points for opening it. If the robot fails to open the door, it must clearly state it and request the referee to open it.

	\item \textbf{Objects:} Some of the objects are placed behind the door and cannot be accessed unless the door is open.

	\item \textbf{Table:} The table can have up to 10 objects, but never less than 5.
	In small tables, objects will be added as the robot frees up space.
\end{enumerate}


% %% %%%%%%%%%%%%%%%%%%%%%%%%%%%%%%%%%%%%%%%%%%%%%%%%%%%%%%
%
% Additional Rules
%
% %% %%%%%%%%%%%%%%%%%%%%%%%%%%%%%%%%%%%%%%%%%%%%%%%%%%%%%%
\subsection{Additional rules and remarks}
\begin{enumerate}
	\item \textbf{Clear area:} The robot may assume that the working area is clear, (i.e. can move slightly backwards for its task).

	\item \textbf{Object distribution:} The 10 objects to be moved are evenly distributed in random fashion including
	4 known objects,
	4 alike objects, and
	2 unknown objects.
	Among these, the robot will always find
	a heavy object,
	a tiny object, and 
	an amorphous object.

	\item \textbf{Table} The table's rough location will be announced beforehand, having its position either left, right, or behind the robot.
\end{enumerate}

\newpage
\subsection{OC instructions}

\textbf{2 hours before the test}
\begin{itemize}
	\item Announce which table will be used in the test.
	\item Announce a rough location for the table.
\end{itemize}

\subsection{Referee instructions}
The referee needs to
\begin{itemize}
	\item Place the objects in the shelf, grouping them by likeliness.
	\item Open the door of the shelf.
	\item Place objects on the table.
\end{itemize}


% \newpage
% \subsection{Score sheet}
% 
The maximum time for this test is 5 minutes.

\begin{scorelist}[attempts=4,%
datarecording=true,%
datarecordingbonus=5000,%
outstanding=true,%
outstandingpc=20,%
]
	\scoreheading{Main Goal}
	\scoreitem{500}{Move 5 objects next to their peers in the shelf}
	\penaltyitem[5]{-30}{Receiving human help (point at target location)}
	\penaltyitem[5]{-100}{Receiving human help (move object)}

	\scoreheading{Bonus rewards}
	\scoreitem{300}{Opening the shelf door without human help}
	\scoreitem{100}{Moving a \emph{tiny} object}
	\scoreitem{100}{Moving a \emph{heavy} object}

	% No longer necessary, computes automatically
	% \setTotalScore{1000}
\end{scorelist}


% Local Variables:
% TeX-master: "Rulebook"
% End:


% Local Variables:
% TeX-master: "Rulebook"
% End:
