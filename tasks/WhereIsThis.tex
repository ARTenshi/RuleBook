\section{Where is This? [Party Host]}
\label{test:where-is-this}
The robot has to show people where they can find places in the arena (e.g.~\emph{Where is the TV?}). The robot has to tell the operator how to get there (in a summarized fashion) and then physically take the operator there in a type of tour guide.


\subsection*{Main Goal}
Give accurate directions and guide at least 3 people.

\noindent\textbf{Reward:} 300pts (100pts per direction/guidance).

\subsection*{Bonus rewards}
\begin{enumerate}[nosep]
	\item Give accurate directions and guide 3 more people 600pts (200pts each).
	\item Provide new instructions based on the previous ones 300pts (100pts each).
	\item Provide directions to a naive operator 300pts (50pts each).
\end{enumerate}

\subsection*{Setup}
\begin{itemize}[nosep]
	\item \textbf{Locations:} All predefined locations inside the arena.

	\item \textbf{Start Location}: The robot starts outside the arena. When the door opens, the robot navigates to the \emph{Information Point}.
\end{itemize}

\subsection*{Additional rules and remarks}
\begin{enumerate}[nosep]
	\item \textbf{Deus ex Machina:}
	\begin{itemize}[nosep]
		\item \textbf{Bypassing speech recognition:} Bypassing ASR causes a score reduction of 50pts per guest.

		\item \textbf{Follower's assistance:} Helping the robot while guiding a guest (e.g.~waving, shouting) causes a score reduction of 50pts.
	\end{itemize}

	\item \textbf{Information Point:} The robot meets guests and gives directions at the \emph{Information Point} and then guides them to the location. The information point is not known beforehand. Right before a run at the entrance door, the team is told a position where they should start the robot.
	After guiding a person, it must return to the information point.

	\item \textbf{Naive Operators:} Optionally, questions can be asked by a \emph{Naive Operator}, i.e.~a person from the audience with no background on robotics. \emph{Naive Operators} are allowed to rephrase commands given by the referees.
	\\\textbf{Remark:} Referees are not allowed to instruct naive operators on how to operate the robot. \textbf{Teams attempting to instruct or bias the operator will be disqualified}.

	\item \textbf{Custom operator fallback:} If the robot consistently fails to understand the naive/professional operator (e.g.~3 times or more), teams can default to a custom operator.

	\item \textbf{Guiding people:} The guiding phase of the test requires the robot to provide a type of tour guide of its explanation (pointing out reference locations at each step e.g.~\emph{The couch is at your right.}). This information can later be used as reference for later explanations. Blocked pathways may be encountered and should be used to make subsequent guides faster.

	\item \textbf{Returning guests:} All guests will come back later to ask for other locations.
	If the robot is able to guide an operator using information based on previous tours (i.e.~it remembers where the person was sent to or avoids blocked paths), bonus points are awarded.
\end{enumerate}

\subsection*{OC instructions}
2 hours before the test:
\begin{itemize}
	\item Generate pairs of linked destinations
	\item Recruit at least 5 volunteers for the test
	\item Instruct volunteers where things are on the house
	\item Announce the location of the \emph{Information Point}.
\end{itemize}

\subsection*{Referee instructions}
The referee needs to
\begin{itemize}
	\item Assign names to each volunteer.
	\item Assign 2 linked destinations to each volunteer.
\end{itemize}

\subsection*{Score sheet}

The maximum time for this test is 10 minutes.

\begin{scorelist}
	\scoreheading{Main Goal}
	\scoreitem[3]{100}{Describe and show the requested location accurately}
	\scoreitem[3]{50}{Using naive operator}
	\penaltyitem[3]{50}{Using custom operator}

	\scoreheading{Bonus rewards}
	\scoreitem[3]{200}{Guide the 4th guest on}
	\scoreitem[3]{50}{Using naive operator}
	\penaltyitem[3]{100}{Using custom operator}
	\scoreitem[3]{100}{Recognize a person and give further instructions.}
	\scoreitem[6]{100}{Provide audio recording and transcripts}
\end{scorelist}

% Local Variables:
% TeX-master: "Rulebook"
% End:

